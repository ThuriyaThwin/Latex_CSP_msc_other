\section{Extensions and improvements}
Two variants of the algorithm have been discussed earlier in the thesis, with different purpose and functionality.
Benchmark data was produced for these variants as well, but on a limited subset of the problems tailored for the purpose of the specific variant.
These results will be reviewed to determine if the particular variant offers the expected improvement.

% [todo] - review and rewrite this
% [todo] - explaining why sets were chosen
% [todo] - analyzing the results
% [todo] - were results as expected?
% [todo] - is a given improvement useful?

\subsection{The \enquote{push} operation}
The purpose of the \enquote{push} operation is to increase solution quality of the approximative algorithm once a feasible solution has been found.
Six problem sets from \cref{tab:comparative-results} were therefore chosen for this benchmark, all with comparatively bad solutions (with solution differences close to or above \SI{1}{\percent}) and competitive runtimes.
The expectation was to obtain better solutions while maintaining competitive runtimes.

\Cref{tab:push-results} shows the results of benchmarking the \enquote{push} operation on the selected problems.
Surprisingly, the solution quality did not improve for a majority of the problems.
In fact, for some sets the solution quality was decreased, and the runtime of the algorithm improved instead (which, given the already competitive runtime of the standard algorithm, is an unwanted result).
In fact, the only problem set for which the excpected result was obained is the \gls{cfn} \emph{Pedigree} set.



Due to these results, the \enquote{push} operation is not as interesting when applied to the max-sum algorithm as it is in the original \gls{lp} formulation.
% [todo] - analyze reason of unexpected results!

% Interesting cactus plots: Mainly MaxCSP/BlackHole when compared to standard

\begin{table}
	\centering
	% Updated 2014-05-24
	\caption{
		Solution quality and runtime using the \enquote{push} operation.
		For several chosen problem sets, the \enquote{push} variant runtime is compared to the results obtained by the standard algorithm (see \cref{tab:comparative-results}).
	}
	% [todo] - double-check all table data!
	\label{tab:push-results}
	\begin{figcenter}
	\begin{tabular}{xySSS[round-mode=places,round-precision=3,scientific-notation=fixed,fixed-exponent=0]
				     S[round-mode=places,round-precision=3,scientific-notation=fixed,fixed-exponent=0]
				     S[round-mode=places,round-precision=2,scientific-notation=fixed,fixed-exponent=0]
				     S[round-mode=places,round-precision=2,scientific-notation=fixed,fixed-exponent=0]}
		\toprule
			{} & {} & \multicolumn{2}{c}{\(\#\) solved} & \multicolumn{2}{c}{Sol. diff. (\si{\percent})} & \multicolumn{2}{c}{Mean time (\si{\second})} \\
			\cmidrule(rl){3-4} \cmidrule(rl){5-6} \cmidrule(rl){7-8}
			{\normalsize Category} & {\normalsize Set} & {Std.} & {\enquote{Push}} & {Std.} & {\enquote{Push}} & {Std.} & {\enquote{Push}} \\
		\midrule
\acrshort{cfn}	&	Pedigree	&	10	&	10	&	1.804874e-00	&	1.51249400	&	2.3750	&	5.6070 \\
\acrshort{cvpr}	&	GeomSurf	&	600	&	600	&	2.091307e-00	&	2.09130700	&	0.0460	&	0.0425 \\
				&	SceneDecomp	&	715	&	715	&	7.545481e+01	&	75.4547890	&	0.0210	&	0.0170 \\
Max-\acrshort{csp}	&	BlackHole	&	37	&	37	&	9.009009e-01	&	1.08108100	&	58.8900	&	31.3310 \\
				&	Langford	&	4	&	4	&	1.311265e-00	&	1.55417600	&	70.7775	&	56.2925 \\
				&	QCP	&	60	&	60	&	1.292034e-00	&	1.30359500	&	43.2575	&	38.0705 \\
\acrshort{mrf}	&	ObjectDetection	&	37	&	37	&	6.465565e-00	&	6.46556500	&	279.8620	&	167.0170 \\
		\bottomrule
	\end{tabular}
	\end{figcenter}
\end{table}


\subsection{The greedy DP update}
The greedy DP update, obtained by fixing \(\alpha=1\) of the fractional DP update, should theoretically improve convergence at the expense of the optimality guarantee.
Here, a large number of problems from \cref{tab:comparative-results} were selected, all exhibiting good solution quality and a reasonable but uncompetitive runtime.
The expectation was to decrease runtime at the expense of solution quality.

\Cref{tab:greedy-dp-results} shows the results of benchmarking the greedy DP algorithm against the selected problems.
As expected, the runtime of all sets (except the \emph{Auction} set) was improved significantly --- between \num{4} and \num{350} times --- with little or no decrease in solution quality.
The runtime improvements are most significant for the \gls{cfn} and \gls{mrf} problems, where the greedy DP variant is faster than all three other solvers.


% [todo] - analyze reason of CFN/Auction not working!

With these results in mind, the greedy algorithm may be very useful when exact solutions aren't required.
% [todo] - mention fields where a "good enough" solution is satisfactory and why
This makes the algorithm with greedy updates extremely competitive in such cases.

% Note: for some problems, including CFN/ProteinDesign, no loss in accuracy but great decrease in time - discuss this!

% Good as part of a broader strategy, to quickly obtain upper bounds?

% Interesting cactus plots: MaxCSP/BlackHole, MaxCSP/QCP, MRF/DBN

\begin{table}
	\centering
	% Updated 2014-05-24
	\caption{
		Solution quality and runtime using the greedy DP update (setting \(\alpha=1\)).
		For several chosen problem sets, the greedy DP runtime is compared to the results obtained by the standard algorithm (see \cref{tab:comparative-results}).
		Problem sets marked with \textdagger{} include unsolved problems (no feasible solution found by the greedy DP update), and n/a values indicate that none of the problems in the set were solved.
	}
	% [todo] - double-check all table data!
	% [todo] - include CP/ParityLearning here?
	\label{tab:greedy-dp-results}
	\begin{figcenter}
	\begin{tabular}{xySSS[round-mode=places,round-precision=3,scientific-notation=fixed,fixed-exponent=0]
				     S[round-mode=places,round-precision=3,scientific-notation=fixed,fixed-exponent=0]
				     S[round-mode=places,round-precision=2,scientific-notation=fixed,fixed-exponent=0]
				     S[round-mode=places,round-precision=2,scientific-notation=fixed,fixed-exponent=0]}
		\toprule
			{} & {} & \multicolumn{2}{c}{\(\#\) solved} & \multicolumn{2}{c}{Sol. diff. (\si{\percent})} & \multicolumn{2}{c}{Mean time (\si{\second})} \\
			\cmidrule(rl){3-4} \cmidrule(rl){5-6} \cmidrule(rl){7-8}
			{\normalsize Category} & {\normalsize Set} & {Std.} & {\(\alpha=1\)} & {Std.} & {\(\alpha=1\)} & {Std.} & {\(\alpha=1\)} \\
		\midrule
\acrshort{cfn}	&	Auction\textdagger	&	102	&	0	&	0.000000e+00	&	{\textcolor{gray}{n/a}}	&	82.8575	&	{\textcolor{gray}{n/a}} \\
				&	CELAR\textdagger	&	10	&	4	&	9.081260e-07	&	0.00000000	&	193.3445	&	3.8775 \\
				&	ProteinDesign\textdagger	&	10	&	9	&	0.000000e+00	&	0.00000000	&	43.3995	&	0.7220 \\
				&	Warehouse\textdagger	&	38	&	53	&	0.000000e+00	&	0.00000000	&	55.8550	&	0.6780 \\
\acrshort{cvpr}	&	Matching	&	4	&	4	&	0.000000e+00	&	0.00000000	&	17.9275	&	4.5525 \\
Max-\acrshort{csp}	&	BlackHole\textdagger	&	37	&	36	&	9.009009e-01	&	0.01081081	&	58.8900	&	13.1665 \\
				&	Composed	&	80	&	80	&	1.342282e-01	&	0.00000000	&	20.3400	&	0.9980 \\
				&	Geometric	&	100	&	100	&	1.082434e-00	&	0.94082100	&	98.9760	&	13.3705 \\
				&	Langford	&	4	&	4	&	1.311265e-00	&	0.96711800	&	70.7775	&	7.3910 \\
				&	QCP	&	60	&	60	&	1.292034e-00	&	2.12260500	&	43.2575	&	4.4240 \\
\acrshort{mrf}	&	DBN	&	108	&	108	&	0.000000e+00	&	0.00000000	&	37.9040	&	0.4465 \\
				&	Linkage\textdagger	&	8	&	10	&	0.000000e+00	&	0.00000000	&	41.0700	&	3.6005 \\
				&	ObjectDetection	&	37	&	37	&	6.465565e-00	&	6.46556500	&		279.8620	&	0.8380 \\
		\bottomrule
	\end{tabular}
	\end{figcenter}
\end{table}
