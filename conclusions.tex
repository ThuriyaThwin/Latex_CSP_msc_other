In this final chapter I will attempt to consolidate the results of the thesis work, conclusively addressing the objective and purpose of the thesis itself.
I will do this by identifying already apparent applications of the in-the-middle algorithm within the field of graphical models while also identifying the limitations and weaknesses of the algorithm, and by pointing out some areas of further research which I believe are tractable and meaningful to pursue when it comes to applying the algorithm to optimization within that field.

\section{Applications}
Since the performance of the algorithm varies widely between different problem sets, one cannot immediately identify any single field in which the algorithm would be decisively competitive.
However, the good overall performance and low optimality gap of the algorithm --- especially when applied to problem sets in which other solvers struggle --- suggest that the algorithm may be a good candidate for a so-called portfolio approach, wherein several solvers are applied in parallel.

The same properties, especially the good optimality gap, suggest that the algorithm may be useful in applications where good upper bounds on the solution are required.
Since the greedy \gls{dp} variant of the algorithm provides such upper bounds quickly, that variant of the algorithm may have applications in branch-and-bound applications.
In combination with an exact solver, the in-the-middle algorithm may serve as an approximative anytime solver providing good approximative solutions while waiting for the exact solver to supply truly optimal solutions.

% [review] - more?

\section{Further research}
The algorithm shows promise in several categories of those included in the benchmark.
In these fields it would therefore be of interest to extend the algorithm and develop the underlying theory to further improve performance in these fields.

Previous work, mainly from \gls{inra}, has provided strong theory on graphical model optimization especially with respect to arc consistency \parencite{Cooper10,Cooper08,deGivry06,deGivry05}.
The in-the-middle algorithm, as previously detailed, uses a relatively weak arc consistency property (general arc consistency).
If stronger notions such as existential \parencite{deGivry05} or optimal soft \parencite{Cooper10} arc consistency can be extended to the framework used by the in-the-middle algorithm, it may increase runtime performance and solution quality.

Another area requiring further research is the design of the constraint updates, especially their effect on the optimality guarantee of the algorithm.
A common situation for the algorithm is that the heuristic finds optimal solutions but cannot guarantee the optimality, which in turn means the algorithm wastes time in subsequent trials.
Constructing constraint updates with better guarantees or providing stronger theory on other parts of the algorithm may therefore improve the performance of the algorithm.
Additionally, common constraint types such as the \emph{all-different} constraint may benefit from specialized constraint component updates.

There are also many tricks, some of them used in \gls{lp} implementations of the in-the-middle heuristic, that may be applied to the algorithm.
In this benchmark, only tie-breaking using noise was introduced.
Futher modifications that would require further research include not updating constraint components for which the associated variable components are unchanged and using the existing \gls{lp} or set-partitioning constraint update \parencite[\pno~102]{Wedelin08} for constraints of that type.
Both these modifications may improve runtime for some problems.

Finally, specializations of the implementation may prove interesting in some areas.
For example, the \gls{wpms} category of problems would benefit from a formulation where the binary variables are stored in a single, shared variable component element instead of two different ones as in the current implementation.
Similar simplifications based on various assumptions of the problem structure may be introduced in the constraint component updates.



