% arara: xelatex
% arara: biber
% arara: makeglossaries
% arara: xelatex
% arara: xelatex
\documentclass[paper=A4,DIV=calc]{scrbook}
\usepackage{amsmath}
%% FONTS (fold)
\usepackage{ifxetex}
\ifxetex
	\usepackage[no-math]{fontspec}
	\usepackage[oldstyle,tabular,semibold]{libertine}
	\usepackage[libertine,slantedGreek,liby,bigdelims]{newtxmath}
	\setmonofont[Scale=0.9]{Anonymous Pro}
	\usepackage{polyglossia}
	\setmainlanguage{english}
	\setotherlanguage{swedish}
\else
	\usepackage[utf8]{inputenc}
	\usepackage[T1]{fontenc}
	\usepackage[oldstyle,tabular,semibold]{libertine}
	\usepackage[libertine,slantedGreek,liby,bigdelims]{newtxmath}
	\usepackage[ttdefault,scale=0.9]{AnonymousPro}
	\usepackage[swedish,english]{babel}
\fi
%% (end)
% tikz stuff
\usepackage{tikz}
\usetikzlibrary{calc,arrows,positioning,shapes}
\tikzset{>=stealth,pointer/.style={->,thick,shorten <=2pt,shorten >=2pt}}
% subcaptions
\usepackage{caption,subfig}
\captionsetup[subfigure]{font={it,small},labelfont={bf,small}}
% more packages
\usepackage{chs-msc-thesis}
\usepackage{csquotes}
\usepackage{array}
%\usepackage{tabu}
\usepackage{booktabs}
\usepackage[algoruled,algochapter,linesnumbered]{algorithm2e}
\usepackage{ntheorem,xfrac}
\usepackage{mathtools}
\usepackage[binary-units=true]{siunitx}
\usepackage[authordate,backend=biber,firstinits=true,useprefix=true]{biblatex-chicago} % noibid?
\usepackage[acronym,nomain,nonumberlist,acronymlists={hidden}]{glossaries}
\usepackage{hyperref}
\usepackage{cleveref}

\let\C\undefined
\usepackage[commonsets]{skmath}
\makeatletter
% http://tex.stackexchange.com/a/70498/66
\newcommand*\placeaccent[2]{%
	\begingroup
		\def\acc@hat{\mbox{\raisebox{-1.27ex}[0ex][0ex]{\^{}}}}
		\def\acc@dot{\kern-0.08em.\kern-0.08em}%
		\def\acc@skip{%
			\ifx\macc@style\displaystyle0.32
			\else\ifx\macc@style\textstyle0.32
			\else\ifx\macc@style\scriptstyle0.22
			\else0.15\fi\fi\fi ex
		}%
		\def\mathaccent##1##2{%
			\setbox6\hbox{$\m@th\macc@style#1$}%
			\@tempdima\wd4
			\advance\@tempdima\macc@kerna
			\advance\@tempdima-\wd6
			\divide\@tempdima\tw@
			\@tempdimb\z@
			\ifdim\@tempdima<\z@ \@tempdimb-\@tempdima \@tempdima\z@ \fi
			\vbox{%
				\offinterlineskip
				\moveright\@tempdima\box6
				\kern\acc@skip
				\moveright\@tempdimb\box4
			}%
		}%
		\macc@depth\@ne
		\let\math@bgroup\@empty \let\math@egroup\macc@set@skewchar
		\mathsurround\z@ \frozen@everymath{\mathgroup\macc@group\relax}%
		\macc@set@skewchar\relax
		\let\mathaccentV\macc@nested@a
		\macc@nested@a\relax111{#2}%
	\endgroup
}
\renewcommand*\dot[1]{%
	\placeaccent{\acc@dot}{#1}%
}
\renewcommand*\ddot[1]{%
	\placeaccent{\acc@dot\mkern1.4mu\acc@dot}{#1}%
}
\renewcommand*\hat[1]{%
	\placeaccent{\acc@hat}{#1}%
}
% http://tex.stackexchange.com/a/22134/66
\renewcommand*\bar[1]{%
	\mkern 1.5mu\overline{\mkern-1.5mu#1\mkern-1.5mu}\mkern 1.5mu%
}
\makeatother

%\NewDocumentCommand\R{}{\ensuremath{\mathbb{R}}}
\NewDocumentCommand\Nplus{}{\ensuremath{\N_{\geq0}}}
\NewDocumentCommand\Nminus{}{\ensuremath{\N_{\leq0}}}
\NewDocumentCommand\Rplus{}{\ensuremath{\R_{\geq0}}}
\NewDocumentCommand\Rminus{}{\ensuremath{\R_{\leq0}}}
\DeclareMathOperator\costoperator{\mathsf{cost}}
\NewDocumentCommand\cost{m}{\costoperator\left(#1\right)}
\DeclareMathOperator\sgnoperator{\mathrm{sgn}}
\NewDocumentCommand\sgn{m}{\sgnoperator\left(#1\right)}

\newcolumntype{H}{>{\setbox0=\hbox\bgroup}c<{\egroup}@{}}
\newcommand*\rowcolor[1]{\rowfont{\color{#1}}}


\newacronym{csp}{CSP}{constraint satisfaction problem}
\newacronym{wcsp}{WCSP}{weighted constraint satisfaction problem}
\newacronym{vcsp}{VCSP}{valued constraint satisfaction problem}
\newacronym{cfn}{CFN}{cost function network}
\newacronym{lp}{LP}{linear programming}
\newacronym{mip}{MIP}{mixed integer programming}
\newacronym{mrf}{MRF}{Markov random field}
\newacronym{wpms}{WPMS}{weighted partial max-SAT}
\newacronym{cp}{CP}{constraint programming}
\newacronym{cvpr}{CVPR}{Computer Vision and Pattern Recognition}
\newacronym{map}{MAP}{maximum posterior}
\newacronym{sat}{SAT}{boolean satisfiability}
\newacronym{maxsat}{Max-SAT}{maximum satisfiability}
\newacronym{scsp}{SCSP}{semiring-based constraint satisfaction problem}

% [todo] - make sure acronyms throughout the report use glossaries


\theoremstyle{plain}
\newtheorem{theorem}{Theorem}
\theoremstyle{definition}
\newtheorem{definition}{Definition}
%\newtheorem{example}{Example}

\DontPrintSemicolon
\sisetup{locale=DE,detect-all}

\makeatletter
% equivalent to dashed=false in standard biblatex styles
\AtEveryBibitem{\global\undef\bbx@lasthash}
%% fixes the "strange" (apparently normal) order of names in author list
%\DeclareNameAlias{sortname}{last-first}
% fixes doi formatting
\DeclareFieldFormat*{doi}{\textsc{doi}\addcolon\space\textsc{#1}}
\makeatother
\ExplSyntaxOn
% circumvents lack of italic small-caps by redefining title styles
\titleformat{\section}[block]{
	\vspace*{.25\baselineskip}\centering
	\__unless_usekomafont:n{\large}
	\__maybe_usekomafont:n{section}
}{\thesection}{1ex}{\__maybe_textssc:n{\MakeLowercase{#1}}}[\vspace{-.5\baselineskip}]
%\cs_if_exist_use:NT\addtokomafont{{section}{\sffamily}}
%\cs_if_exist_use:NT\addtokomafont{{subsection}{\sffamily\normalsize}}
%\titleformat{\subsection}[hang]{
%	\__unless_usekomafont:n{\large}
%	\__maybe_usekomafont:n{subsection}
%}{\llap{\thesubsection\hspace{1ex}}}{0pt}{{#1}}[\vspace{-.5\baselineskip}]
%\cs_if_exist_use:NT\addtokomafont{{subsubsection}{\sffamily}}
%\titleformat{\subsubsection}[hang]{
%	\__maybe_usekomafont:n{subsubsection}
%}{\llap{\thesubsubsection\hspace{1ex}}}{0pt}{\textit{#1}}[\vspace{-.5\baselineskip}]
\ExplSyntaxOff

\SetupMetadata{
	% Required metadata
	%title = {Solving the max-sum problem with the in-the-middle heuristic},
	title = {Evaluating the in-the-middle heuristic in \mbox{Graphical~Model} \mbox{Discrete~Optimization}},
	author = {Simon Sigurdhsson},
	year = {2014},
	subject = {Engineering Mathematics},
	department = {Department of Computer Science and Engineering},
	keywords = {%
	    Combinatorial optimization,
	    Integer linear programming,
	    Undirected graphical models,
	    Markov random fields,
	    Weighted constraint satisfaction,
	    Constraint programming,
	    Wedelin heuristic,
	    Max-SAT,
	    Max-sum
	},
	abstract = {%
		The field of graphical modeling has been very active in recent years, which is unsurprising given the large range of problems which may be described using graphical model.
		While many novel algorithms have been presented, there are still areas of the field in which improvements are possible.
		Recent reviews have uncovered strong links between the linear programming and graphical modelling fields, and it is therefore of interest to survey the possible application of linear programming methods to graphical model optimization.

		The in-the-middle algorithm, an approximative solution method in linear programming which has seen extensive use in the industry, has recently been extended to max-sum problems.
		Graphical models are easily translated into max-sum problems, and the in-the-middle is therefore an ideal candidate in reformulating linear programming algorithms to the field of graphical model optimization.

		This thesis presents an implementation of the in-the-middle algorithm applied to max-sum problems, which may be applied to general graphical model optimization instances.
		The implementation is benchmarked agains three existing high-performance exact solvers in the field, using a problem set consisting of several hundred problems.
		Results indicate that the in-the-middle algorithm may have potential in the field of graphical model optimization, but that further research is required to make the algorithm competitive.
		Several avenues for further research on the algorithm are proposed.
	},
	% Optional metadata
	% [todo] - make sure all this optional metadata is correct
	%issn = {},
	%subtitle = {},
	%publisher = {},
	series = {2014:?},
	%cover-caption = {Bildtext till framsidebilden},
	%cover-image = {\includegraphics{...}}
}

\addbibresource{references.bib}

\begin{document}
	\frontmatter
	\makefrontmatter

	% [review] - include acknowledgements?
	% \cleardoublepage
	% \section*{Acknowledgements}

	\cleardoublepage
	\tableofcontents
	\cleardoublepage
	\printglossary[type=\acronymtype,]
	
	% [fix] - check grammatical tense, grammatical person, etc. (check writing guide)
	% [fix] - use "I" for first-person
	% [fix] - check overall conformance to writing guide
	% [fix] - spell-check entire report (use aspell?)
	% [fix] - make one last pass re-ordering figures and tables

	\mainmatter

	\chapter{Introduction}
	Optimization is a very relevant topic in modern industry, with applications to planning, cost-efficiency and many other areas.
% [todo] - expand above paragraph(s)

% [todo] - insert purpose / objective

% [review] - this is repeated later under background?
This thesis will expand a known \acrlong{lp} omptimization algorithm which has seen extensive use in industry, the in-the-middle algorithm\footnote{Originally proposed by \textcite{Wedelin95}.} (and its accompanying heuristic), to a wider set of problems in \acrlong{cp} and \acrlong{csp}.
This will be done by applying a recently developed reformulation of the algorithm which applies to max-sum problems, with a suitable transformation of \acrlong{cp} and \acrlong{csp} instances into max-sum instances.

An implementation of this reformulated algorithm will be implemented and benchmarked against solvers in the \acrlong{cp} and \acrlong{csp} fields.
This will provide valuable insights which will determine the usefulness of this reformulated algorithm in the fields previously mentioned.

% [todo] - limitations

	\section{Background}
The in-the-middle algorithm \parencite{Wedelin95} for \acrlong{lp} problems can be interpreted as a coordinate descent algorithm, and with a simple heuristic it has been shown to solve integer programming problems with very good performance and quality in large scale applications.
Recent work has extended this algorithm to also apply to max-sum problems \parencites{Wedelin08}{Wedelin13}, but this application has not yet been benchmarked against other algorithms and software in that field.
The primary aim of this thesis is therefore to implement and benchmark the in-the-middle algorithm applied to max-sum problems.

When formulated to solve max-sum problems, the algorithm is applicable to a host of problems which may be restated as such problems, including but not limited to (weighted, valued and regular) \gls{csp}, \gls{cfn} and \gls{wpms} problems.
Additonally, though a simple logarithmic transformation of the problem, \gls{mrf} problems may also be restated as max-sum problems.
This makes the algorithm (including the approximative heuristic variant) interesting in fields ranging from scheduling, protein folding and \(n\)-queen puzzles to classification and pattern regconition.

The \gls{csp} and \gls{maxsat} communities have been very active in recent years, producing benchmarks of solvers in those fields at the \emph{Interntational Conference on Theory and Applications of Satisfiability Testing} \parencite{Argelich11} and the \emph{Annual Congress of the French Society of Operations Research and Decision Support} \parencite{Allouche14b}.
Using the results of such benchmarks, the in-the-middle algorithm on max-sum problems may be compared to existing algorithms in the mentioned fields, determining the interest of developing, analyzing or specializing the algorithm further for such problems.

	\section{Related work}
Since the purpose of this thesis is to extend an existing algorithm and benchmark it against known solvers, it is interesting to note previous work both related to the algorithm itself (either its \gls{lp} formulation or the max-sum variant discussed in this thesis) as well as previous benchmarks and recent algorithms in the related fields.

Previous work on the algorithm itself, especially the \gls{lp} variant, will be implemented and examined to determine if known modifications of the algorithm apply to the max-sum variant as well.
Benchmarks in the \gls{csp} field will be surveyed to collect suitable problem sets and to pick appropriate solvers to benchmark the algorithm against.

\subsection{The in-the-middle algorithm}
Although the in-the-middle algorithm has mostly been discussed by its principal author \parencites{Wedelin95}{Wedelin08}{Wedelin13}{Alefragis00}, there is some litterature evaluating and improving upon the algorithm available.

\Textcite{Bastert10} notes the practical relevance of the field of heuristics for integer programming in recent years, and evaluates the in-the-middle algorithm applied to such problems. In addition to explicitly providing the generalizations mentioned by \textcite{Wedelin95}, they also introduce a \enquote{push} operation intended to improve the quality of the approximative solutions. Finally, they also compare the algorithm with commercial software, with fairly favourable results.

\Textcite{Ernst05} applied a method based on the Lagrangian relaxation used by the in-the-middle algorithm to the task allocation problem, with favourable results compared to the commercial CPLEX \gls{lp} solver.
Similarly, \textcite{Mason01} applied a specialized variant of the in-the-middle algorithm to personnel scheduling, outperforming CPLEX on difficult commercial rostering problems.

\subsection{Max-sum and constraint satisfaction problems}
% [todo] - review this section
Graphical models allow the modelling of a range of NP-hard optimization problems in a consistent manner \parencite{deGivry14}, and recent articles on the subject expose strong connections between linear programming and graphical models \parencites{Werner07}{Kolmogorov13}.
As such, there is reason to believe that the application of linear programming algorithms, especially approximative ones, to graphical models may have practical relevance.

Although several specialized solvers --- \emph{e.g.} Toulbar2 \parencite{Allouche10}, WPM2 \parencite{Ansotegui13b}, MaxHS \parencite{Davies13}, Max-DPLL \parencite{Larrosa08} and several others --- have been developed for graphical models in recent years, benchmarks still indicate that there are areas of graphical model optimization in which \gls{lp} solvers perform better, and areas where no existing solvers excel.

\Textcites{Allouche14b}{deGivry14} evaluate several exact optimizers to graphical model optimization problems, and the results indicate that there may be several problem domains (most notably \acrshort{mrf}, \acrshort{wpms} and Max-\acrshort{csp}) which could benefit from fast, approximative solvers.
Additionally, several large problem sets from each domains are included in the benchmark, and these problem sets have also been made available online by the authors\footnote{\url{http://genoweb.toulouse.inra.fr/~degivry/evalgm} \parencite[\pno~7]{Allouche14b}.}.


	% Reset glossaries "first use" for all acronyms after the introduction
	\glsresetall

	\chapter{Theory}
	This chapter will introduce some of the theory behind max-sum problems, \glspl{csp} and related fields.
We will see that there are significant similarities between these two problems, and a detailed translation of \gls{csp} (and similar problems) to equivalent max-sum formulations will be provided.
This translation will be used to apply the in-the-middle algorithm to \glspl{csp}.

With this background given, the in-the-middle algorithm will be introduced. First, the original \gls{lp} formulation will be briefly described.
Then, the max-sum variant will be thoroughly explained\footnote{The algorithm will be described in theory, with practical implementation issues and considerations being discussed in the next chapter.}, and several extensions and modifications of the algorithm will be introduced.
Finally, the theoretical framework surrounding the algorithm will be compared to theoretical results in a \gls{csp} context.

% [review] - review this entire file

	\section{Max-sum problems and constraint satisfaction}
The formulation of the in-the-middle algorithm provided by \textcite{Wedelin08} is based on the definition of a max-sum problem.
In order to apply this algorithm to general \gls{wcsp}, it is necessary to provide a link between the max-sum problem and problem formulations within the graphical model optimization field, where \gls{wcsp} is one of the more general formulations.
% [fix] - Wedelin: "what are the fundamental problem formulations really?"
This will be done by first introducing the max-sum problem along with some useful interpretations in other fields, followed by a similar introduction of the \gls{wcsp} family.
Finally, a theoretical link between the two fields will be provided, along with an explicit method of translation between the formulations.

\subsection{Max-sum problems}
The general max-sum problem is an NP-hard optimization problem with many applications in fields ranging from statistical physics to artificial intelligence and pattern recognition.
In general, and constrained optimization problem (crucially, several variants of constraint satisfaction) may be restated as max-sum problems.
Formally, the following definition \parencite[based on that of][\pno~11]{Wedelin13} may be used:
\begin{definition}[Max-sum problem] \label{def:max-sum}
	The max-sum problem is the optimization problem
	\begin{equation*}
		\max*[x] f(x) = \sum_{g_k\in G} g_k(x^k),
	\end{equation*}
	where \(g_k(x^k) \in \R\) are distinct arbitrary functions over \(x^k \subseteq x\).
\end{definition}

The max-sum problem then has three distinct components by which it is defined:
\begin{itemize}
	\item A finite set of \emph{variables} \(X = \{x_1, \dotsc, x_n\}\). Let \(X^k \subseteq X\) denote a specific subset of \(X\), and let \(x^k \in X^k\) and \(x \in X\) be \emph{assignments} of the variables.
	\item \emph{Domains} of the variables, \(D = \{D_1, \dotsc, D_n\}\), such that \(x_i \in D_i, \forall i\). Subsets \(D^k \subseteq D\) may be defined analogously to the variable subsets.
	\item A finite set \(G\) of \emph{cost functions} or \emph{components}. Every cost function \(g_k \in G\) is defined over a variable subset \(X^k\), \emph{i.e.} it is a map \(g_k : D^k \mapsto \Rminus \cup \{-\infty\}\).
\end{itemize}
The cost functions may additionally be divided into three kinds: those defined on the empty set (\emph{constants}), those defined on singleton subsets \(X^k = \{x_i\} \subseteq X\) (\emph{variable components}) and those defined on larger subsets (\emph{constraint components}) --- this is the \emph{component model}\label{pg:component-model} introduced by \textcite[\pno~98]{Wedelin08}.

One should also note that the co-domains of \(g_k\) need not be restricted to \(\Rminus \cup \{-\infty\}\), but for the purposes of this thesis that is the chosen output.
% [fix] - The above is undelined by Wedelin, why?
This allows solutions \(x\) to the max-sum problem to be ordered by their cost, where the cost is defined as \(\cost{x} = f(x) = \sum_k g_k(x^k)\), and additionally allows the definition of \emph{infeasible} solutions to the max-sum problem as those for which \(\cost{x} = -\infty\).
This will be useful in the translation between \glspl{wcsp} and max-sum problems.

There are several algorithms available for solving max-sum problems, with \textcite{Werner07} mentioning the \emph{augmented DAG algorithm}, the \emph{max-sum diffusion algorithm} and a \gls{lp} relaxation method.
In addition to those direct methods, the relation to \gls{csp} provides many more (which will be mentioned later), and algorithms such as \emph{belief propagation} and \emph{message passing} are also applicable to some max-sum problems.

\subsubsection{Markov Random Fields}
A restricted variant of the max-sum problem called the \emph{binary max-sum labelling problem} has direct applications to artificial intelligence and pattern recognition, where the problem is known as computing the \gls{map} configuration of \acrlongpl{mrf} \parencite[\pno~1165]{Werner07}.

\subsection{Weighted constraint satisfaction problems}
\Glspl{csp} are very general decision problems, defined through a set of objects which must satisfy a set of constraints.
Many problems in artificial intelligence and operations research (including planning and resource allocation) may be stated as \glspl{csp}, as well as several academic problems such as \gls{sat}, queens puzzles and map colouring.
One of the corresponding combinatorial optimization problems\footnote{Both the \gls{vcsp} and \gls{scsp} frameworks may be seen as the corresponding optimization problems \parencites{Meseguer06}{Bistarelli99}, but \glspl{wcsp} may be described using either.}, the \gls{wcsp}\footnote{In some literature these problems are called \glspl{cfn}, but the definitions are in essence equivalent.}, additionally introduces \emph{weights} on the constraints, and classify these as \emph{hard} (must be satisfied) or \emph{soft}.
All \glspl{csp} may of course be restated as \glspl{wcsp} with only hard constraints, while the relaxed objective of the combinatorial optimization approach allows even over-constrained \glspl{csp} to be \enquote{solved}.
Additionally, many problems in complexity theory such as \gls{maxsat}, max-clique, max-cut and minimal vertex cover may be modelled using \glspl{wcsp} \parencite[\pno~315]{Meseguer06}.
Due to these facts, \glspl{wcsp} may be regarded as more interesting than \glspl{csp}, especially in an optimization context.

There are several ways to formally define a \gls{wcsp}, but the one used here closely matches the definition of a max-sum problem, which simplifies the formal translation between the two. The definition is based on that presented by \textcite{Meseguer06}, which defines \gls{wcsp} \parencite[\pno~284]{Meseguer06} in terms of a regular \gls{csp} \parencite[\pno~281]{Meseguer06}:
\begin{definition}[\Acrlong{csp}] \label{def:csp}
	A \gls{csp} is a decision problem consisting of three parts:
	\begin{itemize}
		\item A finite set of \emph{variables} \(X = \{x_1, \dotsc, x_n\}\). Let \(V \subseteq X\) denote a specific subset of \(X\).
		\item \emph{Domains} of the variables, \(D = \{D_1, \dotsc, D_n\}\), such that \(x_i \in D_i, \forall i\). Subsets \(D^V \subseteq D\) may be defined analogously to the variable subsets.
		\item A finite set \(C\) of \emph{constraints} \(R_V\in C\) defined by a \emph{relation} \(R\) defined on a subset of variables \(V\subseteq X\), which specify the assignments of \(V\) allowed by the constraint.
	\end{itemize}
	The problem is to find an \emph{assignment} \(t\) which is allowed by all constraints \(R_V\in C\).
\end{definition}

The reformulation as an optimization problem mainly concerns the introduction of \emph{weights}, and a reformulation of the objective:
\begin{definition}[\Acrlong{wcsp}] \label{def:wcsp}
	A \gls{wcsp} (denoted by \textcite[\pno~284]{Meseguer06} as a \emph{k-weighted constraint network}) is a 4-tuple \(\langle X, D, C, k\rangle\), where \(X\) and \(D\) are variables and domains as in \cref{def:csp}, \(C\) is a set of \emph{weighted} constraints and \(k\) is an upper bound.
	A weighted constraint \(f_V\in C\) maps a subset \(V\) of variables to the set \([0,k]\), \emph{i.e.} \(f_V : D^V \mapsto [0,k]\).
	The \emph{cost} of an assignment \(t\) is defined as the (bounded) sum of all \(f_V\), and the optimization problem amounts to
	\begin{equation*}
		\min*[t] \cost{t} = \sum_{f_V\in C} f_V(t^V).
	\end{equation*}
\end{definition}

The attentive reader will notice the similarity between \cref{def:wcsp} and \cref{def:max-sum}.
Using this definition, \emph{feasible} solutions are assignments \(t\) for which \(\cost{t} < k\).
One may also make a distinction between \emph{hard} constraints (\(f_V(t^V) = k\) for some assignment(s) \(t^V\)) and \emph{soft} constraints.

\subsubsection{Max-CSP and (weighted) max-SAT}
The special case of \gls{wcsp} where all constraints have either unit or zero cost (\emph{i.e.} \(f_V : D^V \mapsto {0,1}\), regardless of choice of \(k\)) is normally referred to as max-\gls{csp} \parencite[\pno~284]{Meseguer06}.
Here, the objective value is exactly the number of violated clauses, and as such it is the most natural formulation of existing \gls{csp} instances as optimization problems.

The special case where all constraints are clauses of a Boolean formula (and weights are unrestricted) yields the \gls{wpms} problem \parencite[\pno~2]{deGivry14}.
If the weights are restricted to unit or zero costs as above, the problem becomes the well-known \gls{maxsat} problem \parencite[\pno~284]{Meseguer06}.

This makes explicit the fact that many real-world and academic problems in satisfiability and operations research may be restated as \gls{wcsp} (and therefore max-sum) problems, and consequently that the in-the-middle algorithm may present a viable alternative to solving these problems using \gls{lp} formulations \parencites{Ansotegui13a}{Davies13} or special algorithms \parencites{Ansotegui13b}{Larrosa08}.

\subsection{Translating WCSP to max-sum}
The translation from \gls{wcsp} to max-sum, when using the definitions given above, is fairly straight-forward.
In addition to the superficial similarity between \cref{def:wcsp,def:max-sum}, the two problems have deep connections and are in a sense equivalent.
When regarding the \gls{wcsp} formulation as an instance of \gls{scsp} \parencite[\pno~285\psq]{Meseguer06}, the \gls{wcsp} has an associated ordered semiring \(\langle [0,k],+^k,\min*,\leq\rangle\) \parencite[\pno~290]{Meseguer06}.
\Textcite[\pno~1167]{Werner07} noted that the max-sum problem is also associated with a similar ordered semiring structure \(\langle (-\infty,\infty),+,\max*,\geq\rangle\) (although in our case the set is \(\Rminus \cup \{-\infty\}\)).
\Citeauthor{Werner07} additionally provides a connection between max-sum and the regular \gls{csp} through a labelling problem, of which they are both special cases.

In fact, it seems that the only differences between the two problems as stated is the definition of the set included in the semiring --- while the \gls{wcsp} from \cref{def:wcsp} considers a domain \(\{0,\dotsc,k\}\), the max-sum problem according to \cref{def:max-sum} has domain \(\Rminus \cup \{-\infty\}\) --- and the fact that one is a minimization problem while the other is a maximization problem.
However, by transforming the weighted constraints of the \gls{wcsp} problem in such a way that the ordering is preserved but reversed (\emph{i.e.} by negating all costs), and additionally setting \(k=-\infty\) (and changing the costs of all hard constraints accordingly), a formulation which has a semiring \(\langle\Rminus \cup \{-\infty\},+,\max*,\geq\rangle\) is obtained, which is exactly what the max-sum problem has.
From there, it is only a matter of translating the constraints of the \gls{wcsp} into corresponding cost functions in the max-sum formulation.

The translation from \gls{wcsp} to the equivalent max-sum problem may be summarized by a few steps:
\begin{enumerate}
	\item Variable sets and domains are kept in the translation; the sets \(D\), \(X\) and \(X^V\) in \cref{def:wcsp,def:csp} correspond directly to the sets \(D\), \(X\) and \(X^k\) of \cref{def:max-sum}. Additionally, the subsets \(V\) of the \gls{wcsp} problem correspond to the subsets \(k\) in \cref{def:max-sum}.
	\item The weighted constraints \(f_V\) of \cref{def:wcsp} are replaced by new constraints \(f'_V\) defined as
	\begin{equation*}
		f'_V(t) = \begin{cases}
			-\infty &\quad \text{if \(f_V(t) < k\)} \\
			-f_V(t) &\quad \text{otherwise}
		\end{cases}, \quad \forall t.
	\end{equation*}
	\item The cost functions \(g_k\) of the max-sum formulation are constructed from the new constraints, \emph{i.e.} \(g_k = f'_V\), as appropriate.
\end{enumerate}

	\section{The in-the-middle algorithm}
\subsection{Original LP formulation}
\subsection{Max-sum formulation}
\subsection{Extensions and improvements}
\subsubsection{The \enquote{push} operation}
\subsection{Interpretations in other fields}


	\chapter{Method}
	In this chapter, specific implementation details relating to the algorithm as used in the benchmark will be presented.
This includes considerations which are of great importance when implementing efficient variants of the algorithm, including memory considerations as well as computational ones.
The implementation details are intended to facilitate independent implementations of the algorithm.

Additionaly, the method used in the benchmark will be explained (including hardware used).
The problem sets used in the benchmark will be introduced, including a short background and some characteristics of each set.
Finally, the other solvers used in the benchmark will be briefly introduced including a short review of their underlying algorithms.

	\section{Implementation}
The algorithm --- including reading routines for the \textsc{wcsp} file format \parencite{wcspformat}, parameter sweep strategy and timing facilities --- was implemented in C++11.
The implementation is partly based on an existing framework for the original \gls{lp} in-the-middle algorithm.
The code was compiled using version 5.1 of the LLVM compiler, with all safe optimizations enabled (\emph{i.e.} \texttt{-O3}).

There are several implementation details which are highly relevant to the performance of the algorithm, and this section will explore such details in depth.
In particular, the choice of data structure for constraint component data as well as implementation of constraint updates makes significant impact on the runtime of the algorithm.

A parameter sweep strategy used in conjunction with the algorithm, which controls the \(\alpha\) parameter of the fractional \gls{dp} update, will also be introduced and explained in further detail.

\subsection{Constraint component design decisions}
Several design decisions in the implementation of the constraint components have significant impact on the performance of the constraint component updates.
These design decisions mostly relate to the data structure representing costs inside the constraint component, and the main concern in selecting this data structure is quick access in the update loop.
Since the constraint component is kept constant through all iterations, and the temporary cost table \(h\) can be made implicit, this is the only major concern.

Another implementation detail to note is the storage of the (modified) variable components \(\bar{\vcomp}_i\).
Storing these sequentially in memory is a good choice, but care needs to be taken when ordering the variable components in memory --- due to CPU cache characteristics, storing variable components used in the same constraint component next to each other is highly beneficial.
However, the implementation used here does not reorder variable components in this manner, instead storing them in the order they have been defined in the problem input.

A good data structure for the constraint components \(\ccomp_k\) is a vector with a sparse representation of the table costs (omitting infeasible values, \emph{i.e.} \(\ccomp_k(x^k) = -\infty\)).
These values may be implicitly represented by proper initialization of variables in the constraint update.
The vector representation may be described as a list of pairs \(\langle x^k, \ccomp_k(x^k)\rangle\), where the (fixed) variable values \(x^k\) additionally may be used to access corresponding variable component values \(\bar{\vcomp}_i(x^k_i)\).
In the actual implementation, \(x^k_i\) are represented as pointers to the values \(\bar{\vcomp}_i(x^k_i)\).

\Cref{proc:frac-dp-update-fast} highlights these implementation details, and also shows the use of an invariant transformation of the variable components \(\bar{\vcomp}_i(x^k_i)\) that allows the use of existing code to detect solution changes by counting sign changes.

\begin{algorithm}[tbp]
	\SetKwFunction{UpdateConstraint}{UpdateConstraint}
	\Fn{\UpdateConstraint{\(\ccomp_k\), \(\bar{\vcomp}^k\), \(s^k\)}}{
		\KwData{A constraint component \(\ccomp_k\), (pointers to) variable components \(\bar{\vcomp}^k\) and offsets \(s^k\)}
		\KwResult{Updated variable components \(\bar{\vcomp}^k\) and offsets \(s^k\), number of sign changes \(c\)}
		\(c \leftarrow 0\)
		\ForAll{\(\bar{\vcomp}_j\in \bar{\vcomp}^k\)}{
			\(r^k_j \leftarrow \bar{\vcomp}_j - s^k_j\) \;
			\(s^k_j \leftarrow -\infty\) \tct*[r]{Allows omitting \(\ccomp_k(x^k) = -\infty\)}
			\(\vcomp^+_j, \vcomp^-_j \leftarrow -\infty\) \tct*[r]{Used in transforming \(\bar{\vcomp}_j\)}
		}
		\ForAll{pairs \(\langle x^k, \ccomp_k(x^k)\rangle\)}{
			\(v \leftarrow \sum_{x^k_i \in x^k} \vcomp_i(x^k_i)\) \;
			\(v \leftarrow \alpha(v + \ccomp_k(x^k))\) \;
			\ForAll{\(x_j\in x^k\)}{
				\(s^k_j \leftarrow \max{s^k_j, v}\) \;
				\uIf{\(v > \vcomp^+_j\)}{
					\(\vcomp^-_j \leftarrow \vcomp^+_j\) \;
					\(\vcomp^+_j \leftarrow v\) \;
				}
				\ElseIf{\(v > \vcomp^-_j\)}{
					\(\vcomp^-_j \leftarrow v\) \;
				}
			}
		}
		\lForAll(\tct*[r]{Transforms \(\bar{\vcomp}_j\)}){\(\bar{\vcomp}_j\in \bar{\vcomp}^k\)}{
			\(\bar{\vcomp}_j \leftarrow \bar{\vcomp}_j - (\vcomp^+_j + \vcomp^-_j)/2\)
		}
		\ForAll{\(\bar{\vcomp}_j\in \bar{\vcomp}^k\)}{
			Increment \(c\) by \(\#\{\bar{\vcomp}_j : \bar{\vcomp}_j \cdot s^k_j \leq 0\}\) \tct*[r]{Counts number of variable components changed by this update} % [review] - unclear notation?
			\(\bar{\vcomp}_j \leftarrow s^k_j\) \;
			\(s^k_j \leftarrow s^k_j - r^k_j\) \;
		}
		\Return{\(\bar{\vcomp}^k,s^k,c\)}
	}
	
	\caption{
		Fast implementation of the fractional \gls{dp} update described in \cref{proc:dp-update}.
	}
	\label{proc:frac-dp-update-fast}
\end{algorithm}

\subsection{Parameter sweep strategy}
As explained earlier, the fractional \gls{dp} update of constraint components depends on a parameter \(\alpha\), which dictates the amount \enquote{moved out} of the constraint component.
Choosing this parameter is difficult, but some of the results discussed earlier may be used to create a strategy for a parameter sweep.
Knowing that values \(\alpha\leq n^{-1}\) (where \(n\) is the arity of a constraint) guarantee that any solution found is optimal, \(n^{-1}\) may be chosen as a lower limit for the parameter.
A reasonable upper limit fo the parameter is \(\alpha=1\), which in essence corresponds to the regular, non-fractional \gls{dp} update.

To vary the parameter between these two values, a sweep strategy is employed.
\Cref{fig:khappa-plot} shows the value of a parameter \(\kappa\) (defined so that \(\alpha = n^{-1}\left(1 + \kappa(n - 1)\right)\), \emph{i.e.} mapping \(\left[0,1\right]\) to \(\left[n^{-1},1\right]\) \emph{individually} for each constraint component) over the first \emph{trial} for two different problems, along with the number of sign changes which is used as a termination criterion.

\begin{figure}[p]
	\centering
	\subfloat[\label{fig:khappa-plot:comp}A max-\gls{csp} problem from the \enquote{Composed} set.]{% Created by tikzDevice version 0.7.0 on 2014-05-16 17:50:41
% !TEX encoding = UTF-8 Unicode
\begin{tikzpicture}[x=1pt,y=1pt]
\definecolor[named]{fillColor}{rgb}{1.00,1.00,1.00}
\path[use as bounding box,fill=fillColor,fill opacity=0.00] (0,0) rectangle (289.08,180.67);
\begin{scope}
\path[clip] (  0.00,  0.00) rectangle (289.08,180.67);
\definecolor[named]{drawColor}{rgb}{1.00,1.00,1.00}
\definecolor[named]{fillColor}{rgb}{1.00,1.00,1.00}

\path[draw=drawColor,line width= 0.6pt,line join=round,line cap=round,fill=fillColor] ( -0.00,  0.00) rectangle (289.08,180.68);
\end{scope}
\begin{scope}
\path[clip] ( 41.82,102.84) rectangle (264.40,168.63);
\definecolor[named]{fillColor}{rgb}{0.90,0.90,0.90}

\path[fill=fillColor] ( 41.82,102.84) rectangle (264.40,168.63);
\definecolor[named]{drawColor}{rgb}{0.95,0.95,0.95}

\path[draw=drawColor,line width= 0.3pt,line join=round] ( 41.82,116.12) --
	(264.40,116.12);

\path[draw=drawColor,line width= 0.3pt,line join=round] ( 41.82,136.69) --
	(264.40,136.69);

\path[draw=drawColor,line width= 0.3pt,line join=round] ( 41.82,157.27) --
	(264.40,157.27);

\path[draw=drawColor,line width= 0.3pt,line join=round] ( 88.59,102.84) --
	( 88.59,168.63);

\path[draw=drawColor,line width= 0.3pt,line join=round] (161.91,102.84) --
	(161.91,168.63);

\path[draw=drawColor,line width= 0.3pt,line join=round] (235.22,102.84) --
	(235.22,168.63);
\definecolor[named]{drawColor}{rgb}{1.00,1.00,1.00}

\path[draw=drawColor,line width= 0.6pt,line join=round] ( 41.82,105.83) --
	(264.40,105.83);

\path[draw=drawColor,line width= 0.6pt,line join=round] ( 41.82,126.41) --
	(264.40,126.41);

\path[draw=drawColor,line width= 0.6pt,line join=round] ( 41.82,146.98) --
	(264.40,146.98);

\path[draw=drawColor,line width= 0.6pt,line join=round] ( 41.82,167.56) --
	(264.40,167.56);

\path[draw=drawColor,line width= 0.6pt,line join=round] ( 51.94,102.84) --
	( 51.94,168.63);

\path[draw=drawColor,line width= 0.6pt,line join=round] (125.25,102.84) --
	(125.25,168.63);

\path[draw=drawColor,line width= 0.6pt,line join=round] (198.56,102.84) --
	(198.56,168.63);
\definecolor[named]{drawColor}{rgb}{0.00,0.00,0.00}

\path[draw=drawColor,line width= 0.6pt,line join=round] ( 51.94,105.83) --
	( 52.67,105.83) --
	( 53.40,105.83) --
	( 54.14,105.83) --
	( 54.87,105.83) --
	( 55.60,105.83) --
	( 56.34,105.83) --
	( 57.07,105.83) --
	( 57.80,105.83) --
	( 58.53,105.83) --
	( 59.27,105.83) --
	( 60.00,105.83) --
	( 60.73,105.83) --
	( 61.47,105.83) --
	( 62.20,105.83) --
	( 62.93,105.83) --
	( 63.67,105.83) --
	( 64.40,105.83) --
	( 65.13,105.83) --
	( 65.87,105.83) --
	( 66.60,105.83) --
	( 67.33,105.83) --
	( 68.07,105.83) --
	( 68.80,105.83) --
	( 69.53,105.83) --
	( 70.27,105.83) --
	( 71.00,105.83) --
	( 71.73,105.83) --
	( 72.46,105.83) --
	( 73.20,105.83) --
	( 73.93,105.83) --
	( 74.66,105.83) --
	( 75.40,105.83) --
	( 76.13,105.83) --
	( 76.86,105.83) --
	( 77.60,105.83) --
	( 78.33,105.83) --
	( 79.06,105.83) --
	( 79.80,105.83) --
	( 80.53,105.83) --
	( 81.26,105.83) --
	( 82.00,105.83) --
	( 82.73,105.83) --
	( 83.46,105.83) --
	( 84.19,105.83) --
	( 84.93,105.83) --
	( 85.66,105.83) --
	( 86.39,105.83) --
	( 87.13,105.83) --
	( 87.86,105.83) --
	( 88.59,105.83) --
	( 89.33,105.83) --
	( 90.06,105.83) --
	( 90.79,105.83) --
	( 91.53,105.83) --
	( 92.26,105.83) --
	( 92.99,105.83) --
	( 93.73,105.83) --
	( 94.46,105.83) --
	( 95.19,105.83) --
	( 95.93,105.83) --
	( 96.66,105.83) --
	( 97.39,105.83) --
	( 98.12,105.83) --
	( 98.86,105.83) --
	( 99.59,105.83) --
	(100.32,105.83) --
	(101.06,105.83) --
	(101.79,105.83) --
	(102.52,105.83) --
	(103.26,105.83) --
	(103.99,105.83) --
	(104.72,105.83) --
	(105.46,105.83) --
	(106.19,105.83) --
	(106.92,105.83) --
	(107.66,105.83) --
	(108.39,105.83) --
	(109.12,105.83) --
	(109.85,105.83) --
	(110.59,105.83) --
	(111.32,105.83) --
	(112.05,105.83) --
	(112.79,105.83) --
	(113.52,105.83) --
	(114.25,105.83) --
	(114.99,105.83) --
	(115.72,105.83) --
	(116.45,105.83) --
	(117.19,105.83) --
	(117.92,105.83) --
	(118.65,105.83) --
	(119.39,105.83) --
	(120.12,105.83) --
	(120.85,105.83) --
	(121.59,105.83) --
	(122.32,105.83) --
	(123.05,105.83) --
	(123.78,105.83) --
	(124.52,105.83) --
	(125.25,105.83) --
	(125.98,105.83) --
	(126.72,105.83) --
	(127.45,105.83) --
	(128.18,105.83) --
	(128.92,105.83) --
	(129.65,105.83) --
	(130.38,105.83) --
	(131.12,105.83) --
	(131.85,105.83) --
	(132.58,105.83) --
	(133.32,105.83) --
	(134.05,105.83) --
	(134.78,105.83) --
	(135.51,105.83) --
	(136.25,105.83) --
	(136.98,105.83) --
	(137.71,105.83) --
	(138.45,105.83) --
	(139.18,105.83) --
	(139.91,105.83) --
	(140.65,105.83) --
	(141.38,105.83) --
	(142.11,105.83) --
	(142.85,105.83) --
	(143.58,105.83) --
	(144.31,105.83) --
	(145.05,105.83) --
	(145.78,105.83) --
	(146.51,105.83) --
	(147.24,105.83) --
	(147.98,105.83) --
	(148.71,105.83) --
	(149.44,105.83) --
	(150.18,105.83) --
	(150.91,105.83) --
	(151.64,105.83) --
	(152.38,105.83) --
	(153.11,105.83) --
	(153.84,105.83) --
	(154.58,105.83) --
	(155.31,105.83) --
	(156.04,105.83) --
	(156.78,105.83) --
	(157.51,105.83) --
	(158.24,105.83) --
	(158.98,105.83) --
	(159.71,105.83) --
	(160.44,105.83) --
	(161.17,105.83) --
	(161.91,106.34) --
	(162.64,106.86) --
	(163.37,107.37) --
	(164.11,107.89) --
	(164.84,108.40) --
	(165.57,108.92) --
	(166.31,109.43) --
	(167.04,109.94) --
	(167.77,110.46) --
	(168.51,110.97) --
	(169.24,111.49) --
	(169.97,112.00) --
	(170.71,112.52) --
	(171.44,113.03) --
	(172.17,113.55) --
	(172.90,114.06) --
	(173.64,114.57) --
	(174.37,115.09) --
	(175.10,115.60) --
	(175.84,116.12) --
	(176.57,116.63) --
	(177.30,117.15) --
	(178.04,117.66) --
	(178.77,118.18) --
	(179.50,118.69) --
	(180.24,119.20) --
	(180.97,119.72) --
	(181.70,120.23) --
	(182.44,120.75) --
	(183.17,121.26) --
	(183.90,121.78) --
	(184.64,122.29) --
	(185.37,122.81) --
	(186.10,123.32) --
	(186.83,123.83) --
	(187.57,124.35) --
	(188.30,124.86) --
	(189.03,125.38) --
	(189.77,125.89) --
	(190.50,126.41) --
	(191.23,126.92) --
	(191.97,127.43) --
	(192.70,127.95) --
	(193.43,128.46) --
	(194.17,128.98) --
	(194.90,129.49) --
	(195.63,130.01) --
	(196.37,130.52) --
	(197.10,131.04) --
	(197.83,131.55) --
	(198.56,132.06) --
	(199.30,132.58) --
	(200.03,133.09) --
	(200.76,133.61) --
	(201.50,134.12) --
	(202.23,134.64) --
	(202.96,135.15) --
	(203.70,135.67) --
	(204.43,136.18) --
	(205.16,136.69) --
	(205.90,137.21) --
	(206.63,137.72) --
	(207.36,138.24) --
	(208.10,138.75) --
	(208.83,139.27) --
	(209.56,139.78) --
	(210.29,140.30) --
	(211.03,140.81) --
	(211.76,141.32) --
	(212.49,141.84) --
	(213.23,142.35) --
	(213.96,142.87) --
	(214.69,143.38) --
	(215.43,143.90) --
	(216.16,144.41) --
	(216.89,144.93) --
	(217.63,145.44) --
	(218.36,145.95) --
	(219.09,146.47) --
	(219.83,146.98) --
	(220.56,147.50) --
	(221.29,148.01) --
	(222.03,148.53) --
	(222.76,149.04) --
	(223.49,149.56) --
	(224.22,150.07) --
	(224.96,150.58) --
	(225.69,151.10) --
	(226.42,151.61) --
	(227.16,152.13) --
	(227.89,152.64) --
	(228.62,153.16) --
	(229.36,153.67) --
	(230.09,154.19) --
	(230.82,154.70) --
	(231.56,155.21) --
	(232.29,155.73) --
	(233.02,156.24) --
	(233.76,156.76) --
	(234.49,157.27) --
	(235.22,157.79) --
	(235.95,158.30) --
	(236.69,158.82) --
	(237.42,159.33) --
	(238.15,159.84) --
	(238.89,160.36) --
	(239.62,160.87) --
	(240.35,161.39) --
	(241.09,161.90) --
	(241.82,162.42) --
	(242.55,162.93) --
	(243.29,163.45) --
	(244.02,163.96) --
	(244.75,164.47) --
	(245.49,164.99) --
	(246.22,165.50) --
	(246.95,165.52) --
	(247.69,165.53) --
	(248.42,165.54) --
	(249.15,165.55) --
	(249.88,165.57) --
	(250.62,165.58) --
	(251.35,165.59) --
	(252.08,165.60) --
	(252.82,165.61) --
	(253.55,165.63) --
	(254.28,165.64);
\end{scope}
\begin{scope}
\path[clip] ( 41.82, 34.03) rectangle (264.40, 99.83);
\definecolor[named]{fillColor}{rgb}{0.90,0.90,0.90}

\path[fill=fillColor] ( 41.82, 34.03) rectangle (264.40, 99.83);
\definecolor[named]{drawColor}{rgb}{0.95,0.95,0.95}

\path[draw=drawColor,line width= 0.3pt,line join=round] ( 41.82, 44.83) --
	(264.40, 44.83);

\path[draw=drawColor,line width= 0.3pt,line join=round] ( 41.82, 60.45) --
	(264.40, 60.45);

\path[draw=drawColor,line width= 0.3pt,line join=round] ( 41.82, 76.07) --
	(264.40, 76.07);

\path[draw=drawColor,line width= 0.3pt,line join=round] ( 41.82, 91.68) --
	(264.40, 91.68);

\path[draw=drawColor,line width= 0.3pt,line join=round] ( 88.59, 34.03) --
	( 88.59, 99.83);

\path[draw=drawColor,line width= 0.3pt,line join=round] (161.91, 34.03) --
	(161.91, 99.83);

\path[draw=drawColor,line width= 0.3pt,line join=round] (235.22, 34.03) --
	(235.22, 99.83);
\definecolor[named]{drawColor}{rgb}{1.00,1.00,1.00}

\path[draw=drawColor,line width= 0.6pt,line join=round] ( 41.82, 37.03) --
	(264.40, 37.03);

\path[draw=drawColor,line width= 0.6pt,line join=round] ( 41.82, 52.64) --
	(264.40, 52.64);

\path[draw=drawColor,line width= 0.6pt,line join=round] ( 41.82, 68.26) --
	(264.40, 68.26);

\path[draw=drawColor,line width= 0.6pt,line join=round] ( 41.82, 83.87) --
	(264.40, 83.87);

\path[draw=drawColor,line width= 0.6pt,line join=round] ( 41.82, 99.49) --
	(264.40, 99.49);

\path[draw=drawColor,line width= 0.6pt,line join=round] ( 51.94, 34.03) --
	( 51.94, 99.83);

\path[draw=drawColor,line width= 0.6pt,line join=round] (125.25, 34.03) --
	(125.25, 99.83);

\path[draw=drawColor,line width= 0.6pt,line join=round] (198.56, 34.03) --
	(198.56, 99.83);
\definecolor[named]{drawColor}{rgb}{0.00,0.00,0.00}

\path[draw=drawColor,line width= 0.6pt,line join=round] ( 51.94, 93.71) --
	( 52.67, 65.60) --
	( 53.40, 79.35) --
	( 54.14, 79.97) --
	( 54.87, 74.82) --
	( 55.60, 81.69) --
	( 56.34, 81.06) --
	( 57.07, 72.47) --
	( 57.80, 71.69) --
	( 58.53, 83.41) --
	( 59.27, 77.94) --
	( 60.00, 77.00) --
	( 60.73, 82.00) --
	( 61.47, 78.25) --
	( 62.20, 85.90) --
	( 62.93, 81.38) --
	( 63.67, 86.84) --
	( 64.40, 83.41) --
	( 65.13, 91.53) --
	( 65.87, 89.65) --
	( 66.60, 77.94) --
	( 67.33, 85.28) --
	( 68.07, 84.50) --
	( 68.80, 83.41) --
	( 69.53, 81.06) --
	( 70.27, 80.13) --
	( 71.00, 82.63) --
	( 71.73, 77.00) --
	( 72.46, 75.44) --
	( 73.20, 86.06) --
	( 73.93, 79.03) --
	( 74.66, 83.87) --
	( 75.40, 78.10) --
	( 76.13, 79.97) --
	( 76.86, 72.94) --
	( 77.60, 73.57) --
	( 78.33, 76.53) --
	( 79.06, 68.26) --
	( 79.80, 65.92) --
	( 80.53, 68.26) --
	( 81.26, 64.82) --
	( 82.00, 72.16) --
	( 82.73, 72.79) --
	( 83.46, 70.60) --
	( 84.19, 69.66) --
	( 84.93, 70.76) --
	( 85.66, 68.10) --
	( 86.39, 69.04) --
	( 87.13, 73.41) --
	( 87.86, 72.79) --
	( 88.59, 80.75) --
	( 89.33, 67.01) --
	( 90.06, 73.10) --
	( 90.79, 75.91) --
	( 91.53, 87.15) --
	( 92.26, 82.47) --
	( 92.99, 84.19) --
	( 93.73, 75.29) --
	( 94.46, 80.60) --
	( 95.19, 72.16) --
	( 95.93, 72.63) --
	( 96.66, 70.29) --
	( 97.39, 69.98) --
	( 98.12, 71.85) --
	( 98.86, 72.47) --
	( 99.59, 74.19) --
	(100.32, 69.04) --
	(101.06, 65.13) --
	(101.79, 74.66) --
	(102.52, 75.60) --
	(103.26, 80.28) --
	(103.99, 83.72) --
	(104.72, 75.91) --
	(105.46, 68.41) --
	(106.19, 73.10) --
	(106.92, 77.00) --
	(107.66, 74.50) --
	(108.39, 81.06) --
	(109.12, 84.19) --
	(109.85, 80.44) --
	(110.59, 76.07) --
	(111.32, 76.38) --
	(112.05, 82.94) --
	(112.79, 80.60) --
	(113.52, 75.91) --
	(114.25, 77.47) --
	(114.99, 84.19) --
	(115.72, 77.94) --
	(116.45, 80.44) --
	(117.19, 80.13) --
	(117.92, 77.47) --
	(118.65, 80.91) --
	(119.39, 85.28) --
	(120.12, 86.53) --
	(120.85, 86.22) --
	(121.59, 84.66) --
	(122.32, 90.59) --
	(123.05, 80.91) --
	(123.78, 84.50) --
	(124.52, 74.82) --
	(125.25, 83.41) --
	(125.98, 86.37) --
	(126.72, 82.78) --
	(127.45, 88.25) --
	(128.18, 86.53) --
	(128.92, 84.81) --
	(129.65, 83.41) --
	(130.38, 84.19) --
	(131.12, 78.72) --
	(131.85, 83.25) --
	(132.58, 83.25) --
	(133.32, 82.00) --
	(134.05, 84.03) --
	(134.78, 82.63) --
	(135.51, 90.75) --
	(136.25, 79.97) --
	(136.98, 82.63) --
	(137.71, 83.87) --
	(138.45, 84.50) --
	(139.18, 85.12) --
	(139.91, 80.13) --
	(140.65, 82.00) --
	(141.38, 78.10) --
	(142.11, 78.56) --
	(142.85, 73.41) --
	(143.58, 75.91) --
	(144.31, 72.32) --
	(145.05, 77.32) --
	(145.78, 76.22) --
	(146.51, 74.66) --
	(147.24, 75.13) --
	(147.98, 75.29) --
	(148.71, 72.16) --
	(149.44, 77.16) --
	(150.18, 87.78) --
	(150.91, 88.25) --
	(151.64, 85.90) --
	(152.38, 79.50) --
	(153.11, 85.28) --
	(153.84, 76.07) --
	(154.58, 78.72) --
	(155.31, 73.26) --
	(156.04, 75.60) --
	(156.78, 78.88) --
	(157.51, 76.53) --
	(158.24, 79.35) --
	(158.98, 76.38) --
	(159.71, 79.50) --
	(160.44, 83.87) --
	(161.17, 78.25) --
	(161.91, 85.90) --
	(162.64, 86.22) --
	(163.37, 80.75) --
	(164.11, 80.44) --
	(164.84, 87.15) --
	(165.57, 81.84) --
	(166.31, 79.66) --
	(167.04, 75.75) --
	(167.77, 84.03) --
	(168.51, 80.75) --
	(169.24, 85.44) --
	(169.97, 80.13) --
	(170.71, 81.38) --
	(171.44, 86.37) --
	(172.17, 78.56) --
	(172.90, 76.69) --
	(173.64, 73.72) --
	(174.37, 69.98) --
	(175.10, 64.82) --
	(175.84, 67.32) --
	(176.57, 71.38) --
	(177.30, 69.19) --
	(178.04, 74.04) --
	(178.77, 73.10) --
	(179.50, 64.82) --
	(180.24, 62.32) --
	(180.97, 72.94) --
	(181.70, 73.10) --
	(182.44, 71.38) --
	(183.17, 63.89) --
	(183.90, 65.29) --
	(184.64, 69.98) --
	(185.37, 67.01) --
	(186.10, 66.85) --
	(186.83, 67.01) --
	(187.57, 69.98) --
	(188.30, 70.91) --
	(189.03, 67.63) --
	(189.77, 63.42) --
	(190.50, 64.51) --
	(191.23, 67.79) --
	(191.97, 63.10) --
	(192.70, 63.42) --
	(193.43, 59.36) --
	(194.17, 63.73) --
	(194.90, 63.57) --
	(195.63, 64.67) --
	(196.37, 62.17) --
	(197.10, 61.86) --
	(197.83, 64.20) --
	(198.56, 62.95) --
	(199.30, 67.48) --
	(200.03, 74.82) --
	(200.76, 73.72) --
	(201.50, 72.01) --
	(202.23, 71.07) --
	(202.96, 72.32) --
	(203.70, 70.13) --
	(204.43, 71.23) --
	(205.16, 71.38) --
	(205.90, 73.26) --
	(206.63, 69.19) --
	(207.36, 74.82) --
	(208.10, 70.13) --
	(208.83, 77.63) --
	(209.56, 66.07) --
	(210.29, 71.69) --
	(211.03, 64.67) --
	(211.76, 63.26) --
	(212.49, 61.39) --
	(213.23, 60.29) --
	(213.96, 56.23) --
	(214.69, 59.51) --
	(215.43, 67.16) --
	(216.16, 64.35) --
	(216.89, 55.14) --
	(217.63, 53.42) --
	(218.36, 53.11) --
	(219.09, 54.52) --
	(219.83, 50.46) --
	(220.56, 48.43) --
	(221.29, 45.30) --
	(222.03, 46.24) --
	(222.76, 48.89) --
	(223.49, 49.52) --
	(224.22, 47.33) --
	(224.96, 45.93) --
	(225.69, 48.58) --
	(226.42, 50.46) --
	(227.16, 49.52) --
	(227.89, 47.64) --
	(228.62, 42.96) --
	(229.36, 45.46) --
	(230.09, 42.02) --
	(230.82, 44.36) --
	(231.56, 43.58) --
	(232.29, 45.46) --
	(233.02, 42.02) --
	(233.76, 41.40) --
	(234.49, 41.71) --
	(235.22, 47.33) --
	(235.95, 48.74) --
	(236.69, 40.15) --
	(237.42, 42.33) --
	(238.15, 40.46) --
	(238.89, 40.46) --
	(239.62, 38.74) --
	(240.35, 44.52) --
	(241.09, 40.15) --
	(241.82, 42.33) --
	(242.55, 45.77) --
	(243.29, 42.80) --
	(244.02, 39.52) --
	(244.75, 38.27) --
	(245.49, 37.65) --
	(246.22, 37.03) --
	(246.95, 37.03) --
	(247.69, 37.03) --
	(248.42, 37.03) --
	(249.15, 37.03) --
	(249.88, 37.03) --
	(250.62, 37.03) --
	(251.35, 37.03) --
	(252.08, 37.03) --
	(252.82, 37.03) --
	(253.55, 37.03) --
	(254.28, 37.03);
\end{scope}
\begin{scope}
\path[clip] (  0.00,  0.00) rectangle (289.08,180.67);
\definecolor[named]{drawColor}{rgb}{0.50,0.50,0.50}

\node[text=drawColor,anchor=base east,inner sep=0pt, outer sep=0pt, scale=  0.96] at ( 34.71,102.52) {0.0};

\node[text=drawColor,anchor=base east,inner sep=0pt, outer sep=0pt, scale=  0.96] at ( 34.71,123.10) {0.1};

\node[text=drawColor,anchor=base east,inner sep=0pt, outer sep=0pt, scale=  0.96] at ( 34.71,143.68) {0.2};

\node[text=drawColor,anchor=base east,inner sep=0pt, outer sep=0pt, scale=  0.96] at ( 34.71,164.26) {0.3};
\end{scope}
\begin{scope}
\path[clip] (  0.00,  0.00) rectangle (289.08,180.67);
\definecolor[named]{drawColor}{rgb}{0.50,0.50,0.50}

\path[draw=drawColor,line width= 0.6pt,line join=round] ( 37.55,105.83) --
	( 41.82,105.83);

\path[draw=drawColor,line width= 0.6pt,line join=round] ( 37.55,126.41) --
	( 41.82,126.41);

\path[draw=drawColor,line width= 0.6pt,line join=round] ( 37.55,146.98) --
	( 41.82,146.98);

\path[draw=drawColor,line width= 0.6pt,line join=round] ( 37.55,167.56) --
	( 41.82,167.56);
\end{scope}
\begin{scope}
\path[clip] (  0.00,  0.00) rectangle (289.08,180.67);
\definecolor[named]{drawColor}{rgb}{0.50,0.50,0.50}

\node[text=drawColor,anchor=base east,inner sep=0pt, outer sep=0pt, scale=  0.96] at ( 34.71, 33.72) {0};

\node[text=drawColor,anchor=base east,inner sep=0pt, outer sep=0pt, scale=  0.96] at ( 34.71, 49.34) {100};

\node[text=drawColor,anchor=base east,inner sep=0pt, outer sep=0pt, scale=  0.96] at ( 34.71, 64.95) {200};

\node[text=drawColor,anchor=base east,inner sep=0pt, outer sep=0pt, scale=  0.96] at ( 34.71, 80.57) {300};

\node[text=drawColor,anchor=base east,inner sep=0pt, outer sep=0pt, scale=  0.96] at ( 34.71, 96.19) {400};
\end{scope}
\begin{scope}
\path[clip] (  0.00,  0.00) rectangle (289.08,180.67);
\definecolor[named]{drawColor}{rgb}{0.50,0.50,0.50}

\path[draw=drawColor,line width= 0.6pt,line join=round] ( 37.55, 37.03) --
	( 41.82, 37.03);

\path[draw=drawColor,line width= 0.6pt,line join=round] ( 37.55, 52.64) --
	( 41.82, 52.64);

\path[draw=drawColor,line width= 0.6pt,line join=round] ( 37.55, 68.26) --
	( 41.82, 68.26);

\path[draw=drawColor,line width= 0.6pt,line join=round] ( 37.55, 83.87) --
	( 41.82, 83.87);

\path[draw=drawColor,line width= 0.6pt,line join=round] ( 37.55, 99.49) --
	( 41.82, 99.49);
\end{scope}
\begin{scope}
\path[clip] (264.40,102.84) rectangle (277.04,168.63);
\definecolor[named]{fillColor}{rgb}{0.80,0.80,0.80}

\path[fill=fillColor] (264.40,102.84) rectangle (277.04,168.63);
\definecolor[named]{drawColor}{rgb}{0.00,0.00,0.00}

\node[text=drawColor,rotate=270.00,anchor=base,inner sep=0pt, outer sep=0pt, scale=  0.96] at (267.41,135.73) {\(\kappa\)};
\end{scope}
\begin{scope}
\path[clip] (264.40, 34.03) rectangle (277.04, 99.83);
\definecolor[named]{fillColor}{rgb}{0.80,0.80,0.80}

\path[fill=fillColor] (264.40, 34.03) rectangle (277.04, 99.83);
\definecolor[named]{drawColor}{rgb}{0.00,0.00,0.00}

\node[text=drawColor,rotate=270.00,anchor=base,inner sep=0pt, outer sep=0pt, scale=  0.96] at (267.41, 66.93) {sign changes};
\end{scope}
\begin{scope}
\path[clip] (  0.00,  0.00) rectangle (289.08,180.67);
\definecolor[named]{drawColor}{rgb}{0.50,0.50,0.50}

\path[draw=drawColor,line width= 0.6pt,line join=round] ( 51.94, 29.77) --
	( 51.94, 34.03);

\path[draw=drawColor,line width= 0.6pt,line join=round] (125.25, 29.77) --
	(125.25, 34.03);

\path[draw=drawColor,line width= 0.6pt,line join=round] (198.56, 29.77) --
	(198.56, 34.03);
\end{scope}
\begin{scope}
\path[clip] (  0.00,  0.00) rectangle (289.08,180.67);
\definecolor[named]{drawColor}{rgb}{0.50,0.50,0.50}

\node[text=drawColor,anchor=base,inner sep=0pt, outer sep=0pt, scale=  0.96] at ( 51.94, 20.31) {0};

\node[text=drawColor,anchor=base,inner sep=0pt, outer sep=0pt, scale=  0.96] at (125.25, 20.31) {100};

\node[text=drawColor,anchor=base,inner sep=0pt, outer sep=0pt, scale=  0.96] at (198.56, 20.31) {200};
\end{scope}
\begin{scope}
\path[clip] (  0.00,  0.00) rectangle (289.08,180.67);
\definecolor[named]{drawColor}{rgb}{0.00,0.00,0.00}

\node[text=drawColor,anchor=base,inner sep=0pt, outer sep=0pt, scale=  1] at (153.11,  8.53) {Iterations};
\end{scope}
\end{tikzpicture}
}
	\\
	\subfloat[\label{fig:khappa-plot:deer}A \gls{mrf} problem from the \enquote{Object Detection} set.]{% Created by tikzDevice version 0.7.0 on 2014-05-12 14:43:36
% !TEX encoding = UTF-8 Unicode
\begin{tikzpicture}[x=1pt,y=1pt]
\definecolor[named]{fillColor}{rgb}{1.00,1.00,1.00}
\path[use as bounding box,fill=fillColor,fill opacity=0.00] (0,0) rectangle (361.35,216.81);
\begin{scope}
\path[clip] (  0.00,  0.00) rectangle (361.35,216.81);
\definecolor[named]{drawColor}{rgb}{1.00,1.00,1.00}
\definecolor[named]{fillColor}{rgb}{1.00,1.00,1.00}

\path[draw=drawColor,line width= 0.6pt,line join=round,line cap=round,fill=fillColor] (  0.00,  0.00) rectangle (361.35,216.81);
\end{scope}
\begin{scope}
\path[clip] ( 46.62,120.91) rectangle (336.67,204.77);
\definecolor[named]{fillColor}{rgb}{0.90,0.90,0.90}

\path[fill=fillColor] ( 46.62,120.91) rectangle (336.67,204.77);
\definecolor[named]{drawColor}{rgb}{0.95,0.95,0.95}

\path[draw=drawColor,line width= 0.3pt,line join=round] ( 46.62,134.40) --
	(336.67,134.40);

\path[draw=drawColor,line width= 0.3pt,line join=round] ( 46.62,153.75) --
	(336.67,153.75);

\path[draw=drawColor,line width= 0.3pt,line join=round] ( 46.62,173.11) --
	(336.67,173.11);

\path[draw=drawColor,line width= 0.3pt,line join=round] ( 46.62,192.47) --
	(336.67,192.47);

\path[draw=drawColor,line width= 0.3pt,line join=round] ( 85.40,120.91) --
	( 85.40,204.77);

\path[draw=drawColor,line width= 0.3pt,line join=round] (136.60,120.91) --
	(136.60,204.77);

\path[draw=drawColor,line width= 0.3pt,line join=round] (187.80,120.91) --
	(187.80,204.77);

\path[draw=drawColor,line width= 0.3pt,line join=round] (239.01,120.91) --
	(239.01,204.77);

\path[draw=drawColor,line width= 0.3pt,line join=round] (290.21,120.91) --
	(290.21,204.77);
\definecolor[named]{drawColor}{rgb}{1.00,1.00,1.00}

\path[draw=drawColor,line width= 0.6pt,line join=round] ( 46.62,124.72) --
	(336.67,124.72);

\path[draw=drawColor,line width= 0.6pt,line join=round] ( 46.62,144.08) --
	(336.67,144.08);

\path[draw=drawColor,line width= 0.6pt,line join=round] ( 46.62,163.43) --
	(336.67,163.43);

\path[draw=drawColor,line width= 0.6pt,line join=round] ( 46.62,182.79) --
	(336.67,182.79);

\path[draw=drawColor,line width= 0.6pt,line join=round] ( 46.62,202.15) --
	(336.67,202.15);

\path[draw=drawColor,line width= 0.6pt,line join=round] ( 59.80,120.91) --
	( 59.80,204.77);

\path[draw=drawColor,line width= 0.6pt,line join=round] (111.00,120.91) --
	(111.00,204.77);

\path[draw=drawColor,line width= 0.6pt,line join=round] (162.20,120.91) --
	(162.20,204.77);

\path[draw=drawColor,line width= 0.6pt,line join=round] (213.40,120.91) --
	(213.40,204.77);

\path[draw=drawColor,line width= 0.6pt,line join=round] (264.61,120.91) --
	(264.61,204.77);

\path[draw=drawColor,line width= 0.6pt,line join=round] (315.81,120.91) --
	(315.81,204.77);
\definecolor[named]{drawColor}{rgb}{0.00,0.00,0.00}

\path[draw=drawColor,line width= 0.6pt,line join=round] ( 59.80,124.72) --
	( 59.85,124.72) --
	( 59.90,124.72) --
	( 59.96,124.72) --
	( 60.01,124.72) --
	( 60.06,124.72) --
	( 60.11,124.72) --
	( 60.16,124.72) --
	( 60.21,124.72) --
	( 60.26,124.72) --
	( 60.31,124.72) --
	( 60.37,124.72) --
	( 60.42,124.72) --
	( 60.47,124.72) --
	( 60.52,124.72) --
	( 60.57,124.72) --
	( 60.62,124.72) --
	( 60.67,124.72) --
	( 60.72,124.72) --
	( 60.78,124.72) --
	( 60.83,124.72) --
	( 60.88,124.72) --
	( 60.93,124.72) --
	( 60.98,124.72) --
	( 61.03,124.72) --
	( 61.08,124.72) --
	( 61.13,124.72) --
	( 61.18,124.72) --
	( 61.24,124.72) --
	( 61.29,124.72) --
	( 61.34,124.72) --
	( 61.39,124.72) --
	( 61.44,124.72) --
	( 61.49,124.72) --
	( 61.54,124.72) --
	( 61.59,124.72) --
	( 61.65,124.72) --
	( 61.70,124.72) --
	( 61.75,124.72) --
	( 61.80,124.72) --
	( 61.85,124.72) --
	( 61.90,124.72) --
	( 61.95,124.72) --
	( 62.00,124.72) --
	( 62.06,124.72) --
	( 62.11,124.72) --
	( 62.16,124.72) --
	( 62.21,124.72) --
	( 62.26,124.72) --
	( 62.31,124.72) --
	( 62.36,124.72) --
	( 62.41,124.72) --
	( 62.46,124.72) --
	( 62.52,124.72) --
	( 62.57,124.72) --
	( 62.62,124.72) --
	( 62.67,124.72) --
	( 62.72,124.72) --
	( 62.77,124.72) --
	( 62.82,124.72) --
	( 62.87,124.72) --
	( 62.93,124.72) --
	( 62.98,124.72) --
	( 63.03,124.72) --
	( 63.08,124.72) --
	( 63.13,124.72) --
	( 63.18,124.72) --
	( 63.23,124.72) --
	( 63.28,124.72) --
	( 63.34,124.72) --
	( 63.39,124.72) --
	( 63.44,124.72) --
	( 63.49,124.72) --
	( 63.54,124.72) --
	( 63.59,124.72) --
	( 63.64,124.72) --
	( 63.69,124.72) --
	( 63.74,124.72) --
	( 63.80,124.72) --
	( 63.85,124.72) --
	( 63.90,124.72) --
	( 63.95,124.72) --
	( 64.00,124.72) --
	( 64.05,124.72) --
	( 64.10,124.72) --
	( 64.15,124.72) --
	( 64.21,124.72) --
	( 64.26,124.72) --
	( 64.31,124.72) --
	( 64.36,124.72) --
	( 64.41,124.72) --
	( 64.46,124.72) --
	( 64.51,124.72) --
	( 64.56,124.72) --
	( 64.62,124.72) --
	( 64.67,124.72) --
	( 64.72,124.72) --
	( 64.77,124.72) --
	( 64.82,124.72) --
	( 64.87,124.72) --
	( 64.92,124.72) --
	( 64.97,124.72) --
	( 65.02,124.72) --
	( 65.08,124.72) --
	( 65.13,124.72) --
	( 65.18,124.72) --
	( 65.23,124.72) --
	( 65.28,124.72) --
	( 65.33,124.72) --
	( 65.38,124.72) --
	( 65.43,124.72) --
	( 65.49,124.72) --
	( 65.54,124.72) --
	( 65.59,124.72) --
	( 65.64,124.72) --
	( 65.69,124.72) --
	( 65.74,124.72) --
	( 65.79,124.72) --
	( 65.84,124.72) --
	( 65.90,124.72) --
	( 65.95,124.72) --
	( 66.00,124.72) --
	( 66.05,124.72) --
	( 66.10,124.72) --
	( 66.15,124.72) --
	( 66.20,124.72) --
	( 66.25,124.72) --
	( 66.30,124.72) --
	( 66.36,124.72) --
	( 66.41,124.72) --
	( 66.46,124.72) --
	( 66.51,124.72) --
	( 66.56,124.72) --
	( 66.61,124.72) --
	( 66.66,124.72) --
	( 66.71,124.72) --
	( 66.77,124.72) --
	( 66.82,124.72) --
	( 66.87,124.72) --
	( 66.92,124.72) --
	( 66.97,124.72) --
	( 67.02,124.72) --
	( 67.07,124.72) --
	( 67.12,124.72) --
	( 67.18,124.72) --
	( 67.23,124.72) --
	( 67.28,124.72) --
	( 67.33,124.72) --
	( 67.38,124.72) --
	( 67.43,124.72) --
	( 67.48,124.91) --
	( 67.53,125.10) --
	( 67.58,125.30) --
	( 67.64,125.49) --
	( 67.69,125.69) --
	( 67.74,125.88) --
	( 67.79,126.07) --
	( 67.84,126.27) --
	( 67.89,126.46) --
	( 67.94,126.65) --
	( 67.99,126.85) --
	( 68.05,127.04) --
	( 68.10,127.23) --
	( 68.15,127.43) --
	( 68.20,127.62) --
	( 68.25,127.81) --
	( 68.30,128.01) --
	( 68.35,128.20) --
	( 68.40,128.40) --
	( 68.46,128.59) --
	( 68.51,128.78) --
	( 68.56,128.98) --
	( 68.61,129.17) --
	( 68.66,129.36) --
	( 68.71,129.56) --
	( 68.76,129.75) --
	( 68.81,129.94) --
	( 68.86,130.14) --
	( 68.92,130.33) --
	( 68.97,130.52) --
	( 69.02,130.72) --
	( 69.07,130.91) --
	( 69.12,131.11) --
	( 69.17,131.30) --
	( 69.22,131.49) --
	( 69.27,131.69) --
	( 69.33,131.88) --
	( 69.38,132.07) --
	( 69.43,132.27) --
	( 69.48,132.46) --
	( 69.53,132.65) --
	( 69.58,132.85) --
	( 69.63,133.04) --
	( 69.68,133.23) --
	( 69.74,133.43) --
	( 69.79,133.62) --
	( 69.84,133.82) --
	( 69.89,134.01) --
	( 69.94,134.20) --
	( 69.99,134.40) --
	( 70.04,134.59) --
	( 70.09,134.78) --
	( 70.15,134.98) --
	( 70.20,135.17) --
	( 70.25,135.36) --
	( 70.30,135.56) --
	( 70.35,135.75) --
	( 70.40,135.95) --
	( 70.45,136.14) --
	( 70.50,136.33) --
	( 70.55,136.53) --
	( 70.61,136.72) --
	( 70.66,136.91) --
	( 70.71,137.11) --
	( 70.76,137.30) --
	( 70.81,137.49) --
	( 70.86,137.69) --
	( 70.91,137.88) --
	( 70.96,138.07) --
	( 71.02,138.27) --
	( 71.07,138.46) --
	( 71.12,138.66) --
	( 71.17,138.85) --
	( 71.22,139.04) --
	( 71.27,139.24) --
	( 71.32,139.43) --
	( 71.37,139.62) --
	( 71.43,139.82) --
	( 71.48,140.01) --
	( 71.53,140.20) --
	( 71.58,140.40) --
	( 71.63,140.59) --
	( 71.68,140.78) --
	( 71.73,140.98) --
	( 71.78,141.17) --
	( 71.83,141.37) --
	( 71.89,141.56) --
	( 71.94,141.75) --
	( 71.99,141.95) --
	( 72.04,142.14) --
	( 72.09,142.33) --
	( 72.14,142.53) --
	( 72.19,142.72) --
	( 72.24,142.91) --
	( 72.30,143.11) --
	( 72.35,143.30) --
	( 72.40,143.49) --
	( 72.45,143.69) --
	( 72.50,143.88) --
	( 72.55,144.08) --
	( 72.60,144.27) --
	( 72.65,144.46) --
	( 72.71,144.66) --
	( 72.76,144.85) --
	( 72.81,145.04) --
	( 72.86,145.24) --
	( 72.91,145.43) --
	( 72.96,145.62) --
	( 73.01,145.82) --
	( 73.06,146.01) --
	( 73.11,146.20) --
	( 73.17,146.40) --
	( 73.22,146.59) --
	( 73.27,146.79) --
	( 73.32,146.98) --
	( 73.37,147.17) --
	( 73.42,147.37) --
	( 73.47,147.56) --
	( 73.52,147.75) --
	( 73.58,147.95) --
	( 73.63,148.14) --
	( 73.68,148.33) --
	( 73.73,148.53) --
	( 73.78,148.72) --
	( 73.83,148.92) --
	( 73.88,149.11) --
	( 73.93,149.30) --
	( 73.99,149.50) --
	( 74.04,149.69) --
	( 74.09,149.88) --
	( 74.14,150.08) --
	( 74.19,150.27) --
	( 74.24,150.46) --
	( 74.29,150.66) --
	( 74.34,150.85) --
	( 74.39,151.04) --
	( 74.45,151.24) --
	( 74.50,151.43) --
	( 74.55,151.63) --
	( 74.60,151.82) --
	( 74.65,152.01) --
	( 74.70,152.21) --
	( 74.75,152.40) --
	( 74.80,152.59) --
	( 74.86,152.79) --
	( 74.91,152.98) --
	( 74.96,153.17) --
	( 75.01,153.37) --
	( 75.06,153.56) --
	( 75.11,153.75) --
	( 75.16,153.95) --
	( 75.21,154.14) --
	( 75.27,154.34) --
	( 75.32,154.53) --
	( 75.37,154.72) --
	( 75.42,154.92) --
	( 75.47,155.11) --
	( 75.52,155.30) --
	( 75.57,155.50) --
	( 75.62,155.69) --
	( 75.67,155.88) --
	( 75.73,156.08) --
	( 75.78,156.27) --
	( 75.83,156.46) --
	( 75.88,156.66) --
	( 75.93,156.85) --
	( 75.98,157.05) --
	( 76.03,157.24) --
	( 76.08,157.43) --
	( 76.14,157.63) --
	( 76.19,157.82) --
	( 76.24,158.01) --
	( 76.29,158.21) --
	( 76.34,158.40) --
	( 76.39,158.59) --
	( 76.44,158.79) --
	( 76.49,158.98) --
	( 76.55,159.18) --
	( 76.60,159.37) --
	( 76.65,159.56) --
	( 76.70,159.76) --
	( 76.75,159.95) --
	( 76.80,160.14) --
	( 76.85,160.34) --
	( 76.90,160.53) --
	( 76.95,160.72) --
	( 77.01,160.92) --
	( 77.06,161.11) --
	( 77.11,161.30) --
	( 77.16,161.50) --
	( 77.21,161.69) --
	( 77.26,161.89) --
	( 77.31,162.08) --
	( 77.36,162.27) --
	( 77.42,162.47) --
	( 77.47,162.66) --
	( 77.52,162.85) --
	( 77.57,163.05) --
	( 77.62,163.24) --
	( 77.67,163.43) --
	( 77.72,163.63) --
	( 77.77,163.82) --
	( 77.83,164.01) --
	( 77.88,164.21) --
	( 77.93,164.40) --
	( 77.98,164.60) --
	( 78.03,164.79) --
	( 78.08,164.98) --
	( 78.13,165.18) --
	( 78.18,165.37) --
	( 78.23,165.56) --
	( 78.29,165.76) --
	( 78.34,165.95) --
	( 78.39,166.14) --
	( 78.44,166.34) --
	( 78.49,166.53) --
	( 78.54,166.72) --
	( 78.59,166.92) --
	( 78.64,167.11) --
	( 78.70,167.31) --
	( 78.75,167.50) --
	( 78.80,167.69) --
	( 78.85,167.89) --
	( 78.90,168.08) --
	( 78.95,168.27) --
	( 79.00,168.47) --
	( 79.05,168.66) --
	( 79.11,168.85) --
	( 79.16,169.05) --
	( 79.21,169.24) --
	( 79.26,169.43) --
	( 79.31,169.63) --
	( 79.36,169.82) --
	( 79.41,170.02) --
	( 79.46,170.21) --
	( 79.51,170.40) --
	( 79.57,170.60) --
	( 79.62,170.79) --
	( 79.67,170.98) --
	( 79.72,171.18) --
	( 79.77,171.37) --
	( 79.82,171.56) --
	( 79.87,171.76) --
	( 79.92,171.95) --
	( 79.98,172.15) --
	( 80.03,172.34) --
	( 80.08,172.53) --
	( 80.13,172.73) --
	( 80.18,172.92) --
	( 80.23,173.11) --
	( 80.28,173.31) --
	( 80.33,173.50) --
	( 80.39,173.69) --
	( 80.44,173.89) --
	( 80.49,174.08) --
	( 80.54,174.27) --
	( 80.59,174.47) --
	( 80.64,174.66) --
	( 80.69,174.86) --
	( 80.74,175.05) --
	( 80.79,175.24) --
	( 80.85,175.44) --
	( 80.90,175.63) --
	( 80.95,175.82) --
	( 81.00,176.02) --
	( 81.05,176.21) --
	( 81.10,176.40) --
	( 81.15,176.60) --
	( 81.20,176.79) --
	( 81.26,176.98) --
	( 81.31,177.18) --
	( 81.36,177.37) --
	( 81.41,177.57) --
	( 81.46,177.76) --
	( 81.51,177.95) --
	( 81.56,178.15) --
	( 81.61,178.34) --
	( 81.67,178.53) --
	( 81.72,178.73) --
	( 81.77,178.92) --
	( 81.82,178.92) --
	( 81.87,178.93) --
	( 81.92,178.93) --
	( 81.97,178.93) --
	( 82.02,178.94) --
	( 82.07,178.94) --
	( 82.13,178.95) --
	( 82.18,178.95) --
	( 82.23,178.96) --
	( 82.28,178.96) --
	( 82.33,178.97) --
	( 82.38,178.97) --
	( 82.43,178.98) --
	( 82.48,178.98) --
	( 82.54,178.99) --
	( 82.59,178.99) --
	( 82.64,178.99) --
	( 82.69,179.00) --
	( 82.74,179.00) --
	( 82.79,179.01) --
	( 82.84,179.01) --
	( 82.89,179.02) --
	( 82.95,179.02) --
	( 83.00,179.03) --
	( 83.05,179.03) --
	( 83.10,179.04) --
	( 83.15,179.04) --
	( 83.20,179.05) --
	( 83.25,179.05) --
	( 83.30,179.06) --
	( 83.35,179.06) --
	( 83.41,179.06) --
	( 83.46,179.07) --
	( 83.51,179.07) --
	( 83.56,179.08) --
	( 83.61,179.08) --
	( 83.66,179.09) --
	( 83.71,179.09) --
	( 83.76,179.10) --
	( 83.82,179.10) --
	( 83.87,179.11) --
	( 83.92,179.11) --
	( 83.97,179.12) --
	( 84.02,179.12) --
	( 84.07,179.12) --
	( 84.12,179.13) --
	( 84.17,179.13) --
	( 84.23,179.14) --
	( 84.28,179.14) --
	( 84.33,179.15) --
	( 84.38,179.15) --
	( 84.43,179.16) --
	( 84.48,179.16) --
	( 84.53,179.17) --
	( 84.58,179.17) --
	( 84.63,179.18) --
	( 84.69,179.18) --
	( 84.74,179.19) --
	( 84.79,179.19) --
	( 84.84,179.19) --
	( 84.89,179.20) --
	( 84.94,179.20) --
	( 84.99,179.21) --
	( 85.04,179.21) --
	( 85.10,179.22) --
	( 85.15,179.22) --
	( 85.20,179.23) --
	( 85.25,179.23) --
	( 85.30,179.24) --
	( 85.35,179.24) --
	( 85.40,179.25) --
	( 85.45,179.25) --
	( 85.51,179.26) --
	( 85.56,179.26) --
	( 85.61,179.26) --
	( 85.66,179.27) --
	( 85.71,179.27) --
	( 85.76,179.28) --
	( 85.81,179.28) --
	( 85.86,179.29) --
	( 85.91,179.29) --
	( 85.97,179.30) --
	( 86.02,179.30) --
	( 86.07,179.31) --
	( 86.12,179.31) --
	( 86.17,179.32) --
	( 86.22,179.32) --
	( 86.27,179.32) --
	( 86.32,179.33) --
	( 86.38,179.33) --
	( 86.43,179.34) --
	( 86.48,179.34) --
	( 86.53,179.35) --
	( 86.58,179.35) --
	( 86.63,179.36) --
	( 86.68,179.36) --
	( 86.73,179.37) --
	( 86.79,179.37) --
	( 86.84,179.38) --
	( 86.89,179.38) --
	( 86.94,179.39) --
	( 86.99,179.39) --
	( 87.04,179.39) --
	( 87.09,179.40) --
	( 87.14,179.40) --
	( 87.19,179.41) --
	( 87.25,179.41) --
	( 87.30,179.42) --
	( 87.35,179.42) --
	( 87.40,179.43) --
	( 87.45,179.43) --
	( 87.50,179.44) --
	( 87.55,179.44) --
	( 87.60,179.45) --
	( 87.66,179.45) --
	( 87.71,179.45) --
	( 87.76,179.46) --
	( 87.81,179.46) --
	( 87.86,179.47) --
	( 87.91,179.47) --
	( 87.96,179.48) --
	( 88.01,179.48) --
	( 88.07,179.49) --
	( 88.12,179.49) --
	( 88.17,179.50) --
	( 88.22,179.50) --
	( 88.27,179.51) --
	( 88.32,179.51) --
	( 88.37,179.52) --
	( 88.42,179.52) --
	( 88.47,179.52) --
	( 88.53,179.53) --
	( 88.58,179.53) --
	( 88.63,179.54) --
	( 88.68,179.54) --
	( 88.73,179.55) --
	( 88.78,179.55) --
	( 88.83,179.56) --
	( 88.88,179.56) --
	( 88.94,179.57) --
	( 88.99,179.57) --
	( 89.04,179.58) --
	( 89.09,179.58) --
	( 89.14,179.58) --
	( 89.19,179.59) --
	( 89.24,179.59) --
	( 89.29,179.60) --
	( 89.35,179.60) --
	( 89.40,179.61) --
	( 89.45,179.61) --
	( 89.50,179.62) --
	( 89.55,179.62) --
	( 89.60,179.63) --
	( 89.65,179.63) --
	( 89.70,179.64) --
	( 89.75,179.64) --
	( 89.81,179.65) --
	( 89.86,179.65) --
	( 89.91,179.65) --
	( 89.96,179.66) --
	( 90.01,179.66) --
	( 90.06,179.67) --
	( 90.11,179.67) --
	( 90.16,179.68) --
	( 90.22,179.68) --
	( 90.27,179.69) --
	( 90.32,179.69) --
	( 90.37,179.70) --
	( 90.42,179.70) --
	( 90.47,179.71) --
	( 90.52,179.71) --
	( 90.57,179.71) --
	( 90.63,179.72) --
	( 90.68,179.72) --
	( 90.73,179.73) --
	( 90.78,179.73) --
	( 90.83,179.74) --
	( 90.88,179.74) --
	( 90.93,179.75) --
	( 90.98,179.75) --
	( 91.03,179.76) --
	( 91.09,179.76) --
	( 91.14,179.77) --
	( 91.19,179.77) --
	( 91.24,179.78) --
	( 91.29,179.78) --
	( 91.34,179.78) --
	( 91.39,179.79) --
	( 91.44,179.79) --
	( 91.50,179.80) --
	( 91.55,179.80) --
	( 91.60,179.81) --
	( 91.65,179.81) --
	( 91.70,179.82) --
	( 91.75,179.82) --
	( 91.80,179.83) --
	( 91.85,179.83) --
	( 91.91,179.84) --
	( 91.96,179.84) --
	( 92.01,179.85) --
	( 92.06,179.85) --
	( 92.11,179.85) --
	( 92.16,179.86) --
	( 92.21,179.86) --
	( 92.26,179.87) --
	( 92.31,179.87) --
	( 92.37,179.88) --
	( 92.42,179.88) --
	( 92.47,179.89) --
	( 92.52,179.89) --
	( 92.57,179.90) --
	( 92.62,179.90) --
	( 92.67,179.91) --
	( 92.72,179.91) --
	( 92.78,179.91) --
	( 92.83,179.92) --
	( 92.88,179.92) --
	( 92.93,179.93) --
	( 92.98,179.93) --
	( 93.03,179.94) --
	( 93.08,179.94) --
	( 93.13,179.95) --
	( 93.19,179.95) --
	( 93.24,179.96) --
	( 93.29,179.96) --
	( 93.34,179.97) --
	( 93.39,179.97) --
	( 93.44,179.98) --
	( 93.49,179.98) --
	( 93.54,179.98) --
	( 93.59,179.99) --
	( 93.65,179.99) --
	( 93.70,180.00) --
	( 93.75,180.00) --
	( 93.80,180.01) --
	( 93.85,180.01) --
	( 93.90,180.02) --
	( 93.95,180.02) --
	( 94.00,180.03) --
	( 94.06,180.03) --
	( 94.11,180.04) --
	( 94.16,180.04) --
	( 94.21,180.04) --
	( 94.26,180.05) --
	( 94.31,180.05) --
	( 94.36,180.06) --
	( 94.41,180.06) --
	( 94.47,180.07) --
	( 94.52,180.07) --
	( 94.57,180.08) --
	( 94.62,180.08) --
	( 94.67,180.09) --
	( 94.72,180.09) --
	( 94.77,180.10) --
	( 94.82,180.10) --
	( 94.87,180.11) --
	( 94.93,180.11) --
	( 94.98,180.11) --
	( 95.03,180.12) --
	( 95.08,180.12) --
	( 95.13,180.13) --
	( 95.18,180.13) --
	( 95.23,180.14) --
	( 95.28,180.14) --
	( 95.34,180.15) --
	( 95.39,180.15) --
	( 95.44,180.16) --
	( 95.49,180.16) --
	( 95.54,180.17) --
	( 95.59,180.17) --
	( 95.64,180.17) --
	( 95.69,180.18) --
	( 95.75,180.18) --
	( 95.80,180.19) --
	( 95.85,180.19) --
	( 95.90,180.20) --
	( 95.95,180.20) --
	( 96.00,180.21) --
	( 96.05,180.21) --
	( 96.10,180.22) --
	( 96.16,180.22) --
	( 96.21,180.23) --
	( 96.26,180.23) --
	( 96.31,180.24) --
	( 96.36,180.24) --
	( 96.41,180.24) --
	( 96.46,180.25) --
	( 96.51,180.25) --
	( 96.56,180.26) --
	( 96.62,180.26) --
	( 96.67,180.27) --
	( 96.72,180.27) --
	( 96.77,180.28) --
	( 96.82,180.28) --
	( 96.87,180.29) --
	( 96.92,180.29) --
	( 96.97,180.30) --
	( 97.03,180.30) --
	( 97.08,180.31) --
	( 97.13,180.31) --
	( 97.18,180.31) --
	( 97.23,180.32) --
	( 97.28,180.32) --
	( 97.33,180.33) --
	( 97.38,180.33) --
	( 97.44,180.34) --
	( 97.49,180.34) --
	( 97.54,180.35) --
	( 97.59,180.35) --
	( 97.64,180.36) --
	( 97.69,180.36) --
	( 97.74,180.37) --
	( 97.79,180.37) --
	( 97.84,180.37) --
	( 97.90,180.38) --
	( 97.95,180.38) --
	( 98.00,180.39) --
	( 98.05,180.39) --
	( 98.10,180.40) --
	( 98.15,180.40) --
	( 98.20,180.41) --
	( 98.25,180.41) --
	( 98.31,180.42) --
	( 98.36,180.42) --
	( 98.41,180.43) --
	( 98.46,180.43) --
	( 98.51,180.44) --
	( 98.56,180.44) --
	( 98.61,180.44) --
	( 98.66,180.45) --
	( 98.72,180.45) --
	( 98.77,180.46) --
	( 98.82,180.46) --
	( 98.87,180.47) --
	( 98.92,180.47) --
	( 98.97,180.48) --
	( 99.02,180.48) --
	( 99.07,180.49) --
	( 99.12,180.49) --
	( 99.18,180.50) --
	( 99.23,180.50) --
	( 99.28,180.50) --
	( 99.33,180.51) --
	( 99.38,180.51) --
	( 99.43,180.52) --
	( 99.48,180.52) --
	( 99.53,180.53) --
	( 99.59,180.53) --
	( 99.64,180.54) --
	( 99.69,180.54) --
	( 99.74,180.55) --
	( 99.79,180.55) --
	( 99.84,180.56) --
	( 99.89,180.56) --
	( 99.94,180.57) --
	(100.00,180.57) --
	(100.05,180.57) --
	(100.10,180.58) --
	(100.15,180.58) --
	(100.20,180.59) --
	(100.25,180.59) --
	(100.30,180.60) --
	(100.35,180.60) --
	(100.40,180.61) --
	(100.46,180.61) --
	(100.51,180.62) --
	(100.56,180.62) --
	(100.61,180.63) --
	(100.66,180.63) --
	(100.71,180.63) --
	(100.76,180.64) --
	(100.81,180.64) --
	(100.87,180.65) --
	(100.92,180.65) --
	(100.97,180.66) --
	(101.02,180.66) --
	(101.07,180.67) --
	(101.12,180.67) --
	(101.17,180.68) --
	(101.22,180.68) --
	(101.28,180.69) --
	(101.33,180.69) --
	(101.38,180.70) --
	(101.43,180.70) --
	(101.48,180.70) --
	(101.53,180.71) --
	(101.58,180.71) --
	(101.63,180.72) --
	(101.68,180.72) --
	(101.74,180.73) --
	(101.79,180.73) --
	(101.84,180.74) --
	(101.89,180.74) --
	(101.94,180.75) --
	(101.99,180.75) --
	(102.04,180.76) --
	(102.09,180.76) --
	(102.15,180.76) --
	(102.20,180.77) --
	(102.25,180.77) --
	(102.30,180.78) --
	(102.35,180.78) --
	(102.40,180.79) --
	(102.45,180.79) --
	(102.50,180.80) --
	(102.56,180.80) --
	(102.61,180.81) --
	(102.66,180.81) --
	(102.71,180.82) --
	(102.76,180.82) --
	(102.81,180.83) --
	(102.86,180.83) --
	(102.91,180.83) --
	(102.96,180.84) --
	(103.02,180.84) --
	(103.07,180.85) --
	(103.12,180.85) --
	(103.17,180.86) --
	(103.22,180.86) --
	(103.27,180.87) --
	(103.32,180.87) --
	(103.37,180.88) --
	(103.43,180.88) --
	(103.48,180.89) --
	(103.53,180.89) --
	(103.58,180.90) --
	(103.63,180.90) --
	(103.68,180.90) --
	(103.73,180.91) --
	(103.78,180.91) --
	(103.84,180.92) --
	(103.89,180.92) --
	(103.94,180.93) --
	(103.99,180.93) --
	(104.04,180.94) --
	(104.09,180.94) --
	(104.14,180.95) --
	(104.19,180.95) --
	(104.24,180.96) --
	(104.30,180.96) --
	(104.35,180.96) --
	(104.40,180.97) --
	(104.45,180.97) --
	(104.50,180.98) --
	(104.55,180.98) --
	(104.60,180.99) --
	(104.65,180.99) --
	(104.71,181.00) --
	(104.76,181.00) --
	(104.81,181.01) --
	(104.86,181.01) --
	(104.91,181.02) --
	(104.96,181.02) --
	(105.01,181.03) --
	(105.06,181.03) --
	(105.12,181.03) --
	(105.17,181.04) --
	(105.22,181.04) --
	(105.27,181.05) --
	(105.32,181.05) --
	(105.37,181.06) --
	(105.42,181.06) --
	(105.47,181.07) --
	(105.52,181.07) --
	(105.58,181.08) --
	(105.63,181.08) --
	(105.68,181.09) --
	(105.73,181.09) --
	(105.78,181.09) --
	(105.83,181.10) --
	(105.88,181.10) --
	(105.93,181.11) --
	(105.99,181.11) --
	(106.04,181.12) --
	(106.09,181.12) --
	(106.14,181.13) --
	(106.19,181.13) --
	(106.24,181.14) --
	(106.29,181.14) --
	(106.34,181.15) --
	(106.40,181.15) --
	(106.45,181.16) --
	(106.50,181.16) --
	(106.55,181.16) --
	(106.60,181.17) --
	(106.65,181.17) --
	(106.70,181.18) --
	(106.75,181.18) --
	(106.80,181.19) --
	(106.86,181.19) --
	(106.91,181.20) --
	(106.96,181.20) --
	(107.01,181.21) --
	(107.06,181.21) --
	(107.11,181.22) --
	(107.16,181.22) --
	(107.21,181.22) --
	(107.27,181.23) --
	(107.32,181.23) --
	(107.37,181.24) --
	(107.42,181.24) --
	(107.47,181.25) --
	(107.52,181.25) --
	(107.57,181.26) --
	(107.62,181.26) --
	(107.68,181.27) --
	(107.73,181.27) --
	(107.78,181.28) --
	(107.83,181.28) --
	(107.88,181.29) --
	(107.93,181.29) --
	(107.98,181.29) --
	(108.03,181.30) --
	(108.08,181.30) --
	(108.14,181.31) --
	(108.19,181.31) --
	(108.24,181.32) --
	(108.29,181.32) --
	(108.34,181.33) --
	(108.39,181.33) --
	(108.44,181.34) --
	(108.49,181.34) --
	(108.55,181.35) --
	(108.60,181.35) --
	(108.65,181.35) --
	(108.70,181.36) --
	(108.75,181.36) --
	(108.80,181.37) --
	(108.85,181.37) --
	(108.90,181.38) --
	(108.96,181.38) --
	(109.01,181.39) --
	(109.06,181.39) --
	(109.11,181.40) --
	(109.16,181.40) --
	(109.21,181.41) --
	(109.26,181.41) --
	(109.31,181.42) --
	(109.36,181.42) --
	(109.42,181.42) --
	(109.47,181.43) --
	(109.52,181.43) --
	(109.57,181.44) --
	(109.62,181.44) --
	(109.67,181.45) --
	(109.72,181.45) --
	(109.77,181.46) --
	(109.83,181.46) --
	(109.88,181.47) --
	(109.93,181.47) --
	(109.98,181.48) --
	(110.03,181.48) --
	(110.08,181.49) --
	(110.13,181.49) --
	(110.18,181.49) --
	(110.24,181.50) --
	(110.29,181.50) --
	(110.34,181.51) --
	(110.39,181.51) --
	(110.44,181.52) --
	(110.49,181.52) --
	(110.54,181.53) --
	(110.59,181.53) --
	(110.64,181.54) --
	(110.70,181.54) --
	(110.75,181.55) --
	(110.80,181.55) --
	(110.85,181.55) --
	(110.90,181.56) --
	(110.95,181.56) --
	(111.00,181.57) --
	(111.05,181.57) --
	(111.11,181.58) --
	(111.16,181.58) --
	(111.21,181.59) --
	(111.26,181.59) --
	(111.31,181.60) --
	(111.36,181.60) --
	(111.41,181.61) --
	(111.46,181.61) --
	(111.52,181.62) --
	(111.57,181.62) --
	(111.62,181.62) --
	(111.67,181.63) --
	(111.72,181.63) --
	(111.77,181.64) --
	(111.82,181.64) --
	(111.87,181.65) --
	(111.92,181.65) --
	(111.98,181.66) --
	(112.03,181.66) --
	(112.08,181.67) --
	(112.13,181.67) --
	(112.18,181.68) --
	(112.23,181.68) --
	(112.28,181.68) --
	(112.33,181.69) --
	(112.39,181.69) --
	(112.44,181.70) --
	(112.49,181.70) --
	(112.54,181.71) --
	(112.59,181.71) --
	(112.64,181.72) --
	(112.69,181.72) --
	(112.74,181.73) --
	(112.80,181.73) --
	(112.85,181.74) --
	(112.90,181.74) --
	(112.95,181.75) --
	(113.00,181.75) --
	(113.05,181.75) --
	(113.10,181.76) --
	(113.15,181.76) --
	(113.20,181.77) --
	(113.26,181.77) --
	(113.31,181.78) --
	(113.36,181.78) --
	(113.41,181.79) --
	(113.46,181.79) --
	(113.51,181.80) --
	(113.56,181.80) --
	(113.61,181.81) --
	(113.67,181.81) --
	(113.72,181.81) --
	(113.77,181.82) --
	(113.82,181.82) --
	(113.87,181.83) --
	(113.92,181.83) --
	(113.97,181.84) --
	(114.02,181.84) --
	(114.08,181.85) --
	(114.13,181.85) --
	(114.18,181.86) --
	(114.23,181.86) --
	(114.28,181.87) --
	(114.33,181.87) --
	(114.38,181.88) --
	(114.43,181.88) --
	(114.48,181.88) --
	(114.54,181.89) --
	(114.59,181.89) --
	(114.64,181.90) --
	(114.69,181.90) --
	(114.74,181.91) --
	(114.79,181.91) --
	(114.84,181.92) --
	(114.89,181.92) --
	(114.95,181.93) --
	(115.00,181.93) --
	(115.05,181.94) --
	(115.10,181.94) --
	(115.15,181.95) --
	(115.20,181.95) --
	(115.25,181.95) --
	(115.30,181.96) --
	(115.36,181.96) --
	(115.41,181.97) --
	(115.46,181.97) --
	(115.51,181.98) --
	(115.56,181.98) --
	(115.61,181.99) --
	(115.66,181.99) --
	(115.71,182.00) --
	(115.76,182.00) --
	(115.82,182.01) --
	(115.87,182.01) --
	(115.92,182.01) --
	(115.97,182.02) --
	(116.02,182.02) --
	(116.07,182.03) --
	(116.12,182.03) --
	(116.17,182.04) --
	(116.23,182.04) --
	(116.28,182.05) --
	(116.33,182.05) --
	(116.38,182.06) --
	(116.43,182.06) --
	(116.48,182.07) --
	(116.53,182.07) --
	(116.58,182.08) --
	(116.64,182.08) --
	(116.69,182.08) --
	(116.74,182.09) --
	(116.79,182.09) --
	(116.84,182.10) --
	(116.89,182.10) --
	(116.94,182.11) --
	(116.99,182.11) --
	(117.04,182.12) --
	(117.10,182.12) --
	(117.15,182.13) --
	(117.20,182.13) --
	(117.25,182.14) --
	(117.30,182.14) --
	(117.35,182.14) --
	(117.40,182.15) --
	(117.45,182.15) --
	(117.51,182.16) --
	(117.56,182.16) --
	(117.61,182.17) --
	(117.66,182.17) --
	(117.71,182.18) --
	(117.76,182.18) --
	(117.81,182.19) --
	(117.86,182.19) --
	(117.92,182.20) --
	(117.97,182.20) --
	(118.02,182.21) --
	(118.07,182.21) --
	(118.12,182.21) --
	(118.17,182.22) --
	(118.22,182.22) --
	(118.27,182.23) --
	(118.32,182.23) --
	(118.38,182.24) --
	(118.43,182.24) --
	(118.48,182.25) --
	(118.53,182.25) --
	(118.58,182.26) --
	(118.63,182.26) --
	(118.68,182.27) --
	(118.73,182.27) --
	(118.79,182.27) --
	(118.84,182.28) --
	(118.89,182.28) --
	(118.94,182.29) --
	(118.99,182.29) --
	(119.04,182.30) --
	(119.09,182.30) --
	(119.14,182.31) --
	(119.20,182.31) --
	(119.25,182.32) --
	(119.30,182.32) --
	(119.35,182.33) --
	(119.40,182.33) --
	(119.45,182.34) --
	(119.50,182.34) --
	(119.55,182.34) --
	(119.60,182.35) --
	(119.66,182.35) --
	(119.71,182.36) --
	(119.76,182.36) --
	(119.81,182.37) --
	(119.86,182.37) --
	(119.91,182.38) --
	(119.96,182.38) --
	(120.01,182.39) --
	(120.07,182.39) --
	(120.12,182.40) --
	(120.17,182.40) --
	(120.22,182.40) --
	(120.27,182.41) --
	(120.32,182.41) --
	(120.37,182.42) --
	(120.42,182.42) --
	(120.48,182.43) --
	(120.53,182.43) --
	(120.58,182.44) --
	(120.63,182.44) --
	(120.68,182.45) --
	(120.73,182.45) --
	(120.78,182.46) --
	(120.83,182.46) --
	(120.88,182.47) --
	(120.94,182.47) --
	(120.99,182.47) --
	(121.04,182.48) --
	(121.09,182.48) --
	(121.14,182.49) --
	(121.19,182.49) --
	(121.24,182.50) --
	(121.29,182.50) --
	(121.35,182.51) --
	(121.40,182.51) --
	(121.45,182.52) --
	(121.50,182.52) --
	(121.55,182.53) --
	(121.60,182.53) --
	(121.65,182.54) --
	(121.70,182.54) --
	(121.76,182.54) --
	(121.81,182.55) --
	(121.86,182.55) --
	(121.91,182.56) --
	(121.96,182.56) --
	(122.01,182.57) --
	(122.06,182.57) --
	(122.11,182.58) --
	(122.17,182.58) --
	(122.22,182.59) --
	(122.27,182.59) --
	(122.32,182.60) --
	(122.37,182.60) --
	(122.42,182.60) --
	(122.47,182.61) --
	(122.52,182.61) --
	(122.57,182.62) --
	(122.63,182.62) --
	(122.68,182.63) --
	(122.73,182.63) --
	(122.78,182.64) --
	(122.83,182.64) --
	(122.88,182.65) --
	(122.93,182.65) --
	(122.98,182.66) --
	(123.04,182.66) --
	(123.09,182.67) --
	(123.14,182.67) --
	(123.19,182.67) --
	(123.24,182.68) --
	(123.29,182.68) --
	(123.34,182.69) --
	(123.39,182.69) --
	(123.45,182.70) --
	(123.50,182.70) --
	(123.55,182.71) --
	(123.60,182.71) --
	(123.65,182.72) --
	(123.70,182.72) --
	(123.75,182.73) --
	(123.80,182.73) --
	(123.85,182.73) --
	(123.91,182.74) --
	(123.96,182.74) --
	(124.01,182.75) --
	(124.06,182.75) --
	(124.11,182.76) --
	(124.16,182.76) --
	(124.21,182.77) --
	(124.26,182.77) --
	(124.32,182.78) --
	(124.37,182.78) --
	(124.42,182.79) --
	(124.47,182.79) --
	(124.52,182.80) --
	(124.57,182.80) --
	(124.62,182.80) --
	(124.67,182.81) --
	(124.73,182.81) --
	(124.78,182.82) --
	(124.83,182.82) --
	(124.88,182.83) --
	(124.93,182.83) --
	(124.98,182.84) --
	(125.03,182.84) --
	(125.08,182.85) --
	(125.13,182.85) --
	(125.19,182.86) --
	(125.24,182.86) --
	(125.29,182.86) --
	(125.34,182.87) --
	(125.39,182.87) --
	(125.44,182.88) --
	(125.49,182.88) --
	(125.54,182.89) --
	(125.60,182.89) --
	(125.65,182.90) --
	(125.70,182.90) --
	(125.75,182.91) --
	(125.80,182.91) --
	(125.85,182.92) --
	(125.90,182.92) --
	(125.95,182.93) --
	(126.01,182.93) --
	(126.06,182.93) --
	(126.11,182.94) --
	(126.16,182.94) --
	(126.21,182.95) --
	(126.26,182.95) --
	(126.31,182.96) --
	(126.36,182.96) --
	(126.41,182.97) --
	(126.47,182.97) --
	(126.52,182.98) --
	(126.57,182.98) --
	(126.62,182.99) --
	(126.67,182.99) --
	(126.72,183.00) --
	(126.77,183.00) --
	(126.82,183.00) --
	(126.88,183.01) --
	(126.93,183.01) --
	(126.98,183.02) --
	(127.03,183.02) --
	(127.08,183.03) --
	(127.13,183.03) --
	(127.18,183.04) --
	(127.23,183.04) --
	(127.29,183.05) --
	(127.34,183.05) --
	(127.39,183.06) --
	(127.44,183.06) --
	(127.49,183.06) --
	(127.54,183.07) --
	(127.59,183.07) --
	(127.64,183.08) --
	(127.69,183.08) --
	(127.75,183.09) --
	(127.80,183.09) --
	(127.85,183.10) --
	(127.90,183.10) --
	(127.95,183.11) --
	(128.00,183.11) --
	(128.05,183.12) --
	(128.10,183.12) --
	(128.16,183.13) --
	(128.21,183.13) --
	(128.26,183.13) --
	(128.31,183.14) --
	(128.36,183.14) --
	(128.41,183.15) --
	(128.46,183.15) --
	(128.51,183.16) --
	(128.57,183.16) --
	(128.62,183.17) --
	(128.67,183.17) --
	(128.72,183.18) --
	(128.77,183.18) --
	(128.82,183.19) --
	(128.87,183.19) --
	(128.92,183.19) --
	(128.97,183.20) --
	(129.03,183.20) --
	(129.08,183.21) --
	(129.13,183.21) --
	(129.18,183.22) --
	(129.23,183.22) --
	(129.28,183.23) --
	(129.33,183.23) --
	(129.38,183.24) --
	(129.44,183.24) --
	(129.49,183.25) --
	(129.54,183.25) --
	(129.59,183.26) --
	(129.64,183.26) --
	(129.69,183.26) --
	(129.74,183.27) --
	(129.79,183.27) --
	(129.85,183.28) --
	(129.90,183.28) --
	(129.95,183.29) --
	(130.00,183.29) --
	(130.05,183.30) --
	(130.10,183.30) --
	(130.15,183.31) --
	(130.20,183.31) --
	(130.25,183.32) --
	(130.31,183.32) --
	(130.36,183.32) --
	(130.41,183.33) --
	(130.46,183.33) --
	(130.51,183.34) --
	(130.56,183.34) --
	(130.61,183.35) --
	(130.66,183.35) --
	(130.72,183.36) --
	(130.77,183.36) --
	(130.82,183.37) --
	(130.87,183.37) --
	(130.92,183.38) --
	(130.97,183.38) --
	(131.02,183.39) --
	(131.07,183.39) --
	(131.13,183.39) --
	(131.18,183.40) --
	(131.23,183.40) --
	(131.28,183.41) --
	(131.33,183.41) --
	(131.38,183.42) --
	(131.43,183.42) --
	(131.48,183.43) --
	(131.53,183.43) --
	(131.59,183.44) --
	(131.64,183.44) --
	(131.69,183.45) --
	(131.74,183.45) --
	(131.79,183.45) --
	(131.84,183.46) --
	(131.89,183.46) --
	(131.94,183.47) --
	(132.00,183.47) --
	(132.05,183.48) --
	(132.10,183.48) --
	(132.15,183.49) --
	(132.20,183.49) --
	(132.25,183.50) --
	(132.30,183.50) --
	(132.35,183.51) --
	(132.41,183.51) --
	(132.46,183.52) --
	(132.51,183.52) --
	(132.56,183.52) --
	(132.61,183.53) --
	(132.66,183.53) --
	(132.71,183.54) --
	(132.76,183.54) --
	(132.81,183.55) --
	(132.87,183.55) --
	(132.92,183.56) --
	(132.97,183.56) --
	(133.02,183.57) --
	(133.07,183.57) --
	(133.12,183.58) --
	(133.17,183.58) --
	(133.22,183.59) --
	(133.28,183.59) --
	(133.33,183.59) --
	(133.38,183.60) --
	(133.43,183.60) --
	(133.48,183.61) --
	(133.53,183.61) --
	(133.58,183.62) --
	(133.63,183.62) --
	(133.69,183.63) --
	(133.74,183.63) --
	(133.79,183.64) --
	(133.84,183.64) --
	(133.89,183.65) --
	(133.94,183.65) --
	(133.99,183.65) --
	(134.04,183.66) --
	(134.09,183.66) --
	(134.15,183.67) --
	(134.20,183.67) --
	(134.25,183.68) --
	(134.30,183.68) --
	(134.35,183.69) --
	(134.40,183.69) --
	(134.45,183.70) --
	(134.50,183.70) --
	(134.56,183.71) --
	(134.61,183.71) --
	(134.66,183.72) --
	(134.71,183.72) --
	(134.76,183.72) --
	(134.81,183.73) --
	(134.86,183.73) --
	(134.91,183.74) --
	(134.97,183.74) --
	(135.02,183.75) --
	(135.07,183.75) --
	(135.12,183.76) --
	(135.17,183.76) --
	(135.22,183.77) --
	(135.27,183.77) --
	(135.32,183.78) --
	(135.37,183.78) --
	(135.43,183.78) --
	(135.48,183.79) --
	(135.53,183.79) --
	(135.58,183.80) --
	(135.63,183.80) --
	(135.68,183.81) --
	(135.73,183.81) --
	(135.78,183.82) --
	(135.84,183.82) --
	(135.89,183.83) --
	(135.94,183.83) --
	(135.99,183.84) --
	(136.04,183.84) --
	(136.09,183.85) --
	(136.14,183.85) --
	(136.19,183.85) --
	(136.25,183.86) --
	(136.30,183.86) --
	(136.35,183.87) --
	(136.40,183.87) --
	(136.45,183.88) --
	(136.50,183.88) --
	(136.55,183.89) --
	(136.60,183.89) --
	(136.65,183.90) --
	(136.71,183.90) --
	(136.76,183.91) --
	(136.81,183.91) --
	(136.86,183.91) --
	(136.91,183.92) --
	(136.96,183.92) --
	(137.01,183.93) --
	(137.06,183.93) --
	(137.12,183.94) --
	(137.17,183.94) --
	(137.22,183.95) --
	(137.27,183.95) --
	(137.32,183.96) --
	(137.37,183.96) --
	(137.42,183.97) --
	(137.47,183.97) --
	(137.53,183.98) --
	(137.58,183.98) --
	(137.63,183.98) --
	(137.68,183.99) --
	(137.73,183.99) --
	(137.78,184.00) --
	(137.83,184.00) --
	(137.88,184.01) --
	(137.93,184.01) --
	(137.99,184.02) --
	(138.04,184.02) --
	(138.09,184.03) --
	(138.14,184.03) --
	(138.19,184.04) --
	(138.24,184.04) --
	(138.29,184.05) --
	(138.34,184.05) --
	(138.40,184.05) --
	(138.45,184.06) --
	(138.50,184.06) --
	(138.55,184.07) --
	(138.60,184.07) --
	(138.65,184.08) --
	(138.70,184.08) --
	(138.75,184.09) --
	(138.81,184.09) --
	(138.86,184.10) --
	(138.91,184.10) --
	(138.96,184.11) --
	(139.01,184.11) --
	(139.06,184.11) --
	(139.11,184.12) --
	(139.16,184.12) --
	(139.21,184.13) --
	(139.27,184.13) --
	(139.32,184.14) --
	(139.37,184.14) --
	(139.42,184.15) --
	(139.47,184.15) --
	(139.52,184.16) --
	(139.57,184.16) --
	(139.62,184.17) --
	(139.68,184.17) --
	(139.73,184.18) --
	(139.78,184.18) --
	(139.83,184.18) --
	(139.88,184.19) --
	(139.93,184.19) --
	(139.98,184.20) --
	(140.03,184.20) --
	(140.09,184.21) --
	(140.14,184.21) --
	(140.19,184.22) --
	(140.24,184.22) --
	(140.29,184.23) --
	(140.34,184.23) --
	(140.39,184.24) --
	(140.44,184.24) --
	(140.49,184.24) --
	(140.55,184.25) --
	(140.60,184.25) --
	(140.65,184.26) --
	(140.70,184.26) --
	(140.75,184.27) --
	(140.80,184.27) --
	(140.85,184.28) --
	(140.90,184.28) --
	(140.96,184.29) --
	(141.01,184.29) --
	(141.06,184.30) --
	(141.11,184.30) --
	(141.16,184.31) --
	(141.21,184.31) --
	(141.26,184.31) --
	(141.31,184.32) --
	(141.37,184.32) --
	(141.42,184.33) --
	(141.47,184.33) --
	(141.52,184.34) --
	(141.57,184.34) --
	(141.62,184.35) --
	(141.67,184.35) --
	(141.72,184.36) --
	(141.77,184.36) --
	(141.83,184.37) --
	(141.88,184.37) --
	(141.93,184.37) --
	(141.98,184.38) --
	(142.03,184.38) --
	(142.08,184.39) --
	(142.13,184.39) --
	(142.18,184.40) --
	(142.24,184.40) --
	(142.29,184.41) --
	(142.34,184.41) --
	(142.39,184.42) --
	(142.44,184.42) --
	(142.49,184.43) --
	(142.54,184.43) --
	(142.59,184.44) --
	(142.65,184.44) --
	(142.70,184.44) --
	(142.75,184.45) --
	(142.80,184.45) --
	(142.85,184.46) --
	(142.90,184.46) --
	(142.95,184.47) --
	(143.00,184.47) --
	(143.05,184.48) --
	(143.11,184.48) --
	(143.16,184.49) --
	(143.21,184.49) --
	(143.26,184.50) --
	(143.31,184.50) --
	(143.36,184.50) --
	(143.41,184.51) --
	(143.46,184.51) --
	(143.52,184.52) --
	(143.57,184.52) --
	(143.62,184.53) --
	(143.67,184.53) --
	(143.72,184.54) --
	(143.77,184.54) --
	(143.82,184.55) --
	(143.87,184.55) --
	(143.93,184.56) --
	(143.98,184.56) --
	(144.03,184.57) --
	(144.08,184.57) --
	(144.13,184.57) --
	(144.18,184.58) --
	(144.23,184.58) --
	(144.28,184.59) --
	(144.33,184.59) --
	(144.39,184.60) --
	(144.44,184.60) --
	(144.49,184.61) --
	(144.54,184.61) --
	(144.59,184.62) --
	(144.64,184.62) --
	(144.69,184.63) --
	(144.74,184.63) --
	(144.80,184.64) --
	(144.85,184.64) --
	(144.90,184.64) --
	(144.95,184.65) --
	(145.00,184.65) --
	(145.05,184.66) --
	(145.10,184.66) --
	(145.15,184.67) --
	(145.21,184.67) --
	(145.26,184.68) --
	(145.31,184.68) --
	(145.36,184.69) --
	(145.41,184.69) --
	(145.46,184.70) --
	(145.51,184.70) --
	(145.56,184.70) --
	(145.61,184.71) --
	(145.67,184.71) --
	(145.72,184.72) --
	(145.77,184.72) --
	(145.82,184.73) --
	(145.87,184.73) --
	(145.92,184.74) --
	(145.97,184.74) --
	(146.02,184.75) --
	(146.08,184.75) --
	(146.13,184.76) --
	(146.18,184.76) --
	(146.23,184.77) --
	(146.28,184.77) --
	(146.33,184.77) --
	(146.38,184.78) --
	(146.43,184.78) --
	(146.49,184.79) --
	(146.54,184.79) --
	(146.59,184.80) --
	(146.64,184.80) --
	(146.69,184.81) --
	(146.74,184.81) --
	(146.79,184.82) --
	(146.84,184.82) --
	(146.89,184.83) --
	(146.95,184.83) --
	(147.00,184.83) --
	(147.05,184.84) --
	(147.10,184.84) --
	(147.15,184.85) --
	(147.20,184.85) --
	(147.25,184.86) --
	(147.30,184.86) --
	(147.36,184.87) --
	(147.41,184.87) --
	(147.46,184.88) --
	(147.51,184.88) --
	(147.56,184.89) --
	(147.61,184.89) --
	(147.66,184.90) --
	(147.71,184.90) --
	(147.77,184.90) --
	(147.82,184.91) --
	(147.87,184.91) --
	(147.92,184.92) --
	(147.97,184.92) --
	(148.02,184.93) --
	(148.07,184.93) --
	(148.12,184.94) --
	(148.18,184.94) --
	(148.23,184.95) --
	(148.28,184.95) --
	(148.33,184.96) --
	(148.38,184.96) --
	(148.43,184.96) --
	(148.48,184.97) --
	(148.53,184.97) --
	(148.58,184.98) --
	(148.64,184.98) --
	(148.69,184.99) --
	(148.74,184.99) --
	(148.79,185.00) --
	(148.84,185.00) --
	(148.89,185.01) --
	(148.94,185.01) --
	(148.99,185.02) --
	(149.05,185.02) --
	(149.10,185.03) --
	(149.15,185.03) --
	(149.20,185.03) --
	(149.25,185.04) --
	(149.30,185.04) --
	(149.35,185.05) --
	(149.40,185.05) --
	(149.46,185.06) --
	(149.51,185.06) --
	(149.56,185.07) --
	(149.61,185.07) --
	(149.66,185.08) --
	(149.71,185.08) --
	(149.76,185.09) --
	(149.81,185.09) --
	(149.86,185.10) --
	(149.92,185.10) --
	(149.97,185.10) --
	(150.02,185.11) --
	(150.07,185.11) --
	(150.12,185.12) --
	(150.17,185.12) --
	(150.22,185.13) --
	(150.27,185.13) --
	(150.33,185.14) --
	(150.38,185.14) --
	(150.43,185.15) --
	(150.48,185.15) --
	(150.53,185.16) --
	(150.58,185.16) --
	(150.63,185.16) --
	(150.68,185.17) --
	(150.74,185.17) --
	(150.79,185.18) --
	(150.84,185.18) --
	(150.89,185.19) --
	(150.94,185.19) --
	(150.99,185.20) --
	(151.04,185.20) --
	(151.09,185.21) --
	(151.14,185.21) --
	(151.20,185.22) --
	(151.25,185.22) --
	(151.30,185.23) --
	(151.35,185.23) --
	(151.40,185.23) --
	(151.45,185.24) --
	(151.50,185.24) --
	(151.55,185.25) --
	(151.61,185.25) --
	(151.66,185.26) --
	(151.71,185.26) --
	(151.76,185.27) --
	(151.81,185.27) --
	(151.86,185.28) --
	(151.91,185.28) --
	(151.96,185.29) --
	(152.02,185.29) --
	(152.07,185.29) --
	(152.12,185.30) --
	(152.17,185.30) --
	(152.22,185.31) --
	(152.27,185.31) --
	(152.32,185.32) --
	(152.37,185.32) --
	(152.42,185.33) --
	(152.48,185.33) --
	(152.53,185.34) --
	(152.58,185.34) --
	(152.63,185.35) --
	(152.68,185.35) --
	(152.73,185.36) --
	(152.78,185.36) --
	(152.83,185.36) --
	(152.89,185.37) --
	(152.94,185.37) --
	(152.99,185.38) --
	(153.04,185.38) --
	(153.09,185.39) --
	(153.14,185.39) --
	(153.19,185.40) --
	(153.24,185.40) --
	(153.30,185.41) --
	(153.35,185.41) --
	(153.40,185.42) --
	(153.45,185.42) --
	(153.50,185.42) --
	(153.55,185.43) --
	(153.60,185.43) --
	(153.65,185.44) --
	(153.70,185.44) --
	(153.76,185.45) --
	(153.81,185.45) --
	(153.86,185.46) --
	(153.91,185.46) --
	(153.96,185.47) --
	(154.01,185.47) --
	(154.06,185.48) --
	(154.11,185.48) --
	(154.17,185.49) --
	(154.22,185.49) --
	(154.27,185.49) --
	(154.32,185.50) --
	(154.37,185.50) --
	(154.42,185.51) --
	(154.47,185.51) --
	(154.52,185.52) --
	(154.58,185.52) --
	(154.63,185.53) --
	(154.68,185.53) --
	(154.73,185.54) --
	(154.78,185.54) --
	(154.83,185.55) --
	(154.88,185.55) --
	(154.93,185.55) --
	(154.98,185.56) --
	(155.04,185.56) --
	(155.09,185.57) --
	(155.14,185.57) --
	(155.19,185.58) --
	(155.24,185.58) --
	(155.29,185.59) --
	(155.34,185.59) --
	(155.39,185.60) --
	(155.45,185.60) --
	(155.50,185.61) --
	(155.55,185.61) --
	(155.60,185.62) --
	(155.65,185.62) --
	(155.70,185.62) --
	(155.75,185.63) --
	(155.80,185.63) --
	(155.86,185.64) --
	(155.91,185.64) --
	(155.96,185.65) --
	(156.01,185.65) --
	(156.06,185.66) --
	(156.11,185.66) --
	(156.16,185.67) --
	(156.21,185.67) --
	(156.26,185.68) --
	(156.32,185.68) --
	(156.37,185.69) --
	(156.42,185.69) --
	(156.47,185.69) --
	(156.52,185.70) --
	(156.57,185.70) --
	(156.62,185.71) --
	(156.67,185.71) --
	(156.73,185.72) --
	(156.78,185.72) --
	(156.83,185.73) --
	(156.88,185.73) --
	(156.93,185.74) --
	(156.98,185.74) --
	(157.03,185.75) --
	(157.08,185.75) --
	(157.14,185.75) --
	(157.19,185.76) --
	(157.24,185.76) --
	(157.29,185.77) --
	(157.34,185.77) --
	(157.39,185.78) --
	(157.44,185.78) --
	(157.49,185.79) --
	(157.54,185.79) --
	(157.60,185.80) --
	(157.65,185.80) --
	(157.70,185.81) --
	(157.75,185.81) --
	(157.80,185.82) --
	(157.85,185.82) --
	(157.90,185.82) --
	(157.95,185.83) --
	(158.01,185.83) --
	(158.06,185.84) --
	(158.11,185.84) --
	(158.16,185.85) --
	(158.21,185.85) --
	(158.26,185.86) --
	(158.31,185.86) --
	(158.36,185.87) --
	(158.42,185.87) --
	(158.47,185.88) --
	(158.52,185.88) --
	(158.57,185.88) --
	(158.62,185.89) --
	(158.67,185.89) --
	(158.72,185.90) --
	(158.77,185.90) --
	(158.82,185.91) --
	(158.88,185.91) --
	(158.93,185.92) --
	(158.98,185.92) --
	(159.03,185.93) --
	(159.08,185.93) --
	(159.13,185.94) --
	(159.18,185.94) --
	(159.23,185.95) --
	(159.29,185.95) --
	(159.34,185.95) --
	(159.39,185.96) --
	(159.44,185.96) --
	(159.49,185.97) --
	(159.54,185.97) --
	(159.59,185.98) --
	(159.64,185.98) --
	(159.70,185.99) --
	(159.75,185.99) --
	(159.80,186.00) --
	(159.85,186.00) --
	(159.90,186.01) --
	(159.95,186.01) --
	(160.00,186.01) --
	(160.05,186.02) --
	(160.10,186.02) --
	(160.16,186.03) --
	(160.21,186.03) --
	(160.26,186.04) --
	(160.31,186.04) --
	(160.36,186.05) --
	(160.41,186.05) --
	(160.46,186.06) --
	(160.51,186.06) --
	(160.57,186.07) --
	(160.62,186.07) --
	(160.67,186.08) --
	(160.72,186.08) --
	(160.77,186.08) --
	(160.82,186.09) --
	(160.87,186.09) --
	(160.92,186.10) --
	(160.98,186.10) --
	(161.03,186.11) --
	(161.08,186.11) --
	(161.13,186.12) --
	(161.18,186.12) --
	(161.23,186.13) --
	(161.28,186.13) --
	(161.33,186.14) --
	(161.38,186.14) --
	(161.44,186.15) --
	(161.49,186.15) --
	(161.54,186.15) --
	(161.59,186.16) --
	(161.64,186.16) --
	(161.69,186.17) --
	(161.74,186.17) --
	(161.79,186.18) --
	(161.85,186.18) --
	(161.90,186.19) --
	(161.95,186.19) --
	(162.00,186.20) --
	(162.05,186.20) --
	(162.10,186.21) --
	(162.15,186.21) --
	(162.20,186.21) --
	(162.26,186.22) --
	(162.31,186.22) --
	(162.36,186.23) --
	(162.41,186.23) --
	(162.46,186.24) --
	(162.51,186.24) --
	(162.56,186.25) --
	(162.61,186.25) --
	(162.66,186.26) --
	(162.72,186.26) --
	(162.77,186.27) --
	(162.82,186.27) --
	(162.87,186.28) --
	(162.92,186.28) --
	(162.97,186.28) --
	(163.02,186.29) --
	(163.07,186.29) --
	(163.13,186.30) --
	(163.18,186.30) --
	(163.23,186.31) --
	(163.28,186.31) --
	(163.33,186.32) --
	(163.38,186.32) --
	(163.43,186.33) --
	(163.48,186.33) --
	(163.54,186.34) --
	(163.59,186.34) --
	(163.64,186.34) --
	(163.69,186.35) --
	(163.74,186.35) --
	(163.79,186.36) --
	(163.84,186.36) --
	(163.89,186.37) --
	(163.94,186.37) --
	(164.00,186.38) --
	(164.05,186.38) --
	(164.10,186.39) --
	(164.15,186.39) --
	(164.20,186.40) --
	(164.25,186.40) --
	(164.30,186.41) --
	(164.35,186.41) --
	(164.41,186.41) --
	(164.46,186.42) --
	(164.51,186.42) --
	(164.56,186.43) --
	(164.61,186.43) --
	(164.66,186.44) --
	(164.71,186.44) --
	(164.76,186.45) --
	(164.82,186.45) --
	(164.87,186.46) --
	(164.92,186.46) --
	(164.97,186.47) --
	(165.02,186.47) --
	(165.07,186.47) --
	(165.12,186.48) --
	(165.17,186.48) --
	(165.22,186.49) --
	(165.28,186.49) --
	(165.33,186.50) --
	(165.38,186.50) --
	(165.43,186.51) --
	(165.48,186.51) --
	(165.53,186.52) --
	(165.58,186.52) --
	(165.63,186.53) --
	(165.69,186.53) --
	(165.74,186.54) --
	(165.79,186.54) --
	(165.84,186.54) --
	(165.89,186.55) --
	(165.94,186.55) --
	(165.99,186.56) --
	(166.04,186.56) --
	(166.10,186.57) --
	(166.15,186.57) --
	(166.20,186.58) --
	(166.25,186.58) --
	(166.30,186.59) --
	(166.35,186.59) --
	(166.40,186.60) --
	(166.45,186.60) --
	(166.50,186.60) --
	(166.56,186.61) --
	(166.61,186.61) --
	(166.66,186.62) --
	(166.71,186.62) --
	(166.76,186.63) --
	(166.81,186.63) --
	(166.86,186.64) --
	(166.91,186.64) --
	(166.97,186.65) --
	(167.02,186.65) --
	(167.07,186.66) --
	(167.12,186.66) --
	(167.17,186.67) --
	(167.22,186.67) --
	(167.27,186.67) --
	(167.32,186.68) --
	(167.38,186.68) --
	(167.43,186.69) --
	(167.48,186.69) --
	(167.53,186.70) --
	(167.58,186.70) --
	(167.63,186.71) --
	(167.68,186.71) --
	(167.73,186.72) --
	(167.78,186.72) --
	(167.84,186.73) --
	(167.89,186.73) --
	(167.94,186.74) --
	(167.99,186.74) --
	(168.04,186.74) --
	(168.09,186.75) --
	(168.14,186.75) --
	(168.19,186.76) --
	(168.25,186.76) --
	(168.30,186.77) --
	(168.35,186.77) --
	(168.40,186.78) --
	(168.45,186.78) --
	(168.50,186.79) --
	(168.55,186.79) --
	(168.60,186.80) --
	(168.66,186.80) --
	(168.71,186.80) --
	(168.76,186.81) --
	(168.81,186.81) --
	(168.86,186.82) --
	(168.91,186.82) --
	(168.96,186.83) --
	(169.01,186.83) --
	(169.06,186.84) --
	(169.12,186.84) --
	(169.17,186.85) --
	(169.22,186.85) --
	(169.27,186.86) --
	(169.32,186.86) --
	(169.37,186.87) --
	(169.42,186.87) --
	(169.47,186.87) --
	(169.53,186.88) --
	(169.58,186.88) --
	(169.63,186.89) --
	(169.68,186.89) --
	(169.73,186.90) --
	(169.78,186.90) --
	(169.83,186.91) --
	(169.88,186.91) --
	(169.94,186.92) --
	(169.99,186.92) --
	(170.04,186.93) --
	(170.09,186.93) --
	(170.14,186.93) --
	(170.19,186.94) --
	(170.24,186.94) --
	(170.29,186.95) --
	(170.34,186.95) --
	(170.40,186.96) --
	(170.45,186.96) --
	(170.50,186.97) --
	(170.55,186.97) --
	(170.60,186.98) --
	(170.65,186.98) --
	(170.70,186.99) --
	(170.75,186.99) --
	(170.81,187.00) --
	(170.86,187.00) --
	(170.91,187.00) --
	(170.96,187.01) --
	(171.01,187.01) --
	(171.06,187.02) --
	(171.11,187.02) --
	(171.16,187.03) --
	(171.22,187.03) --
	(171.27,187.04) --
	(171.32,187.04) --
	(171.37,187.05) --
	(171.42,187.05) --
	(171.47,187.06) --
	(171.52,187.06) --
	(171.57,187.06) --
	(171.62,187.07) --
	(171.68,187.07) --
	(171.73,187.08) --
	(171.78,187.08) --
	(171.83,187.09) --
	(171.88,187.09) --
	(171.93,187.10) --
	(171.98,187.10) --
	(172.03,187.11) --
	(172.09,187.11) --
	(172.14,187.12) --
	(172.19,187.12) --
	(172.24,187.13) --
	(172.29,187.13) --
	(172.34,187.13) --
	(172.39,187.14) --
	(172.44,187.14) --
	(172.50,187.15) --
	(172.55,187.15) --
	(172.60,187.16) --
	(172.65,187.16) --
	(172.70,187.17) --
	(172.75,187.17) --
	(172.80,187.18) --
	(172.85,187.18) --
	(172.90,187.19) --
	(172.96,187.19) --
	(173.01,187.20) --
	(173.06,187.20) --
	(173.11,187.20) --
	(173.16,187.21) --
	(173.21,187.21) --
	(173.26,187.22) --
	(173.31,187.22) --
	(173.37,187.23) --
	(173.42,187.23) --
	(173.47,187.24) --
	(173.52,187.24) --
	(173.57,187.25) --
	(173.62,187.25) --
	(173.67,187.26) --
	(173.72,187.26) --
	(173.78,187.26) --
	(173.83,187.27) --
	(173.88,187.27) --
	(173.93,187.28) --
	(173.98,187.28) --
	(174.03,187.29) --
	(174.08,187.29) --
	(174.13,187.30) --
	(174.19,187.30) --
	(174.24,187.31) --
	(174.29,187.31) --
	(174.34,187.32) --
	(174.39,187.32) --
	(174.44,187.33) --
	(174.49,187.33) --
	(174.54,187.33) --
	(174.59,187.34) --
	(174.65,187.34) --
	(174.70,187.35) --
	(174.75,187.35) --
	(174.80,187.36) --
	(174.85,187.36) --
	(174.90,187.37) --
	(174.95,187.37) --
	(175.00,187.38) --
	(175.06,187.38) --
	(175.11,187.39) --
	(175.16,187.39) --
	(175.21,187.39) --
	(175.26,187.40) --
	(175.31,187.40) --
	(175.36,187.41) --
	(175.41,187.41) --
	(175.47,187.42) --
	(175.52,187.42) --
	(175.57,187.43) --
	(175.62,187.43) --
	(175.67,187.44) --
	(175.72,187.44) --
	(175.77,187.45) --
	(175.82,187.45) --
	(175.87,187.46) --
	(175.93,187.46) --
	(175.98,187.46) --
	(176.03,187.47) --
	(176.08,187.47) --
	(176.13,187.48) --
	(176.18,187.48) --
	(176.23,187.49) --
	(176.28,187.49) --
	(176.34,187.50) --
	(176.39,187.50) --
	(176.44,187.51) --
	(176.49,187.51) --
	(176.54,187.52) --
	(176.59,187.52) --
	(176.64,187.52) --
	(176.69,187.53) --
	(176.75,187.53) --
	(176.80,187.54) --
	(176.85,187.54) --
	(176.90,187.55) --
	(176.95,187.55) --
	(177.00,187.56) --
	(177.05,187.56) --
	(177.10,187.57) --
	(177.15,187.57) --
	(177.21,187.58) --
	(177.26,187.58) --
	(177.31,187.59) --
	(177.36,187.59) --
	(177.41,187.59) --
	(177.46,187.60) --
	(177.51,187.60) --
	(177.56,187.61) --
	(177.62,187.61) --
	(177.67,187.62) --
	(177.72,187.62) --
	(177.77,187.63) --
	(177.82,187.63) --
	(177.87,187.64) --
	(177.92,187.64) --
	(177.97,187.65) --
	(178.03,187.65) --
	(178.08,187.65) --
	(178.13,187.66) --
	(178.18,187.66) --
	(178.23,187.67) --
	(178.28,187.67) --
	(178.33,187.68) --
	(178.38,187.68) --
	(178.43,187.69) --
	(178.49,187.69) --
	(178.54,187.70) --
	(178.59,187.70) --
	(178.64,187.71) --
	(178.69,187.71) --
	(178.74,187.72) --
	(178.79,187.72) --
	(178.84,187.72) --
	(178.90,187.73) --
	(178.95,187.73) --
	(179.00,187.74) --
	(179.05,187.74) --
	(179.10,187.75) --
	(179.15,187.75) --
	(179.20,187.76) --
	(179.25,187.76) --
	(179.31,187.77) --
	(179.36,187.77) --
	(179.41,187.78) --
	(179.46,187.78) --
	(179.51,187.79) --
	(179.56,187.79) --
	(179.61,187.79) --
	(179.66,187.80) --
	(179.71,187.80) --
	(179.77,187.81) --
	(179.82,187.81) --
	(179.87,187.82) --
	(179.92,187.82) --
	(179.97,187.83) --
	(180.02,187.83) --
	(180.07,187.84) --
	(180.12,187.84) --
	(180.18,187.85) --
	(180.23,187.85) --
	(180.28,187.85) --
	(180.33,187.86) --
	(180.38,187.86) --
	(180.43,187.87) --
	(180.48,187.87) --
	(180.53,187.88) --
	(180.59,187.88) --
	(180.64,187.89) --
	(180.69,187.89) --
	(180.74,187.90) --
	(180.79,187.90) --
	(180.84,187.91) --
	(180.89,187.91) --
	(180.94,187.92) --
	(180.99,187.92) --
	(181.05,187.92) --
	(181.10,187.93) --
	(181.15,187.93) --
	(181.20,187.94) --
	(181.25,187.94) --
	(181.30,187.95) --
	(181.35,187.95) --
	(181.40,187.96) --
	(181.46,187.96) --
	(181.51,187.97) --
	(181.56,187.97) --
	(181.61,187.98) --
	(181.66,187.98) --
	(181.71,187.98) --
	(181.76,187.99) --
	(181.81,187.99) --
	(181.87,188.00) --
	(181.92,188.00) --
	(181.97,188.01) --
	(182.02,188.01) --
	(182.07,188.02) --
	(182.12,188.02) --
	(182.17,188.03) --
	(182.22,188.03) --
	(182.27,188.04) --
	(182.33,188.04) --
	(182.38,188.05) --
	(182.43,188.05) --
	(182.48,188.05) --
	(182.53,188.06) --
	(182.58,188.06) --
	(182.63,188.07) --
	(182.68,188.07) --
	(182.74,188.08) --
	(182.79,188.08) --
	(182.84,188.09) --
	(182.89,188.09) --
	(182.94,188.10) --
	(182.99,188.10) --
	(183.04,188.11) --
	(183.09,188.11) --
	(183.15,188.11) --
	(183.20,188.12) --
	(183.25,188.12) --
	(183.30,188.13) --
	(183.35,188.13) --
	(183.40,188.14) --
	(183.45,188.14) --
	(183.50,188.15) --
	(183.55,188.15) --
	(183.61,188.16) --
	(183.66,188.16) --
	(183.71,188.17) --
	(183.76,188.17) --
	(183.81,188.18) --
	(183.86,188.18) --
	(183.91,188.18) --
	(183.96,188.19) --
	(184.02,188.19) --
	(184.07,188.20) --
	(184.12,188.20) --
	(184.17,188.21) --
	(184.22,188.21) --
	(184.27,188.22) --
	(184.32,188.22) --
	(184.37,188.23) --
	(184.43,188.23) --
	(184.48,188.24) --
	(184.53,188.24) --
	(184.58,188.25) --
	(184.63,188.25) --
	(184.68,188.25) --
	(184.73,188.26) --
	(184.78,188.26) --
	(184.83,188.27) --
	(184.89,188.27) --
	(184.94,188.28) --
	(184.99,188.28) --
	(185.04,188.29) --
	(185.09,188.29) --
	(185.14,188.30) --
	(185.19,188.30) --
	(185.24,188.31) --
	(185.30,188.31) --
	(185.35,188.31) --
	(185.40,188.32) --
	(185.45,188.32) --
	(185.50,188.33) --
	(185.55,188.33) --
	(185.60,188.34) --
	(185.65,188.34) --
	(185.71,188.35) --
	(185.76,188.35) --
	(185.81,188.36) --
	(185.86,188.36) --
	(185.91,188.37) --
	(185.96,188.37) --
	(186.01,188.38) --
	(186.06,188.38) --
	(186.11,188.38) --
	(186.17,188.39) --
	(186.22,188.39) --
	(186.27,188.40) --
	(186.32,188.40) --
	(186.37,188.41) --
	(186.42,188.41) --
	(186.47,188.42) --
	(186.52,188.42) --
	(186.58,188.43) --
	(186.63,188.43) --
	(186.68,188.44) --
	(186.73,188.44) --
	(186.78,188.44) --
	(186.83,188.45) --
	(186.88,188.45) --
	(186.93,188.46) --
	(186.99,188.46) --
	(187.04,188.47) --
	(187.09,188.47) --
	(187.14,188.48) --
	(187.19,188.48) --
	(187.24,188.49) --
	(187.29,188.49) --
	(187.34,188.50) --
	(187.39,188.50) --
	(187.45,188.51) --
	(187.50,188.51) --
	(187.55,188.51) --
	(187.60,188.52) --
	(187.65,188.52) --
	(187.70,188.53) --
	(187.75,188.53) --
	(187.80,188.54) --
	(187.86,188.54) --
	(187.91,188.55) --
	(187.96,188.55) --
	(188.01,188.56) --
	(188.06,188.56) --
	(188.11,188.57) --
	(188.16,188.57) --
	(188.21,188.57) --
	(188.27,188.58) --
	(188.32,188.58) --
	(188.37,188.59) --
	(188.42,188.59) --
	(188.47,188.60) --
	(188.52,188.60) --
	(188.57,188.61) --
	(188.62,188.61) --
	(188.67,188.62) --
	(188.73,188.62) --
	(188.78,188.63) --
	(188.83,188.63) --
	(188.88,188.64) --
	(188.93,188.64) --
	(188.98,188.64) --
	(189.03,188.65) --
	(189.08,188.65) --
	(189.14,188.66) --
	(189.19,188.66) --
	(189.24,188.67) --
	(189.29,188.67) --
	(189.34,188.68) --
	(189.39,188.68) --
	(189.44,188.69) --
	(189.49,188.69) --
	(189.55,188.70) --
	(189.60,188.70) --
	(189.65,188.70) --
	(189.70,188.71) --
	(189.75,188.71) --
	(189.80,188.72) --
	(189.85,188.72) --
	(189.90,188.73) --
	(189.95,188.73) --
	(190.01,188.74) --
	(190.06,188.74) --
	(190.11,188.75) --
	(190.16,188.75) --
	(190.21,188.76) --
	(190.26,188.76) --
	(190.31,188.77) --
	(190.36,188.77) --
	(190.42,188.77) --
	(190.47,188.78) --
	(190.52,188.78) --
	(190.57,188.79) --
	(190.62,188.79) --
	(190.67,188.80) --
	(190.72,188.80) --
	(190.77,188.81) --
	(190.83,188.81) --
	(190.88,188.82) --
	(190.93,188.82) --
	(190.98,188.83) --
	(191.03,188.83) --
	(191.08,188.84) --
	(191.13,188.84) --
	(191.18,188.84) --
	(191.23,188.85) --
	(191.29,188.85) --
	(191.34,188.86) --
	(191.39,188.86) --
	(191.44,188.87) --
	(191.49,188.87) --
	(191.54,188.88) --
	(191.59,188.88) --
	(191.64,188.89) --
	(191.70,188.89) --
	(191.75,188.90) --
	(191.80,188.90) --
	(191.85,188.90) --
	(191.90,188.91) --
	(191.95,188.91) --
	(192.00,188.92) --
	(192.05,188.92) --
	(192.11,188.93) --
	(192.16,188.93) --
	(192.21,188.94) --
	(192.26,188.94) --
	(192.31,188.95) --
	(192.36,188.95) --
	(192.41,188.96) --
	(192.46,188.96) --
	(192.51,188.97) --
	(192.57,188.97) --
	(192.62,188.97) --
	(192.67,188.98) --
	(192.72,188.98) --
	(192.77,188.99) --
	(192.82,188.99) --
	(192.87,189.00) --
	(192.92,189.00) --
	(192.98,189.01) --
	(193.03,189.01) --
	(193.08,189.02) --
	(193.13,189.02) --
	(193.18,189.03) --
	(193.23,189.03) --
	(193.28,189.03) --
	(193.33,189.04) --
	(193.39,189.04) --
	(193.44,189.05) --
	(193.49,189.05) --
	(193.54,189.06) --
	(193.59,189.06) --
	(193.64,189.07) --
	(193.69,189.07) --
	(193.74,189.08) --
	(193.79,189.08) --
	(193.85,189.09) --
	(193.90,189.09) --
	(193.95,189.10) --
	(194.00,189.10) --
	(194.05,189.10) --
	(194.10,189.11) --
	(194.15,189.11) --
	(194.20,189.12) --
	(194.26,189.12) --
	(194.31,189.13) --
	(194.36,189.13) --
	(194.41,189.14) --
	(194.46,189.14) --
	(194.51,189.15) --
	(194.56,189.15) --
	(194.61,189.16) --
	(194.67,189.16) --
	(194.72,189.16) --
	(194.77,189.17) --
	(194.82,189.17) --
	(194.87,189.18) --
	(194.92,189.18) --
	(194.97,189.19) --
	(195.02,189.19) --
	(195.07,189.20) --
	(195.13,189.20) --
	(195.18,189.21) --
	(195.23,189.21) --
	(195.28,189.22) --
	(195.33,189.22) --
	(195.38,189.23) --
	(195.43,189.23) --
	(195.48,189.23) --
	(195.54,189.24) --
	(195.59,189.24) --
	(195.64,189.25) --
	(195.69,189.25) --
	(195.74,189.26) --
	(195.79,189.26) --
	(195.84,189.27) --
	(195.89,189.27) --
	(195.95,189.28) --
	(196.00,189.28) --
	(196.05,189.29) --
	(196.10,189.29) --
	(196.15,189.30) --
	(196.20,189.30) --
	(196.25,189.30) --
	(196.30,189.31) --
	(196.35,189.31) --
	(196.41,189.32) --
	(196.46,189.32) --
	(196.51,189.33) --
	(196.56,189.33) --
	(196.61,189.34) --
	(196.66,189.34) --
	(196.71,189.35) --
	(196.76,189.35) --
	(196.82,189.36) --
	(196.87,189.36) --
	(196.92,189.36) --
	(196.97,189.37) --
	(197.02,189.37) --
	(197.07,189.38) --
	(197.12,189.38) --
	(197.17,189.39) --
	(197.23,189.39) --
	(197.28,189.40) --
	(197.33,189.40) --
	(197.38,189.41) --
	(197.43,189.41) --
	(197.48,189.42) --
	(197.53,189.42) --
	(197.58,189.43) --
	(197.63,189.43) --
	(197.69,189.43) --
	(197.74,189.44) --
	(197.79,189.44) --
	(197.84,189.45) --
	(197.89,189.45) --
	(197.94,189.46) --
	(197.99,189.46) --
	(198.04,189.47) --
	(198.10,189.47) --
	(198.15,189.48) --
	(198.20,189.48) --
	(198.25,189.49) --
	(198.30,189.49) --
	(198.35,189.49) --
	(198.40,189.50) --
	(198.45,189.50) --
	(198.51,189.51) --
	(198.56,189.51) --
	(198.61,189.52) --
	(198.66,189.52) --
	(198.71,189.53) --
	(198.76,189.53) --
	(198.81,189.54) --
	(198.86,189.54) --
	(198.92,189.55) --
	(198.97,189.55) --
	(199.02,189.56) --
	(199.07,189.56) --
	(199.12,189.56) --
	(199.17,189.57) --
	(199.22,189.57) --
	(199.27,189.58) --
	(199.32,189.58) --
	(199.38,189.59) --
	(199.43,189.59) --
	(199.48,189.60) --
	(199.53,189.60) --
	(199.58,189.61) --
	(199.63,189.61) --
	(199.68,189.62) --
	(199.73,189.62) --
	(199.79,189.62) --
	(199.84,189.63) --
	(199.89,189.63) --
	(199.94,189.64) --
	(199.99,189.64) --
	(200.04,189.65) --
	(200.09,189.65) --
	(200.14,189.66) --
	(200.20,189.66) --
	(200.25,189.67) --
	(200.30,189.67) --
	(200.35,189.68) --
	(200.40,189.68) --
	(200.45,189.69) --
	(200.50,189.69) --
	(200.55,189.69) --
	(200.60,189.70) --
	(200.66,189.70) --
	(200.71,189.71) --
	(200.76,189.71) --
	(200.81,189.72) --
	(200.86,189.72) --
	(200.91,189.73) --
	(200.96,189.73) --
	(201.01,189.74) --
	(201.07,189.74) --
	(201.12,189.75) --
	(201.17,189.75) --
	(201.22,189.75) --
	(201.27,189.76) --
	(201.32,189.76) --
	(201.37,189.77) --
	(201.42,189.77) --
	(201.48,189.78) --
	(201.53,189.78) --
	(201.58,189.79) --
	(201.63,189.79) --
	(201.68,189.80) --
	(201.73,189.80) --
	(201.78,189.81) --
	(201.83,189.81) --
	(201.88,189.82) --
	(201.94,189.82) --
	(201.99,189.82) --
	(202.04,189.83) --
	(202.09,189.83) --
	(202.14,189.84) --
	(202.19,189.84) --
	(202.24,189.85) --
	(202.29,189.85) --
	(202.35,189.86) --
	(202.40,189.86) --
	(202.45,189.87) --
	(202.50,189.87) --
	(202.55,189.88) --
	(202.60,189.88) --
	(202.65,189.89) --
	(202.70,189.89) --
	(202.76,189.89) --
	(202.81,189.90) --
	(202.86,189.90) --
	(202.91,189.91) --
	(202.96,189.91) --
	(203.01,189.92) --
	(203.06,189.92) --
	(203.11,189.93) --
	(203.16,189.93) --
	(203.22,189.94) --
	(203.27,189.94) --
	(203.32,189.95) --
	(203.37,189.95) --
	(203.42,189.95) --
	(203.47,189.96) --
	(203.52,189.96) --
	(203.57,189.97) --
	(203.63,189.97) --
	(203.68,189.98) --
	(203.73,189.98) --
	(203.78,189.99) --
	(203.83,189.99) --
	(203.88,190.00) --
	(203.93,190.00) --
	(203.98,190.01) --
	(204.04,190.01) --
	(204.09,190.02) --
	(204.14,190.02) --
	(204.19,190.02) --
	(204.24,190.03) --
	(204.29,190.03) --
	(204.34,190.04) --
	(204.39,190.04) --
	(204.44,190.05) --
	(204.50,190.05) --
	(204.55,190.06) --
	(204.60,190.06) --
	(204.65,190.07) --
	(204.70,190.07) --
	(204.75,190.08) --
	(204.80,190.08) --
	(204.85,190.08) --
	(204.91,190.09) --
	(204.96,190.09) --
	(205.01,190.10) --
	(205.06,190.10) --
	(205.11,190.11) --
	(205.16,190.11) --
	(205.21,190.12) --
	(205.26,190.12) --
	(205.32,190.13) --
	(205.37,190.13) --
	(205.42,190.14) --
	(205.47,190.14) --
	(205.52,190.15) --
	(205.57,190.15) --
	(205.62,190.15) --
	(205.67,190.16) --
	(205.72,190.16) --
	(205.78,190.17) --
	(205.83,190.17) --
	(205.88,190.18) --
	(205.93,190.18) --
	(205.98,190.19) --
	(206.03,190.19) --
	(206.08,190.20) --
	(206.13,190.20) --
	(206.19,190.21) --
	(206.24,190.21) --
	(206.29,190.21) --
	(206.34,190.22) --
	(206.39,190.22) --
	(206.44,190.23) --
	(206.49,190.23) --
	(206.54,190.24) --
	(206.60,190.24) --
	(206.65,190.25) --
	(206.70,190.25) --
	(206.75,190.26) --
	(206.80,190.26) --
	(206.85,190.27) --
	(206.90,190.27) --
	(206.95,190.28) --
	(207.00,190.28) --
	(207.06,190.28) --
	(207.11,190.29) --
	(207.16,190.29) --
	(207.21,190.30) --
	(207.26,190.30) --
	(207.31,190.31) --
	(207.36,190.31) --
	(207.41,190.32) --
	(207.47,190.32) --
	(207.52,190.33) --
	(207.57,190.33) --
	(207.62,190.34) --
	(207.67,190.34) --
	(207.72,190.34) --
	(207.77,190.35) --
	(207.82,190.35) --
	(207.88,190.36) --
	(207.93,190.36) --
	(207.98,190.37) --
	(208.03,190.37) --
	(208.08,190.38) --
	(208.13,190.38) --
	(208.18,190.39) --
	(208.23,190.39) --
	(208.28,190.40) --
	(208.34,190.40) --
	(208.39,190.41) --
	(208.44,190.41) --
	(208.49,190.41) --
	(208.54,190.42) --
	(208.59,190.42) --
	(208.64,190.43) --
	(208.69,190.43) --
	(208.75,190.44) --
	(208.80,190.44) --
	(208.85,190.45) --
	(208.90,190.45) --
	(208.95,190.46) --
	(209.00,190.46) --
	(209.05,190.47) --
	(209.10,190.47) --
	(209.16,190.48) --
	(209.21,190.48) --
	(209.26,190.48) --
	(209.31,190.49) --
	(209.36,190.49) --
	(209.41,190.50) --
	(209.46,190.50) --
	(209.51,190.51) --
	(209.56,190.51) --
	(209.62,190.52) --
	(209.67,190.52) --
	(209.72,190.53) --
	(209.77,190.53) --
	(209.82,190.54) --
	(209.87,190.54) --
	(209.92,190.54) --
	(209.97,190.55) --
	(210.03,190.55) --
	(210.08,190.56) --
	(210.13,190.56) --
	(210.18,190.57) --
	(210.23,190.57) --
	(210.28,190.58) --
	(210.33,190.58) --
	(210.38,190.59) --
	(210.44,190.59) --
	(210.49,190.60) --
	(210.54,190.60) --
	(210.59,190.61) --
	(210.64,190.61) --
	(210.69,190.61) --
	(210.74,190.62) --
	(210.79,190.62) --
	(210.84,190.63) --
	(210.90,190.63) --
	(210.95,190.64) --
	(211.00,190.64) --
	(211.05,190.65) --
	(211.10,190.65) --
	(211.15,190.66) --
	(211.20,190.66) --
	(211.25,190.67) --
	(211.31,190.67) --
	(211.36,190.67) --
	(211.41,190.68) --
	(211.46,190.68) --
	(211.51,190.69) --
	(211.56,190.69) --
	(211.61,190.70) --
	(211.66,190.70) --
	(211.72,190.71) --
	(211.77,190.71) --
	(211.82,190.72) --
	(211.87,190.72) --
	(211.92,190.73) --
	(211.97,190.73) --
	(212.02,190.74) --
	(212.07,190.74) --
	(212.12,190.74) --
	(212.18,190.75) --
	(212.23,190.75) --
	(212.28,190.76) --
	(212.33,190.76) --
	(212.38,190.77) --
	(212.43,190.77) --
	(212.48,190.78) --
	(212.53,190.78) --
	(212.59,190.79) --
	(212.64,190.79) --
	(212.69,190.80) --
	(212.74,190.80) --
	(212.79,190.80) --
	(212.84,190.81) --
	(212.89,190.81) --
	(212.94,190.82) --
	(213.00,190.82) --
	(213.05,190.83) --
	(213.10,190.83) --
	(213.15,190.84) --
	(213.20,190.84) --
	(213.25,190.85) --
	(213.30,190.85) --
	(213.35,190.86) --
	(213.40,190.86) --
	(213.46,190.87) --
	(213.51,190.87) --
	(213.56,190.87) --
	(213.61,190.88) --
	(213.66,190.88) --
	(213.71,190.89) --
	(213.76,190.89) --
	(213.81,190.90) --
	(213.87,190.90) --
	(213.92,190.91) --
	(213.97,190.91) --
	(214.02,190.92) --
	(214.07,190.92) --
	(214.12,190.93) --
	(214.17,190.93) --
	(214.22,190.94) --
	(214.28,190.94) --
	(214.33,190.94) --
	(214.38,190.95) --
	(214.43,190.95) --
	(214.48,190.96) --
	(214.53,190.96) --
	(214.58,190.97) --
	(214.63,190.97) --
	(214.68,190.98) --
	(214.74,190.98) --
	(214.79,190.99) --
	(214.84,190.99) --
	(214.89,191.00) --
	(214.94,191.00) --
	(214.99,191.00) --
	(215.04,191.01) --
	(215.09,191.01) --
	(215.15,191.02) --
	(215.20,191.02) --
	(215.25,191.03) --
	(215.30,191.03) --
	(215.35,191.04) --
	(215.40,191.04) --
	(215.45,191.05) --
	(215.50,191.05) --
	(215.56,191.06) --
	(215.61,191.06) --
	(215.66,191.07) --
	(215.71,191.07) --
	(215.76,191.07) --
	(215.81,191.08) --
	(215.86,191.08) --
	(215.91,191.09) --
	(215.96,191.09) --
	(216.02,191.10) --
	(216.07,191.10) --
	(216.12,191.11) --
	(216.17,191.11) --
	(216.22,191.12) --
	(216.27,191.12) --
	(216.32,191.13) --
	(216.37,191.13) --
	(216.43,191.13) --
	(216.48,191.14) --
	(216.53,191.14) --
	(216.58,191.15) --
	(216.63,191.15) --
	(216.68,191.16) --
	(216.73,191.16) --
	(216.78,191.17) --
	(216.84,191.17) --
	(216.89,191.18) --
	(216.94,191.18) --
	(216.99,191.19) --
	(217.04,191.19) --
	(217.09,191.20) --
	(217.14,191.20) --
	(217.19,191.20) --
	(217.24,191.21) --
	(217.30,191.21) --
	(217.35,191.22) --
	(217.40,191.22) --
	(217.45,191.23) --
	(217.50,191.23) --
	(217.55,191.24) --
	(217.60,191.24) --
	(217.65,191.25) --
	(217.71,191.25) --
	(217.76,191.26) --
	(217.81,191.26) --
	(217.86,191.26) --
	(217.91,191.27) --
	(217.96,191.27) --
	(218.01,191.28) --
	(218.06,191.28) --
	(218.12,191.29) --
	(218.17,191.29) --
	(218.22,191.30) --
	(218.27,191.30) --
	(218.32,191.31) --
	(218.37,191.31) --
	(218.42,191.32) --
	(218.47,191.32) --
	(218.52,191.33) --
	(218.58,191.33) --
	(218.63,191.33) --
	(218.68,191.34) --
	(218.73,191.34) --
	(218.78,191.35) --
	(218.83,191.35) --
	(218.88,191.36) --
	(218.93,191.36) --
	(218.99,191.37) --
	(219.04,191.37) --
	(219.09,191.38) --
	(219.14,191.38) --
	(219.19,191.39) --
	(219.24,191.39) --
	(219.29,191.39) --
	(219.34,191.40) --
	(219.40,191.40) --
	(219.45,191.41) --
	(219.50,191.41) --
	(219.55,191.42) --
	(219.60,191.42) --
	(219.65,191.43) --
	(219.70,191.43) --
	(219.75,191.44) --
	(219.80,191.44) --
	(219.86,191.45) --
	(219.91,191.45) --
	(219.96,191.46) --
	(220.01,191.46) --
	(220.06,191.46) --
	(220.11,191.47) --
	(220.16,191.47) --
	(220.21,191.48) --
	(220.27,191.48) --
	(220.32,191.49) --
	(220.37,191.49) --
	(220.42,191.50) --
	(220.47,191.50) --
	(220.52,191.51) --
	(220.57,191.51) --
	(220.62,191.52) --
	(220.68,191.52) --
	(220.73,191.53) --
	(220.78,191.53) --
	(220.83,191.53) --
	(220.88,191.54) --
	(220.93,191.54) --
	(220.98,191.55) --
	(221.03,191.55) --
	(221.08,191.56) --
	(221.14,191.56) --
	(221.19,191.57) --
	(221.24,191.57) --
	(221.29,191.58) --
	(221.34,191.58) --
	(221.39,191.59) --
	(221.44,191.59) --
	(221.49,191.59) --
	(221.55,191.60) --
	(221.60,191.60) --
	(221.65,191.61) --
	(221.70,191.61) --
	(221.75,191.62) --
	(221.80,191.62) --
	(221.85,191.63) --
	(221.90,191.63) --
	(221.96,191.64) --
	(222.01,191.64) --
	(222.06,191.65) --
	(222.11,191.65) --
	(222.16,191.66) --
	(222.21,191.66) --
	(222.26,191.66) --
	(222.31,191.67) --
	(222.36,191.67) --
	(222.42,191.68) --
	(222.47,191.68) --
	(222.52,191.69) --
	(222.57,191.69) --
	(222.62,191.70) --
	(222.67,191.70) --
	(222.72,191.71) --
	(222.77,191.71) --
	(222.83,191.72) --
	(222.88,191.72) --
	(222.93,191.72) --
	(222.98,191.73) --
	(223.03,191.73) --
	(223.08,191.74) --
	(223.13,191.74) --
	(223.18,191.75) --
	(223.24,191.75) --
	(223.29,191.76) --
	(223.34,191.76) --
	(223.39,191.77) --
	(223.44,191.77) --
	(223.49,191.78) --
	(223.54,191.78) --
	(223.59,191.79) --
	(223.64,191.79) --
	(223.70,191.79) --
	(223.75,191.80) --
	(223.80,191.80) --
	(223.85,191.81) --
	(223.90,191.81) --
	(223.95,191.82) --
	(224.00,191.82) --
	(224.05,191.83) --
	(224.11,191.83) --
	(224.16,191.84) --
	(224.21,191.84) --
	(224.26,191.85) --
	(224.31,191.85) --
	(224.36,191.85) --
	(224.41,191.86) --
	(224.46,191.86) --
	(224.52,191.87) --
	(224.57,191.87) --
	(224.62,191.88) --
	(224.67,191.88) --
	(224.72,191.89) --
	(224.77,191.89) --
	(224.82,191.90) --
	(224.87,191.90) --
	(224.93,191.91) --
	(224.98,191.91) --
	(225.03,191.92) --
	(225.08,191.92) --
	(225.13,191.92) --
	(225.18,191.93) --
	(225.23,191.93) --
	(225.28,191.94) --
	(225.33,191.94) --
	(225.39,191.95) --
	(225.44,191.95) --
	(225.49,191.96) --
	(225.54,191.96) --
	(225.59,191.97) --
	(225.64,191.97) --
	(225.69,191.98) --
	(225.74,191.98) --
	(225.80,191.99) --
	(225.85,191.99) --
	(225.90,191.99) --
	(225.95,192.00) --
	(226.00,192.00) --
	(226.05,192.01) --
	(226.10,192.01) --
	(226.15,192.02) --
	(226.21,192.02) --
	(226.26,192.03) --
	(226.31,192.03) --
	(226.36,192.04) --
	(226.41,192.04) --
	(226.46,192.05) --
	(226.51,192.05) --
	(226.56,192.05) --
	(226.61,192.06) --
	(226.67,192.06) --
	(226.72,192.07) --
	(226.77,192.07) --
	(226.82,192.08) --
	(226.87,192.08) --
	(226.92,192.09) --
	(226.97,192.09) --
	(227.02,192.10) --
	(227.08,192.10) --
	(227.13,192.11) --
	(227.18,192.11) --
	(227.23,192.12) --
	(227.28,192.12) --
	(227.33,192.12) --
	(227.38,192.13) --
	(227.43,192.13) --
	(227.49,192.14) --
	(227.54,192.14) --
	(227.59,192.15) --
	(227.64,192.15) --
	(227.69,192.16) --
	(227.74,192.16) --
	(227.79,192.17) --
	(227.84,192.17) --
	(227.89,192.18) --
	(227.95,192.18) --
	(228.00,192.18) --
	(228.05,192.19) --
	(228.10,192.19) --
	(228.15,192.20) --
	(228.20,192.20) --
	(228.25,192.21) --
	(228.30,192.21) --
	(228.36,192.22) --
	(228.41,192.22) --
	(228.46,192.23) --
	(228.51,192.23) --
	(228.56,192.24) --
	(228.61,192.24) --
	(228.66,192.25) --
	(228.71,192.25) --
	(228.77,192.25) --
	(228.82,192.26) --
	(228.87,192.26) --
	(228.92,192.27) --
	(228.97,192.27) --
	(229.02,192.28) --
	(229.07,192.28) --
	(229.12,192.29) --
	(229.17,192.29) --
	(229.23,192.30) --
	(229.28,192.30) --
	(229.33,192.31) --
	(229.38,192.31) --
	(229.43,192.31) --
	(229.48,192.32) --
	(229.53,192.32) --
	(229.58,192.33) --
	(229.64,192.33) --
	(229.69,192.34) --
	(229.74,192.34) --
	(229.79,192.35) --
	(229.84,192.35) --
	(229.89,192.36) --
	(229.94,192.36) --
	(229.99,192.37) --
	(230.05,192.37) --
	(230.10,192.38) --
	(230.15,192.38) --
	(230.20,192.38) --
	(230.25,192.39) --
	(230.30,192.39) --
	(230.35,192.40) --
	(230.40,192.40) --
	(230.45,192.41) --
	(230.51,192.41) --
	(230.56,192.42) --
	(230.61,192.42) --
	(230.66,192.43) --
	(230.71,192.43) --
	(230.76,192.44) --
	(230.81,192.44) --
	(230.86,192.44) --
	(230.92,192.45) --
	(230.97,192.45) --
	(231.02,192.46) --
	(231.07,192.46) --
	(231.12,192.47) --
	(231.17,192.47) --
	(231.22,192.48) --
	(231.27,192.48) --
	(231.33,192.49) --
	(231.38,192.49) --
	(231.43,192.50) --
	(231.48,192.50) --
	(231.53,192.51) --
	(231.58,192.51) --
	(231.63,192.51) --
	(231.68,192.52) --
	(231.73,192.52) --
	(231.79,192.53) --
	(231.84,192.53) --
	(231.89,192.54) --
	(231.94,192.54) --
	(231.99,192.55) --
	(232.04,192.55) --
	(232.09,192.56) --
	(232.14,192.56) --
	(232.20,192.57) --
	(232.25,192.57) --
	(232.30,192.58) --
	(232.35,192.58) --
	(232.40,192.58) --
	(232.45,192.59) --
	(232.50,192.59) --
	(232.55,192.60) --
	(232.61,192.60) --
	(232.66,192.61) --
	(232.71,192.61) --
	(232.76,192.62) --
	(232.81,192.62) --
	(232.86,192.63) --
	(232.91,192.63) --
	(232.96,192.64) --
	(233.01,192.64) --
	(233.07,192.64) --
	(233.12,192.65) --
	(233.17,192.65) --
	(233.22,192.66) --
	(233.27,192.66) --
	(233.32,192.67) --
	(233.37,192.67) --
	(233.42,192.68) --
	(233.48,192.68) --
	(233.53,192.69) --
	(233.58,192.69) --
	(233.63,192.70) --
	(233.68,192.70) --
	(233.73,192.71) --
	(233.78,192.71) --
	(233.83,192.71) --
	(233.89,192.72) --
	(233.94,192.72) --
	(233.99,192.73) --
	(234.04,192.73) --
	(234.09,192.74) --
	(234.14,192.74) --
	(234.19,192.75) --
	(234.24,192.75) --
	(234.29,192.76) --
	(234.35,192.76) --
	(234.40,192.77) --
	(234.45,192.77) --
	(234.50,192.77) --
	(234.55,192.78) --
	(234.60,192.78) --
	(234.65,192.79) --
	(234.70,192.79) --
	(234.76,192.80) --
	(234.81,192.80) --
	(234.86,192.81) --
	(234.91,192.81) --
	(234.96,192.82) --
	(235.01,192.82) --
	(235.06,192.83) --
	(235.11,192.83) --
	(235.17,192.84) --
	(235.22,192.84) --
	(235.27,192.84) --
	(235.32,192.85) --
	(235.37,192.85) --
	(235.42,192.86) --
	(235.47,192.86) --
	(235.52,192.87) --
	(235.57,192.87) --
	(235.63,192.88) --
	(235.68,192.88) --
	(235.73,192.89) --
	(235.78,192.89) --
	(235.83,192.90) --
	(235.88,192.90) --
	(235.93,192.90) --
	(235.98,192.91) --
	(236.04,192.91) --
	(236.09,192.92) --
	(236.14,192.92) --
	(236.19,192.93) --
	(236.24,192.93) --
	(236.29,192.94) --
	(236.34,192.94) --
	(236.39,192.95) --
	(236.45,192.95) --
	(236.50,192.96) --
	(236.55,192.96) --
	(236.60,192.97) --
	(236.65,192.97) --
	(236.70,192.97) --
	(236.75,192.98) --
	(236.80,192.98) --
	(236.85,192.99) --
	(236.91,192.99) --
	(236.96,193.00) --
	(237.01,193.00) --
	(237.06,193.01) --
	(237.11,193.01) --
	(237.16,193.02) --
	(237.21,193.02) --
	(237.26,193.03) --
	(237.32,193.03) --
	(237.37,193.04) --
	(237.42,193.04) --
	(237.47,193.04) --
	(237.52,193.05) --
	(237.57,193.05) --
	(237.62,193.06) --
	(237.67,193.06) --
	(237.73,193.07) --
	(237.78,193.07) --
	(237.83,193.08) --
	(237.88,193.08) --
	(237.93,193.09) --
	(237.98,193.09) --
	(238.03,193.10) --
	(238.08,193.10) --
	(238.13,193.10) --
	(238.19,193.11) --
	(238.24,193.11) --
	(238.29,193.12) --
	(238.34,193.12) --
	(238.39,193.13) --
	(238.44,193.13) --
	(238.49,193.14) --
	(238.54,193.14) --
	(238.60,193.15) --
	(238.65,193.15) --
	(238.70,193.16) --
	(238.75,193.16) --
	(238.80,193.17) --
	(238.85,193.17) --
	(238.90,193.17) --
	(238.95,193.18) --
	(239.01,193.18) --
	(239.06,193.19) --
	(239.11,193.19) --
	(239.16,193.20) --
	(239.21,193.20) --
	(239.26,193.21) --
	(239.31,193.21) --
	(239.36,193.22) --
	(239.41,193.22) --
	(239.47,193.23) --
	(239.52,193.23) --
	(239.57,193.23) --
	(239.62,193.24) --
	(239.67,193.24) --
	(239.72,193.25) --
	(239.77,193.25) --
	(239.82,193.26) --
	(239.88,193.26) --
	(239.93,193.27) --
	(239.98,193.27) --
	(240.03,193.28) --
	(240.08,193.28) --
	(240.13,193.29) --
	(240.18,193.29) --
	(240.23,193.30) --
	(240.29,193.30) --
	(240.34,193.30) --
	(240.39,193.31) --
	(240.44,193.31) --
	(240.49,193.32) --
	(240.54,193.32) --
	(240.59,193.33) --
	(240.64,193.33) --
	(240.69,193.34) --
	(240.75,193.34) --
	(240.80,193.35) --
	(240.85,193.35) --
	(240.90,193.36) --
	(240.95,193.36) --
	(241.00,193.36) --
	(241.05,193.37) --
	(241.10,193.37) --
	(241.16,193.38) --
	(241.21,193.38) --
	(241.26,193.39) --
	(241.31,193.39) --
	(241.36,193.40) --
	(241.41,193.40) --
	(241.46,193.41) --
	(241.51,193.41) --
	(241.57,193.42) --
	(241.62,193.42) --
	(241.67,193.43) --
	(241.72,193.43) --
	(241.77,193.43) --
	(241.82,193.44) --
	(241.87,193.44) --
	(241.92,193.45) --
	(241.97,193.45) --
	(242.03,193.46) --
	(242.08,193.46) --
	(242.13,193.47) --
	(242.18,193.47) --
	(242.23,193.48) --
	(242.28,193.48) --
	(242.33,193.49) --
	(242.38,193.49) --
	(242.44,193.49) --
	(242.49,193.50) --
	(242.54,193.50) --
	(242.59,193.51) --
	(242.64,193.51) --
	(242.69,193.52) --
	(242.74,193.52) --
	(242.79,193.53) --
	(242.85,193.53) --
	(242.90,193.54) --
	(242.95,193.54) --
	(243.00,193.55) --
	(243.05,193.55) --
	(243.10,193.56) --
	(243.15,193.56) --
	(243.20,193.56) --
	(243.25,193.57) --
	(243.31,193.57) --
	(243.36,193.58) --
	(243.41,193.58) --
	(243.46,193.59) --
	(243.51,193.59) --
	(243.56,193.60) --
	(243.61,193.60) --
	(243.66,193.61) --
	(243.72,193.61) --
	(243.77,193.62) --
	(243.82,193.62) --
	(243.87,193.63) --
	(243.92,193.63) --
	(243.97,193.63) --
	(244.02,193.64) --
	(244.07,193.64) --
	(244.13,193.65) --
	(244.18,193.65) --
	(244.23,193.66) --
	(244.28,193.66) --
	(244.33,193.67) --
	(244.38,193.67) --
	(244.43,193.68) --
	(244.48,193.68) --
	(244.53,193.69) --
	(244.59,193.69) --
	(244.64,193.69) --
	(244.69,193.70) --
	(244.74,193.70) --
	(244.79,193.71) --
	(244.84,193.71) --
	(244.89,193.72) --
	(244.94,193.72) --
	(245.00,193.73) --
	(245.05,193.73) --
	(245.10,193.74) --
	(245.15,193.74) --
	(245.20,193.75) --
	(245.25,193.75) --
	(245.30,193.76) --
	(245.35,193.76) --
	(245.41,193.76) --
	(245.46,193.77) --
	(245.51,193.77) --
	(245.56,193.78) --
	(245.61,193.78) --
	(245.66,193.79) --
	(245.71,193.79) --
	(245.76,193.80) --
	(245.81,193.80) --
	(245.87,193.81) --
	(245.92,193.81) --
	(245.97,193.82) --
	(246.02,193.82) --
	(246.07,193.82) --
	(246.12,193.83) --
	(246.17,193.83) --
	(246.22,193.84) --
	(246.28,193.84) --
	(246.33,193.85) --
	(246.38,193.85) --
	(246.43,193.86) --
	(246.48,193.86) --
	(246.53,193.87) --
	(246.58,193.87) --
	(246.63,193.88) --
	(246.69,193.88) --
	(246.74,193.89) --
	(246.79,193.89) --
	(246.84,193.89) --
	(246.89,193.90) --
	(246.94,193.90) --
	(246.99,193.91) --
	(247.04,193.91) --
	(247.09,193.92) --
	(247.15,193.92) --
	(247.20,193.93) --
	(247.25,193.93) --
	(247.30,193.94) --
	(247.35,193.94) --
	(247.40,193.95) --
	(247.45,193.95) --
	(247.50,193.95) --
	(247.56,193.96) --
	(247.61,193.96) --
	(247.66,193.97) --
	(247.71,193.97) --
	(247.76,193.98) --
	(247.81,193.98) --
	(247.86,193.99) --
	(247.91,193.99) --
	(247.97,194.00) --
	(248.02,194.00) --
	(248.07,194.01) --
	(248.12,194.01) --
	(248.17,194.02) --
	(248.22,194.02) --
	(248.27,194.02) --
	(248.32,194.03) --
	(248.37,194.03) --
	(248.43,194.04) --
	(248.48,194.04) --
	(248.53,194.05) --
	(248.58,194.05) --
	(248.63,194.06) --
	(248.68,194.06) --
	(248.73,194.07) --
	(248.78,194.07) --
	(248.84,194.08) --
	(248.89,194.08) --
	(248.94,194.09) --
	(248.99,194.09) --
	(249.04,194.09) --
	(249.09,194.10) --
	(249.14,194.10) --
	(249.19,194.11) --
	(249.25,194.11) --
	(249.30,194.12) --
	(249.35,194.12) --
	(249.40,194.13) --
	(249.45,194.13) --
	(249.50,194.14) --
	(249.55,194.14) --
	(249.60,194.15) --
	(249.65,194.15) --
	(249.71,194.15) --
	(249.76,194.16) --
	(249.81,194.16) --
	(249.86,194.17) --
	(249.91,194.17) --
	(249.96,194.18) --
	(250.01,194.18) --
	(250.06,194.19) --
	(250.12,194.19) --
	(250.17,194.20) --
	(250.22,194.20) --
	(250.27,194.21) --
	(250.32,194.21) --
	(250.37,194.22) --
	(250.42,194.22) --
	(250.47,194.22) --
	(250.53,194.23) --
	(250.58,194.23) --
	(250.63,194.24) --
	(250.68,194.24) --
	(250.73,194.25) --
	(250.78,194.25) --
	(250.83,194.26) --
	(250.88,194.26) --
	(250.94,194.27) --
	(250.99,194.27) --
	(251.04,194.28) --
	(251.09,194.28) --
	(251.14,194.28) --
	(251.19,194.29) --
	(251.24,194.29) --
	(251.29,194.30) --
	(251.34,194.30) --
	(251.40,194.31) --
	(251.45,194.31) --
	(251.50,194.32) --
	(251.55,194.32) --
	(251.60,194.33) --
	(251.65,194.33) --
	(251.70,194.34) --
	(251.75,194.34) --
	(251.81,194.35) --
	(251.86,194.35) --
	(251.91,194.35) --
	(251.96,194.36) --
	(252.01,194.36) --
	(252.06,194.37) --
	(252.11,194.37) --
	(252.16,194.38) --
	(252.22,194.38) --
	(252.27,194.39) --
	(252.32,194.39) --
	(252.37,194.40) --
	(252.42,194.40) --
	(252.47,194.41) --
	(252.52,194.41) --
	(252.57,194.41) --
	(252.62,194.42) --
	(252.68,194.42) --
	(252.73,194.43) --
	(252.78,194.43) --
	(252.83,194.44) --
	(252.88,194.44) --
	(252.93,194.45) --
	(252.98,194.45) --
	(253.03,194.46) --
	(253.09,194.46) --
	(253.14,194.47) --
	(253.19,194.47) --
	(253.24,194.48) --
	(253.29,194.48) --
	(253.34,194.48) --
	(253.39,194.49) --
	(253.44,194.49) --
	(253.50,194.50) --
	(253.55,194.50) --
	(253.60,194.51) --
	(253.65,194.51) --
	(253.70,194.52) --
	(253.75,194.52) --
	(253.80,194.53) --
	(253.85,194.53) --
	(253.90,194.54) --
	(253.96,194.54) --
	(254.01,194.54) --
	(254.06,194.55) --
	(254.11,194.55) --
	(254.16,194.56) --
	(254.21,194.56) --
	(254.26,194.57) --
	(254.31,194.57) --
	(254.37,194.58) --
	(254.42,194.58) --
	(254.47,194.59) --
	(254.52,194.59) --
	(254.57,194.60) --
	(254.62,194.60) --
	(254.67,194.61) --
	(254.72,194.61) --
	(254.78,194.61) --
	(254.83,194.62) --
	(254.88,194.62) --
	(254.93,194.63) --
	(254.98,194.63) --
	(255.03,194.64) --
	(255.08,194.64) --
	(255.13,194.65) --
	(255.18,194.65) --
	(255.24,194.66) --
	(255.29,194.66) --
	(255.34,194.67) --
	(255.39,194.67) --
	(255.44,194.68) --
	(255.49,194.68) --
	(255.54,194.68) --
	(255.59,194.69) --
	(255.65,194.69) --
	(255.70,194.70) --
	(255.75,194.70) --
	(255.80,194.71) --
	(255.85,194.71) --
	(255.90,194.72) --
	(255.95,194.72) --
	(256.00,194.73) --
	(256.06,194.73) --
	(256.11,194.74) --
	(256.16,194.74) --
	(256.21,194.74) --
	(256.26,194.75) --
	(256.31,194.75) --
	(256.36,194.76) --
	(256.41,194.76) --
	(256.46,194.77) --
	(256.52,194.77) --
	(256.57,194.78) --
	(256.62,194.78) --
	(256.67,194.79) --
	(256.72,194.79) --
	(256.77,194.80) --
	(256.82,194.80) --
	(256.87,194.81) --
	(256.93,194.81) --
	(256.98,194.81) --
	(257.03,194.82) --
	(257.08,194.82) --
	(257.13,194.83) --
	(257.18,194.83) --
	(257.23,194.84) --
	(257.28,194.84) --
	(257.34,194.85) --
	(257.39,194.85) --
	(257.44,194.86) --
	(257.49,194.86) --
	(257.54,194.87) --
	(257.59,194.87) --
	(257.64,194.87) --
	(257.69,194.88) --
	(257.74,194.88) --
	(257.80,194.89) --
	(257.85,194.89) --
	(257.90,194.90) --
	(257.95,194.90) --
	(258.00,194.91) --
	(258.05,194.91) --
	(258.10,194.92) --
	(258.15,194.92) --
	(258.21,194.93) --
	(258.26,194.93) --
	(258.31,194.94) --
	(258.36,194.94) --
	(258.41,194.94) --
	(258.46,194.95) --
	(258.51,194.95) --
	(258.56,194.96) --
	(258.62,194.96) --
	(258.67,194.97) --
	(258.72,194.97) --
	(258.77,194.98) --
	(258.82,194.98) --
	(258.87,194.99) --
	(258.92,194.99) --
	(258.97,195.00) --
	(259.02,195.00) --
	(259.08,195.00) --
	(259.13,195.01) --
	(259.18,195.01) --
	(259.23,195.02) --
	(259.28,195.02) --
	(259.33,195.03) --
	(259.38,195.03) --
	(259.43,195.04) --
	(259.49,195.04) --
	(259.54,195.05) --
	(259.59,195.05) --
	(259.64,195.06) --
	(259.69,195.06) --
	(259.74,195.07) --
	(259.79,195.07) --
	(259.84,195.07) --
	(259.90,195.08) --
	(259.95,195.08) --
	(260.00,195.09) --
	(260.05,195.09) --
	(260.10,195.10) --
	(260.15,195.10) --
	(260.20,195.11) --
	(260.25,195.11) --
	(260.30,195.12) --
	(260.36,195.12) --
	(260.41,195.13) --
	(260.46,195.13) --
	(260.51,195.14) --
	(260.56,195.14) --
	(260.61,195.14) --
	(260.66,195.15) --
	(260.71,195.15) --
	(260.77,195.16) --
	(260.82,195.16) --
	(260.87,195.17) --
	(260.92,195.17) --
	(260.97,195.18) --
	(261.02,195.18) --
	(261.07,195.19) --
	(261.12,195.19) --
	(261.18,195.20) --
	(261.23,195.20) --
	(261.28,195.20) --
	(261.33,195.21) --
	(261.38,195.21) --
	(261.43,195.22) --
	(261.48,195.22) --
	(261.53,195.23) --
	(261.58,195.23) --
	(261.64,195.24) --
	(261.69,195.24) --
	(261.74,195.25) --
	(261.79,195.25) --
	(261.84,195.26) --
	(261.89,195.26) --
	(261.94,195.27) --
	(261.99,195.27) --
	(262.05,195.27) --
	(262.10,195.28) --
	(262.15,195.28) --
	(262.20,195.29) --
	(262.25,195.29) --
	(262.30,195.30) --
	(262.35,195.30) --
	(262.40,195.31) --
	(262.46,195.31) --
	(262.51,195.32) --
	(262.56,195.32) --
	(262.61,195.33) --
	(262.66,195.33) --
	(262.71,195.33) --
	(262.76,195.34) --
	(262.81,195.34) --
	(262.86,195.35) --
	(262.92,195.35) --
	(262.97,195.36) --
	(263.02,195.36) --
	(263.07,195.37) --
	(263.12,195.37) --
	(263.17,195.38) --
	(263.22,195.38) --
	(263.27,195.39) --
	(263.33,195.39) --
	(263.38,195.40) --
	(263.43,195.40) --
	(263.48,195.40) --
	(263.53,195.41) --
	(263.58,195.41) --
	(263.63,195.42) --
	(263.68,195.42) --
	(263.74,195.43) --
	(263.79,195.43) --
	(263.84,195.44) --
	(263.89,195.44) --
	(263.94,195.45) --
	(263.99,195.45) --
	(264.04,195.46) --
	(264.09,195.46) --
	(264.14,195.46) --
	(264.20,195.47) --
	(264.25,195.47) --
	(264.30,195.48) --
	(264.35,195.48) --
	(264.40,195.49) --
	(264.45,195.49) --
	(264.50,195.50) --
	(264.55,195.50) --
	(264.61,195.51) --
	(264.66,195.51) --
	(264.71,195.52) --
	(264.76,195.52) --
	(264.81,195.53) --
	(264.86,195.53) --
	(264.91,195.53) --
	(264.96,195.54) --
	(265.02,195.54) --
	(265.07,195.55) --
	(265.12,195.55) --
	(265.17,195.56) --
	(265.22,195.56) --
	(265.27,195.57) --
	(265.32,195.57) --
	(265.37,195.58) --
	(265.42,195.58) --
	(265.48,195.59) --
	(265.53,195.59) --
	(265.58,195.59) --
	(265.63,195.60) --
	(265.68,195.60) --
	(265.73,195.61) --
	(265.78,195.61) --
	(265.83,195.62) --
	(265.89,195.62) --
	(265.94,195.63) --
	(265.99,195.63) --
	(266.04,195.64) --
	(266.09,195.64) --
	(266.14,195.65) --
	(266.19,195.65) --
	(266.24,195.66) --
	(266.30,195.66) --
	(266.35,195.66) --
	(266.40,195.67) --
	(266.45,195.67) --
	(266.50,195.68) --
	(266.55,195.68) --
	(266.60,195.69) --
	(266.65,195.69) --
	(266.70,195.70) --
	(266.76,195.70) --
	(266.81,195.71) --
	(266.86,195.71) --
	(266.91,195.72) --
	(266.96,195.72) --
	(267.01,195.73) --
	(267.06,195.73) --
	(267.11,195.73) --
	(267.17,195.74) --
	(267.22,195.74) --
	(267.27,195.75) --
	(267.32,195.75) --
	(267.37,195.76) --
	(267.42,195.76) --
	(267.47,195.77) --
	(267.52,195.77) --
	(267.58,195.78) --
	(267.63,195.78) --
	(267.68,195.79) --
	(267.73,195.79) --
	(267.78,195.79) --
	(267.83,195.80) --
	(267.88,195.80) --
	(267.93,195.81) --
	(267.98,195.81) --
	(268.04,195.82) --
	(268.09,195.82) --
	(268.14,195.83) --
	(268.19,195.83) --
	(268.24,195.84) --
	(268.29,195.84) --
	(268.34,195.85) --
	(268.39,195.85) --
	(268.45,195.86) --
	(268.50,195.86) --
	(268.55,195.86) --
	(268.60,195.87) --
	(268.65,195.87) --
	(268.70,195.88) --
	(268.75,195.88) --
	(268.80,195.89) --
	(268.86,195.89) --
	(268.91,195.90) --
	(268.96,195.90) --
	(269.01,195.91) --
	(269.06,195.91) --
	(269.11,195.92) --
	(269.16,195.92) --
	(269.21,195.92) --
	(269.26,195.93) --
	(269.32,195.93) --
	(269.37,195.94) --
	(269.42,195.94) --
	(269.47,195.95) --
	(269.52,195.95) --
	(269.57,195.96) --
	(269.62,195.96) --
	(269.67,195.97) --
	(269.73,195.97) --
	(269.78,195.98) --
	(269.83,195.98) --
	(269.88,195.99) --
	(269.93,195.99) --
	(269.98,195.99) --
	(270.03,196.00) --
	(270.08,196.00) --
	(270.14,196.01) --
	(270.19,196.01) --
	(270.24,196.02) --
	(270.29,196.02) --
	(270.34,196.03) --
	(270.39,196.03) --
	(270.44,196.04) --
	(270.49,196.04) --
	(270.54,196.05) --
	(270.60,196.05) --
	(270.65,196.05) --
	(270.70,196.06) --
	(270.75,196.06) --
	(270.80,196.07) --
	(270.85,196.07) --
	(270.90,196.08) --
	(270.95,196.08) --
	(271.01,196.09) --
	(271.06,196.09) --
	(271.11,196.10) --
	(271.16,196.10) --
	(271.21,196.11) --
	(271.26,196.11) --
	(271.31,196.12) --
	(271.36,196.12) --
	(271.42,196.12) --
	(271.47,196.13) --
	(271.52,196.13) --
	(271.57,196.14) --
	(271.62,196.14) --
	(271.67,196.15) --
	(271.72,196.15) --
	(271.77,196.16) --
	(271.82,196.16) --
	(271.88,196.17) --
	(271.93,196.17) --
	(271.98,196.18) --
	(272.03,196.18) --
	(272.08,196.19) --
	(272.13,196.19) --
	(272.18,196.19) --
	(272.23,196.20) --
	(272.29,196.20) --
	(272.34,196.21) --
	(272.39,196.21) --
	(272.44,196.22) --
	(272.49,196.22) --
	(272.54,196.23) --
	(272.59,196.23) --
	(272.64,196.24) --
	(272.70,196.24) --
	(272.75,196.25) --
	(272.80,196.25) --
	(272.85,196.25) --
	(272.90,196.26) --
	(272.95,196.26) --
	(273.00,196.27) --
	(273.05,196.27) --
	(273.10,196.28) --
	(273.16,196.28) --
	(273.21,196.29) --
	(273.26,196.29) --
	(273.31,196.30) --
	(273.36,196.30) --
	(273.41,196.31) --
	(273.46,196.31) --
	(273.51,196.32) --
	(273.57,196.32) --
	(273.62,196.32) --
	(273.67,196.33) --
	(273.72,196.33) --
	(273.77,196.34) --
	(273.82,196.34) --
	(273.87,196.35) --
	(273.92,196.35) --
	(273.98,196.36) --
	(274.03,196.36) --
	(274.08,196.37) --
	(274.13,196.37) --
	(274.18,196.38) --
	(274.23,196.38) --
	(274.28,196.38) --
	(274.33,196.39) --
	(274.38,196.39) --
	(274.44,196.40) --
	(274.49,196.40) --
	(274.54,196.41) --
	(274.59,196.41) --
	(274.64,196.42) --
	(274.69,196.42) --
	(274.74,196.43) --
	(274.79,196.43) --
	(274.85,196.44) --
	(274.90,196.44) --
	(274.95,196.45) --
	(275.00,196.45) --
	(275.05,196.45) --
	(275.10,196.46) --
	(275.15,196.46) --
	(275.20,196.47) --
	(275.26,196.47) --
	(275.31,196.48) --
	(275.36,196.48) --
	(275.41,196.49) --
	(275.46,196.49) --
	(275.51,196.50) --
	(275.56,196.50) --
	(275.61,196.51) --
	(275.66,196.51) --
	(275.72,196.51) --
	(275.77,196.52) --
	(275.82,196.52) --
	(275.87,196.53) --
	(275.92,196.53) --
	(275.97,196.54) --
	(276.02,196.54) --
	(276.07,196.55) --
	(276.13,196.55) --
	(276.18,196.56) --
	(276.23,196.56) --
	(276.28,196.57) --
	(276.33,196.57) --
	(276.38,196.58) --
	(276.43,196.58) --
	(276.48,196.58) --
	(276.54,196.59) --
	(276.59,196.59) --
	(276.64,196.60) --
	(276.69,196.60) --
	(276.74,196.61) --
	(276.79,196.61) --
	(276.84,196.62) --
	(276.89,196.62) --
	(276.95,196.63) --
	(277.00,196.63) --
	(277.05,196.64) --
	(277.10,196.64) --
	(277.15,196.64) --
	(277.20,196.65) --
	(277.25,196.65) --
	(277.30,196.66) --
	(277.35,196.66) --
	(277.41,196.67) --
	(277.46,196.67) --
	(277.51,196.68) --
	(277.56,196.68) --
	(277.61,196.69) --
	(277.66,196.69) --
	(277.71,196.70) --
	(277.76,196.70) --
	(277.82,196.71) --
	(277.87,196.71) --
	(277.92,196.71) --
	(277.97,196.72) --
	(278.02,196.72) --
	(278.07,196.73) --
	(278.12,196.73) --
	(278.17,196.74) --
	(278.23,196.74) --
	(278.28,196.75) --
	(278.33,196.75) --
	(278.38,196.76) --
	(278.43,196.76) --
	(278.48,196.77) --
	(278.53,196.77) --
	(278.58,196.78) --
	(278.63,196.78) --
	(278.69,196.78) --
	(278.74,196.79) --
	(278.79,196.79) --
	(278.84,196.80) --
	(278.89,196.80) --
	(278.94,196.81) --
	(278.99,196.81) --
	(279.04,196.82) --
	(279.10,196.82) --
	(279.15,196.83) --
	(279.20,196.83) --
	(279.25,196.84) --
	(279.30,196.84) --
	(279.35,196.84) --
	(279.40,196.85) --
	(279.45,196.85) --
	(279.51,196.86) --
	(279.56,196.86) --
	(279.61,196.87) --
	(279.66,196.87) --
	(279.71,196.88) --
	(279.76,196.88) --
	(279.81,196.89) --
	(279.86,196.89) --
	(279.91,196.90) --
	(279.97,196.90) --
	(280.02,196.91) --
	(280.07,196.91) --
	(280.12,196.91) --
	(280.17,196.92) --
	(280.22,196.92) --
	(280.27,196.93) --
	(280.32,196.93) --
	(280.38,196.94) --
	(280.43,196.94) --
	(280.48,196.95) --
	(280.53,196.95) --
	(280.58,196.96) --
	(280.63,196.96) --
	(280.68,196.97) --
	(280.73,196.97) --
	(280.79,196.97) --
	(280.84,196.98) --
	(280.89,196.98) --
	(280.94,196.99) --
	(280.99,196.99) --
	(281.04,197.00) --
	(281.09,197.00) --
	(281.14,197.01) --
	(281.19,197.01) --
	(281.25,197.02) --
	(281.30,197.02) --
	(281.35,197.03) --
	(281.40,197.03) --
	(281.45,197.04) --
	(281.50,197.04) --
	(281.55,197.04) --
	(281.60,197.05) --
	(281.66,197.05) --
	(281.71,197.06) --
	(281.76,197.06) --
	(281.81,197.07) --
	(281.86,197.07) --
	(281.91,197.08) --
	(281.96,197.08) --
	(282.01,197.09) --
	(282.07,197.09) --
	(282.12,197.10) --
	(282.17,197.10) --
	(282.22,197.10) --
	(282.27,197.11) --
	(282.32,197.11) --
	(282.37,197.12) --
	(282.42,197.12) --
	(282.47,197.13) --
	(282.53,197.13) --
	(282.58,197.14) --
	(282.63,197.14) --
	(282.68,197.15) --
	(282.73,197.15) --
	(282.78,197.16) --
	(282.83,197.16) --
	(282.88,197.17) --
	(282.94,197.17) --
	(282.99,197.17) --
	(283.04,197.18) --
	(283.09,197.18) --
	(283.14,197.19) --
	(283.19,197.19) --
	(283.24,197.20) --
	(283.29,197.20) --
	(283.35,197.21) --
	(283.40,197.21) --
	(283.45,197.22) --
	(283.50,197.22) --
	(283.55,197.23) --
	(283.60,197.23) --
	(283.65,197.24) --
	(283.70,197.24) --
	(283.75,197.24) --
	(283.81,197.25) --
	(283.86,197.25) --
	(283.91,197.26) --
	(283.96,197.26) --
	(284.01,197.27) --
	(284.06,197.27) --
	(284.11,197.28) --
	(284.16,197.28) --
	(284.22,197.29) --
	(284.27,197.29) --
	(284.32,197.30) --
	(284.37,197.30) --
	(284.42,197.30) --
	(284.47,197.31) --
	(284.52,197.31) --
	(284.57,197.32) --
	(284.63,197.32) --
	(284.68,197.33) --
	(284.73,197.33) --
	(284.78,197.34) --
	(284.83,197.34) --
	(284.88,197.35) --
	(284.93,197.35) --
	(284.98,197.36) --
	(285.03,197.36) --
	(285.09,197.37) --
	(285.14,197.37) --
	(285.19,197.37) --
	(285.24,197.38) --
	(285.29,197.38) --
	(285.34,197.39) --
	(285.39,197.39) --
	(285.44,197.40) --
	(285.50,197.40) --
	(285.55,197.41) --
	(285.60,197.41) --
	(285.65,197.42) --
	(285.70,197.42) --
	(285.75,197.43) --
	(285.80,197.43) --
	(285.85,197.43) --
	(285.91,197.44) --
	(285.96,197.44) --
	(286.01,197.45) --
	(286.06,197.45) --
	(286.11,197.46) --
	(286.16,197.46) --
	(286.21,197.47) --
	(286.26,197.47) --
	(286.31,197.48) --
	(286.37,197.48) --
	(286.42,197.49) --
	(286.47,197.49) --
	(286.52,197.50) --
	(286.57,197.50) --
	(286.62,197.50) --
	(286.67,197.51) --
	(286.72,197.51) --
	(286.78,197.52) --
	(286.83,197.52) --
	(286.88,197.53) --
	(286.93,197.53) --
	(286.98,197.54) --
	(287.03,197.54) --
	(287.08,197.55) --
	(287.13,197.55) --
	(287.19,197.56) --
	(287.24,197.56) --
	(287.29,197.56) --
	(287.34,197.57) --
	(287.39,197.57) --
	(287.44,197.58) --
	(287.49,197.58) --
	(287.54,197.59) --
	(287.59,197.59) --
	(287.65,197.60) --
	(287.70,197.60) --
	(287.75,197.61) --
	(287.80,197.61) --
	(287.85,197.62) --
	(287.90,197.62) --
	(287.95,197.63) --
	(288.00,197.63) --
	(288.06,197.63) --
	(288.11,197.64) --
	(288.16,197.64) --
	(288.21,197.65) --
	(288.26,197.65) --
	(288.31,197.66) --
	(288.36,197.66) --
	(288.41,197.67) --
	(288.47,197.67) --
	(288.52,197.68) --
	(288.57,197.68) --
	(288.62,197.69) --
	(288.67,197.69) --
	(288.72,197.69) --
	(288.77,197.70) --
	(288.82,197.70) --
	(288.87,197.71) --
	(288.93,197.71) --
	(288.98,197.72) --
	(289.03,197.72) --
	(289.08,197.73) --
	(289.13,197.73) --
	(289.18,197.74) --
	(289.23,197.74) --
	(289.28,197.75) --
	(289.34,197.75) --
	(289.39,197.76) --
	(289.44,197.76) --
	(289.49,197.76) --
	(289.54,197.77) --
	(289.59,197.77) --
	(289.64,197.78) --
	(289.69,197.78) --
	(289.75,197.79) --
	(289.80,197.79) --
	(289.85,197.80) --
	(289.90,197.80) --
	(289.95,197.81) --
	(290.00,197.81) --
	(290.05,197.82) --
	(290.10,197.82) --
	(290.15,197.83) --
	(290.21,197.83) --
	(290.26,197.83) --
	(290.31,197.84) --
	(290.36,197.84) --
	(290.41,197.85) --
	(290.46,197.85) --
	(290.51,197.86) --
	(290.56,197.86) --
	(290.62,197.87) --
	(290.67,197.87) --
	(290.72,197.88) --
	(290.77,197.88) --
	(290.82,197.89) --
	(290.87,197.89) --
	(290.92,197.89) --
	(290.97,197.90) --
	(291.03,197.90) --
	(291.08,197.91) --
	(291.13,197.91) --
	(291.18,197.92) --
	(291.23,197.92) --
	(291.28,197.93) --
	(291.33,197.93) --
	(291.38,197.94) --
	(291.43,197.94) --
	(291.49,197.95) --
	(291.54,197.95) --
	(291.59,197.96) --
	(291.64,197.96) --
	(291.69,197.96) --
	(291.74,197.97) --
	(291.79,197.97) --
	(291.84,197.98) --
	(291.90,197.98) --
	(291.95,197.99) --
	(292.00,197.99) --
	(292.05,198.00) --
	(292.10,198.00) --
	(292.15,198.01) --
	(292.20,198.01) --
	(292.25,198.02) --
	(292.31,198.02) --
	(292.36,198.02) --
	(292.41,198.03) --
	(292.46,198.03) --
	(292.51,198.04) --
	(292.56,198.04) --
	(292.61,198.05) --
	(292.66,198.05) --
	(292.71,198.06) --
	(292.77,198.06) --
	(292.82,198.07) --
	(292.87,198.07) --
	(292.92,198.08) --
	(292.97,198.08) --
	(293.02,198.09) --
	(293.07,198.09) --
	(293.12,198.09) --
	(293.18,198.10) --
	(293.23,198.10) --
	(293.28,198.11) --
	(293.33,198.11) --
	(293.38,198.12) --
	(293.43,198.12) --
	(293.48,198.13) --
	(293.53,198.13) --
	(293.59,198.14) --
	(293.64,198.14) --
	(293.69,198.15) --
	(293.74,198.15) --
	(293.79,198.15) --
	(293.84,198.16) --
	(293.89,198.16) --
	(293.94,198.17) --
	(293.99,198.17) --
	(294.05,198.18) --
	(294.10,198.18) --
	(294.15,198.19) --
	(294.20,198.19) --
	(294.25,198.20) --
	(294.30,198.20) --
	(294.35,198.21) --
	(294.40,198.21) --
	(294.46,198.22) --
	(294.51,198.22) --
	(294.56,198.22) --
	(294.61,198.23) --
	(294.66,198.23) --
	(294.71,198.24) --
	(294.76,198.24) --
	(294.81,198.25) --
	(294.87,198.25) --
	(294.92,198.26) --
	(294.97,198.26) --
	(295.02,198.27) --
	(295.07,198.27) --
	(295.12,198.28) --
	(295.17,198.28) --
	(295.22,198.29) --
	(295.27,198.29) --
	(295.33,198.29) --
	(295.38,198.30) --
	(295.43,198.30) --
	(295.48,198.31) --
	(295.53,198.31) --
	(295.58,198.32) --
	(295.63,198.32) --
	(295.68,198.33) --
	(295.74,198.33) --
	(295.79,198.34) --
	(295.84,198.34) --
	(295.89,198.35) --
	(295.94,198.35) --
	(295.99,198.35) --
	(296.04,198.36) --
	(296.09,198.36) --
	(296.15,198.37) --
	(296.20,198.37) --
	(296.25,198.38) --
	(296.30,198.38) --
	(296.35,198.39) --
	(296.40,198.39) --
	(296.45,198.40) --
	(296.50,198.40) --
	(296.55,198.41) --
	(296.61,198.41) --
	(296.66,198.42) --
	(296.71,198.42) --
	(296.76,198.42) --
	(296.81,198.43) --
	(296.86,198.43) --
	(296.91,198.44) --
	(296.96,198.44) --
	(297.02,198.45) --
	(297.07,198.45) --
	(297.12,198.46) --
	(297.17,198.46) --
	(297.22,198.47) --
	(297.27,198.47) --
	(297.32,198.48) --
	(297.37,198.48) --
	(297.43,198.48) --
	(297.48,198.49) --
	(297.53,198.49) --
	(297.58,198.50) --
	(297.63,198.50) --
	(297.68,198.51) --
	(297.73,198.51) --
	(297.78,198.52) --
	(297.83,198.52) --
	(297.89,198.53) --
	(297.94,198.53) --
	(297.99,198.54) --
	(298.04,198.54) --
	(298.09,198.55) --
	(298.14,198.55) --
	(298.19,198.55) --
	(298.24,198.56) --
	(298.30,198.56) --
	(298.35,198.57) --
	(298.40,198.57) --
	(298.45,198.58) --
	(298.50,198.58) --
	(298.55,198.59) --
	(298.60,198.59) --
	(298.65,198.60) --
	(298.71,198.60) --
	(298.76,198.61) --
	(298.81,198.61) --
	(298.86,198.61) --
	(298.91,198.62) --
	(298.96,198.62) --
	(299.01,198.63) --
	(299.06,198.63) --
	(299.11,198.64) --
	(299.17,198.64) --
	(299.22,198.65) --
	(299.27,198.65) --
	(299.32,198.66) --
	(299.37,198.66) --
	(299.42,198.67) --
	(299.47,198.67) --
	(299.52,198.68) --
	(299.58,198.68) --
	(299.63,198.68) --
	(299.68,198.69) --
	(299.73,198.69) --
	(299.78,198.70) --
	(299.83,198.70) --
	(299.88,198.71) --
	(299.93,198.71) --
	(299.99,198.72) --
	(300.04,198.72) --
	(300.09,198.73) --
	(300.14,198.73) --
	(300.19,198.74) --
	(300.24,198.74) --
	(300.29,198.74) --
	(300.34,198.75) --
	(300.39,198.75) --
	(300.45,198.76) --
	(300.50,198.76) --
	(300.55,198.77) --
	(300.60,198.77) --
	(300.65,198.78) --
	(300.70,198.78) --
	(300.75,198.79) --
	(300.80,198.79) --
	(300.86,198.80) --
	(300.91,198.80) --
	(300.96,198.81) --
	(301.01,198.81) --
	(301.06,198.81) --
	(301.11,198.82) --
	(301.16,198.82) --
	(301.21,198.83) --
	(301.27,198.83) --
	(301.32,198.84) --
	(301.37,198.84) --
	(301.42,198.85) --
	(301.47,198.85) --
	(301.52,198.86) --
	(301.57,198.86) --
	(301.62,198.87) --
	(301.67,198.87) --
	(301.73,198.88) --
	(301.78,198.88) --
	(301.83,198.88) --
	(301.88,198.89) --
	(301.93,198.89) --
	(301.98,198.90) --
	(302.03,198.90) --
	(302.08,198.91) --
	(302.14,198.91) --
	(302.19,198.92) --
	(302.24,198.92) --
	(302.29,198.93) --
	(302.34,198.93) --
	(302.39,198.94) --
	(302.44,198.94) --
	(302.49,198.94) --
	(302.55,198.95) --
	(302.60,198.95) --
	(302.65,198.96) --
	(302.70,198.96) --
	(302.75,198.97) --
	(302.80,198.97) --
	(302.85,198.98) --
	(302.90,198.98) --
	(302.96,198.99) --
	(303.01,198.99) --
	(303.06,199.00) --
	(303.11,199.00) --
	(303.16,199.01) --
	(303.21,199.01) --
	(303.26,199.01) --
	(303.31,199.02) --
	(303.36,199.02) --
	(303.42,199.03) --
	(303.47,199.03) --
	(303.52,199.04) --
	(303.57,199.04) --
	(303.62,199.05) --
	(303.67,199.05) --
	(303.72,199.06) --
	(303.77,199.06) --
	(303.83,199.07) --
	(303.88,199.07) --
	(303.93,199.07) --
	(303.98,199.08) --
	(304.03,199.08) --
	(304.08,199.09) --
	(304.13,199.09) --
	(304.18,199.10) --
	(304.24,199.10) --
	(304.29,199.11) --
	(304.34,199.11) --
	(304.39,199.12) --
	(304.44,199.12) --
	(304.49,199.13) --
	(304.54,199.13) --
	(304.59,199.14) --
	(304.64,199.14) --
	(304.70,199.14) --
	(304.75,199.15) --
	(304.80,199.15) --
	(304.85,199.16) --
	(304.90,199.16) --
	(304.95,199.17) --
	(305.00,199.17) --
	(305.05,199.18) --
	(305.11,199.18) --
	(305.16,199.19) --
	(305.21,199.19) --
	(305.26,199.20) --
	(305.31,199.20) --
	(305.36,199.20) --
	(305.41,199.21) --
	(305.46,199.21) --
	(305.52,199.22) --
	(305.57,199.22) --
	(305.62,199.23) --
	(305.67,199.23) --
	(305.72,199.24) --
	(305.77,199.24) --
	(305.82,199.25) --
	(305.87,199.25) --
	(305.92,199.26) --
	(305.98,199.26) --
	(306.03,199.27) --
	(306.08,199.27) --
	(306.13,199.27) --
	(306.18,199.28) --
	(306.23,199.28) --
	(306.28,199.29) --
	(306.33,199.29) --
	(306.39,199.30) --
	(306.44,199.30) --
	(306.49,199.31) --
	(306.54,199.31) --
	(306.59,199.32) --
	(306.64,199.32) --
	(306.69,199.33) --
	(306.74,199.33) --
	(306.80,199.34) --
	(306.85,199.34) --
	(306.90,199.34) --
	(306.95,199.35) --
	(307.00,199.35) --
	(307.05,199.36) --
	(307.10,199.36) --
	(307.15,199.37) --
	(307.20,199.37) --
	(307.26,199.38) --
	(307.31,199.38) --
	(307.36,199.39) --
	(307.41,199.39) --
	(307.46,199.40) --
	(307.51,199.40) --
	(307.56,199.40) --
	(307.61,199.41) --
	(307.67,199.41) --
	(307.72,199.42) --
	(307.77,199.42) --
	(307.82,199.43) --
	(307.87,199.43) --
	(307.92,199.44) --
	(307.97,199.44) --
	(308.02,199.45) --
	(308.08,199.45) --
	(308.13,199.46) --
	(308.18,199.46) --
	(308.23,199.47) --
	(308.28,199.47) --
	(308.33,199.47) --
	(308.38,199.48) --
	(308.43,199.48) --
	(308.48,199.49) --
	(308.54,199.49) --
	(308.59,199.50) --
	(308.64,199.50) --
	(308.69,199.51) --
	(308.74,199.51) --
	(308.79,199.52) --
	(308.84,199.52) --
	(308.89,199.53) --
	(308.95,199.53) --
	(309.00,199.53) --
	(309.05,199.54) --
	(309.10,199.54) --
	(309.15,199.55) --
	(309.20,199.55) --
	(309.25,199.56) --
	(309.30,199.56) --
	(309.36,199.57) --
	(309.41,199.57) --
	(309.46,199.58) --
	(309.51,199.58) --
	(309.56,199.59) --
	(309.61,199.59) --
	(309.66,199.60) --
	(309.71,199.60) --
	(309.76,199.60) --
	(309.82,199.61) --
	(309.87,199.61) --
	(309.92,199.62) --
	(309.97,199.62) --
	(310.02,199.63) --
	(310.07,199.63) --
	(310.12,199.64) --
	(310.17,199.64) --
	(310.23,199.65) --
	(310.28,199.65) --
	(310.33,199.66) --
	(310.38,199.66) --
	(310.43,199.66) --
	(310.48,199.67) --
	(310.53,199.67) --
	(310.58,199.68) --
	(310.64,199.68) --
	(310.69,199.69) --
	(310.74,199.69) --
	(310.79,199.70) --
	(310.84,199.70) --
	(310.89,199.71) --
	(310.94,199.71) --
	(310.99,199.72) --
	(311.04,199.72) --
	(311.10,199.73) --
	(311.15,199.73) --
	(311.20,199.73) --
	(311.25,199.74) --
	(311.30,199.74) --
	(311.35,199.75) --
	(311.40,199.75) --
	(311.45,199.76) --
	(311.51,199.76) --
	(311.56,199.77) --
	(311.61,199.77) --
	(311.66,199.78) --
	(311.71,199.78) --
	(311.76,199.79) --
	(311.81,199.79) --
	(311.86,199.79) --
	(311.92,199.80) --
	(311.97,199.80) --
	(312.02,199.81) --
	(312.07,199.81) --
	(312.12,199.82) --
	(312.17,199.82) --
	(312.22,199.83) --
	(312.27,199.83) --
	(312.32,199.84) --
	(312.38,199.84) --
	(312.43,199.85) --
	(312.48,199.85) --
	(312.53,199.86) --
	(312.58,199.86) --
	(312.63,199.86) --
	(312.68,199.87) --
	(312.73,199.87) --
	(312.79,199.88) --
	(312.84,199.88) --
	(312.89,199.89) --
	(312.94,199.89) --
	(312.99,199.90) --
	(313.04,199.90) --
	(313.09,199.91) --
	(313.14,199.91) --
	(313.20,199.92) --
	(313.25,199.92) --
	(313.30,199.93) --
	(313.35,199.93) --
	(313.40,199.93) --
	(313.45,199.94) --
	(313.50,199.94) --
	(313.55,199.95) --
	(313.60,199.95) --
	(313.66,199.96) --
	(313.71,199.96) --
	(313.76,199.97) --
	(313.81,199.97) --
	(313.86,199.98) --
	(313.91,199.98) --
	(313.96,199.99) --
	(314.01,199.99) --
	(314.07,199.99) --
	(314.12,200.00) --
	(314.17,200.00) --
	(314.22,200.01) --
	(314.27,200.01) --
	(314.32,200.02) --
	(314.37,200.02) --
	(314.42,200.03) --
	(314.48,200.03) --
	(314.53,200.04) --
	(314.58,200.04) --
	(314.63,200.05) --
	(314.68,200.05) --
	(314.73,200.06) --
	(314.78,200.06) --
	(314.83,200.06) --
	(314.88,200.07) --
	(314.94,200.07) --
	(314.99,200.08) --
	(315.04,200.08) --
	(315.09,200.09) --
	(315.14,200.09) --
	(315.19,200.10) --
	(315.24,200.10) --
	(315.29,200.11) --
	(315.35,200.11) --
	(315.40,200.12) --
	(315.45,200.12) --
	(315.50,200.12) --
	(315.55,200.13) --
	(315.60,200.13) --
	(315.65,200.14) --
	(315.70,200.14) --
	(315.76,200.15) --
	(315.81,200.15) --
	(315.86,200.16) --
	(315.91,200.16) --
	(315.96,200.17) --
	(316.01,200.17) --
	(316.06,200.18) --
	(316.11,200.18) --
	(316.16,200.19) --
	(316.22,200.19) --
	(316.27,200.19) --
	(316.32,200.20) --
	(316.37,200.20) --
	(316.42,200.21) --
	(316.47,200.21) --
	(316.52,200.22) --
	(316.57,200.22) --
	(316.63,200.23) --
	(316.68,200.23) --
	(316.73,200.24) --
	(316.78,200.24) --
	(316.83,200.25) --
	(316.88,200.25) --
	(316.93,200.25) --
	(316.98,200.26) --
	(317.04,200.26) --
	(317.09,200.27) --
	(317.14,200.27) --
	(317.19,200.28) --
	(317.24,200.28) --
	(317.29,200.29) --
	(317.34,200.29) --
	(317.39,200.30) --
	(317.44,200.30) --
	(317.50,200.31) --
	(317.55,200.31) --
	(317.60,200.32) --
	(317.65,200.32) --
	(317.70,200.32) --
	(317.75,200.33) --
	(317.80,200.33) --
	(317.85,200.34) --
	(317.91,200.34) --
	(317.96,200.35) --
	(318.01,200.35) --
	(318.06,200.36) --
	(318.11,200.36) --
	(318.16,200.37) --
	(318.21,200.37) --
	(318.26,200.38) --
	(318.32,200.38) --
	(318.37,200.38) --
	(318.42,200.39) --
	(318.47,200.39) --
	(318.52,200.40) --
	(318.57,200.40) --
	(318.62,200.41) --
	(318.67,200.41) --
	(318.72,200.42) --
	(318.78,200.42) --
	(318.83,200.43) --
	(318.88,200.43) --
	(318.93,200.44) --
	(318.98,200.44) --
	(319.03,200.45) --
	(319.08,200.45) --
	(319.13,200.45) --
	(319.19,200.46) --
	(319.24,200.46) --
	(319.29,200.47) --
	(319.34,200.47) --
	(319.39,200.48) --
	(319.44,200.48) --
	(319.49,200.49) --
	(319.54,200.49) --
	(319.60,200.50) --
	(319.65,200.50) --
	(319.70,200.51) --
	(319.75,200.51) --
	(319.80,200.52) --
	(319.85,200.52) --
	(319.90,200.52) --
	(319.95,200.53) --
	(320.00,200.53) --
	(320.06,200.54) --
	(320.11,200.54) --
	(320.16,200.55) --
	(320.21,200.55) --
	(320.26,200.56) --
	(320.31,200.56) --
	(320.36,200.57) --
	(320.41,200.57) --
	(320.47,200.58) --
	(320.52,200.58) --
	(320.57,200.58) --
	(320.62,200.59) --
	(320.67,200.59) --
	(320.72,200.60) --
	(320.77,200.60) --
	(320.82,200.61) --
	(320.88,200.61) --
	(320.93,200.62) --
	(320.98,200.62) --
	(321.03,200.63) --
	(321.08,200.63) --
	(321.13,200.64) --
	(321.18,200.64) --
	(321.23,200.65) --
	(321.28,200.65) --
	(321.34,200.65) --
	(321.39,200.66) --
	(321.44,200.66) --
	(321.49,200.67) --
	(321.54,200.67) --
	(321.59,200.68) --
	(321.64,200.68) --
	(321.69,200.69) --
	(321.75,200.69) --
	(321.80,200.70) --
	(321.85,200.70) --
	(321.90,200.71) --
	(321.95,200.71) --
	(322.00,200.71) --
	(322.05,200.72) --
	(322.10,200.72) --
	(322.16,200.73) --
	(322.21,200.73) --
	(322.26,200.74) --
	(322.31,200.74) --
	(322.36,200.75) --
	(322.41,200.75) --
	(322.46,200.76) --
	(322.51,200.76) --
	(322.56,200.77) --
	(322.62,200.77) --
	(322.67,200.78) --
	(322.72,200.78) --
	(322.77,200.78) --
	(322.82,200.79) --
	(322.87,200.79) --
	(322.92,200.80) --
	(322.97,200.80) --
	(323.03,200.81) --
	(323.08,200.81) --
	(323.13,200.82) --
	(323.18,200.82) --
	(323.23,200.83) --
	(323.28,200.83) --
	(323.33,200.84) --
	(323.38,200.84) --
	(323.44,200.84) --
	(323.49,200.85);
\end{scope}
\begin{scope}
\path[clip] ( 46.62, 34.03) rectangle (336.67,117.89);
\definecolor[named]{fillColor}{rgb}{0.90,0.90,0.90}

\path[fill=fillColor] ( 46.62, 34.03) rectangle (336.67,117.89);
\definecolor[named]{drawColor}{rgb}{0.95,0.95,0.95}

\path[draw=drawColor,line width= 0.3pt,line join=round] ( 46.62, 49.08) --
	(336.67, 49.08);

\path[draw=drawColor,line width= 0.3pt,line join=round] ( 46.62, 71.54) --
	(336.67, 71.54);

\path[draw=drawColor,line width= 0.3pt,line join=round] ( 46.62, 94.00) --
	(336.67, 94.00);

\path[draw=drawColor,line width= 0.3pt,line join=round] ( 46.62,116.46) --
	(336.67,116.46);

\path[draw=drawColor,line width= 0.3pt,line join=round] ( 85.40, 34.03) --
	( 85.40,117.89);

\path[draw=drawColor,line width= 0.3pt,line join=round] (136.60, 34.03) --
	(136.60,117.89);

\path[draw=drawColor,line width= 0.3pt,line join=round] (187.80, 34.03) --
	(187.80,117.89);

\path[draw=drawColor,line width= 0.3pt,line join=round] (239.01, 34.03) --
	(239.01,117.89);

\path[draw=drawColor,line width= 0.3pt,line join=round] (290.21, 34.03) --
	(290.21,117.89);
\definecolor[named]{drawColor}{rgb}{1.00,1.00,1.00}

\path[draw=drawColor,line width= 0.6pt,line join=round] ( 46.62, 37.85) --
	(336.67, 37.85);

\path[draw=drawColor,line width= 0.6pt,line join=round] ( 46.62, 60.31) --
	(336.67, 60.31);

\path[draw=drawColor,line width= 0.6pt,line join=round] ( 46.62, 82.77) --
	(336.67, 82.77);

\path[draw=drawColor,line width= 0.6pt,line join=round] ( 46.62,105.23) --
	(336.67,105.23);

\path[draw=drawColor,line width= 0.6pt,line join=round] ( 59.80, 34.03) --
	( 59.80,117.89);

\path[draw=drawColor,line width= 0.6pt,line join=round] (111.00, 34.03) --
	(111.00,117.89);

\path[draw=drawColor,line width= 0.6pt,line join=round] (162.20, 34.03) --
	(162.20,117.89);

\path[draw=drawColor,line width= 0.6pt,line join=round] (213.40, 34.03) --
	(213.40,117.89);

\path[draw=drawColor,line width= 0.6pt,line join=round] (264.61, 34.03) --
	(264.61,117.89);

\path[draw=drawColor,line width= 0.6pt,line join=round] (315.81, 34.03) --
	(315.81,117.89);
\definecolor[named]{drawColor}{rgb}{0.00,0.00,0.00}

\path[draw=drawColor,line width= 0.6pt,line join=round] ( 59.80,104.20) --
	( 59.85,109.81) --
	( 59.90,103.57) --
	( 59.96, 98.18) --
	( 60.01, 92.79) --
	( 60.06, 91.26) --
	( 60.11, 86.14) --
	( 60.16, 84.70) --
	( 60.21, 83.22) --
	( 60.26, 81.29) --
	( 60.31, 82.95) --
	( 60.37, 82.86) --
	( 60.42, 84.57) --
	( 60.47, 80.97) --
	( 60.52, 83.13) --
	( 60.57, 82.23) --
	( 60.62, 81.24) --
	( 60.67, 79.09) --
	( 60.72, 83.98) --
	( 60.78, 81.42) --
	( 60.83, 82.59) --
	( 60.88, 79.94) --
	( 60.93, 79.45) --
	( 60.98, 81.11) --
	( 61.03, 80.43) --
	( 61.08, 81.29) --
	( 61.13, 81.87) --
	( 61.18, 81.38) --
	( 61.24, 81.83) --
	( 61.29, 80.16) --
	( 61.34, 79.58) --
	( 61.39, 80.48) --
	( 61.44, 82.46) --
	( 61.49, 81.51) --
	( 61.54, 80.52) --
	( 61.59, 82.41) --
	( 61.65, 81.20) --
	( 61.70, 80.79) --
	( 61.75, 80.52) --
	( 61.80, 82.28) --
	( 61.85, 81.78) --
	( 61.90, 82.01) --
	( 61.95, 81.33) --
	( 62.00, 81.24) --
	( 62.06, 80.12) --
	( 62.11, 79.76) --
	( 62.16, 80.34) --
	( 62.21, 79.31) --
	( 62.26, 81.96) --
	( 62.31, 82.77) --
	( 62.36, 81.11) --
	( 62.41, 81.87) --
	( 62.46, 82.37) --
	( 62.52, 82.10) --
	( 62.57, 80.97) --
	( 62.62, 82.73) --
	( 62.67, 81.47) --
	( 62.72, 80.66) --
	( 62.77, 81.11) --
	( 62.82, 80.93) --
	( 62.87, 81.60) --
	( 62.93, 80.93) --
	( 62.98, 82.28) --
	( 63.03, 83.98) --
	( 63.08, 80.05) --
	( 63.13, 81.74) --
	( 63.18, 81.38) --
	( 63.23, 81.51) --
	( 63.28, 80.84) --
	( 63.34, 81.38) --
	( 63.39, 81.74) --
	( 63.44, 82.73) --
	( 63.49, 80.59) --
	( 63.54, 82.86) --
	( 63.59, 82.19) --
	( 63.64, 80.95) --
	( 63.69, 82.14) --
	( 63.74, 81.29) --
	( 63.80, 78.64) --
	( 63.85, 81.38) --
	( 63.90, 79.24) --
	( 63.95, 83.02) --
	( 64.00, 79.56) --
	( 64.05, 81.06) --
	( 64.10, 81.20) --
	( 64.15, 82.19) --
	( 64.21, 81.60) --
	( 64.26, 80.14) --
	( 64.31, 80.41) --
	( 64.36, 81.74) --
	( 64.41, 80.10) --
	( 64.46, 78.01) --
	( 64.51, 83.40) --
	( 64.56, 81.78) --
	( 64.62, 82.03) --
	( 64.67, 81.56) --
	( 64.72, 77.42) --
	( 64.77, 81.47) --
	( 64.82, 81.20) --
	( 64.87, 79.72) --
	( 64.92, 80.70) --
	( 64.97, 81.13) --
	( 65.02, 80.79) --
	( 65.08, 81.09) --
	( 65.13, 81.65) --
	( 65.18, 80.05) --
	( 65.23, 81.65) --
	( 65.28, 79.38) --
	( 65.33, 80.52) --
	( 65.38, 80.77) --
	( 65.43, 79.27) --
	( 65.49, 80.91) --
	( 65.54, 82.88) --
	( 65.59, 79.22) --
	( 65.64, 77.36) --
	( 65.69, 80.70) --
	( 65.74, 80.25) --
	( 65.79, 79.54) --
	( 65.84, 81.71) --
	( 65.90, 80.82) --
	( 65.95, 82.37) --
	( 66.00, 82.07) --
	( 66.05, 80.66) --
	( 66.10, 81.69) --
	( 66.15, 81.02) --
	( 66.20, 80.88) --
	( 66.25, 81.65) --
	( 66.30, 80.75) --
	( 66.36, 80.79) --
	( 66.41, 81.29) --
	( 66.46, 84.21) --
	( 66.51, 81.40) --
	( 66.56, 80.73) --
	( 66.61, 83.56) --
	( 66.66, 84.05) --
	( 66.71, 81.60) --
	( 66.77, 80.68) --
	( 66.82, 81.65) --
	( 66.87, 80.32) --
	( 66.92, 81.67) --
	( 66.97, 81.31) --
	( 67.02, 80.30) --
	( 67.07, 81.06) --
	( 67.12, 81.60) --
	( 67.18, 81.00) --
	( 67.23, 82.91) --
	( 67.28, 78.77) --
	( 67.33, 80.82) --
	( 67.38, 81.85) --
	( 67.43, 82.30) --
	( 67.48, 81.83) --
	( 67.53, 82.32) --
	( 67.58, 79.65) --
	( 67.64, 81.00) --
	( 67.69, 83.13) --
	( 67.74, 81.49) --
	( 67.79, 81.45) --
	( 67.84, 82.64) --
	( 67.89, 81.15) --
	( 67.94, 81.85) --
	( 67.99, 81.15) --
	( 68.05, 83.04) --
	( 68.10, 82.10) --
	( 68.15, 81.80) --
	( 68.20, 80.37) --
	( 68.25, 81.33) --
	( 68.30, 83.89) --
	( 68.35, 81.53) --
	( 68.40, 85.35) --
	( 68.46, 80.91) --
	( 68.51, 82.19) --
	( 68.56, 81.58) --
	( 68.61, 79.72) --
	( 68.66, 82.19) --
	( 68.71, 83.15) --
	( 68.76, 80.64) --
	( 68.81, 84.50) --
	( 68.86, 82.84) --
	( 68.92, 82.21) --
	( 68.97, 83.42) --
	( 69.02, 85.04) --
	( 69.07, 84.25) --
	( 69.12, 84.39) --
	( 69.17, 84.30) --
	( 69.22, 81.33) --
	( 69.27, 82.95) --
	( 69.33, 83.17) --
	( 69.38, 83.38) --
	( 69.43, 82.41) --
	( 69.48, 84.61) --
	( 69.53, 82.03) --
	( 69.58, 85.38) --
	( 69.63, 84.25) --
	( 69.68, 81.29) --
	( 69.74, 83.53) --
	( 69.79, 83.44) --
	( 69.84, 83.87) --
	( 69.89, 82.43) --
	( 69.94, 83.89) --
	( 69.99, 82.50) --
	( 70.04, 84.57) --
	( 70.09, 84.57) --
	( 70.15, 81.60) --
	( 70.20, 84.10) --
	( 70.25, 86.61) --
	( 70.30, 84.79) --
	( 70.35, 82.59) --
	( 70.40, 80.59) --
	( 70.45, 86.54) --
	( 70.50, 85.11) --
	( 70.55, 82.79) --
	( 70.61, 84.77) --
	( 70.66, 85.49) --
	( 70.71, 83.89) --
	( 70.76, 86.21) --
	( 70.81, 83.92) --
	( 70.86, 86.61) --
	( 70.91, 86.34) --
	( 70.96, 85.85) --
	( 71.02, 84.63) --
	( 71.07, 86.18) --
	( 71.12, 83.13) --
	( 71.17, 85.65) --
	( 71.22, 86.16) --
	( 71.27, 82.41) --
	( 71.32, 87.89) --
	( 71.37, 84.19) --
	( 71.43, 85.40) --
	( 71.48, 84.52) --
	( 71.53, 84.63) --
	( 71.58, 83.74) --
	( 71.63, 85.47) --
	( 71.68, 83.89) --
	( 71.73, 84.19) --
	( 71.78, 88.88) --
	( 71.83, 83.83) --
	( 71.89, 86.88) --
	( 71.94, 85.56) --
	( 71.99, 88.00) --
	( 72.04, 85.78) --
	( 72.09, 84.75) --
	( 72.14, 84.72) --
	( 72.19, 86.75) --
	( 72.24, 85.65) --
	( 72.30, 84.25) --
	( 72.35, 85.04) --
	( 72.40, 86.34) --
	( 72.45, 83.67) --
	( 72.50, 83.49) --
	( 72.55, 85.56) --
	( 72.60, 85.85) --
	( 72.65, 84.61) --
	( 72.71, 85.02) --
	( 72.76, 82.50) --
	( 72.81, 86.97) --
	( 72.86, 82.14) --
	( 72.91, 85.94) --
	( 72.96, 84.93) --
	( 73.01, 87.13) --
	( 73.06, 83.78) --
	( 73.11, 86.70) --
	( 73.17, 85.80) --
	( 73.22, 86.14) --
	( 73.27, 84.99) --
	( 73.32, 85.60) --
	( 73.37, 83.04) --
	( 73.42, 85.74) --
	( 73.47, 85.31) --
	( 73.52, 84.54) --
	( 73.58, 84.07) --
	( 73.63, 86.50) --
	( 73.68, 85.80) --
	( 73.73, 87.91) --
	( 73.78, 83.78) --
	( 73.83, 85.02) --
	( 73.88, 85.87) --
	( 73.93, 86.36) --
	( 73.99, 85.42) --
	( 74.04, 85.60) --
	( 74.09, 86.45) --
	( 74.14, 85.08) --
	( 74.19, 86.50) --
	( 74.24, 85.69) --
	( 74.29, 82.91) --
	( 74.34, 85.96) --
	( 74.39, 89.51) --
	( 74.45, 83.62) --
	( 74.50, 87.08) --
	( 74.55, 86.45) --
	( 74.60, 88.16) --
	( 74.65, 89.10) --
	( 74.70, 90.41) --
	( 74.75, 89.15) --
	( 74.80, 86.36) --
	( 74.86, 86.72) --
	( 74.91, 87.08) --
	( 74.96, 89.87) --
	( 75.01, 88.16) --
	( 75.06, 90.41) --
	( 75.11, 89.01) --
	( 75.16, 88.88) --
	( 75.21, 91.22) --
	( 75.27, 90.45) --
	( 75.32, 90.18) --
	( 75.37, 93.42) --
	( 75.42, 89.64) --
	( 75.47, 92.61) --
	( 75.52, 90.14) --
	( 75.57, 91.58) --
	( 75.62, 87.85) --
	( 75.67, 92.92) --
	( 75.73, 90.41) --
	( 75.78, 93.19) --
	( 75.83, 87.58) --
	( 75.88, 90.18) --
	( 75.93, 90.05) --
	( 75.98, 89.37) --
	( 76.03, 90.32) --
	( 76.08, 90.45) --
	( 76.14, 89.46) --
	( 76.19, 92.74) --
	( 76.24, 90.95) --
	( 76.29, 91.76) --
	( 76.34, 90.90) --
	( 76.39, 89.51) --
	( 76.44, 87.62) --
	( 76.49, 91.67) --
	( 76.55, 90.45) --
	( 76.60, 88.79) --
	( 76.65, 90.54) --
	( 76.70, 91.35) --
	( 76.75, 89.33) --
	( 76.80, 91.53) --
	( 76.85, 86.81) --
	( 76.90, 91.84) --
	( 76.95, 86.59) --
	( 77.01, 89.64) --
	( 77.06, 86.00) --
	( 77.11, 90.77) --
	( 77.16, 88.52) --
	( 77.21, 91.76) --
	( 77.26, 87.62) --
	( 77.31, 88.66) --
	( 77.36, 87.67) --
	( 77.42, 87.85) --
	( 77.47, 87.67) --
	( 77.52, 90.14) --
	( 77.57, 89.06) --
	( 77.62, 88.79) --
	( 77.67, 88.66) --
	( 77.72, 90.09) --
	( 77.77, 89.55) --
	( 77.83, 90.72) --
	( 77.88, 86.50) --
	( 77.93, 89.60) --
	( 77.98, 84.93) --
	( 78.03, 89.28) --
	( 78.08, 85.47) --
	( 78.13, 90.23) --
	( 78.18, 86.77) --
	( 78.23, 86.50) --
	( 78.29, 84.84) --
	( 78.34, 87.85) --
	( 78.39, 86.81) --
	( 78.44, 87.80) --
	( 78.49, 84.39) --
	( 78.54, 86.50) --
	( 78.59, 86.32) --
	( 78.64, 87.58) --
	( 78.70, 87.31) --
	( 78.75, 89.82) --
	( 78.80, 87.58) --
	( 78.85, 87.76) --
	( 78.90, 85.56) --
	( 78.95, 88.66) --
	( 79.00, 86.81) --
	( 79.05, 86.50) --
	( 79.11, 82.64) --
	( 79.16, 86.50) --
	( 79.21, 84.25) --
	( 79.26, 86.27) --
	( 79.31, 84.43) --
	( 79.36, 85.65) --
	( 79.41, 83.80) --
	( 79.46, 88.07) --
	( 79.51, 87.71) --
	( 79.57, 87.40) --
	( 79.62, 85.69) --
	( 79.67, 85.33) --
	( 79.72, 83.04) --
	( 79.77, 86.32) --
	( 79.82, 79.90) --
	( 79.87, 83.62) --
	( 79.92, 84.39) --
	( 79.98, 82.23) --
	( 80.03, 86.41) --
	( 80.08, 82.41) --
	( 80.13, 82.64) --
	( 80.18, 84.25) --
	( 80.23, 84.12) --
	( 80.28, 83.89) --
	( 80.33, 84.25) --
	( 80.39, 86.05) --
	( 80.44, 79.85) --
	( 80.49, 81.69) --
	( 80.54, 83.04) --
	( 80.59, 85.83) --
	( 80.64, 82.50) --
	( 80.69, 83.80) --
	( 80.74, 80.61) --
	( 80.79, 86.95) --
	( 80.85, 81.24) --
	( 80.90, 79.90) --
	( 80.95, 80.39) --
	( 81.00, 79.85) --
	( 81.05, 80.39) --
	( 81.10, 78.05) --
	( 81.15, 80.88) --
	( 81.20, 79.36) --
	( 81.26, 80.88) --
	( 81.31, 79.81) --
	( 81.36, 82.01) --
	( 81.41, 81.42) --
	( 81.46, 78.73) --
	( 81.51, 79.76) --
	( 81.56, 80.39) --
	( 81.61, 77.96) --
	( 81.67, 77.24) --
	( 81.72, 79.63) --
	( 81.77, 79.63) --
	( 81.82, 78.95) --
	( 81.87, 77.11) --
	( 81.92, 81.56) --
	( 81.97, 78.05) --
	( 82.02, 80.84) --
	( 82.07, 80.52) --
	( 82.13, 80.43) --
	( 82.18, 78.05) --
	( 82.23, 77.69) --
	( 82.28, 82.46) --
	( 82.33, 77.07) --
	( 82.38, 77.74) --
	( 82.43, 81.20) --
	( 82.48, 79.99) --
	( 82.54, 80.25) --
	( 82.59, 80.57) --
	( 82.64, 77.29) --
	( 82.69, 80.07) --
	( 82.74, 80.30) --
	( 82.79, 80.39) --
	( 82.84, 79.09) --
	( 82.89, 74.82) --
	( 82.95, 81.38) --
	( 83.00, 78.32) --
	( 83.05, 78.28) --
	( 83.10, 81.96) --
	( 83.15, 78.77) --
	( 83.20, 78.73) --
	( 83.25, 83.58) --
	( 83.30, 78.37) --
	( 83.35, 80.52) --
	( 83.41, 79.09) --
	( 83.46, 78.55) --
	( 83.51, 79.18) --
	( 83.56, 79.00) --
	( 83.61, 80.66) --
	( 83.66, 76.26) --
	( 83.71, 78.41) --
	( 83.76, 76.84) --
	( 83.82, 78.28) --
	( 83.87, 78.28) --
	( 83.92, 77.83) --
	( 83.97, 79.49) --
	( 84.02, 77.56) --
	( 84.07, 76.66) --
	( 84.12, 79.00) --
	( 84.17, 79.54) --
	( 84.23, 79.27) --
	( 84.28, 76.71) --
	( 84.33, 79.18) --
	( 84.38, 77.87) --
	( 84.43, 79.04) --
	( 84.48, 74.95) --
	( 84.53, 81.15) --
	( 84.58, 75.58) --
	( 84.63, 82.64) --
	( 84.69, 81.47) --
	( 84.74, 77.33) --
	( 84.79, 78.10) --
	( 84.84, 80.03) --
	( 84.89, 78.01) --
	( 84.94, 79.85) --
	( 84.99, 79.81) --
	( 85.04, 78.91) --
	( 85.10, 77.65) --
	( 85.15, 78.19) --
	( 85.20, 77.83) --
	( 85.25, 78.82) --
	( 85.30, 79.22) --
	( 85.35, 79.72) --
	( 85.40, 80.39) --
	( 85.45, 80.61) --
	( 85.51, 80.75) --
	( 85.56, 83.13) --
	( 85.61, 83.26) --
	( 85.66, 77.92) --
	( 85.71, 78.86) --
	( 85.76, 80.16) --
	( 85.81, 81.60) --
	( 85.86, 78.14) --
	( 85.91, 81.11) --
	( 85.97, 78.37) --
	( 86.02, 80.03) --
	( 86.07, 78.55) --
	( 86.12, 77.47) --
	( 86.17, 78.68) --
	( 86.22, 79.49) --
	( 86.27, 78.46) --
	( 86.32, 81.24) --
	( 86.38, 78.19) --
	( 86.43, 82.50) --
	( 86.48, 78.46) --
	( 86.53, 79.13) --
	( 86.58, 78.91) --
	( 86.63, 79.27) --
	( 86.68, 77.87) --
	( 86.73, 76.98) --
	( 86.79, 79.36) --
	( 86.84, 80.12) --
	( 86.89, 80.61) --
	( 86.94, 79.31) --
	( 86.99, 80.88) --
	( 87.04, 79.76) --
	( 87.09, 76.48) --
	( 87.14, 76.26) --
	( 87.19, 80.66) --
	( 87.25, 76.75) --
	( 87.30, 80.16) --
	( 87.35, 77.69) --
	( 87.40, 80.61) --
	( 87.45, 77.11) --
	( 87.50, 82.50) --
	( 87.55, 78.41) --
	( 87.60, 78.59) --
	( 87.66, 80.75) --
	( 87.71, 80.93) --
	( 87.76, 79.49) --
	( 87.81, 79.13) --
	( 87.86, 75.04) --
	( 87.91, 80.30) --
	( 87.96, 81.20) --
	( 88.01, 81.11) --
	( 88.07, 76.39) --
	( 88.12, 80.70) --
	( 88.17, 80.03) --
	( 88.22, 79.36) --
	( 88.27, 77.83) --
	( 88.32, 78.14) --
	( 88.37, 77.47) --
	( 88.42, 79.67) --
	( 88.47, 78.01) --
	( 88.53, 79.67) --
	( 88.58, 77.56) --
	( 88.63, 82.46) --
	( 88.68, 77.65) --
	( 88.73, 79.81) --
	( 88.78, 76.44) --
	( 88.83, 78.64) --
	( 88.88, 78.28) --
	( 88.94, 76.84) --
	( 88.99, 81.83) --
	( 89.04, 78.05) --
	( 89.09, 74.73) --
	( 89.14, 79.58) --
	( 89.19, 79.27) --
	( 89.24, 78.46) --
	( 89.29, 82.82) --
	( 89.35, 78.73) --
	( 89.40, 80.70) --
	( 89.45, 78.10) --
	( 89.50, 81.02) --
	( 89.55, 78.91) --
	( 89.60, 79.49) --
	( 89.65, 75.58) --
	( 89.70, 84.07) --
	( 89.75, 76.75) --
	( 89.81, 79.63) --
	( 89.86, 76.84) --
	( 89.91, 81.56) --
	( 89.96, 78.05) --
	( 90.01, 83.22) --
	( 90.06, 78.46) --
	( 90.11, 79.40) --
	( 90.16, 77.15) --
	( 90.22, 79.54) --
	( 90.27, 78.91) --
	( 90.32, 76.89) --
	( 90.37, 79.31) --
	( 90.42, 78.14) --
	( 90.47, 78.05) --
	( 90.52, 79.76) --
	( 90.57, 79.22) --
	( 90.63, 79.04) --
	( 90.68, 78.55) --
	( 90.73, 77.11) --
	( 90.78, 75.09) --
	( 90.83, 79.76) --
	( 90.88, 80.16) --
	( 90.93, 79.90) --
	( 90.98, 81.65) --
	( 91.03, 80.34) --
	( 91.09, 76.71) --
	( 91.14, 77.24) --
	( 91.19, 78.32) --
	( 91.24, 78.01) --
	( 91.29, 79.22) --
	( 91.34, 77.33) --
	( 91.39, 77.24) --
	( 91.44, 80.97) --
	( 91.50, 80.48) --
	( 91.55, 79.00) --
	( 91.60, 76.03) --
	( 91.65, 77.07) --
	( 91.70, 78.95) --
	( 91.75, 77.51) --
	( 91.80, 76.08) --
	( 91.85, 80.12) --
	( 91.91, 78.77) --
	( 91.96, 79.99) --
	( 92.01, 80.79) --
	( 92.06, 76.98) --
	( 92.11, 77.51) --
	( 92.16, 79.94) --
	( 92.21, 78.32) --
	( 92.26, 78.32) --
	( 92.31, 77.33) --
	( 92.37, 77.78) --
	( 92.42, 77.78) --
	( 92.47, 79.76) --
	( 92.52, 80.39) --
	( 92.57, 78.19) --
	( 92.62, 79.99) --
	( 92.67, 75.85) --
	( 92.72, 78.95) --
	( 92.78, 76.08) --
	( 92.83, 78.86) --
	( 92.88, 77.38) --
	( 92.93, 78.01) --
	( 92.98, 78.23) --
	( 93.03, 80.48) --
	( 93.08, 77.83) --
	( 93.13, 81.29) --
	( 93.19, 77.65) --
	( 93.24, 79.45) --
	( 93.29, 77.92) --
	( 93.34, 80.12) --
	( 93.39, 77.87) --
	( 93.44, 80.39) --
	( 93.49, 79.67) --
	( 93.54, 80.84) --
	( 93.59, 80.52) --
	( 93.65, 80.61) --
	( 93.70, 78.46) --
	( 93.75, 80.61) --
	( 93.80, 78.77) --
	( 93.85, 76.39) --
	( 93.90, 78.50) --
	( 93.95, 77.24) --
	( 94.00, 79.49) --
	( 94.06, 78.23) --
	( 94.11, 77.65) --
	( 94.16, 78.32) --
	( 94.21, 78.86) --
	( 94.26, 79.27) --
	( 94.31, 82.19) --
	( 94.36, 80.16) --
	( 94.41, 81.24) --
	( 94.47, 75.90) --
	( 94.52, 76.53) --
	( 94.57, 76.98) --
	( 94.62, 76.30) --
	( 94.67, 78.55) --
	( 94.72, 78.41) --
	( 94.77, 79.00) --
	( 94.82, 79.99) --
	( 94.87, 78.73) --
	( 94.93, 79.22) --
	( 94.98, 78.73) --
	( 95.03, 76.66) --
	( 95.08, 79.67) --
	( 95.13, 77.29) --
	( 95.18, 78.14) --
	( 95.23, 76.66) --
	( 95.28, 78.10) --
	( 95.34, 77.11) --
	( 95.39, 81.15) --
	( 95.44, 78.37) --
	( 95.49, 80.61) --
	( 95.54, 77.51) --
	( 95.59, 79.94) --
	( 95.64, 78.19) --
	( 95.69, 78.86) --
	( 95.75, 76.30) --
	( 95.80, 79.85) --
	( 95.85, 80.25) --
	( 95.90, 80.12) --
	( 95.95, 78.77) --
	( 96.00, 79.40) --
	( 96.05, 77.78) --
	( 96.10, 77.42) --
	( 96.16, 77.29) --
	( 96.21, 79.49) --
	( 96.26, 76.17) --
	( 96.31, 79.94) --
	( 96.36, 80.88) --
	( 96.41, 75.49) --
	( 96.46, 80.21) --
	( 96.51, 80.25) --
	( 96.56, 78.77) --
	( 96.62, 75.72) --
	( 96.67, 80.43) --
	( 96.72, 78.10) --
	( 96.77, 80.79) --
	( 96.82, 78.86) --
	( 96.87, 78.55) --
	( 96.92, 76.30) --
	( 96.97, 81.60) --
	( 97.03, 79.54) --
	( 97.08, 79.49) --
	( 97.13, 77.20) --
	( 97.18, 76.35) --
	( 97.23, 76.48) --
	( 97.28, 77.20) --
	( 97.33, 75.40) --
	( 97.38, 77.42) --
	( 97.44, 79.85) --
	( 97.49, 80.03) --
	( 97.54, 78.64) --
	( 97.59, 79.58) --
	( 97.64, 81.38) --
	( 97.69, 76.89) --
	( 97.74, 76.53) --
	( 97.79, 77.20) --
	( 97.84, 77.42) --
	( 97.90, 80.12) --
	( 97.95, 80.52) --
	( 98.00, 77.47) --
	( 98.05, 78.05) --
	( 98.10, 79.45) --
	( 98.15, 77.07) --
	( 98.20, 78.01) --
	( 98.25, 78.59) --
	( 98.31, 75.54) --
	( 98.36, 80.79) --
	( 98.41, 78.37) --
	( 98.46, 78.37) --
	( 98.51, 76.75) --
	( 98.56, 77.83) --
	( 98.61, 78.05) --
	( 98.66, 81.02) --
	( 98.72, 78.37) --
	( 98.77, 78.10) --
	( 98.82, 77.96) --
	( 98.87, 75.99) --
	( 98.92, 78.77) --
	( 98.97, 80.25) --
	( 99.02, 77.20) --
	( 99.07, 77.47) --
	( 99.12, 78.50) --
	( 99.18, 79.45) --
	( 99.23, 77.20) --
	( 99.28, 77.83) --
	( 99.33, 77.29) --
	( 99.38, 77.87) --
	( 99.43, 77.42) --
	( 99.48, 75.90) --
	( 99.53, 79.45) --
	( 99.59, 76.03) --
	( 99.64, 77.92) --
	( 99.69, 75.54) --
	( 99.74, 78.95) --
	( 99.79, 75.54) --
	( 99.84, 78.37) --
	( 99.89, 76.80) --
	( 99.94, 77.69) --
	(100.00, 76.12) --
	(100.05, 77.02) --
	(100.10, 75.81) --
	(100.15, 79.90) --
	(100.20, 76.89) --
	(100.25, 77.51) --
	(100.30, 76.35) --
	(100.35, 79.76) --
	(100.40, 78.14) --
	(100.46, 79.27) --
	(100.51, 78.95) --
	(100.56, 78.64) --
	(100.61, 74.46) --
	(100.66, 79.00) --
	(100.71, 79.18) --
	(100.76, 74.64) --
	(100.81, 79.99) --
	(100.87, 80.21) --
	(100.92, 77.33) --
	(100.97, 77.69) --
	(101.02, 73.97) --
	(101.07, 75.09) --
	(101.12, 78.32) --
	(101.17, 77.56) --
	(101.22, 78.23) --
	(101.28, 76.26) --
	(101.33, 78.68) --
	(101.38, 78.46) --
	(101.43, 77.42) --
	(101.48, 78.23) --
	(101.53, 77.11) --
	(101.58, 78.37) --
	(101.63, 77.33) --
	(101.68, 77.65) --
	(101.74, 80.93) --
	(101.79, 82.55) --
	(101.84, 74.91) --
	(101.89, 78.82) --
	(101.94, 75.72) --
	(101.99, 77.24) --
	(102.04, 79.18) --
	(102.09, 75.67) --
	(102.15, 79.94) --
	(102.20, 76.12) --
	(102.25, 78.55) --
	(102.30, 76.66) --
	(102.35, 76.30) --
	(102.40, 78.73) --
	(102.45, 78.95) --
	(102.50, 78.23) --
	(102.56, 80.03) --
	(102.61, 76.66) --
	(102.66, 78.50) --
	(102.71, 75.94) --
	(102.76, 76.44) --
	(102.81, 76.84) --
	(102.86, 79.04) --
	(102.91, 75.58) --
	(102.96, 82.23) --
	(103.02, 77.29) --
	(103.07, 75.27) --
	(103.12, 76.57) --
	(103.17, 78.46) --
	(103.22, 76.39) --
	(103.27, 78.91) --
	(103.32, 76.71) --
	(103.37, 82.28) --
	(103.43, 74.86) --
	(103.48, 80.93) --
	(103.53, 78.32) --
	(103.58, 81.06) --
	(103.63, 76.26) --
	(103.68, 78.23) --
	(103.73, 75.76) --
	(103.78, 78.77) --
	(103.84, 77.74) --
	(103.89, 78.86) --
	(103.94, 77.24) --
	(103.99, 80.12) --
	(104.04, 77.07) --
	(104.09, 77.92) --
	(104.14, 79.40) --
	(104.19, 78.23) --
	(104.24, 79.45) --
	(104.30, 78.37) --
	(104.35, 78.10) --
	(104.40, 77.96) --
	(104.45, 79.31) --
	(104.50, 78.86) --
	(104.55, 74.50) --
	(104.60, 77.83) --
	(104.65, 76.12) --
	(104.71, 77.96) --
	(104.76, 78.59) --
	(104.81, 77.47) --
	(104.86, 76.35) --
	(104.91, 76.48) --
	(104.96, 74.64) --
	(105.01, 79.76) --
	(105.06, 78.55) --
	(105.12, 75.22) --
	(105.17, 77.69) --
	(105.22, 76.93) --
	(105.27, 78.82) --
	(105.32, 78.46) --
	(105.37, 76.66) --
	(105.42, 80.57) --
	(105.47, 82.10) --
	(105.52, 75.94) --
	(105.58, 77.07) --
	(105.63, 80.21) --
	(105.68, 79.72) --
	(105.73, 78.32) --
	(105.78, 75.27) --
	(105.83, 78.05) --
	(105.88, 77.87) --
	(105.93, 78.95) --
	(105.99, 78.73) --
	(106.04, 76.08) --
	(106.09, 80.79) --
	(106.14, 81.02) --
	(106.19, 75.99) --
	(106.24, 75.45) --
	(106.29, 77.87) --
	(106.34, 77.56) --
	(106.40, 78.46) --
	(106.45, 76.44) --
	(106.50, 77.51) --
	(106.55, 78.28) --
	(106.60, 76.39) --
	(106.65, 77.83) --
	(106.70, 77.92) --
	(106.75, 76.53) --
	(106.80, 78.68) --
	(106.86, 76.08) --
	(106.91, 79.58) --
	(106.96, 78.86) --
	(107.01, 76.93) --
	(107.06, 76.57) --
	(107.11, 76.98) --
	(107.16, 77.83) --
	(107.21, 79.72) --
	(107.27, 76.93) --
	(107.32, 80.84) --
	(107.37, 78.46) --
	(107.42, 79.49) --
	(107.47, 78.41) --
	(107.52, 80.57) --
	(107.57, 77.11) --
	(107.62, 74.41) --
	(107.68, 75.45) --
	(107.73, 76.12) --
	(107.78, 75.54) --
	(107.83, 76.71) --
	(107.88, 78.91) --
	(107.93, 75.76) --
	(107.98, 76.12) --
	(108.03, 74.37) --
	(108.08, 79.09) --
	(108.14, 75.67) --
	(108.19, 77.47) --
	(108.24, 76.12) --
	(108.29, 74.32) --
	(108.34, 77.11) --
	(108.39, 75.94) --
	(108.44, 79.54) --
	(108.49, 76.26) --
	(108.55, 75.54) --
	(108.60, 76.62) --
	(108.65, 77.42) --
	(108.70, 79.94) --
	(108.75, 76.03) --
	(108.80, 76.44) --
	(108.85, 79.27) --
	(108.90, 77.47) --
	(108.96, 78.19) --
	(109.01, 81.87) --
	(109.06, 79.67) --
	(109.11, 79.31) --
	(109.16, 76.48) --
	(109.21, 75.27) --
	(109.26, 73.38) --
	(109.31, 77.74) --
	(109.36, 75.72) --
	(109.42, 77.92) --
	(109.47, 80.75) --
	(109.52, 78.19) --
	(109.57, 75.81) --
	(109.62, 74.77) --
	(109.67, 77.51) --
	(109.72, 76.89) --
	(109.77, 77.87) --
	(109.83, 80.61) --
	(109.88, 78.10) --
	(109.93, 81.47) --
	(109.98, 77.33) --
	(110.03, 77.20) --
	(110.08, 76.75) --
	(110.13, 76.12) --
	(110.18, 76.30) --
	(110.24, 77.07) --
	(110.29, 75.67) --
	(110.34, 76.44) --
	(110.39, 78.59) --
	(110.44, 75.99) --
	(110.49, 79.99) --
	(110.54, 77.42) --
	(110.59, 76.84) --
	(110.64, 74.73) --
	(110.70, 81.42) --
	(110.75, 73.56) --
	(110.80, 79.45) --
	(110.85, 75.63) --
	(110.90, 79.00) --
	(110.95, 78.41) --
	(111.00, 77.96) --
	(111.05, 76.12) --
	(111.11, 75.94) --
	(111.16, 77.65) --
	(111.21, 76.71) --
	(111.26, 77.33) --
	(111.31, 78.86) --
	(111.36, 78.05) --
	(111.41, 77.47) --
	(111.46, 76.62) --
	(111.52, 73.97) --
	(111.57, 75.04) --
	(111.62, 78.77) --
	(111.67, 77.11) --
	(111.72, 76.89) --
	(111.77, 74.32) --
	(111.82, 77.96) --
	(111.87, 75.00) --
	(111.92, 76.80) --
	(111.98, 81.83) --
	(112.03, 76.30) --
	(112.08, 76.48) --
	(112.13, 78.59) --
	(112.18, 76.75) --
	(112.23, 77.33) --
	(112.28, 78.32) --
	(112.33, 80.21) --
	(112.39, 76.53) --
	(112.44, 77.02) --
	(112.49, 78.41) --
	(112.54, 73.79) --
	(112.59, 74.01) --
	(112.64, 79.27) --
	(112.69, 77.51) --
	(112.74, 78.55) --
	(112.80, 78.41) --
	(112.85, 73.97) --
	(112.90, 74.37) --
	(112.95, 74.37) --
	(113.00, 78.05) --
	(113.05, 73.02) --
	(113.10, 78.41) --
	(113.15, 77.74) --
	(113.20, 76.21) --
	(113.26, 77.78) --
	(113.31, 78.73) --
	(113.36, 76.21) --
	(113.41, 76.12) --
	(113.46, 78.86) --
	(113.51, 80.16) --
	(113.56, 76.12) --
	(113.61, 77.42) --
	(113.67, 73.83) --
	(113.72, 76.48) --
	(113.77, 79.40) --
	(113.82, 75.09) --
	(113.87, 76.35) --
	(113.92, 77.33) --
	(113.97, 77.20) --
	(114.02, 74.64) --
	(114.08, 76.08) --
	(114.13, 75.22) --
	(114.18, 75.40) --
	(114.23, 78.73) --
	(114.28, 75.63) --
	(114.33, 76.08) --
	(114.38, 74.73) --
	(114.43, 77.60) --
	(114.48, 78.19) --
	(114.54, 74.23) --
	(114.59, 74.41) --
	(114.64, 77.83) --
	(114.69, 76.39) --
	(114.74, 78.14) --
	(114.79, 74.91) --
	(114.84, 75.81) --
	(114.89, 75.00) --
	(114.95, 78.46) --
	(115.00, 73.34) --
	(115.05, 78.82) --
	(115.10, 73.88) --
	(115.15, 77.51) --
	(115.20, 74.68) --
	(115.25, 75.90) --
	(115.30, 80.75) --
	(115.36, 75.36) --
	(115.41, 77.33) --
	(115.46, 76.48) --
	(115.51, 76.26) --
	(115.56, 75.90) --
	(115.61, 72.84) --
	(115.66, 78.14) --
	(115.71, 74.41) --
	(115.76, 77.74) --
	(115.82, 75.85) --
	(115.87, 74.19) --
	(115.92, 76.71) --
	(115.97, 76.17) --
	(116.02, 74.06) --
	(116.07, 78.14) --
	(116.12, 76.80) --
	(116.17, 77.69) --
	(116.23, 77.20) --
	(116.28, 79.67) --
	(116.33, 76.53) --
	(116.38, 78.73) --
	(116.43, 76.66) --
	(116.48, 75.09) --
	(116.53, 75.72) --
	(116.58, 74.19) --
	(116.64, 78.95) --
	(116.69, 76.53) --
	(116.74, 77.96) --
	(116.79, 75.81) --
	(116.84, 80.52) --
	(116.89, 76.17) --
	(116.94, 79.76) --
	(116.99, 77.96) --
	(117.04, 77.29) --
	(117.10, 76.39) --
	(117.15, 74.32) --
	(117.20, 77.20) --
	(117.25, 75.13) --
	(117.30, 76.89) --
	(117.35, 79.00) --
	(117.40, 77.87) --
	(117.45, 77.15) --
	(117.51, 77.87) --
	(117.56, 78.14) --
	(117.61, 77.60) --
	(117.66, 74.86) --
	(117.71, 74.28) --
	(117.76, 77.92) --
	(117.81, 75.94) --
	(117.86, 73.92) --
	(117.92, 72.98) --
	(117.97, 73.79) --
	(118.02, 77.47) --
	(118.07, 76.57) --
	(118.12, 77.60) --
	(118.17, 76.89) --
	(118.22, 76.17) --
	(118.27, 75.45) --
	(118.32, 76.93) --
	(118.38, 76.21) --
	(118.43, 78.82) --
	(118.48, 74.50) --
	(118.53, 73.38) --
	(118.58, 77.42) --
	(118.63, 76.12) --
	(118.68, 78.86) --
	(118.73, 72.12) --
	(118.79, 76.57) --
	(118.84, 75.81) --
	(118.89, 78.41) --
	(118.94, 76.75) --
	(118.99, 77.83) --
	(119.04, 76.21) --
	(119.09, 77.65) --
	(119.14, 75.22) --
	(119.20, 78.68) --
	(119.25, 75.09) --
	(119.30, 80.21) --
	(119.35, 77.87) --
	(119.40, 76.98) --
	(119.45, 76.71) --
	(119.50, 75.04) --
	(119.55, 78.55) --
	(119.60, 75.27) --
	(119.66, 76.53) --
	(119.71, 76.48) --
	(119.76, 74.55) --
	(119.81, 79.22) --
	(119.86, 76.75) --
	(119.91, 75.67) --
	(119.96, 78.41) --
	(120.01, 74.01) --
	(120.07, 75.67) --
	(120.12, 77.87) --
	(120.17, 73.47) --
	(120.22, 75.72) --
	(120.27, 75.72) --
	(120.32, 73.07) --
	(120.37, 78.73) --
	(120.42, 73.61) --
	(120.48, 77.47) --
	(120.53, 74.86) --
	(120.58, 75.99) --
	(120.63, 77.07) --
	(120.68, 75.49) --
	(120.73, 77.96) --
	(120.78, 79.18) --
	(120.83, 78.86) --
	(120.88, 76.62) --
	(120.94, 77.24) --
	(120.99, 76.12) --
	(121.04, 78.91) --
	(121.09, 72.66) --
	(121.14, 75.99) --
	(121.19, 76.26) --
	(121.24, 77.02) --
	(121.29, 76.71) --
	(121.35, 73.92) --
	(121.40, 79.40) --
	(121.45, 78.10) --
	(121.50, 76.08) --
	(121.55, 75.72) --
	(121.60, 77.69) --
	(121.65, 77.51) --
	(121.70, 75.31) --
	(121.76, 74.14) --
	(121.81, 74.46) --
	(121.86, 74.77) --
	(121.91, 78.41) --
	(121.96, 71.45) --
	(122.01, 77.83) --
	(122.06, 73.83) --
	(122.11, 77.07) --
	(122.17, 77.38) --
	(122.22, 74.95) --
	(122.27, 77.02) --
	(122.32, 76.35) --
	(122.37, 79.09) --
	(122.42, 76.44) --
	(122.47, 78.32) --
	(122.52, 80.16) --
	(122.57, 75.45) --
	(122.63, 78.77) --
	(122.68, 73.43) --
	(122.73, 76.98) --
	(122.78, 77.02) --
	(122.83, 78.01) --
	(122.88, 76.84) --
	(122.93, 78.91) --
	(122.98, 75.81) --
	(123.04, 79.04) --
	(123.09, 75.04) --
	(123.14, 79.54) --
	(123.19, 76.30) --
	(123.24, 75.81) --
	(123.29, 73.34) --
	(123.34, 73.38) --
	(123.39, 75.27) --
	(123.45, 75.09) --
	(123.50, 73.02) --
	(123.55, 77.29) --
	(123.60, 73.92) --
	(123.65, 77.15) --
	(123.70, 77.92) --
	(123.75, 78.37) --
	(123.80, 74.28) --
	(123.85, 77.07) --
	(123.91, 74.01) --
	(123.96, 75.49) --
	(124.01, 78.64) --
	(124.06, 79.18) --
	(124.11, 74.10) --
	(124.16, 76.66) --
	(124.21, 77.15) --
	(124.26, 76.17) --
	(124.32, 72.66) --
	(124.37, 79.27) --
	(124.42, 74.41) --
	(124.47, 75.27) --
	(124.52, 75.31) --
	(124.57, 75.99) --
	(124.62, 78.73) --
	(124.67, 79.31) --
	(124.73, 75.45) --
	(124.78, 76.44) --
	(124.83, 75.63) --
	(124.88, 76.39) --
	(124.93, 72.53) --
	(124.98, 76.84) --
	(125.03, 75.99) --
	(125.08, 77.74) --
	(125.13, 73.74) --
	(125.19, 77.02) --
	(125.24, 79.09) --
	(125.29, 75.63) --
	(125.34, 72.80) --
	(125.39, 75.63) --
	(125.44, 74.19) --
	(125.49, 78.86) --
	(125.54, 75.81) --
	(125.60, 74.91) --
	(125.65, 77.42) --
	(125.70, 75.58) --
	(125.75, 75.09) --
	(125.80, 76.84) --
	(125.85, 73.11) --
	(125.90, 76.26) --
	(125.95, 74.06) --
	(126.01, 74.77) --
	(126.06, 75.58) --
	(126.11, 76.30) --
	(126.16, 74.68) --
	(126.21, 75.18) --
	(126.26, 79.63) --
	(126.31, 77.29) --
	(126.36, 73.16) --
	(126.41, 76.08) --
	(126.47, 76.21) --
	(126.52, 75.31) --
	(126.57, 78.28) --
	(126.62, 77.29) --
	(126.67, 78.10) --
	(126.72, 75.81) --
	(126.77, 75.45) --
	(126.82, 74.77) --
	(126.88, 78.37) --
	(126.93, 74.82) --
	(126.98, 76.26) --
	(127.03, 76.21) --
	(127.08, 74.64) --
	(127.13, 76.62) --
	(127.18, 75.40) --
	(127.23, 76.44) --
	(127.29, 73.79) --
	(127.34, 78.28) --
	(127.39, 76.98) --
	(127.44, 81.38) --
	(127.49, 76.03) --
	(127.54, 77.42) --
	(127.59, 74.19) --
	(127.64, 79.27) --
	(127.69, 75.67) --
	(127.75, 77.56) --
	(127.80, 71.36) --
	(127.85, 75.54) --
	(127.90, 75.22) --
	(127.95, 77.38) --
	(128.00, 77.20) --
	(128.05, 75.81) --
	(128.10, 76.53) --
	(128.16, 77.42) --
	(128.21, 75.13) --
	(128.26, 74.55) --
	(128.31, 76.21) --
	(128.36, 73.83) --
	(128.41, 77.29) --
	(128.46, 74.68) --
	(128.51, 75.85) --
	(128.57, 75.13) --
	(128.62, 77.69) --
	(128.67, 75.58) --
	(128.72, 77.02) --
	(128.77, 76.98) --
	(128.82, 75.27) --
	(128.87, 76.39) --
	(128.92, 74.10) --
	(128.97, 79.54) --
	(129.03, 74.28) --
	(129.08, 77.92) --
	(129.13, 81.96) --
	(129.18, 71.72) --
	(129.23, 74.59) --
	(129.28, 72.44) --
	(129.33, 73.20) --
	(129.38, 76.03) --
	(129.44, 75.58) --
	(129.49, 72.26) --
	(129.54, 77.51) --
	(129.59, 75.63) --
	(129.64, 78.46) --
	(129.69, 74.46) --
	(129.74, 77.38) --
	(129.79, 75.63) --
	(129.85, 79.99) --
	(129.90, 75.36) --
	(129.95, 76.21) --
	(130.00, 75.99) --
	(130.05, 76.75) --
	(130.10, 74.32) --
	(130.15, 73.88) --
	(130.20, 72.98) --
	(130.25, 76.75) --
	(130.31, 72.75) --
	(130.36, 74.77) --
	(130.41, 76.21) --
	(130.46, 76.30) --
	(130.51, 73.97) --
	(130.56, 78.77) --
	(130.61, 75.63) --
	(130.66, 75.85) --
	(130.72, 75.27) --
	(130.77, 72.57) --
	(130.82, 74.91) --
	(130.87, 73.92) --
	(130.92, 76.03) --
	(130.97, 76.03) --
	(131.02, 75.09) --
	(131.07, 75.09) --
	(131.13, 74.28) --
	(131.18, 74.06) --
	(131.23, 77.07) --
	(131.28, 75.63) --
	(131.33, 73.79) --
	(131.38, 76.08) --
	(131.43, 74.23) --
	(131.48, 71.05) --
	(131.53, 75.54) --
	(131.59, 77.96) --
	(131.64, 73.20) --
	(131.69, 74.28) --
	(131.74, 74.55) --
	(131.79, 74.14) --
	(131.84, 75.94) --
	(131.89, 74.19) --
	(131.94, 73.25) --
	(132.00, 75.36) --
	(132.05, 77.83) --
	(132.10, 75.58) --
	(132.15, 76.30) --
	(132.20, 74.41) --
	(132.25, 80.21) --
	(132.30, 76.66) --
	(132.35, 73.88) --
	(132.41, 78.01) --
	(132.46, 74.06) --
	(132.51, 77.02) --
	(132.56, 73.34) --
	(132.61, 76.03) --
	(132.66, 74.50) --
	(132.71, 76.35) --
	(132.76, 75.54) --
	(132.81, 74.91) --
	(132.87, 75.13) --
	(132.92, 75.04) --
	(132.97, 75.54) --
	(133.02, 73.74) --
	(133.07, 75.90) --
	(133.12, 76.93) --
	(133.17, 75.13) --
	(133.22, 71.40) --
	(133.28, 77.33) --
	(133.33, 76.12) --
	(133.38, 74.59) --
	(133.43, 76.89) --
	(133.48, 74.01) --
	(133.53, 75.18) --
	(133.58, 71.18) --
	(133.63, 78.28) --
	(133.69, 71.49) --
	(133.74, 76.35) --
	(133.79, 74.01) --
	(133.84, 77.47) --
	(133.89, 73.02) --
	(133.94, 77.47) --
	(133.99, 74.46) --
	(134.04, 72.62) --
	(134.09, 76.03) --
	(134.15, 77.56) --
	(134.20, 74.86) --
	(134.25, 76.03) --
	(134.30, 73.88) --
	(134.35, 73.25) --
	(134.40, 74.14) --
	(134.45, 73.43) --
	(134.50, 75.85) --
	(134.56, 75.76) --
	(134.61, 72.89) --
	(134.66, 74.32) --
	(134.71, 75.99) --
	(134.76, 76.66) --
	(134.81, 72.44) --
	(134.86, 72.93) --
	(134.91, 75.40) --
	(134.97, 76.03) --
	(135.02, 74.32) --
	(135.07, 77.83) --
	(135.12, 73.79) --
	(135.17, 77.74) --
	(135.22, 74.10) --
	(135.27, 73.97) --
	(135.32, 75.58) --
	(135.37, 75.49) --
	(135.43, 74.95) --
	(135.48, 72.35) --
	(135.53, 76.17) --
	(135.58, 73.92) --
	(135.63, 75.76) --
	(135.68, 76.03) --
	(135.73, 77.78) --
	(135.78, 74.14) --
	(135.84, 76.17) --
	(135.89, 73.34) --
	(135.94, 77.29) --
	(135.99, 71.63) --
	(136.04, 75.63) --
	(136.09, 72.35) --
	(136.14, 76.71) --
	(136.19, 73.74) --
	(136.25, 74.01) --
	(136.30, 76.44) --
	(136.35, 76.98) --
	(136.40, 73.25) --
	(136.45, 77.60) --
	(136.50, 76.93) --
	(136.55, 74.50) --
	(136.60, 74.41) --
	(136.65, 73.70) --
	(136.71, 75.31) --
	(136.76, 74.14) --
	(136.81, 74.23) --
	(136.86, 76.98) --
	(136.91, 74.41) --
	(136.96, 73.61) --
	(137.01, 75.40) --
	(137.06, 77.69) --
	(137.12, 76.53) --
	(137.17, 74.01) --
	(137.22, 76.57) --
	(137.27, 75.81) --
	(137.32, 77.65) --
	(137.37, 75.90) --
	(137.42, 73.52) --
	(137.47, 73.61) --
	(137.53, 76.30) --
	(137.58, 75.58) --
	(137.63, 78.23) --
	(137.68, 74.01) --
	(137.73, 72.80) --
	(137.78, 77.51) --
	(137.83, 71.72) --
	(137.88, 74.64) --
	(137.93, 72.62) --
	(137.99, 77.65) --
	(138.04, 70.15) --
	(138.09, 75.67) --
	(138.14, 76.08) --
	(138.19, 75.22) --
	(138.24, 72.71) --
	(138.29, 75.00) --
	(138.34, 77.33) --
	(138.40, 73.92) --
	(138.45, 71.99) --
	(138.50, 72.53) --
	(138.55, 76.03) --
	(138.60, 73.97) --
	(138.65, 75.72) --
	(138.70, 74.95) --
	(138.75, 73.16) --
	(138.81, 74.41) --
	(138.86, 72.84) --
	(138.91, 76.93) --
	(138.96, 75.36) --
	(139.01, 77.69) --
	(139.06, 77.02) --
	(139.11, 77.69) --
	(139.16, 74.64) --
	(139.21, 75.09) --
	(139.27, 74.86) --
	(139.32, 72.66) --
	(139.37, 78.95) --
	(139.42, 74.73) --
	(139.47, 77.60) --
	(139.52, 76.75) --
	(139.57, 76.89) --
	(139.62, 76.62) --
	(139.68, 73.70) --
	(139.73, 73.29) --
	(139.78, 74.64) --
	(139.83, 75.58) --
	(139.88, 75.67) --
	(139.93, 72.08) --
	(139.98, 74.86) --
	(140.03, 75.13) --
	(140.09, 76.71) --
	(140.14, 71.81) --
	(140.19, 75.54) --
	(140.24, 73.29) --
	(140.29, 76.17) --
	(140.34, 74.86) --
	(140.39, 74.10) --
	(140.44, 75.63) --
	(140.49, 74.01) --
	(140.55, 77.87) --
	(140.60, 76.26) --
	(140.65, 77.29) --
	(140.70, 75.09) --
	(140.75, 75.81) --
	(140.80, 77.07) --
	(140.85, 73.92) --
	(140.90, 76.66) --
	(140.96, 73.16) --
	(141.01, 77.15) --
	(141.06, 73.29) --
	(141.11, 78.59) --
	(141.16, 71.94) --
	(141.21, 76.35) --
	(141.26, 75.18) --
	(141.31, 74.95) --
	(141.37, 72.39) --
	(141.42, 73.02) --
	(141.47, 76.35) --
	(141.52, 74.41) --
	(141.57, 73.97) --
	(141.62, 75.09) --
	(141.67, 74.77) --
	(141.72, 74.55) --
	(141.77, 71.00) --
	(141.83, 74.86) --
	(141.88, 73.16) --
	(141.93, 73.97) --
	(141.98, 75.76) --
	(142.03, 73.47) --
	(142.08, 75.85) --
	(142.13, 76.53) --
	(142.18, 75.72) --
	(142.24, 74.19) --
	(142.29, 76.71) --
	(142.34, 75.67) --
	(142.39, 78.23) --
	(142.44, 75.67) --
	(142.49, 76.30) --
	(142.54, 77.24) --
	(142.59, 75.22) --
	(142.65, 74.01) --
	(142.70, 73.02) --
	(142.75, 76.93) --
	(142.80, 73.07) --
	(142.85, 75.81) --
	(142.90, 76.35) --
	(142.95, 73.79) --
	(143.00, 71.81) --
	(143.05, 75.58) --
	(143.11, 75.54) --
	(143.16, 76.44) --
	(143.21, 74.46) --
	(143.26, 75.00) --
	(143.31, 74.82) --
	(143.36, 72.26) --
	(143.41, 75.58) --
	(143.46, 75.00) --
	(143.52, 72.57) --
	(143.57, 73.88) --
	(143.62, 75.00) --
	(143.67, 75.40) --
	(143.72, 74.55) --
	(143.77, 76.17) --
	(143.82, 73.47) --
	(143.87, 73.34) --
	(143.93, 76.21) --
	(143.98, 73.56) --
	(144.03, 76.71) --
	(144.08, 72.03) --
	(144.13, 72.62) --
	(144.18, 74.59) --
	(144.23, 74.41) --
	(144.28, 75.22) --
	(144.33, 74.82) --
	(144.39, 75.31) --
	(144.44, 72.80) --
	(144.49, 73.25) --
	(144.54, 79.36) --
	(144.59, 74.10) --
	(144.64, 72.44) --
	(144.69, 74.37) --
	(144.74, 75.00) --
	(144.80, 75.27) --
	(144.85, 76.71) --
	(144.90, 73.52) --
	(144.95, 73.43) --
	(145.00, 78.19) --
	(145.05, 73.38) --
	(145.10, 76.30) --
	(145.15, 75.45) --
	(145.21, 76.53) --
	(145.26, 73.43) --
	(145.31, 75.40) --
	(145.36, 74.23) --
	(145.41, 75.45) --
	(145.46, 75.54) --
	(145.51, 72.44) --
	(145.56, 73.29) --
	(145.61, 72.62) --
	(145.67, 75.90) --
	(145.72, 73.29) --
	(145.77, 74.10) --
	(145.82, 75.22) --
	(145.87, 75.09) --
	(145.92, 74.77) --
	(145.97, 71.54) --
	(146.02, 72.75) --
	(146.08, 72.98) --
	(146.13, 74.50) --
	(146.18, 71.27) --
	(146.23, 76.98) --
	(146.28, 74.41) --
	(146.33, 70.91) --
	(146.38, 76.35) --
	(146.43, 74.23) --
	(146.49, 74.73) --
	(146.54, 75.45) --
	(146.59, 73.79) --
	(146.64, 73.74) --
	(146.69, 76.48) --
	(146.74, 73.34) --
	(146.79, 76.30) --
	(146.84, 72.75) --
	(146.89, 74.23) --
	(146.95, 74.91) --
	(147.00, 74.95) --
	(147.05, 74.95) --
	(147.10, 74.73) --
	(147.15, 73.74) --
	(147.20, 73.25) --
	(147.25, 77.11) --
	(147.30, 71.85) --
	(147.36, 74.28) --
	(147.41, 74.59) --
	(147.46, 75.45) --
	(147.51, 75.27) --
	(147.56, 73.47) --
	(147.61, 76.12) --
	(147.66, 74.77) --
	(147.71, 75.18) --
	(147.77, 70.87) --
	(147.82, 72.80) --
	(147.87, 70.69) --
	(147.92, 72.62) --
	(147.97, 75.49) --
	(148.02, 73.74) --
	(148.07, 76.26) --
	(148.12, 75.49) --
	(148.18, 74.08) --
	(148.23, 69.56) --
	(148.28, 74.50) --
	(148.33, 77.74) --
	(148.38, 76.26) --
	(148.43, 75.85) --
	(148.48, 75.49) --
	(148.53, 74.91) --
	(148.58, 74.59) --
	(148.64, 74.41) --
	(148.69, 74.14) --
	(148.74, 72.53) --
	(148.79, 78.05) --
	(148.84, 74.23) --
	(148.89, 75.99) --
	(148.94, 72.17) --
	(148.99, 76.21) --
	(149.05, 75.99) --
	(149.10, 72.80) --
	(149.15, 74.50) --
	(149.20, 75.49) --
	(149.25, 76.62) --
	(149.30, 71.81) --
	(149.35, 77.87) --
	(149.40, 73.92) --
	(149.46, 72.44) --
	(149.51, 73.74) --
	(149.56, 74.68) --
	(149.61, 72.03) --
	(149.66, 76.53) --
	(149.71, 72.26) --
	(149.76, 75.22) --
	(149.81, 78.50) --
	(149.86, 71.99) --
	(149.92, 77.51) --
	(149.97, 72.21) --
	(150.02, 77.15) --
	(150.07, 74.41) --
	(150.12, 75.49) --
	(150.17, 75.99) --
	(150.22, 74.32) --
	(150.27, 74.82) --
	(150.33, 74.37) --
	(150.38, 77.69) --
	(150.43, 70.78) --
	(150.48, 73.52) --
	(150.53, 75.81) --
	(150.58, 73.79) --
	(150.63, 72.62) --
	(150.68, 75.81) --
	(150.74, 74.37) --
	(150.79, 75.90) --
	(150.84, 74.10) --
	(150.89, 72.57) --
	(150.94, 73.29) --
	(150.99, 75.45) --
	(151.04, 73.79) --
	(151.09, 74.59) --
	(151.14, 72.30) --
	(151.20, 73.07) --
	(151.25, 74.28) --
	(151.30, 70.51) --
	(151.35, 75.94) --
	(151.40, 73.97) --
	(151.45, 75.54) --
	(151.50, 71.63) --
	(151.55, 74.91) --
	(151.61, 72.48) --
	(151.66, 77.96) --
	(151.71, 73.38) --
	(151.76, 73.38) --
	(151.81, 75.63) --
	(151.86, 77.02) --
	(151.91, 71.85) --
	(151.96, 76.17) --
	(152.02, 70.78) --
	(152.07, 73.07) --
	(152.12, 74.41) --
	(152.17, 73.52) --
	(152.22, 72.98) --
	(152.27, 72.57) --
	(152.32, 73.92) --
	(152.37, 72.44) --
	(152.42, 73.74) --
	(152.48, 73.02) --
	(152.53, 71.99) --
	(152.58, 72.75) --
	(152.63, 75.31) --
	(152.68, 76.93) --
	(152.73, 71.72) --
	(152.78, 74.77) --
	(152.83, 71.36) --
	(152.89, 73.70) --
	(152.94, 76.30) --
	(152.99, 72.71) --
	(153.04, 75.49) --
	(153.09, 77.38) --
	(153.14, 70.96) --
	(153.19, 75.40) --
	(153.24, 74.50) --
	(153.30, 74.86) --
	(153.35, 73.88) --
	(153.40, 72.89) --
	(153.45, 77.92) --
	(153.50, 72.08) --
	(153.55, 75.90) --
	(153.60, 68.39) --
	(153.65, 73.79) --
	(153.70, 71.63) --
	(153.76, 70.78) --
	(153.81, 73.20) --
	(153.86, 74.64) --
	(153.91, 74.46) --
	(153.96, 73.88) --
	(154.01, 70.15) --
	(154.06, 72.98) --
	(154.11, 73.25) --
	(154.17, 78.14) --
	(154.22, 71.18) --
	(154.27, 76.35) --
	(154.32, 72.62) --
	(154.37, 73.70) --
	(154.42, 76.84) --
	(154.47, 74.19) --
	(154.52, 72.89) --
	(154.58, 74.32) --
	(154.63, 72.93) --
	(154.68, 74.55) --
	(154.73, 75.09) --
	(154.78, 71.22) --
	(154.83, 72.98) --
	(154.88, 72.71) --
	(154.93, 75.81) --
	(154.98, 73.83) --
	(155.04, 71.49) --
	(155.09, 73.43) --
	(155.14, 74.82) --
	(155.19, 72.17) --
	(155.24, 71.36) --
	(155.29, 72.12) --
	(155.34, 78.28) --
	(155.39, 72.98) --
	(155.45, 71.54) --
	(155.50, 75.13) --
	(155.55, 73.34) --
	(155.60, 74.64) --
	(155.65, 70.15) --
	(155.70, 71.40) --
	(155.75, 72.03) --
	(155.80, 75.31) --
	(155.86, 70.51) --
	(155.91, 71.27) --
	(155.96, 69.83) --
	(156.01, 72.53) --
	(156.06, 76.89) --
	(156.11, 69.52) --
	(156.16, 78.59) --
	(156.21, 77.42) --
	(156.26, 72.08) --
	(156.32, 73.74) --
	(156.37, 73.20) --
	(156.42, 73.02) --
	(156.47, 71.94) --
	(156.52, 74.86) --
	(156.57, 68.57) --
	(156.62, 75.40) --
	(156.67, 71.99) --
	(156.73, 72.80) --
	(156.78, 72.12) --
	(156.83, 69.20) --
	(156.88, 74.32) --
	(156.93, 73.29) --
	(156.98, 71.36) --
	(157.03, 74.73) --
	(157.08, 73.07) --
	(157.14, 75.45) --
	(157.19, 74.46) --
	(157.24, 74.19) --
	(157.29, 74.68) --
	(157.34, 72.44) --
	(157.39, 71.72) --
	(157.44, 70.28) --
	(157.49, 72.62) --
	(157.54, 70.15) --
	(157.60, 72.89) --
	(157.65, 73.61) --
	(157.70, 75.04) --
	(157.75, 72.84) --
	(157.80, 75.27) --
	(157.85, 72.17) --
	(157.90, 71.31) --
	(157.95, 72.89) --
	(158.01, 76.89) --
	(158.06, 76.53) --
	(158.11, 71.63) --
	(158.16, 71.14) --
	(158.21, 70.15) --
	(158.26, 72.21) --
	(158.31, 73.70) --
	(158.36, 75.04) --
	(158.42, 73.83) --
	(158.47, 72.62) --
	(158.52, 74.01) --
	(158.57, 73.02) --
	(158.62, 74.95) --
	(158.67, 72.17) --
	(158.72, 74.06) --
	(158.77, 74.06) --
	(158.82, 73.07) --
	(158.88, 70.42) --
	(158.93, 74.91) --
	(158.98, 69.11) --
	(159.03, 74.77) --
	(159.08, 72.03) --
	(159.13, 73.38) --
	(159.18, 74.41) --
	(159.23, 72.75) --
	(159.29, 74.10) --
	(159.34, 72.98) --
	(159.39, 71.94) --
	(159.44, 73.47) --
	(159.49, 73.29) --
	(159.54, 72.57) --
	(159.59, 72.44) --
	(159.64, 72.17) --
	(159.70, 72.30) --
	(159.75, 72.93) --
	(159.80, 74.19) --
	(159.85, 73.47) --
	(159.90, 74.82) --
	(159.95, 74.01) --
	(160.00, 71.81) --
	(160.05, 71.85) --
	(160.10, 74.91) --
	(160.16, 73.70) --
	(160.21, 70.24) --
	(160.26, 72.26) --
	(160.31, 75.22) --
	(160.36, 72.57) --
	(160.41, 71.99) --
	(160.46, 73.56) --
	(160.51, 74.23) --
	(160.57, 73.92) --
	(160.62, 72.89) --
	(160.67, 70.96) --
	(160.72, 72.03) --
	(160.77, 72.08) --
	(160.82, 71.05) --
	(160.87, 72.12) --
	(160.92, 70.33) --
	(160.98, 76.21) --
	(161.03, 72.17) --
	(161.08, 76.08) --
	(161.13, 68.13) --
	(161.18, 72.44) --
	(161.23, 73.65) --
	(161.28, 75.40) --
	(161.33, 73.38) --
	(161.38, 71.18) --
	(161.44, 74.59) --
	(161.49, 74.14) --
	(161.54, 74.64) --
	(161.59, 74.64) --
	(161.64, 70.73) --
	(161.69, 74.32) --
	(161.74, 71.67) --
	(161.79, 71.94) --
	(161.85, 71.36) --
	(161.90, 74.19) --
	(161.95, 70.19) --
	(162.00, 71.09) --
	(162.05, 71.45) --
	(162.10, 76.21) --
	(162.15, 71.31) --
	(162.20, 74.06) --
	(162.26, 72.84) --
	(162.31, 72.98) --
	(162.36, 69.97) --
	(162.41, 74.10) --
	(162.46, 73.47) --
	(162.51, 72.66) --
	(162.56, 72.35) --
	(162.61, 73.70) --
	(162.66, 73.07) --
	(162.72, 72.12) --
	(162.77, 76.30) --
	(162.82, 72.21) --
	(162.87, 73.47) --
	(162.92, 73.16) --
	(162.97, 71.85) --
	(163.02, 72.98) --
	(163.07, 72.12) --
	(163.13, 72.44) --
	(163.18, 68.62) --
	(163.23, 75.58) --
	(163.28, 71.81) --
	(163.33, 72.98) --
	(163.38, 72.93) --
	(163.43, 78.10) --
	(163.48, 72.17) --
	(163.54, 74.14) --
	(163.59, 72.26) --
	(163.64, 73.29) --
	(163.69, 71.76) --
	(163.74, 76.17) --
	(163.79, 73.61) --
	(163.84, 73.65) --
	(163.89, 70.78) --
	(163.94, 72.66) --
	(164.00, 72.44) --
	(164.05, 76.08) --
	(164.10, 72.39) --
	(164.15, 69.29) --
	(164.20, 70.60) --
	(164.25, 72.57) --
	(164.30, 72.12) --
	(164.35, 71.85) --
	(164.41, 77.42) --
	(164.46, 70.60) --
	(164.51, 69.11) --
	(164.56, 75.22) --
	(164.61, 73.47) --
	(164.66, 74.32) --
	(164.71, 70.24) --
	(164.76, 75.99) --
	(164.82, 69.07) --
	(164.87, 76.66) --
	(164.92, 73.61) --
	(164.97, 71.90) --
	(165.02, 71.76) --
	(165.07, 70.01) --
	(165.12, 72.89) --
	(165.17, 74.86) --
	(165.22, 73.29) --
	(165.28, 71.63) --
	(165.33, 71.63) --
	(165.38, 75.00) --
	(165.43, 73.83) --
	(165.48, 75.63) --
	(165.53, 71.99) --
	(165.58, 72.39) --
	(165.63, 72.66) --
	(165.69, 72.48) --
	(165.74, 71.27) --
	(165.79, 72.98) --
	(165.84, 72.26) --
	(165.89, 68.98) --
	(165.94, 72.93) --
	(165.99, 72.35) --
	(166.04, 70.60) --
	(166.10, 71.00) --
	(166.15, 72.89) --
	(166.20, 68.89) --
	(166.25, 72.26) --
	(166.30, 74.19) --
	(166.35, 72.75) --
	(166.40, 71.49) --
	(166.45, 71.99) --
	(166.50, 72.62) --
	(166.56, 68.98) --
	(166.61, 73.56) --
	(166.66, 72.08) --
	(166.71, 72.03) --
	(166.76, 71.36) --
	(166.81, 73.25) --
	(166.86, 71.18) --
	(166.91, 72.39) --
	(166.97, 73.02) --
	(167.02, 70.33) --
	(167.07, 71.76) --
	(167.12, 71.45) --
	(167.17, 75.18) --
	(167.22, 73.29) --
	(167.27, 71.58) --
	(167.32, 71.81) --
	(167.38, 72.93) --
	(167.43, 72.71) --
	(167.48, 70.69) --
	(167.53, 75.49) --
	(167.58, 69.65) --
	(167.63, 76.71) --
	(167.68, 72.44) --
	(167.73, 71.31) --
	(167.78, 75.54) --
	(167.84, 70.78) --
	(167.89, 72.44) --
	(167.94, 74.55) --
	(167.99, 74.68) --
	(168.04, 69.56) --
	(168.09, 73.38) --
	(168.14, 73.16) --
	(168.19, 71.05) --
	(168.25, 68.89) --
	(168.30, 72.03) --
	(168.35, 71.72) --
	(168.40, 74.41) --
	(168.45, 72.57) --
	(168.50, 74.01) --
	(168.55, 72.08) --
	(168.60, 70.33) --
	(168.66, 75.18) --
	(168.71, 69.92) --
	(168.76, 68.89) --
	(168.81, 73.16) --
	(168.86, 71.00) --
	(168.91, 73.16) --
	(168.96, 68.98) --
	(169.01, 72.39) --
	(169.06, 74.23) --
	(169.12, 70.42) --
	(169.17, 72.84) --
	(169.22, 68.57) --
	(169.27, 76.03) --
	(169.32, 73.07) --
	(169.37, 72.35) --
	(169.42, 76.03) --
	(169.47, 70.91) --
	(169.53, 72.53) --
	(169.58, 69.16) --
	(169.63, 71.81) --
	(169.68, 72.93) --
	(169.73, 73.38) --
	(169.78, 73.38) --
	(169.83, 70.51) --
	(169.88, 70.73) --
	(169.94, 71.76) --
	(169.99, 69.70) --
	(170.04, 69.97) --
	(170.09, 70.73) --
	(170.14, 76.21) --
	(170.19, 70.55) --
	(170.24, 73.29) --
	(170.29, 68.48) --
	(170.34, 74.59) --
	(170.40, 76.03) --
	(170.45, 71.18) --
	(170.50, 70.28) --
	(170.55, 72.44) --
	(170.60, 70.64) --
	(170.65, 75.36) --
	(170.70, 72.80) --
	(170.75, 70.42) --
	(170.81, 70.28) --
	(170.86, 71.94) --
	(170.91, 70.64) --
	(170.96, 71.90) --
	(171.01, 70.42) --
	(171.06, 69.88) --
	(171.11, 69.43) --
	(171.16, 74.86) --
	(171.22, 69.70) --
	(171.27, 73.65) --
	(171.32, 71.54) --
	(171.37, 70.37) --
	(171.42, 69.79) --
	(171.47, 70.91) --
	(171.52, 71.09) --
	(171.57, 73.56) --
	(171.62, 69.29) --
	(171.68, 73.07) --
	(171.73, 72.89) --
	(171.78, 71.54) --
	(171.83, 71.27) --
	(171.88, 70.24) --
	(171.93, 72.84) --
	(171.98, 73.29) --
	(172.03, 73.56) --
	(172.09, 73.11) --
	(172.14, 73.20) --
	(172.19, 71.09) --
	(172.24, 69.92) --
	(172.29, 70.15) --
	(172.34, 71.18) --
	(172.39, 72.03) --
	(172.44, 69.25) --
	(172.50, 73.43) --
	(172.55, 74.28) --
	(172.60, 71.09) --
	(172.65, 74.14) --
	(172.70, 70.78) --
	(172.75, 69.79) --
	(172.80, 69.34) --
	(172.85, 71.14) --
	(172.90, 67.95) --
	(172.96, 74.91) --
	(173.01, 70.55) --
	(173.06, 72.66) --
	(173.11, 71.14) --
	(173.16, 74.37) --
	(173.21, 68.93) --
	(173.26, 76.48) --
	(173.31, 69.25) --
	(173.37, 72.30) --
	(173.42, 71.90) --
	(173.47, 75.27) --
	(173.52, 72.17) --
	(173.57, 71.90) --
	(173.62, 72.75) --
	(173.67, 70.91) --
	(173.72, 72.30) --
	(173.78, 73.07) --
	(173.83, 73.52) --
	(173.88, 75.00) --
	(173.93, 71.54) --
	(173.98, 72.57) --
	(174.03, 71.49) --
	(174.08, 73.92) --
	(174.13, 71.36) --
	(174.19, 70.91) --
	(174.24, 71.14) --
	(174.29, 72.17) --
	(174.34, 70.96) --
	(174.39, 72.75) --
	(174.44, 71.54) --
	(174.49, 69.65) --
	(174.54, 74.19) --
	(174.59, 68.35) --
	(174.65, 75.72) --
	(174.70, 71.36) --
	(174.75, 72.48) --
	(174.80, 71.81) --
	(174.85, 68.93) --
	(174.90, 75.22) --
	(174.95, 72.80) --
	(175.00, 75.00) --
	(175.06, 68.57) --
	(175.11, 68.93) --
	(175.16, 70.78) --
	(175.21, 71.22) --
	(175.26, 68.89) --
	(175.31, 71.94) --
	(175.36, 70.15) --
	(175.41, 72.30) --
	(175.47, 71.05) --
	(175.52, 68.30) --
	(175.57, 72.03) --
	(175.62, 71.45) --
	(175.67, 68.98) --
	(175.72, 73.70) --
	(175.77, 70.69) --
	(175.82, 68.44) --
	(175.87, 74.14) --
	(175.93, 67.72) --
	(175.98, 72.35) --
	(176.03, 72.17) --
	(176.08, 70.15) --
	(176.13, 73.79) --
	(176.18, 69.29) --
	(176.23, 69.74) --
	(176.28, 73.29) --
	(176.34, 69.74) --
	(176.39, 71.72) --
	(176.44, 73.11) --
	(176.49, 71.85) --
	(176.54, 73.56) --
	(176.59, 69.92) --
	(176.64, 72.44) --
	(176.69, 68.80) --
	(176.75, 70.28) --
	(176.80, 72.89) --
	(176.85, 70.10) --
	(176.90, 71.18) --
	(176.95, 73.61) --
	(177.00, 72.62) --
	(177.05, 71.40) --
	(177.10, 77.11) --
	(177.15, 68.35) --
	(177.21, 74.46) --
	(177.26, 70.55) --
	(177.31, 75.94) --
	(177.36, 70.91) --
	(177.41, 69.65) --
	(177.46, 68.89) --
	(177.51, 73.20) --
	(177.56, 73.83) --
	(177.62, 69.20) --
	(177.67, 70.10) --
	(177.72, 68.75) --
	(177.77, 68.08) --
	(177.82, 70.64) --
	(177.87, 68.75) --
	(177.92, 76.44) --
	(177.97, 69.38) --
	(178.03, 74.28) --
	(178.08, 74.64) --
	(178.13, 71.63) --
	(178.18, 75.63) --
	(178.23, 70.73) --
	(178.28, 71.85) --
	(178.33, 68.53) --
	(178.38, 70.10) --
	(178.43, 68.71) --
	(178.49, 68.62) --
	(178.54, 70.78) --
	(178.59, 71.31) --
	(178.64, 68.75) --
	(178.69, 73.74) --
	(178.74, 71.58) --
	(178.79, 72.53) --
	(178.84, 71.58) --
	(178.90, 70.37) --
	(178.95, 71.00) --
	(179.00, 71.85) --
	(179.05, 71.49) --
	(179.10, 73.11) --
	(179.15, 70.06) --
	(179.20, 73.88) --
	(179.25, 70.42) --
	(179.31, 67.23) --
	(179.36, 72.98) --
	(179.41, 72.98) --
	(179.46, 70.91) --
	(179.51, 72.48) --
	(179.56, 75.49) --
	(179.61, 72.89) --
	(179.66, 72.17) --
	(179.71, 72.08) --
	(179.77, 71.14) --
	(179.82, 70.73) --
	(179.87, 71.99) --
	(179.92, 70.06) --
	(179.97, 69.47) --
	(180.02, 75.22) --
	(180.07, 71.67) --
	(180.12, 69.74) --
	(180.18, 71.18) --
	(180.23, 68.62) --
	(180.28, 69.52) --
	(180.33, 72.35) --
	(180.38, 68.93) --
	(180.43, 69.16) --
	(180.48, 69.34) --
	(180.53, 68.75) --
	(180.59, 70.24) --
	(180.64, 74.01) --
	(180.69, 71.67) --
	(180.74, 70.15) --
	(180.79, 70.69) --
	(180.84, 72.98) --
	(180.89, 73.16) --
	(180.94, 70.01) --
	(180.99, 72.62) --
	(181.05, 69.83) --
	(181.10, 67.32) --
	(181.15, 66.91) --
	(181.20, 69.97) --
	(181.25, 72.30) --
	(181.30, 71.18) --
	(181.35, 71.36) --
	(181.40, 69.25) --
	(181.46, 70.06) --
	(181.51, 70.10) --
	(181.56, 67.81) --
	(181.61, 69.88) --
	(181.66, 72.98) --
	(181.71, 69.70) --
	(181.76, 69.47) --
	(181.81, 71.67) --
	(181.87, 70.19) --
	(181.92, 72.57) --
	(181.97, 68.98) --
	(182.02, 71.85) --
	(182.07, 72.35) --
	(182.12, 72.89) --
	(182.17, 69.79) --
	(182.22, 68.71) --
	(182.27, 69.92) --
	(182.33, 72.53) --
	(182.38, 68.17) --
	(182.43, 69.74) --
	(182.48, 70.69) --
	(182.53, 69.61) --
	(182.58, 69.65) --
	(182.63, 69.79) --
	(182.68, 73.92) --
	(182.74, 69.07) --
	(182.79, 69.65) --
	(182.84, 75.67) --
	(182.89, 66.64) --
	(182.94, 71.18) --
	(182.99, 64.94) --
	(183.04, 74.50) --
	(183.09, 67.27) --
	(183.15, 70.06) --
	(183.20, 68.62) --
	(183.25, 72.26) --
	(183.30, 70.69) --
	(183.35, 72.75) --
	(183.40, 71.58) --
	(183.45, 76.30) --
	(183.50, 70.42) --
	(183.55, 72.17) --
	(183.61, 70.19) --
	(183.66, 70.24) --
	(183.71, 70.46) --
	(183.76, 68.98) --
	(183.81, 73.34) --
	(183.86, 68.98) --
	(183.91, 72.57) --
	(183.96, 72.53) --
	(184.02, 71.63) --
	(184.07, 72.21) --
	(184.12, 71.45) --
	(184.17, 70.96) --
	(184.22, 68.84) --
	(184.27, 71.14) --
	(184.32, 71.36) --
	(184.37, 72.62) --
	(184.43, 70.28) --
	(184.48, 69.29) --
	(184.53, 68.21) --
	(184.58, 73.16) --
	(184.63, 71.54) --
	(184.68, 69.11) --
	(184.73, 68.44) --
	(184.78, 67.27) --
	(184.83, 73.47) --
	(184.89, 68.71) --
	(184.94, 69.11) --
	(184.99, 69.92) --
	(185.04, 70.28) --
	(185.09, 72.89) --
	(185.14, 68.71) --
	(185.19, 68.62) --
	(185.24, 70.96) --
	(185.30, 71.67) --
	(185.35, 71.05) --
	(185.40, 71.85) --
	(185.45, 71.14) --
	(185.50, 69.70) --
	(185.55, 70.55) --
	(185.60, 70.01) --
	(185.65, 68.44) --
	(185.71, 69.70) --
	(185.76, 73.29) --
	(185.81, 73.02) --
	(185.86, 68.89) --
	(185.91, 72.75) --
	(185.96, 71.14) --
	(186.01, 70.06) --
	(186.06, 69.43) --
	(186.11, 68.39) --
	(186.17, 71.94) --
	(186.22, 72.98) --
	(186.27, 71.36) --
	(186.32, 74.10) --
	(186.37, 66.42) --
	(186.42, 73.52) --
	(186.47, 68.98) --
	(186.52, 69.25) --
	(186.58, 72.84) --
	(186.63, 69.83) --
	(186.68, 68.75) --
	(186.73, 68.48) --
	(186.78, 70.10) --
	(186.83, 69.92) --
	(186.88, 72.12) --
	(186.93, 71.94) --
	(186.99, 75.13) --
	(187.04, 66.55) --
	(187.09, 69.56) --
	(187.14, 69.47) --
	(187.19, 72.12) --
	(187.24, 67.81) --
	(187.29, 69.88) --
	(187.34, 70.46) --
	(187.39, 72.89) --
	(187.45, 73.92) --
	(187.50, 69.34) --
	(187.55, 70.55) --
	(187.60, 69.34) --
	(187.65, 71.85) --
	(187.70, 69.02) --
	(187.75, 70.55) --
	(187.80, 71.05) --
	(187.86, 69.52) --
	(187.91, 69.25) --
	(187.96, 72.57) --
	(188.01, 70.24) --
	(188.06, 69.29) --
	(188.11, 69.79) --
	(188.16, 67.63) --
	(188.21, 69.56) --
	(188.27, 69.11) --
	(188.32, 70.15) --
	(188.37, 68.80) --
	(188.42, 70.69) --
	(188.47, 72.62) --
	(188.52, 70.06) --
	(188.57, 70.46) --
	(188.62, 72.21) --
	(188.67, 70.78) --
	(188.73, 72.30) --
	(188.78, 70.10) --
	(188.83, 70.91) --
	(188.88, 69.88) --
	(188.93, 68.75) --
	(188.98, 72.35) --
	(189.03, 69.47) --
	(189.08, 72.75) --
	(189.14, 73.47) --
	(189.19, 70.96) --
	(189.24, 71.76) --
	(189.29, 70.96) --
	(189.34, 70.51) --
	(189.39, 69.74) --
	(189.44, 68.30) --
	(189.49, 69.47) --
	(189.55, 72.71) --
	(189.60, 70.10) --
	(189.65, 68.57) --
	(189.70, 68.44) --
	(189.75, 74.28) --
	(189.80, 72.57) --
	(189.85, 72.89) --
	(189.90, 71.40) --
	(189.95, 71.14) --
	(190.01, 70.24) --
	(190.06, 69.38) --
	(190.11, 69.47) --
	(190.16, 67.86) --
	(190.21, 70.51) --
	(190.26, 70.69) --
	(190.31, 68.89) --
	(190.36, 71.72) --
	(190.42, 68.57) --
	(190.47, 71.18) --
	(190.52, 70.91) --
	(190.57, 72.21) --
	(190.62, 67.63) --
	(190.67, 65.16) --
	(190.72, 73.70) --
	(190.77, 69.25) --
	(190.83, 70.37) --
	(190.88, 70.60) --
	(190.93, 70.37) --
	(190.98, 70.24) --
	(191.03, 71.72) --
	(191.08, 70.87) --
	(191.13, 68.75) --
	(191.18, 69.65) --
	(191.23, 67.54) --
	(191.29, 70.51) --
	(191.34, 71.49) --
	(191.39, 65.83) --
	(191.44, 72.57) --
	(191.49, 71.27) --
	(191.54, 69.29) --
	(191.59, 70.64) --
	(191.64, 72.62) --
	(191.70, 69.02) --
	(191.75, 72.26) --
	(191.80, 70.28) --
	(191.85, 70.51) --
	(191.90, 71.45) --
	(191.95, 68.21) --
	(192.00, 71.05) --
	(192.05, 73.16) --
	(192.11, 67.95) --
	(192.16, 73.61) --
	(192.21, 69.25) --
	(192.26, 70.82) --
	(192.31, 67.45) --
	(192.36, 66.73) --
	(192.41, 70.37) --
	(192.46, 70.42) --
	(192.51, 72.44) --
	(192.57, 71.40) --
	(192.62, 68.53) --
	(192.67, 66.15) --
	(192.72, 71.00) --
	(192.77, 70.91) --
	(192.82, 68.13) --
	(192.87, 72.08) --
	(192.92, 69.38) --
	(192.98, 69.20) --
	(193.03, 67.99) --
	(193.08, 69.11) --
	(193.13, 71.67) --
	(193.18, 69.88) --
	(193.23, 71.45) --
	(193.28, 69.25) --
	(193.33, 72.03) --
	(193.39, 69.92) --
	(193.44, 68.30) --
	(193.49, 68.71) --
	(193.54, 69.16) --
	(193.59, 69.61) --
	(193.64, 69.83) --
	(193.69, 72.89) --
	(193.74, 68.21) --
	(193.79, 69.43) --
	(193.85, 71.45) --
	(193.90, 70.37) --
	(193.95, 69.34) --
	(194.00, 70.87) --
	(194.05, 70.33) --
	(194.10, 68.93) --
	(194.15, 70.82) --
	(194.20, 68.89) --
	(194.26, 73.43) --
	(194.31, 67.50) --
	(194.36, 67.54) --
	(194.41, 70.64) --
	(194.46, 69.25) --
	(194.51, 67.32) --
	(194.56, 68.84) --
	(194.61, 71.72) --
	(194.67, 70.69) --
	(194.72, 69.56) --
	(194.77, 67.41) --
	(194.82, 67.36) --
	(194.87, 71.09) --
	(194.92, 70.87) --
	(194.97, 70.69) --
	(195.02, 67.27) --
	(195.07, 64.94) --
	(195.13, 69.61) --
	(195.18, 67.95) --
	(195.23, 69.70) --
	(195.28, 70.15) --
	(195.33, 70.82) --
	(195.38, 68.98) --
	(195.43, 67.63) --
	(195.48, 68.21) --
	(195.54, 67.41) --
	(195.59, 69.83) --
	(195.64, 68.21) --
	(195.69, 67.77) --
	(195.74, 72.80) --
	(195.79, 71.18) --
	(195.84, 67.27) --
	(195.89, 72.39) --
	(195.95, 68.21) --
	(196.00, 71.81) --
	(196.05, 63.90) --
	(196.10, 69.52) --
	(196.15, 68.75) --
	(196.20, 69.97) --
	(196.25, 73.88) --
	(196.30, 69.70) --
	(196.35, 72.35) --
	(196.41, 68.98) --
	(196.46, 71.22) --
	(196.51, 66.73) --
	(196.56, 69.56) --
	(196.61, 66.91) --
	(196.66, 68.48) --
	(196.71, 71.76) --
	(196.76, 69.43) --
	(196.82, 71.36) --
	(196.87, 68.66) --
	(196.92, 65.52) --
	(196.97, 69.65) --
	(197.02, 68.89) --
	(197.07, 73.16) --
	(197.12, 70.42) --
	(197.17, 65.43) --
	(197.23, 70.01) --
	(197.28, 67.41) --
	(197.33, 68.75) --
	(197.38, 66.60) --
	(197.43, 66.64) --
	(197.48, 67.09) --
	(197.53, 69.29) --
	(197.58, 67.77) --
	(197.63, 69.47) --
	(197.69, 68.84) --
	(197.74, 67.32) --
	(197.79, 70.73) --
	(197.84, 67.05) --
	(197.89, 69.29) --
	(197.94, 70.82) --
	(197.99, 66.55) --
	(198.04, 69.97) --
	(198.10, 67.90) --
	(198.15, 64.76) --
	(198.20, 68.13) --
	(198.25, 69.52) --
	(198.30, 69.70) --
	(198.35, 64.26) --
	(198.40, 71.14) --
	(198.45, 68.39) --
	(198.51, 71.85) --
	(198.56, 71.27) --
	(198.61, 73.20) --
	(198.66, 74.59) --
	(198.71, 70.82) --
	(198.76, 69.92) --
	(198.81, 70.37) --
	(198.86, 71.09) --
	(198.92, 67.09) --
	(198.97, 68.30) --
	(199.02, 69.74) --
	(199.07, 70.33) --
	(199.12, 71.09) --
	(199.17, 71.31) --
	(199.22, 69.88) --
	(199.27, 71.18) --
	(199.32, 69.88) --
	(199.38, 69.88) --
	(199.43, 69.47) --
	(199.48, 69.61) --
	(199.53, 70.10) --
	(199.58, 67.50) --
	(199.63, 70.64) --
	(199.68, 70.24) --
	(199.73, 72.93) --
	(199.79, 68.66) --
	(199.84, 70.73) --
	(199.89, 68.39) --
	(199.94, 69.92) --
	(199.99, 70.46) --
	(200.04, 65.88) --
	(200.09, 70.51) --
	(200.14, 69.70) --
	(200.20, 69.61) --
	(200.25, 68.62) --
	(200.30, 70.69) --
	(200.35, 68.48) --
	(200.40, 68.04) --
	(200.45, 67.05) --
	(200.50, 68.04) --
	(200.55, 68.48) --
	(200.60, 68.39) --
	(200.66, 67.27) --
	(200.71, 69.38) --
	(200.76, 68.57) --
	(200.81, 70.28) --
	(200.86, 72.12) --
	(200.91, 66.37) --
	(200.96, 71.05) --
	(201.01, 68.13) --
	(201.07, 67.45) --
	(201.12, 68.13) --
	(201.17, 69.61) --
	(201.22, 71.14) --
	(201.27, 68.57) --
	(201.32, 72.12) --
	(201.37, 64.67) --
	(201.42, 68.39) --
	(201.48, 73.02) --
	(201.53, 68.84) --
	(201.58, 70.28) --
	(201.63, 65.16) --
	(201.68, 67.90) --
	(201.73, 70.69) --
	(201.78, 70.37) --
	(201.83, 69.61) --
	(201.88, 67.32) --
	(201.94, 69.88) --
	(201.99, 68.53) --
	(202.04, 66.19) --
	(202.09, 69.16) --
	(202.14, 67.36) --
	(202.19, 66.51) --
	(202.24, 68.66) --
	(202.29, 67.77) --
	(202.35, 67.27) --
	(202.40, 70.60) --
	(202.45, 71.90) --
	(202.50, 66.64) --
	(202.55, 68.75) --
	(202.60, 67.05) --
	(202.65, 67.27) --
	(202.70, 72.48) --
	(202.76, 66.78) --
	(202.81, 65.79) --
	(202.86, 70.87) --
	(202.91, 65.47) --
	(202.96, 69.43) --
	(203.01, 68.26) --
	(203.06, 69.11) --
	(203.11, 70.42) --
	(203.16, 68.71) --
	(203.22, 69.61) --
	(203.27, 68.21) --
	(203.32, 68.80) --
	(203.37, 67.54) --
	(203.42, 70.78) --
	(203.47, 65.52) --
	(203.52, 64.98) --
	(203.57, 72.30) --
	(203.63, 67.59) --
	(203.68, 71.94) --
	(203.73, 67.59) --
	(203.78, 65.97) --
	(203.83, 66.73) --
	(203.88, 66.87) --
	(203.93, 67.41) --
	(203.98, 68.84) --
	(204.04, 68.26) --
	(204.09, 69.61) --
	(204.14, 69.11) --
	(204.19, 67.05) --
	(204.24, 70.01) --
	(204.29, 70.69) --
	(204.34, 69.25) --
	(204.39, 66.82) --
	(204.44, 70.42) --
	(204.50, 65.21) --
	(204.55, 68.08) --
	(204.60, 69.16) --
	(204.65, 68.98) --
	(204.70, 66.24) --
	(204.75, 69.16) --
	(204.80, 69.74) --
	(204.85, 68.62) --
	(204.91, 72.57) --
	(204.96, 69.52) --
	(205.01, 66.82) --
	(205.06, 67.14) --
	(205.11, 72.39) --
	(205.16, 70.24) --
	(205.21, 67.45) --
	(205.26, 67.54) --
	(205.32, 68.30) --
	(205.37, 71.90) --
	(205.42, 66.42) --
	(205.47, 70.10) --
	(205.52, 67.32) --
	(205.57, 70.55) --
	(205.62, 67.18) --
	(205.67, 71.09) --
	(205.72, 68.26) --
	(205.78, 70.01) --
	(205.83, 68.62) --
	(205.88, 68.75) --
	(205.93, 69.47) --
	(205.98, 69.07) --
	(206.03, 67.14) --
	(206.08, 68.13) --
	(206.13, 67.77) --
	(206.19, 67.36) --
	(206.24, 67.27) --
	(206.29, 67.59) --
	(206.34, 66.73) --
	(206.39, 69.07) --
	(206.44, 68.13) --
	(206.49, 71.27) --
	(206.54, 65.38) --
	(206.60, 69.20) --
	(206.65, 68.57) --
	(206.70, 66.15) --
	(206.75, 69.74) --
	(206.80, 68.26) --
	(206.85, 68.35) --
	(206.90, 69.74) --
	(206.95, 67.41) --
	(207.00, 69.34) --
	(207.06, 69.02) --
	(207.11, 64.85) --
	(207.16, 70.60) --
	(207.21, 66.69) --
	(207.26, 71.72) --
	(207.31, 66.69) --
	(207.36, 66.96) --
	(207.41, 68.17) --
	(207.47, 66.10) --
	(207.52, 69.97) --
	(207.57, 67.32) --
	(207.62, 66.87) --
	(207.67, 66.15) --
	(207.72, 72.93) --
	(207.77, 69.20) --
	(207.82, 68.66) --
	(207.88, 72.62) --
	(207.93, 68.93) --
	(207.98, 68.30) --
	(208.03, 70.87) --
	(208.08, 67.41) --
	(208.13, 68.75) --
	(208.18, 67.32) --
	(208.23, 65.07) --
	(208.28, 66.78) --
	(208.34, 68.53) --
	(208.39, 69.70) --
	(208.44, 64.53) --
	(208.49, 66.42) --
	(208.54, 67.99) --
	(208.59, 66.51) --
	(208.64, 68.80) --
	(208.69, 64.26) --
	(208.75, 70.73) --
	(208.80, 65.29) --
	(208.85, 67.14) --
	(208.90, 67.09) --
	(208.95, 64.85) --
	(209.00, 67.27) --
	(209.05, 65.29) --
	(209.10, 71.05) --
	(209.16, 67.09) --
	(209.21, 69.47) --
	(209.26, 62.87) --
	(209.31, 68.53) --
	(209.36, 70.37) --
	(209.41, 65.34) --
	(209.46, 69.61) --
	(209.51, 67.09) --
	(209.56, 66.78) --
	(209.62, 66.24) --
	(209.67, 68.80) --
	(209.72, 64.80) --
	(209.77, 66.15) --
	(209.82, 67.86) --
	(209.87, 67.99) --
	(209.92, 65.65) --
	(209.97, 64.71) --
	(210.03, 69.47) --
	(210.08, 65.92) --
	(210.13, 69.34) --
	(210.18, 64.98) --
	(210.23, 72.93) --
	(210.28, 65.38) --
	(210.33, 68.57) --
	(210.38, 70.82) --
	(210.44, 63.50) --
	(210.49, 69.52) --
	(210.54, 66.87) --
	(210.59, 64.62) --
	(210.64, 68.71) --
	(210.69, 65.70) --
	(210.74, 67.36) --
	(210.79, 69.61) --
	(210.84, 62.24) --
	(210.90, 65.52) --
	(210.95, 70.82) --
	(211.00, 63.32) --
	(211.05, 68.35) --
	(211.10, 65.83) --
	(211.15, 67.72) --
	(211.20, 69.02) --
	(211.25, 65.74) --
	(211.31, 68.44) --
	(211.36, 65.47) --
	(211.41, 65.47) --
	(211.46, 69.16) --
	(211.51, 64.35) --
	(211.56, 67.23) --
	(211.61, 66.28) --
	(211.66, 66.87) --
	(211.72, 69.61) --
	(211.77, 68.93) --
	(211.82, 70.64) --
	(211.87, 66.10) --
	(211.92, 66.64) --
	(211.97, 69.25) --
	(212.02, 68.13) --
	(212.07, 67.77) --
	(212.12, 67.18) --
	(212.18, 67.99) --
	(212.23, 68.26) --
	(212.28, 66.91) --
	(212.33, 64.49) --
	(212.38, 68.17) --
	(212.43, 66.69) --
	(212.48, 65.74) --
	(212.53, 70.91) --
	(212.59, 68.26) --
	(212.64, 66.78) --
	(212.69, 67.90) --
	(212.74, 67.63) --
	(212.79, 68.44) --
	(212.84, 68.35) --
	(212.89, 68.48) --
	(212.94, 70.46) --
	(213.00, 64.89) --
	(213.05, 67.09) --
	(213.10, 66.37) --
	(213.15, 66.51) --
	(213.20, 67.63) --
	(213.25, 69.47) --
	(213.30, 69.97) --
	(213.35, 72.26) --
	(213.40, 70.82) --
	(213.46, 67.45) --
	(213.51, 68.80) --
	(213.56, 65.12) --
	(213.61, 68.26) --
	(213.66, 68.30) --
	(213.71, 66.64) --
	(213.76, 68.30) --
	(213.81, 66.69) --
	(213.87, 67.77) --
	(213.92, 67.23) --
	(213.97, 66.91) --
	(214.02, 66.69) --
	(214.07, 64.49) --
	(214.12, 67.27) --
	(214.17, 66.51) --
	(214.22, 70.60) --
	(214.28, 64.94) --
	(214.33, 68.04) --
	(214.38, 68.04) --
	(214.43, 67.90) --
	(214.48, 67.27) --
	(214.53, 64.22) --
	(214.58, 66.19) --
	(214.63, 65.83) --
	(214.68, 66.37) --
	(214.74, 63.81) --
	(214.79, 67.09) --
	(214.84, 67.99) --
	(214.89, 66.42) --
	(214.94, 67.14) --
	(214.99, 67.63) --
	(215.04, 66.28) --
	(215.09, 64.40) --
	(215.15, 66.46) --
	(215.20, 65.74) --
	(215.25, 70.73) --
	(215.30, 63.27) --
	(215.35, 69.47) --
	(215.40, 65.88) --
	(215.45, 67.59) --
	(215.50, 66.55) --
	(215.56, 64.40) --
	(215.61, 68.89) --
	(215.66, 68.48) --
	(215.71, 68.89) --
	(215.76, 66.96) --
	(215.81, 66.82) --
	(215.86, 64.13) --
	(215.91, 65.29) --
	(215.96, 69.83) --
	(216.02, 66.42) --
	(216.07, 67.59) --
	(216.12, 70.64) --
	(216.17, 64.89) --
	(216.22, 70.46) --
	(216.27, 65.12) --
	(216.32, 67.90) --
	(216.37, 69.20) --
	(216.43, 65.16) --
	(216.48, 63.54) --
	(216.53, 68.71) --
	(216.58, 62.78) --
	(216.63, 65.25) --
	(216.68, 65.79) --
	(216.73, 66.96) --
	(216.78, 66.78) --
	(216.84, 63.50) --
	(216.89, 66.19) --
	(216.94, 63.81) --
	(216.99, 67.90) --
	(217.04, 64.62) --
	(217.09, 66.33) --
	(217.14, 66.19) --
	(217.19, 69.25) --
	(217.24, 69.74) --
	(217.30, 69.61) --
	(217.35, 68.62) --
	(217.40, 66.91) --
	(217.45, 70.01) --
	(217.50, 67.14) --
	(217.55, 66.37) --
	(217.60, 66.01) --
	(217.65, 66.82) --
	(217.71, 64.67) --
	(217.76, 64.89) --
	(217.81, 67.32) --
	(217.86, 68.57) --
	(217.91, 67.81) --
	(217.96, 65.07) --
	(218.01, 67.32) --
	(218.06, 72.08) --
	(218.12, 66.37) --
	(218.17, 66.73) --
	(218.22, 65.88) --
	(218.27, 65.97) --
	(218.32, 67.68) --
	(218.37, 69.16) --
	(218.42, 69.43) --
	(218.47, 67.36) --
	(218.52, 65.65) --
	(218.58, 64.26) --
	(218.63, 68.39) --
	(218.68, 67.95) --
	(218.73, 66.55) --
	(218.78, 68.62) --
	(218.83, 64.53) --
	(218.88, 66.60) --
	(218.93, 66.82) --
	(218.99, 66.78) --
	(219.04, 66.64) --
	(219.09, 65.29) --
	(219.14, 67.63) --
	(219.19, 70.28) --
	(219.24, 65.65) --
	(219.29, 65.38) --
	(219.34, 68.93) --
	(219.40, 63.95) --
	(219.45, 71.31) --
	(219.50, 64.62) --
	(219.55, 67.50) --
	(219.60, 63.09) --
	(219.65, 66.24) --
	(219.70, 69.07) --
	(219.75, 65.21) --
	(219.80, 65.52) --
	(219.86, 68.62) --
	(219.91, 65.16) --
	(219.96, 67.23) --
	(220.01, 65.47) --
	(220.06, 66.33) --
	(220.11, 66.19) --
	(220.16, 67.27) --
	(220.21, 66.33) --
	(220.27, 66.33) --
	(220.32, 61.88) --
	(220.37, 66.91) --
	(220.42, 63.14) --
	(220.47, 68.75) --
	(220.52, 65.43) --
	(220.57, 66.78) --
	(220.62, 69.29) --
	(220.68, 67.36) --
	(220.73, 67.45) --
	(220.78, 68.30) --
	(220.83, 64.98) --
	(220.88, 64.85) --
	(220.93, 65.38) --
	(220.98, 69.20) --
	(221.03, 65.34) --
	(221.08, 64.53) --
	(221.14, 68.53) --
	(221.19, 67.36) --
	(221.24, 66.60) --
	(221.29, 68.71) --
	(221.34, 66.64) --
	(221.39, 64.80) --
	(221.44, 67.72) --
	(221.49, 65.25) --
	(221.55, 66.55) --
	(221.60, 65.56) --
	(221.65, 65.03) --
	(221.70, 69.11) --
	(221.75, 65.74) --
	(221.80, 68.75) --
	(221.85, 69.47) --
	(221.90, 66.55) --
	(221.96, 67.23) --
	(222.01, 66.69) --
	(222.06, 66.91) --
	(222.11, 68.98) --
	(222.16, 69.83) --
	(222.21, 65.16) --
	(222.26, 66.19) --
	(222.31, 66.19) --
	(222.36, 67.45) --
	(222.42, 66.10) --
	(222.47, 65.34) --
	(222.52, 66.42) --
	(222.57, 66.01) --
	(222.62, 68.62) --
	(222.67, 64.85) --
	(222.72, 63.81) --
	(222.77, 69.65) --
	(222.83, 67.95) --
	(222.88, 66.64) --
	(222.93, 65.34) --
	(222.98, 65.70) --
	(223.03, 67.14) --
	(223.08, 63.18) --
	(223.13, 65.79) --
	(223.18, 66.15) --
	(223.24, 63.86) --
	(223.29, 67.99) --
	(223.34, 63.63) --
	(223.39, 68.80) --
	(223.44, 63.77) --
	(223.49, 64.44) --
	(223.54, 68.57) --
	(223.59, 67.00) --
	(223.64, 65.74) --
	(223.70, 62.60) --
	(223.75, 65.56) --
	(223.80, 66.64) --
	(223.85, 69.25) --
	(223.90, 65.47) --
	(223.95, 64.67) --
	(224.00, 66.78) --
	(224.05, 67.59) --
	(224.11, 65.12) --
	(224.16, 64.62) --
	(224.21, 65.16) --
	(224.26, 65.16) --
	(224.31, 66.28) --
	(224.36, 63.77) --
	(224.41, 65.61) --
	(224.46, 65.16) --
	(224.52, 64.71) --
	(224.57, 67.14) --
	(224.62, 65.25) --
	(224.67, 62.64) --
	(224.72, 65.43) --
	(224.77, 69.70) --
	(224.82, 68.62) --
	(224.87, 66.01) --
	(224.93, 64.71) --
	(224.98, 65.16) --
	(225.03, 66.78) --
	(225.08, 67.00) --
	(225.13, 66.24) --
	(225.18, 61.30) --
	(225.23, 68.44) --
	(225.28, 66.24) --
	(225.33, 64.98) --
	(225.39, 67.50) --
	(225.44, 66.19) --
	(225.49, 68.30) --
	(225.54, 66.87) --
	(225.59, 66.96) --
	(225.64, 65.83) --
	(225.69, 67.36) --
	(225.74, 67.63) --
	(225.80, 67.99) --
	(225.85, 65.65) --
	(225.90, 64.26) --
	(225.95, 65.52) --
	(226.00, 67.99) --
	(226.05, 69.97) --
	(226.10, 64.17) --
	(226.15, 66.06) --
	(226.21, 66.69) --
	(226.26, 62.91) --
	(226.31, 63.59) --
	(226.36, 63.59) --
	(226.41, 65.74) --
	(226.46, 63.95) --
	(226.51, 65.65) --
	(226.56, 69.11) --
	(226.61, 66.60) --
	(226.67, 65.47) --
	(226.72, 67.32) --
	(226.77, 65.88) --
	(226.82, 64.58) --
	(226.87, 62.20) --
	(226.92, 64.67) --
	(226.97, 67.54) --
	(227.02, 67.81) --
	(227.08, 66.28) --
	(227.13, 63.59) --
	(227.18, 69.20) --
	(227.23, 65.79) --
	(227.28, 62.78) --
	(227.33, 62.60) --
	(227.38, 65.97) --
	(227.43, 67.27) --
	(227.49, 66.42) --
	(227.54, 66.55) --
	(227.59, 67.05) --
	(227.64, 67.86) --
	(227.69, 66.73) --
	(227.74, 67.77) --
	(227.79, 67.59) --
	(227.84, 63.27) --
	(227.89, 63.09) --
	(227.95, 61.48) --
	(228.00, 64.85) --
	(228.05, 64.76) --
	(228.10, 63.63) --
	(228.15, 67.54) --
	(228.20, 66.33) --
	(228.25, 68.89) --
	(228.30, 67.32) --
	(228.36, 64.49) --
	(228.41, 66.15) --
	(228.46, 67.41) --
	(228.51, 64.62) --
	(228.56, 65.88) --
	(228.61, 64.08) --
	(228.66, 63.59) --
	(228.71, 64.89) --
	(228.77, 65.34) --
	(228.82, 61.43) --
	(228.87, 67.41) --
	(228.92, 66.82) --
	(228.97, 62.02) --
	(229.02, 70.06) --
	(229.07, 63.45) --
	(229.12, 64.35) --
	(229.17, 68.93) --
	(229.23, 62.78) --
	(229.28, 66.64) --
	(229.33, 65.25) --
	(229.38, 66.19) --
	(229.43, 64.44) --
	(229.48, 65.47) --
	(229.53, 68.04) --
	(229.58, 67.99) --
	(229.64, 67.05) --
	(229.69, 61.34) --
	(229.74, 63.27) --
	(229.79, 65.52) --
	(229.84, 65.12) --
	(229.89, 67.41) --
	(229.94, 63.09) --
	(229.99, 67.72) --
	(230.05, 66.28) --
	(230.10, 64.22) --
	(230.15, 67.23) --
	(230.20, 64.26) --
	(230.25, 65.65) --
	(230.30, 65.29) --
	(230.35, 66.69) --
	(230.40, 64.71) --
	(230.45, 67.27) --
	(230.51, 60.49) --
	(230.56, 62.69) --
	(230.61, 65.16) --
	(230.66, 62.69) --
	(230.71, 63.14) --
	(230.76, 63.95) --
	(230.81, 63.72) --
	(230.86, 62.15) --
	(230.92, 68.30) --
	(230.97, 65.43) --
	(231.02, 64.80) --
	(231.07, 65.16) --
	(231.12, 63.95) --
	(231.17, 64.80) --
	(231.22, 64.53) --
	(231.27, 67.99) --
	(231.33, 66.69) --
	(231.38, 64.44) --
	(231.43, 66.91) --
	(231.48, 64.44) --
	(231.53, 64.26) --
	(231.58, 62.69) --
	(231.63, 61.84) --
	(231.68, 69.92) --
	(231.73, 70.37) --
	(231.79, 61.30) --
	(231.84, 68.26) --
	(231.89, 61.66) --
	(231.94, 64.58) --
	(231.99, 64.35) --
	(232.04, 65.12) --
	(232.09, 69.92) --
	(232.14, 64.26) --
	(232.20, 65.97) --
	(232.25, 65.29) --
	(232.30, 64.22) --
	(232.35, 63.63) --
	(232.40, 64.76) --
	(232.45, 65.43) --
	(232.50, 64.31) --
	(232.55, 64.04) --
	(232.61, 62.42) --
	(232.66, 67.54) --
	(232.71, 68.13) --
	(232.76, 65.16) --
	(232.81, 62.60) --
	(232.86, 65.47) --
	(232.91, 63.72) --
	(232.96, 65.47) --
	(233.01, 62.06) --
	(233.07, 65.34) --
	(233.12, 61.07) --
	(233.17, 69.92) --
	(233.22, 65.12) --
	(233.27, 64.22) --
	(233.32, 66.82) --
	(233.37, 63.27) --
	(233.42, 69.97) --
	(233.48, 66.24) --
	(233.53, 64.71) --
	(233.58, 66.15) --
	(233.63, 66.10) --
	(233.68, 66.55) --
	(233.73, 66.15) --
	(233.78, 64.98) --
	(233.83, 66.46) --
	(233.89, 63.27) --
	(233.94, 64.31) --
	(233.99, 66.64) --
	(234.04, 63.81) --
	(234.09, 61.79) --
	(234.14, 64.67) --
	(234.19, 62.78) --
	(234.24, 63.00) --
	(234.29, 63.00) --
	(234.35, 64.44) --
	(234.40, 62.02) --
	(234.45, 66.28) --
	(234.50, 66.69) --
	(234.55, 64.80) --
	(234.60, 66.42) --
	(234.65, 67.09) --
	(234.70, 62.37) --
	(234.76, 65.56) --
	(234.81, 65.92) --
	(234.86, 64.04) --
	(234.91, 61.84) --
	(234.96, 62.64) --
	(235.01, 64.71) --
	(235.06, 61.93) --
	(235.11, 61.88) --
	(235.17, 65.79) --
	(235.22, 64.89) --
	(235.27, 64.44) --
	(235.32, 67.77) --
	(235.37, 64.58) --
	(235.42, 65.70) --
	(235.47, 66.37) --
	(235.52, 64.71) --
	(235.57, 64.85) --
	(235.63, 64.17) --
	(235.68, 62.91) --
	(235.73, 66.01) --
	(235.78, 64.17) --
	(235.83, 64.62) --
	(235.88, 66.24) --
	(235.93, 64.17) --
	(235.98, 66.69) --
	(236.04, 70.10) --
	(236.09, 63.23) --
	(236.14, 64.04) --
	(236.19, 63.54) --
	(236.24, 64.94) --
	(236.29, 63.86) --
	(236.34, 63.36) --
	(236.39, 62.42) --
	(236.45, 65.03) --
	(236.50, 62.06) --
	(236.55, 62.82) --
	(236.60, 62.82) --
	(236.65, 64.80) --
	(236.70, 62.46) --
	(236.75, 69.88) --
	(236.80, 61.30) --
	(236.85, 64.76) --
	(236.91, 64.17) --
	(236.96, 61.25) --
	(237.01, 65.16) --
	(237.06, 68.57) --
	(237.11, 61.70) --
	(237.16, 64.53) --
	(237.21, 64.62) --
	(237.26, 64.40) --
	(237.32, 63.05) --
	(237.37, 65.88) --
	(237.42, 65.74) --
	(237.47, 65.52) --
	(237.52, 63.77) --
	(237.57, 67.86) --
	(237.62, 66.15) --
	(237.67, 62.46) --
	(237.73, 63.36) --
	(237.78, 65.25) --
	(237.83, 59.77) --
	(237.88, 63.59) --
	(237.93, 63.45) --
	(237.98, 64.31) --
	(238.03, 66.24) --
	(238.08, 66.64) --
	(238.13, 63.77) --
	(238.19, 59.23) --
	(238.24, 60.58) --
	(238.29, 67.45) --
	(238.34, 62.64) --
	(238.39, 68.80) --
	(238.44, 66.28) --
	(238.49, 63.05) --
	(238.54, 66.15) --
	(238.60, 66.55) --
	(238.65, 64.76) --
	(238.70, 63.05) --
	(238.75, 64.49) --
	(238.80, 67.95) --
	(238.85, 62.69) --
	(238.90, 66.96) --
	(238.95, 64.53) --
	(239.01, 61.25) --
	(239.06, 63.81) --
	(239.11, 63.32) --
	(239.16, 62.78) --
	(239.21, 63.86) --
	(239.26, 65.97) --
	(239.31, 64.35) --
	(239.36, 63.63) --
	(239.41, 64.31) --
	(239.47, 62.06) --
	(239.52, 63.36) --
	(239.57, 59.99) --
	(239.62, 64.22) --
	(239.67, 67.32) --
	(239.72, 64.67) --
	(239.77, 63.59) --
	(239.82, 63.23) --
	(239.88, 62.15) --
	(239.93, 62.87) --
	(239.98, 62.96) --
	(240.03, 62.82) --
	(240.08, 64.49) --
	(240.13, 62.24) --
	(240.18, 66.73) --
	(240.23, 61.93) --
	(240.29, 64.26) --
	(240.34, 68.89) --
	(240.39, 61.25) --
	(240.44, 63.27) --
	(240.49, 63.86) --
	(240.54, 64.31) --
	(240.59, 60.04) --
	(240.64, 65.65) --
	(240.69, 63.99) --
	(240.75, 67.00) --
	(240.80, 65.12) --
	(240.85, 63.23) --
	(240.90, 63.32) --
	(240.95, 65.12) --
	(241.00, 63.41) --
	(241.05, 63.41) --
	(241.10, 66.55) --
	(241.16, 62.29) --
	(241.21, 62.24) --
	(241.26, 63.23) --
	(241.31, 64.22) --
	(241.36, 61.21) --
	(241.41, 68.53) --
	(241.46, 64.13) --
	(241.51, 63.18) --
	(241.57, 64.85) --
	(241.62, 61.48) --
	(241.67, 57.16) --
	(241.72, 65.56) --
	(241.77, 62.69) --
	(241.82, 60.13) --
	(241.87, 64.58) --
	(241.92, 62.91) --
	(241.97, 62.69) --
	(242.03, 61.84) --
	(242.08, 61.12) --
	(242.13, 64.44) --
	(242.18, 63.50) --
	(242.23, 63.77) --
	(242.28, 61.16) --
	(242.33, 60.85) --
	(242.38, 63.41) --
	(242.44, 65.83) --
	(242.49, 62.82) --
	(242.54, 63.81) --
	(242.59, 68.04) --
	(242.64, 62.02) --
	(242.69, 64.17) --
	(242.74, 66.01) --
	(242.79, 62.91) --
	(242.85, 61.79) --
	(242.90, 63.81) --
	(242.95, 61.61) --
	(243.00, 57.16) --
	(243.05, 63.00) --
	(243.10, 61.97) --
	(243.15, 62.55) --
	(243.20, 65.29) --
	(243.25, 64.31) --
	(243.31, 62.06) --
	(243.36, 68.89) --
	(243.41, 63.68) --
	(243.46, 61.52) --
	(243.51, 64.49) --
	(243.56, 62.64) --
	(243.61, 64.67) --
	(243.66, 69.38) --
	(243.72, 61.88) --
	(243.77, 64.98) --
	(243.82, 63.72) --
	(243.87, 58.51) --
	(243.92, 65.43) --
	(243.97, 65.88) --
	(244.02, 59.59) --
	(244.07, 65.47) --
	(244.13, 63.81) --
	(244.18, 63.72) --
	(244.23, 63.27) --
	(244.28, 65.56) --
	(244.33, 63.27) --
	(244.38, 68.44) --
	(244.43, 63.27) --
	(244.48, 64.49) --
	(244.53, 62.11) --
	(244.59, 69.43) --
	(244.64, 60.22) --
	(244.69, 62.33) --
	(244.74, 63.99) --
	(244.79, 63.68) --
	(244.84, 63.41) --
	(244.89, 64.62) --
	(244.94, 61.84) --
	(245.00, 66.33) --
	(245.05, 65.16) --
	(245.10, 61.93) --
	(245.15, 59.41) --
	(245.20, 66.01) --
	(245.25, 62.37) --
	(245.30, 61.75) --
	(245.35, 63.54) --
	(245.41, 61.43) --
	(245.46, 64.49) --
	(245.51, 65.43) --
	(245.56, 65.47) --
	(245.61, 62.02) --
	(245.66, 62.73) --
	(245.71, 66.46) --
	(245.76, 61.03) --
	(245.81, 61.66) --
	(245.87, 65.56) --
	(245.92, 66.64) --
	(245.97, 59.41) --
	(246.02, 61.70) --
	(246.07, 66.15) --
	(246.12, 58.47) --
	(246.17, 64.04) --
	(246.22, 63.68) --
	(246.28, 61.79) --
	(246.33, 61.12) --
	(246.38, 64.58) --
	(246.43, 60.89) --
	(246.48, 62.87) --
	(246.53, 62.91) --
	(246.58, 62.87) --
	(246.63, 59.32) --
	(246.69, 64.98) --
	(246.74, 63.14) --
	(246.79, 61.34) --
	(246.84, 60.17) --
	(246.89, 63.63) --
	(246.94, 61.57) --
	(246.99, 59.28) --
	(247.04, 61.66) --
	(247.09, 63.72) --
	(247.15, 60.58) --
	(247.20, 64.62) --
	(247.25, 66.10) --
	(247.30, 60.76) --
	(247.35, 64.22) --
	(247.40, 63.99) --
	(247.45, 64.08) --
	(247.50, 58.51) --
	(247.56, 64.22) --
	(247.61, 60.80) --
	(247.66, 62.37) --
	(247.71, 62.82) --
	(247.76, 62.42) --
	(247.81, 59.59) --
	(247.86, 64.85) --
	(247.91, 63.99) --
	(247.97, 61.48) --
	(248.02, 62.42) --
	(248.07, 61.30) --
	(248.12, 62.29) --
	(248.17, 61.34) --
	(248.22, 62.87) --
	(248.27, 61.43) --
	(248.32, 63.77) --
	(248.37, 62.33) --
	(248.43, 64.40) --
	(248.48, 63.09) --
	(248.53, 59.05) --
	(248.58, 66.19) --
	(248.63, 59.72) --
	(248.68, 66.78) --
	(248.73, 63.09) --
	(248.78, 62.20) --
	(248.84, 64.44) --
	(248.89, 60.44) --
	(248.94, 64.89) --
	(248.99, 66.01) --
	(249.04, 60.26) --
	(249.09, 65.70) --
	(249.14, 66.46) --
	(249.19, 61.30) --
	(249.25, 61.84) --
	(249.30, 63.50) --
	(249.35, 61.07) --
	(249.40, 57.88) --
	(249.45, 65.07) --
	(249.50, 58.83) --
	(249.55, 60.76) --
	(249.60, 60.76) --
	(249.65, 61.34) --
	(249.71, 62.73) --
	(249.76, 62.64) --
	(249.81, 64.49) --
	(249.86, 61.66) --
	(249.91, 61.88) --
	(249.96, 61.30) --
	(250.01, 59.28) --
	(250.06, 62.96) --
	(250.12, 61.75) --
	(250.17, 63.27) --
	(250.22, 58.87) --
	(250.27, 64.58) --
	(250.32, 62.24) --
	(250.37, 60.26) --
	(250.42, 64.04) --
	(250.47, 64.53) --
	(250.53, 62.02) --
	(250.58, 62.87) --
	(250.63, 61.07) --
	(250.68, 61.66) --
	(250.73, 65.88) --
	(250.78, 60.76) --
	(250.83, 55.91) --
	(250.88, 63.54) --
	(250.94, 59.10) --
	(250.99, 60.85) --
	(251.04, 65.43) --
	(251.09, 59.68) --
	(251.14, 60.53) --
	(251.19, 64.85) --
	(251.24, 63.50) --
	(251.29, 62.82) --
	(251.34, 63.45) --
	(251.40, 62.46) --
	(251.45, 63.41) --
	(251.50, 62.20) --
	(251.55, 60.85) --
	(251.60, 64.31) --
	(251.65, 59.54) --
	(251.70, 58.96) --
	(251.75, 61.25) --
	(251.81, 60.31) --
	(251.86, 63.68) --
	(251.91, 63.54) --
	(251.96, 65.65) --
	(252.01, 59.54) --
	(252.06, 65.03) --
	(252.11, 61.61) --
	(252.16, 63.36) --
	(252.22, 58.92) --
	(252.27, 63.54) --
	(252.32, 59.99) --
	(252.37, 65.16) --
	(252.42, 62.96) --
	(252.47, 59.95) --
	(252.52, 61.88) --
	(252.57, 66.19) --
	(252.62, 61.61) --
	(252.68, 60.67) --
	(252.73, 65.65) --
	(252.78, 58.56) --
	(252.83, 64.49) --
	(252.88, 63.59) --
	(252.93, 61.61) --
	(252.98, 63.81) --
	(253.03, 60.44) --
	(253.09, 60.76) --
	(253.14, 61.79) --
	(253.19, 62.60) --
	(253.24, 62.60) --
	(253.29, 59.72) --
	(253.34, 62.73) --
	(253.39, 61.84) --
	(253.44, 60.22) --
	(253.50, 66.55) --
	(253.55, 63.00) --
	(253.60, 60.98) --
	(253.65, 63.50) --
	(253.70, 60.44) --
	(253.75, 63.50) --
	(253.80, 61.16) --
	(253.85, 61.84) --
	(253.90, 64.49) --
	(253.96, 62.51) --
	(254.01, 67.68) --
	(254.06, 63.32) --
	(254.11, 58.74) --
	(254.16, 60.40) --
	(254.21, 65.56) --
	(254.26, 62.87) --
	(254.31, 61.34) --
	(254.37, 59.81) --
	(254.42, 61.66) --
	(254.47, 63.59) --
	(254.52, 63.72) --
	(254.57, 63.95) --
	(254.62, 61.16) --
	(254.67, 60.53) --
	(254.72, 61.75) --
	(254.78, 59.28) --
	(254.83, 60.98) --
	(254.88, 59.23) --
	(254.93, 59.59) --
	(254.98, 63.09) --
	(255.03, 62.46) --
	(255.08, 61.79) --
	(255.13, 57.43) --
	(255.18, 62.69) --
	(255.24, 61.48) --
	(255.29, 58.51) --
	(255.34, 59.86) --
	(255.39, 59.54) --
	(255.44, 64.89) --
	(255.49, 57.16) --
	(255.54, 57.61) --
	(255.59, 62.29) --
	(255.65, 61.39) --
	(255.70, 58.11) --
	(255.75, 66.01) --
	(255.80, 62.73) --
	(255.85, 58.83) --
	(255.90, 63.23) --
	(255.95, 64.22) --
	(256.00, 59.19) --
	(256.06, 56.71) --
	(256.11, 55.10) --
	(256.16, 65.16) --
	(256.21, 61.34) --
	(256.26, 58.15) --
	(256.31, 62.87) --
	(256.36, 60.89) --
	(256.41, 59.77) --
	(256.46, 61.84) --
	(256.52, 63.09) --
	(256.57, 60.98) --
	(256.62, 60.08) --
	(256.67, 61.61) --
	(256.72, 62.06) --
	(256.77, 59.81) --
	(256.82, 64.94) --
	(256.87, 59.68) --
	(256.93, 59.28) --
	(256.98, 63.68) --
	(257.03, 60.98) --
	(257.08, 58.74) --
	(257.13, 57.61) --
	(257.18, 60.58) --
	(257.23, 60.53) --
	(257.28, 63.18) --
	(257.34, 63.86) --
	(257.39, 61.43) --
	(257.44, 57.57) --
	(257.49, 62.46) --
	(257.54, 59.45) --
	(257.59, 59.54) --
	(257.64, 60.22) --
	(257.69, 59.90) --
	(257.74, 63.54) --
	(257.80, 62.87) --
	(257.85, 57.03) --
	(257.90, 65.07) --
	(257.95, 65.56) --
	(258.00, 59.95) --
	(258.05, 58.47) --
	(258.10, 62.42) --
	(258.15, 64.35) --
	(258.21, 60.76) --
	(258.26, 63.41) --
	(258.31, 59.90) --
	(258.36, 62.82) --
	(258.41, 62.06) --
	(258.46, 61.52) --
	(258.51, 61.88) --
	(258.56, 59.45) --
	(258.62, 55.55) --
	(258.67, 64.98) --
	(258.72, 63.45) --
	(258.77, 55.86) --
	(258.82, 62.60) --
	(258.87, 64.40) --
	(258.92, 58.11) --
	(258.97, 66.73) --
	(259.02, 61.16) --
	(259.08, 57.43) --
	(259.13, 61.12) --
	(259.18, 64.26) --
	(259.23, 59.32) --
	(259.28, 60.44) --
	(259.33, 60.44) --
	(259.38, 60.71) --
	(259.43, 62.42) --
	(259.49, 55.91) --
	(259.54, 63.77) --
	(259.59, 64.53) --
	(259.64, 56.80) --
	(259.69, 56.53) --
	(259.74, 65.38) --
	(259.79, 65.03) --
	(259.84, 59.59) --
	(259.90, 64.85) --
	(259.95, 61.52) --
	(260.00, 61.88) --
	(260.05, 55.73) --
	(260.10, 58.69) --
	(260.15, 59.95) --
	(260.20, 58.24) --
	(260.25, 61.52) --
	(260.30, 62.20) --
	(260.36, 58.06) --
	(260.41, 58.47) --
	(260.46, 62.55) --
	(260.51, 58.83) --
	(260.56, 63.27) --
	(260.61, 64.89) --
	(260.66, 61.48) --
	(260.71, 58.69) --
	(260.77, 61.75) --
	(260.82, 61.61) --
	(260.87, 63.68) --
	(260.92, 60.80) --
	(260.97, 55.05) --
	(261.02, 61.88) --
	(261.07, 60.94) --
	(261.12, 57.16) --
	(261.18, 62.06) --
	(261.23, 62.91) --
	(261.28, 55.10) --
	(261.33, 52.85) --
	(261.38, 63.86) --
	(261.43, 60.40) --
	(261.48, 61.75) --
	(261.53, 57.07) --
	(261.58, 58.02) --
	(261.64, 59.77) --
	(261.69, 61.84) --
	(261.74, 60.71) --
	(261.79, 62.20) --
	(261.84, 55.86) --
	(261.89, 59.14) --
	(261.94, 63.27) --
	(261.99, 61.30) --
	(262.05, 55.64) --
	(262.10, 64.49) --
	(262.15, 59.23) --
	(262.20, 61.07) --
	(262.25, 54.51) --
	(262.30, 63.54) --
	(262.35, 59.01) --
	(262.40, 57.48) --
	(262.46, 62.69) --
	(262.51, 62.46) --
	(262.56, 56.67) --
	(262.61, 55.95) --
	(262.66, 63.23) --
	(262.71, 60.22) --
	(262.76, 58.20) --
	(262.81, 61.93) --
	(262.86, 59.59) --
	(262.92, 60.08) --
	(262.97, 63.41) --
	(263.02, 60.26) --
	(263.07, 58.92) --
	(263.12, 57.12) --
	(263.17, 60.44) --
	(263.22, 58.69) --
	(263.27, 60.22) --
	(263.33, 58.65) --
	(263.38, 59.45) --
	(263.43, 56.58) --
	(263.48, 58.69) --
	(263.53, 63.90) --
	(263.58, 56.53) --
	(263.63, 59.10) --
	(263.68, 59.90) --
	(263.74, 57.61) --
	(263.79, 58.83) --
	(263.84, 58.96) --
	(263.89, 62.20) --
	(263.94, 61.25) --
	(263.99, 59.90) --
	(264.04, 57.70) --
	(264.09, 59.72) --
	(264.14, 58.60) --
	(264.20, 59.95) --
	(264.25, 57.34) --
	(264.30, 54.51) --
	(264.35, 59.36) --
	(264.40, 63.90) --
	(264.45, 59.68) --
	(264.50, 60.17) --
	(264.55, 57.88) --
	(264.61, 59.95) --
	(264.66, 61.66) --
	(264.71, 58.20) --
	(264.76, 60.89) --
	(264.81, 63.77) --
	(264.86, 58.96) --
	(264.91, 59.41) --
	(264.96, 62.73) --
	(265.02, 58.51) --
	(265.07, 52.40) --
	(265.12, 60.04) --
	(265.17, 62.55) --
	(265.22, 62.51) --
	(265.27, 59.45) --
	(265.32, 61.75) --
	(265.37, 61.39) --
	(265.42, 62.37) --
	(265.48, 56.85) --
	(265.53, 59.68) --
	(265.58, 60.94) --
	(265.63, 60.22) --
	(265.68, 57.79) --
	(265.73, 57.21) --
	(265.78, 60.22) --
	(265.83, 61.30) --
	(265.89, 58.87) --
	(265.94, 59.19) --
	(265.99, 61.61) --
	(266.04, 55.77) --
	(266.09, 60.98) --
	(266.14, 61.61) --
	(266.19, 54.69) --
	(266.24, 62.64) --
	(266.30, 60.62) --
	(266.35, 58.56) --
	(266.40, 60.67) --
	(266.45, 58.83) --
	(266.50, 61.97) --
	(266.55, 56.13) --
	(266.60, 57.43) --
	(266.65, 63.18) --
	(266.70, 62.37) --
	(266.76, 59.05) --
	(266.81, 63.86) --
	(266.86, 60.26) --
	(266.91, 53.08) --
	(266.96, 54.56) --
	(267.01, 56.18) --
	(267.06, 59.95) --
	(267.11, 58.69) --
	(267.17, 57.48) --
	(267.22, 57.03) --
	(267.27, 56.22) --
	(267.32, 56.00) --
	(267.37, 64.62) --
	(267.42, 60.80) --
	(267.47, 58.92) --
	(267.52, 63.00) --
	(267.58, 62.69) --
	(267.63, 58.38) --
	(267.68, 61.43) --
	(267.73, 60.44) --
	(267.78, 61.48) --
	(267.83, 61.57) --
	(267.88, 57.12) --
	(267.93, 53.75) --
	(267.98, 57.03) --
	(268.04, 64.31) --
	(268.09, 59.28) --
	(268.14, 58.69) --
	(268.19, 58.15) --
	(268.24, 55.32) --
	(268.29, 60.71) --
	(268.34, 59.81) --
	(268.39, 59.41) --
	(268.45, 61.07) --
	(268.50, 61.25) --
	(268.55, 61.25) --
	(268.60, 58.38) --
	(268.65, 57.88) --
	(268.70, 60.85) --
	(268.75, 59.81) --
	(268.80, 56.40) --
	(268.86, 55.37) --
	(268.91, 57.88) --
	(268.96, 59.45) --
	(269.01, 57.48) --
	(269.06, 59.10) --
	(269.11, 62.06) --
	(269.16, 57.97) --
	(269.21, 56.58) --
	(269.26, 59.10) --
	(269.32, 57.30) --
	(269.37, 57.07) --
	(269.42, 56.22) --
	(269.47, 61.39) --
	(269.52, 59.36) --
	(269.57, 55.55) --
	(269.62, 54.60) --
	(269.67, 55.91) --
	(269.73, 58.92) --
	(269.78, 59.05) --
	(269.83, 56.04) --
	(269.88, 54.74) --
	(269.93, 63.36) --
	(269.98, 59.41) --
	(270.03, 56.49) --
	(270.08, 58.60) --
	(270.14, 60.53) --
	(270.19, 57.16) --
	(270.24, 56.36) --
	(270.29, 54.42) --
	(270.34, 59.19) --
	(270.39, 60.89) --
	(270.44, 55.01) --
	(270.49, 56.22) --
	(270.54, 59.23) --
	(270.60, 60.85) --
	(270.65, 63.72) --
	(270.70, 55.64) --
	(270.75, 57.79) --
	(270.80, 59.10) --
	(270.85, 60.89) --
	(270.90, 56.80) --
	(270.95, 59.45) --
	(271.01, 59.14) --
	(271.06, 59.59) --
	(271.11, 55.28) --
	(271.16, 56.22) --
	(271.21, 56.44) --
	(271.26, 58.56) --
	(271.31, 60.31) --
	(271.36, 54.33) --
	(271.42, 55.10) --
	(271.47, 54.78) --
	(271.52, 53.57) --
	(271.57, 54.78) --
	(271.62, 57.52) --
	(271.67, 58.42) --
	(271.72, 57.39) --
	(271.77, 53.93) --
	(271.82, 51.73) --
	(271.88, 55.01) --
	(271.93, 63.59) --
	(271.98, 60.08) --
	(272.03, 57.48) --
	(272.08, 59.86) --
	(272.13, 60.26) --
	(272.18, 59.28) --
	(272.23, 57.43) --
	(272.29, 59.28) --
	(272.34, 57.88) --
	(272.39, 57.12) --
	(272.44, 55.50) --
	(272.49, 53.26) --
	(272.54, 58.74) --
	(272.59, 59.28) --
	(272.64, 57.88) --
	(272.70, 55.19) --
	(272.75, 55.82) --
	(272.80, 59.45) --
	(272.85, 58.11) --
	(272.90, 56.00) --
	(272.95, 55.28) --
	(273.00, 61.12) --
	(273.05, 61.79) --
	(273.10, 56.36) --
	(273.16, 52.90) --
	(273.21, 60.13) --
	(273.26, 57.57) --
	(273.31, 56.89) --
	(273.36, 53.03) --
	(273.41, 51.23) --
	(273.46, 57.52) --
	(273.51, 63.36) --
	(273.57, 56.22) --
	(273.62, 55.05) --
	(273.67, 58.29) --
	(273.72, 55.10) --
	(273.77, 51.77) --
	(273.82, 58.47) --
	(273.87, 59.72) --
	(273.92, 57.03) --
	(273.98, 58.29) --
	(274.03, 58.78) --
	(274.08, 57.25) --
	(274.13, 53.88) --
	(274.18, 56.62) --
	(274.23, 62.55) --
	(274.28, 59.59) --
	(274.33, 58.83) --
	(274.38, 57.16) --
	(274.44, 57.07) --
	(274.49, 59.41) --
	(274.54, 57.48) --
	(274.59, 60.67) --
	(274.64, 57.75) --
	(274.69, 59.77) --
	(274.74, 57.30) --
	(274.79, 56.58) --
	(274.85, 55.73) --
	(274.90, 61.84) --
	(274.95, 58.24) --
	(275.00, 56.00) --
	(275.05, 59.95) --
	(275.10, 59.10) --
	(275.15, 58.60) --
	(275.20, 59.77) --
	(275.26, 59.41) --
	(275.31, 57.34) --
	(275.36, 56.36) --
	(275.41, 58.87) --
	(275.46, 58.87) --
	(275.51, 55.68) --
	(275.56, 58.29) --
	(275.61, 57.61) --
	(275.66, 56.31) --
	(275.72, 52.04) --
	(275.77, 56.22) --
	(275.82, 63.54) --
	(275.87, 57.12) --
	(275.92, 53.79) --
	(275.97, 63.72) --
	(276.02, 57.52) --
	(276.07, 52.04) --
	(276.13, 55.19) --
	(276.18, 58.83) --
	(276.23, 57.07) --
	(276.28, 54.96) --
	(276.33, 54.92) --
	(276.38, 54.92) --
	(276.43, 51.32) --
	(276.48, 52.13) --
	(276.54, 56.49) --
	(276.59, 56.53) --
	(276.64, 56.31) --
	(276.69, 51.64) --
	(276.74, 52.81) --
	(276.79, 54.78) --
	(276.84, 59.72) --
	(276.89, 54.15) --
	(276.95, 52.58) --
	(277.00, 56.22) --
	(277.05, 52.94) --
	(277.10, 57.79) --
	(277.15, 57.39) --
	(277.20, 62.24) --
	(277.25, 58.74) --
	(277.30, 58.96) --
	(277.35, 54.20) --
	(277.41, 49.80) --
	(277.46, 51.32) --
	(277.51, 57.84) --
	(277.56, 55.68) --
	(277.61, 49.35) --
	(277.66, 51.32) --
	(277.71, 52.76) --
	(277.76, 59.90) --
	(277.82, 62.46) --
	(277.87, 51.82) --
	(277.92, 54.15) --
	(277.97, 57.07) --
	(278.02, 57.43) --
	(278.07, 57.07) --
	(278.12, 53.21) --
	(278.17, 47.33) --
	(278.23, 51.55) --
	(278.28, 50.02) --
	(278.33, 56.36) --
	(278.38, 60.53) --
	(278.43, 59.72) --
	(278.48, 56.00) --
	(278.53, 51.68) --
	(278.58, 59.14) --
	(278.63, 56.18) --
	(278.69, 53.88) --
	(278.74, 55.23) --
	(278.79, 49.44) --
	(278.84, 51.59) --
	(278.89, 52.72) --
	(278.94, 57.52) --
	(278.99, 57.66) --
	(279.04, 59.63) --
	(279.10, 54.38) --
	(279.15, 53.88) --
	(279.20, 51.82) --
	(279.25, 49.39) --
	(279.30, 54.42) --
	(279.35, 53.48) --
	(279.40, 54.38) --
	(279.45, 56.53) --
	(279.51, 54.65) --
	(279.56, 51.82) --
	(279.61, 52.76) --
	(279.66, 56.40) --
	(279.71, 53.43) --
	(279.76, 53.48) --
	(279.81, 51.19) --
	(279.86, 52.76) --
	(279.91, 57.30) --
	(279.97, 57.30) --
	(280.02, 57.03) --
	(280.07, 51.37) --
	(280.12, 55.28) --
	(280.17, 56.71) --
	(280.22, 50.02) --
	(280.27, 56.09) --
	(280.32, 56.62) --
	(280.38, 53.70) --
	(280.43, 55.46) --
	(280.48, 56.89) --
	(280.53, 55.23) --
	(280.58, 52.54) --
	(280.63, 53.57) --
	(280.68, 53.12) --
	(280.73, 50.65) --
	(280.79, 51.59) --
	(280.84, 56.36) --
	(280.89, 57.16) --
	(280.94, 53.43) --
	(280.99, 55.10) --
	(281.04, 53.08) --
	(281.09, 56.80) --
	(281.14, 54.47) --
	(281.19, 51.59) --
	(281.25, 51.46) --
	(281.30, 50.25) --
	(281.35, 53.17) --
	(281.40, 50.51) --
	(281.45, 49.66) --
	(281.50, 55.19) --
	(281.55, 58.87) --
	(281.60, 54.60) --
	(281.66, 49.26) --
	(281.71, 48.40) --
	(281.76, 53.21) --
	(281.81, 55.73) --
	(281.86, 54.60) --
	(281.91, 51.95) --
	(281.96, 49.71) --
	(282.01, 53.97) --
	(282.07, 52.36) --
	(282.12, 53.08) --
	(282.17, 51.50) --
	(282.22, 51.86) --
	(282.27, 57.52) --
	(282.32, 53.61) --
	(282.37, 53.79) --
	(282.42, 52.09) --
	(282.47, 60.62) --
	(282.53, 53.21) --
	(282.58, 50.29) --
	(282.63, 53.35) --
	(282.68, 59.50) --
	(282.73, 60.58) --
	(282.78, 55.86) --
	(282.83, 59.77) --
	(282.88, 54.51) --
	(282.94, 55.32) --
	(282.99, 57.43) --
	(283.04, 53.03) --
	(283.09, 49.66) --
	(283.14, 57.12) --
	(283.19, 58.74) --
	(283.24, 55.01) --
	(283.29, 54.06) --
	(283.35, 54.47) --
	(283.40, 54.29) --
	(283.45, 58.78) --
	(283.50, 58.15) --
	(283.55, 56.00) --
	(283.60, 54.11) --
	(283.65, 60.08) --
	(283.70, 55.14) --
	(283.75, 53.30) --
	(283.81, 50.65) --
	(283.86, 49.26) --
	(283.91, 54.83) --
	(283.96, 57.48) --
	(284.01, 54.96) --
	(284.06, 52.27) --
	(284.11, 50.20) --
	(284.16, 50.29) --
	(284.22, 48.99) --
	(284.27, 59.54) --
	(284.32, 55.86) --
	(284.37, 54.87) --
	(284.42, 57.03) --
	(284.47, 58.24) --
	(284.52, 56.71) --
	(284.57, 55.86) --
	(284.63, 53.39) --
	(284.68, 52.63) --
	(284.73, 52.00) --
	(284.78, 55.14) --
	(284.83, 57.25) --
	(284.88, 55.77) --
	(284.93, 51.23) --
	(284.98, 55.50) --
	(285.03, 56.04) --
	(285.09, 53.57) --
	(285.14, 52.76) --
	(285.19, 55.55) --
	(285.24, 56.49) --
	(285.29, 51.95) --
	(285.34, 49.30) --
	(285.39, 52.94) --
	(285.44, 52.72) --
	(285.50, 51.01) --
	(285.55, 54.51) --
	(285.60, 57.61) --
	(285.65, 56.49) --
	(285.70, 53.12) --
	(285.75, 55.14) --
	(285.80, 50.25) --
	(285.85, 49.08) --
	(285.91, 49.17) --
	(285.96, 52.18) --
	(286.01, 46.97) --
	(286.06, 49.75) --
	(286.11, 50.43) --
	(286.16, 48.36) --
	(286.21, 51.10) --
	(286.26, 57.70) --
	(286.31, 54.29) --
	(286.37, 49.80) --
	(286.42, 53.35) --
	(286.47, 59.36) --
	(286.52, 54.65) --
	(286.57, 51.73) --
	(286.62, 47.55) --
	(286.67, 49.53) --
	(286.72, 52.00) --
	(286.78, 54.20) --
	(286.83, 55.64) --
	(286.88, 56.85) --
	(286.93, 49.66) --
	(286.98, 53.52) --
	(287.03, 52.85) --
	(287.08, 51.82) --
	(287.13, 47.01) --
	(287.19, 55.41) --
	(287.24, 52.76) --
	(287.29, 48.76) --
	(287.34, 49.57) --
	(287.39, 47.42) --
	(287.44, 48.13) --
	(287.49, 45.08) --
	(287.54, 49.89) --
	(287.59, 53.61) --
	(287.65, 50.60) --
	(287.70, 50.07) --
	(287.75, 49.17) --
	(287.80, 53.39) --
	(287.85, 58.06) --
	(287.90, 58.78) --
	(287.95, 52.67) --
	(288.00, 53.12) --
	(288.06, 56.71) --
	(288.11, 55.01) --
	(288.16, 51.68) --
	(288.21, 49.89) --
	(288.26, 55.50) --
	(288.31, 56.18) --
	(288.36, 51.64) --
	(288.41, 48.31) --
	(288.47, 49.66) --
	(288.52, 51.77) --
	(288.57, 48.99) --
	(288.62, 52.58) --
	(288.67, 53.17) --
	(288.72, 49.71) --
	(288.77, 48.58) --
	(288.82, 48.31) --
	(288.87, 49.84) --
	(288.93, 54.60) --
	(288.98, 53.93) --
	(289.03, 50.96) --
	(289.08, 50.02) --
	(289.13, 51.14) --
	(289.18, 48.09) --
	(289.23, 45.53) --
	(289.28, 47.06) --
	(289.34, 56.76) --
	(289.39, 55.37) --
	(289.44, 54.20) --
	(289.49, 49.57) --
	(289.54, 50.74) --
	(289.59, 48.76) --
	(289.64, 49.08) --
	(289.69, 51.77) --
	(289.75, 52.63) --
	(289.80, 50.60) --
	(289.85, 50.02) --
	(289.90, 51.05) --
	(289.95, 49.17) --
	(290.00, 46.43) --
	(290.05, 52.00) --
	(290.10, 51.46) --
	(290.15, 50.51) --
	(290.21, 45.62) --
	(290.26, 51.91) --
	(290.31, 57.48) --
	(290.36, 52.58) --
	(290.41, 46.70) --
	(290.46, 48.99) --
	(290.51, 48.13) --
	(290.56, 45.89) --
	(290.62, 46.25) --
	(290.67, 48.31) --
	(290.72, 48.85) --
	(290.77, 50.07) --
	(290.82, 50.92) --
	(290.87, 49.93) --
	(290.92, 55.10) --
	(290.97, 48.67) --
	(291.03, 49.80) --
	(291.08, 52.54) --
	(291.13, 53.12) --
	(291.18, 53.03) --
	(291.23, 49.80) --
	(291.28, 50.74) --
	(291.33, 50.92) --
	(291.38, 51.55) --
	(291.43, 47.15) --
	(291.49, 48.54) --
	(291.54, 52.99) --
	(291.59, 50.74) --
	(291.64, 46.83) --
	(291.69, 51.23) --
	(291.74, 55.05) --
	(291.79, 51.91) --
	(291.84, 49.17) --
	(291.90, 45.71) --
	(291.95, 46.16) --
	(292.00, 47.77) --
	(292.05, 47.46) --
	(292.10, 44.36) --
	(292.15, 48.99) --
	(292.20, 48.85) --
	(292.25, 46.34) --
	(292.31, 49.12) --
	(292.36, 48.18) --
	(292.41, 52.63) --
	(292.46, 53.52) --
	(292.51, 49.71) --
	(292.56, 48.81) --
	(292.61, 46.74) --
	(292.66, 47.64) --
	(292.71, 48.94) --
	(292.77, 47.10) --
	(292.82, 46.52) --
	(292.87, 46.65) --
	(292.92, 47.42) --
	(292.97, 46.70) --
	(293.02, 47.06) --
	(293.07, 53.17) --
	(293.12, 52.99) --
	(293.18, 49.62) --
	(293.23, 50.25) --
	(293.28, 46.88) --
	(293.33, 46.56) --
	(293.38, 48.90) --
	(293.43, 46.25) --
	(293.48, 46.92) --
	(293.53, 48.40) --
	(293.59, 48.18) --
	(293.64, 44.45) --
	(293.69, 44.54) --
	(293.74, 48.49) --
	(293.79, 49.75) --
	(293.84, 49.30) --
	(293.89, 46.52) --
	(293.94, 48.72) --
	(293.99, 51.86) --
	(294.05, 50.74) --
	(294.10, 46.61) --
	(294.15, 47.19) --
	(294.20, 48.76) --
	(294.25, 48.00) --
	(294.30, 46.11) --
	(294.35, 46.25) --
	(294.40, 48.45) --
	(294.46, 50.25) --
	(294.51, 45.89) --
	(294.56, 44.45) --
	(294.61, 43.87) --
	(294.66, 49.30) --
	(294.71, 48.85) --
	(294.76, 46.83) --
	(294.81, 44.41) --
	(294.87, 46.29) --
	(294.92, 45.62) --
	(294.97, 50.60) --
	(295.02, 51.28) --
	(295.07, 51.41) --
	(295.12, 49.39) --
	(295.17, 46.88) --
	(295.22, 45.53) --
	(295.27, 49.03) --
	(295.33, 48.67) --
	(295.38, 48.99) --
	(295.43, 46.79) --
	(295.48, 49.08) --
	(295.53, 49.12) --
	(295.58, 44.36) --
	(295.63, 47.77) --
	(295.68, 47.91) --
	(295.74, 49.75) --
	(295.79, 49.35) --
	(295.84, 48.00) --
	(295.89, 48.67) --
	(295.94, 48.04) --
	(295.99, 46.56) --
	(296.04, 47.06) --
	(296.09, 46.38) --
	(296.15, 46.61) --
	(296.20, 43.78) --
	(296.25, 46.92) --
	(296.30, 48.22) --
	(296.35, 45.89) --
	(296.40, 44.94) --
	(296.45, 44.72) --
	(296.50, 48.81) --
	(296.55, 47.15) --
	(296.61, 43.69) --
	(296.66, 43.60) --
	(296.71, 50.65) --
	(296.76, 47.68) --
	(296.81, 45.66) --
	(296.86, 43.91) --
	(296.91, 50.38) --
	(296.96, 48.40) --
	(297.02, 45.17) --
	(297.07, 43.96) --
	(297.12, 50.43) --
	(297.17, 48.13) --
	(297.22, 49.93) --
	(297.27, 44.00) --
	(297.32, 46.07) --
	(297.37, 44.72) --
	(297.43, 46.65) --
	(297.48, 44.54) --
	(297.53, 45.30) --
	(297.58, 48.54) --
	(297.63, 49.75) --
	(297.68, 46.74) --
	(297.73, 43.01) --
	(297.78, 45.03) --
	(297.83, 46.07) --
	(297.89, 47.10) --
	(297.94, 46.02) --
	(297.99, 46.56) --
	(298.04, 46.16) --
	(298.09, 49.53) --
	(298.14, 44.00) --
	(298.19, 45.71) --
	(298.24, 51.01) --
	(298.30, 53.03) --
	(298.35, 50.83) --
	(298.40, 46.65) --
	(298.45, 43.91) --
	(298.50, 44.67) --
	(298.55, 45.57) --
	(298.60, 45.75) --
	(298.65, 45.17) --
	(298.71, 46.43) --
	(298.76, 46.65) --
	(298.81, 42.20) --
	(298.86, 45.48) --
	(298.91, 47.24) --
	(298.96, 45.66) --
	(299.01, 45.84) --
	(299.06, 43.69) --
	(299.11, 44.81) --
	(299.17, 47.15) --
	(299.22, 47.46) --
	(299.27, 43.60) --
	(299.32, 43.78) --
	(299.37, 46.20) --
	(299.42, 47.82) --
	(299.47, 44.09) --
	(299.52, 43.96) --
	(299.58, 44.94) --
	(299.63, 47.59) --
	(299.68, 46.92) --
	(299.73, 45.75) --
	(299.78, 42.16) --
	(299.83, 47.06) --
	(299.88, 45.75) --
	(299.93, 47.28) --
	(299.99, 44.54) --
	(300.04, 42.07) --
	(300.09, 45.35) --
	(300.14, 48.99) --
	(300.19, 46.88) --
	(300.24, 45.84) --
	(300.29, 43.15) --
	(300.34, 47.15) --
	(300.39, 47.10) --
	(300.45, 44.90) --
	(300.50, 45.53) --
	(300.55, 43.33) --
	(300.60, 42.70) --
	(300.65, 45.75) --
	(300.70, 46.92) --
	(300.75, 46.25) --
	(300.80, 45.21) --
	(300.86, 41.93) --
	(300.91, 44.27) --
	(300.96, 48.49) --
	(301.01, 44.18) --
	(301.06, 46.43) --
	(301.11, 48.00) --
	(301.16, 43.19) --
	(301.21, 41.31) --
	(301.27, 44.90) --
	(301.32, 44.99) --
	(301.37, 46.52) --
	(301.42, 46.92) --
	(301.47, 46.07) --
	(301.52, 42.47) --
	(301.57, 41.98) --
	(301.62, 42.97) --
	(301.67, 47.33) --
	(301.73, 46.52) --
	(301.78, 45.08) --
	(301.83, 42.11) --
	(301.88, 45.93) --
	(301.93, 46.97) --
	(301.98, 44.94) --
	(302.03, 42.83) --
	(302.08, 42.20) --
	(302.14, 43.55) --
	(302.19, 48.45) --
	(302.24, 49.12) --
	(302.29, 45.30) --
	(302.34, 42.47) --
	(302.39, 43.19) --
	(302.44, 43.51) --
	(302.49, 44.14) --
	(302.55, 43.51) --
	(302.60, 44.50) --
	(302.65, 45.26) --
	(302.70, 45.26) --
	(302.75, 44.45) --
	(302.80, 41.89) --
	(302.85, 43.15) --
	(302.90, 45.89) --
	(302.96, 45.84) --
	(303.01, 46.47) --
	(303.06, 42.70) --
	(303.11, 45.26) --
	(303.16, 45.03) --
	(303.21, 45.17) --
	(303.26, 44.00) --
	(303.31, 44.14) --
	(303.36, 42.11) --
	(303.42, 45.39) --
	(303.47, 42.52) --
	(303.52, 43.73) --
	(303.57, 43.87) --
	(303.62, 42.65) --
	(303.67, 46.20) --
	(303.72, 47.51) --
	(303.77, 46.34) --
	(303.83, 42.29) --
	(303.88, 45.39) --
	(303.93, 45.30) --
	(303.98, 44.05) --
	(304.03, 44.90) --
	(304.08, 43.19) --
	(304.13, 44.23) --
	(304.18, 42.92) --
	(304.24, 42.25) --
	(304.29, 48.18) --
	(304.34, 45.48) --
	(304.39, 47.77) --
	(304.44, 44.63) --
	(304.49, 44.90) --
	(304.54, 40.81) --
	(304.59, 44.09) --
	(304.64, 46.11) --
	(304.70, 44.32) --
	(304.75, 43.55) --
	(304.80, 42.25) --
	(304.85, 40.81) --
	(304.90, 44.14) --
	(304.95, 47.91) --
	(305.00, 47.28) --
	(305.05, 46.88) --
	(305.11, 42.97) --
	(305.16, 41.80) --
	(305.21, 44.67) --
	(305.26, 47.64) --
	(305.31, 46.79) --
	(305.36, 46.70) --
	(305.41, 46.20) --
	(305.46, 48.54) --
	(305.52, 45.30) --
	(305.57, 43.10) --
	(305.62, 44.54) --
	(305.67, 48.04) --
	(305.72, 48.00) --
	(305.77, 44.81) --
	(305.82, 43.46) --
	(305.87, 41.13) --
	(305.92, 44.09) --
	(305.98, 43.46) --
	(306.03, 45.21) --
	(306.08, 44.90) --
	(306.13, 43.55) --
	(306.18, 43.46) --
	(306.23, 45.75) --
	(306.28, 43.10) --
	(306.33, 42.83) --
	(306.39, 41.66) --
	(306.44, 41.93) --
	(306.49, 43.91) --
	(306.54, 43.42) --
	(306.59, 41.53) --
	(306.64, 45.66) --
	(306.69, 41.13) --
	(306.74, 41.93) --
	(306.80, 46.70) --
	(306.85, 46.02) --
	(306.90, 44.23) --
	(306.95, 42.74) --
	(307.00, 43.87) --
	(307.05, 42.34) --
	(307.10, 42.92) --
	(307.15, 44.41) --
	(307.20, 44.58) --
	(307.26, 40.95) --
	(307.31, 44.85) --
	(307.36, 42.97) --
	(307.41, 41.80) --
	(307.46, 43.73) --
	(307.51, 46.65) --
	(307.56, 45.66) --
	(307.61, 42.02) --
	(307.67, 40.72) --
	(307.72, 43.06) --
	(307.77, 44.41) --
	(307.82, 44.81) --
	(307.87, 42.20) --
	(307.92, 41.93) --
	(307.97, 44.14) --
	(308.02, 42.74) --
	(308.08, 43.82) --
	(308.13, 42.25) --
	(308.18, 44.18) --
	(308.23, 43.64) --
	(308.28, 43.60) --
	(308.33, 41.22) --
	(308.38, 45.12) --
	(308.43, 46.56) --
	(308.48, 43.78) --
	(308.54, 46.16) --
	(308.59, 42.16) --
	(308.64, 41.04) --
	(308.69, 41.93) --
	(308.74, 41.75) --
	(308.79, 43.69) --
	(308.84, 43.24) --
	(308.89, 43.46) --
	(308.95, 46.83) --
	(309.00, 43.91) --
	(309.05, 41.84) --
	(309.10, 41.53) --
	(309.15, 41.17) --
	(309.20, 40.54) --
	(309.25, 42.34) --
	(309.30, 41.49) --
	(309.36, 40.81) --
	(309.41, 43.78) --
	(309.46, 44.32) --
	(309.51, 42.92) --
	(309.56, 44.67) --
	(309.61, 44.76) --
	(309.66, 42.88) --
	(309.71, 46.38) --
	(309.76, 41.17) --
	(309.82, 42.25) --
	(309.87, 43.73) --
	(309.92, 40.99) --
	(309.97, 42.20) --
	(310.02, 44.00) --
	(310.07, 42.11) --
	(310.12, 40.77) --
	(310.17, 42.16) --
	(310.23, 44.94) --
	(310.28, 41.71) --
	(310.33, 41.98) --
	(310.38, 42.70) --
	(310.43, 43.33) --
	(310.48, 40.81) --
	(310.53, 40.95) --
	(310.58, 43.42) --
	(310.64, 42.61) --
	(310.69, 42.83) --
	(310.74, 45.08) --
	(310.79, 44.72) --
	(310.84, 41.26) --
	(310.89, 44.41) --
	(310.94, 41.35) --
	(310.99, 41.89) --
	(311.04, 40.81) --
	(311.10, 41.31) --
	(311.15, 40.14) --
	(311.20, 41.98) --
	(311.25, 44.54) --
	(311.30, 43.96) --
	(311.35, 44.18) --
	(311.40, 41.44) --
	(311.45, 40.50) --
	(311.51, 43.46) --
	(311.56, 47.68) --
	(311.61, 44.90) --
	(311.66, 43.10) --
	(311.71, 40.68) --
	(311.76, 42.83) --
	(311.81, 44.45) --
	(311.86, 44.18) --
	(311.92, 42.25) --
	(311.97, 43.37) --
	(312.02, 42.70) --
	(312.07, 44.90) --
	(312.12, 45.12) --
	(312.17, 42.79) --
	(312.22, 44.99) --
	(312.27, 42.34) --
	(312.32, 41.31) --
	(312.38, 40.00) --
	(312.43, 38.97) --
	(312.48, 39.51) --
	(312.53, 41.93) --
	(312.58, 41.75) --
	(312.63, 45.93) --
	(312.68, 42.79) --
	(312.73, 41.13) --
	(312.79, 42.52) --
	(312.84, 44.09) --
	(312.89, 41.04) --
	(312.94, 41.17) --
	(312.99, 40.14) --
	(313.04, 39.33) --
	(313.09, 42.65) --
	(313.14, 42.65) --
	(313.20, 41.58) --
	(313.25, 43.42) --
	(313.30, 42.92) --
	(313.35, 41.35) --
	(313.40, 40.86) --
	(313.45, 40.81) --
	(313.50, 42.25) --
	(313.55, 45.39) --
	(313.60, 42.92) --
	(313.66, 42.88) --
	(313.71, 41.93) --
	(313.76, 41.26) --
	(313.81, 41.13) --
	(313.86, 42.47) --
	(313.91, 41.93) --
	(313.96, 44.76) --
	(314.01, 45.39) --
	(314.07, 45.84) --
	(314.12, 42.52) --
	(314.17, 41.58) --
	(314.22, 40.99) --
	(314.27, 42.97) --
	(314.32, 41.08) --
	(314.37, 41.98) --
	(314.42, 41.31) --
	(314.48, 42.02) --
	(314.53, 41.44) --
	(314.58, 43.19) --
	(314.63, 43.15) --
	(314.68, 43.46) --
	(314.73, 40.99) --
	(314.78, 40.32) --
	(314.83, 38.74) --
	(314.88, 40.59) --
	(314.94, 40.99) --
	(314.99, 40.99) --
	(315.04, 41.80) --
	(315.09, 43.10) --
	(315.14, 42.83) --
	(315.19, 44.45) --
	(315.24, 40.99) --
	(315.29, 40.63) --
	(315.35, 41.31) --
	(315.40, 43.15) --
	(315.45, 40.32) --
	(315.50, 41.26) --
	(315.55, 44.76) --
	(315.60, 43.15) --
	(315.65, 41.17) --
	(315.70, 40.50) --
	(315.76, 38.74) --
	(315.81, 39.06) --
	(315.86, 40.36) --
	(315.91, 39.24) --
	(315.96, 38.57) --
	(316.01, 39.01) --
	(316.06, 42.34) --
	(316.11, 45.44) --
	(316.16, 41.71) --
	(316.22, 40.86) --
	(316.27, 41.89) --
	(316.32, 39.96) --
	(316.37, 39.96) --
	(316.42, 40.45) --
	(316.47, 44.81) --
	(316.52, 43.78) --
	(316.57, 41.62) --
	(316.63, 40.05) --
	(316.68, 39.82) --
	(316.73, 43.73) --
	(316.78, 43.73) --
	(316.83, 40.99) --
	(316.88, 40.59) --
	(316.93, 39.55) --
	(316.98, 39.64) --
	(317.04, 41.89) --
	(317.09, 42.07) --
	(317.14, 43.96) --
	(317.19, 42.70) --
	(317.24, 40.72) --
	(317.29, 39.51) --
	(317.34, 38.66) --
	(317.39, 38.97) --
	(317.44, 38.70) --
	(317.50, 40.41) --
	(317.55, 43.55) --
	(317.60, 43.01) --
	(317.65, 41.44) --
	(317.70, 41.22) --
	(317.75, 39.15) --
	(317.80, 39.33) --
	(317.85, 39.64) --
	(317.91, 40.14) --
	(317.96, 39.87) --
	(318.01, 39.91) --
	(318.06, 41.44) --
	(318.11, 40.09) --
	(318.16, 41.84) --
	(318.21, 41.40) --
	(318.26, 40.36) --
	(318.32, 43.55) --
	(318.37, 42.16) --
	(318.42, 42.97) --
	(318.47, 43.96) --
	(318.52, 40.63) --
	(318.57, 39.87) --
	(318.62, 39.28) --
	(318.67, 39.24) --
	(318.72, 39.19) --
	(318.78, 39.42) --
	(318.83, 40.05) --
	(318.88, 42.56) --
	(318.93, 40.27) --
	(318.98, 40.81) --
	(319.03, 41.62) --
	(319.08, 41.31) --
	(319.13, 40.68) --
	(319.19, 39.78) --
	(319.24, 42.56) --
	(319.29, 41.44) --
	(319.34, 42.11) --
	(319.39, 42.38) --
	(319.44, 40.32) --
	(319.49, 40.09) --
	(319.54, 40.00) --
	(319.60, 40.09) --
	(319.65, 39.96) --
	(319.70, 42.52) --
	(319.75, 44.09) --
	(319.80, 43.28) --
	(319.85, 43.24) --
	(319.90, 40.77) --
	(319.95, 39.42) --
	(320.00, 42.29) --
	(320.06, 42.25) --
	(320.11, 42.07) --
	(320.16, 40.45) --
	(320.21, 39.19) --
	(320.26, 39.60) --
	(320.31, 40.50) --
	(320.36, 42.56) --
	(320.41, 42.07) --
	(320.47, 41.49) --
	(320.52, 42.56) --
	(320.57, 40.27) --
	(320.62, 39.06) --
	(320.67, 39.15) --
	(320.72, 39.60) --
	(320.77, 40.05) --
	(320.82, 38.43) --
	(320.88, 37.89) --
	(320.93, 38.74) --
	(320.98, 38.92) --
	(321.03, 40.81) --
	(321.08, 40.63) --
	(321.13, 40.72) --
	(321.18, 39.33) --
	(321.23, 39.10) --
	(321.28, 39.06) --
	(321.34, 40.18) --
	(321.39, 39.24) --
	(321.44, 39.42) --
	(321.49, 38.92) --
	(321.54, 41.40) --
	(321.59, 44.41) --
	(321.64, 41.31) --
	(321.69, 39.51) --
	(321.75, 39.24) --
	(321.80, 38.61) --
	(321.85, 38.48) --
	(321.90, 38.70) --
	(321.95, 38.07) --
	(322.00, 38.79) --
	(322.05, 40.50) --
	(322.10, 41.04) --
	(322.16, 41.17) --
	(322.21, 40.90) --
	(322.26, 39.51) --
	(322.31, 38.97) --
	(322.36, 41.35) --
	(322.41, 42.56) --
	(322.46, 40.68) --
	(322.51, 40.05) --
	(322.56, 40.36) --
	(322.62, 38.48) --
	(322.67, 38.79) --
	(322.72, 38.16) --
	(322.77, 38.16) --
	(322.82, 38.12) --
	(322.87, 38.07) --
	(322.92, 37.85) --
	(322.97, 37.85) --
	(323.03, 37.85) --
	(323.08, 37.85) --
	(323.13, 37.85) --
	(323.18, 37.85) --
	(323.23, 37.85) --
	(323.28, 37.85) --
	(323.33, 37.85) --
	(323.38, 37.85) --
	(323.44, 37.85) --
	(323.49, 37.85);
\end{scope}
\begin{scope}
\path[clip] (  0.00,  0.00) rectangle (361.35,216.81);
\definecolor[named]{drawColor}{rgb}{0.50,0.50,0.50}

\node[text=drawColor,anchor=base east,inner sep=0pt, outer sep=0pt, scale=  0.96] at ( 39.51,121.41) {0.00};

\node[text=drawColor,anchor=base east,inner sep=0pt, outer sep=0pt, scale=  0.96] at ( 39.51,140.77) {0.25};

\node[text=drawColor,anchor=base east,inner sep=0pt, outer sep=0pt, scale=  0.96] at ( 39.51,160.13) {0.50};

\node[text=drawColor,anchor=base east,inner sep=0pt, outer sep=0pt, scale=  0.96] at ( 39.51,179.49) {0.75};

\node[text=drawColor,anchor=base east,inner sep=0pt, outer sep=0pt, scale=  0.96] at ( 39.51,198.84) {1.00};
\end{scope}
\begin{scope}
\path[clip] (  0.00,  0.00) rectangle (361.35,216.81);
\definecolor[named]{drawColor}{rgb}{0.50,0.50,0.50}

\path[draw=drawColor,line width= 0.6pt,line join=round] ( 42.35,124.72) --
	( 46.62,124.72);

\path[draw=drawColor,line width= 0.6pt,line join=round] ( 42.35,144.08) --
	( 46.62,144.08);

\path[draw=drawColor,line width= 0.6pt,line join=round] ( 42.35,163.43) --
	( 46.62,163.43);

\path[draw=drawColor,line width= 0.6pt,line join=round] ( 42.35,182.79) --
	( 46.62,182.79);

\path[draw=drawColor,line width= 0.6pt,line join=round] ( 42.35,202.15) --
	( 46.62,202.15);
\end{scope}
\begin{scope}
\path[clip] (  0.00,  0.00) rectangle (361.35,216.81);
\definecolor[named]{drawColor}{rgb}{0.50,0.50,0.50}

\node[text=drawColor,anchor=base east,inner sep=0pt, outer sep=0pt, scale=  0.96] at ( 39.51, 34.54) {0};

\node[text=drawColor,anchor=base east,inner sep=0pt, outer sep=0pt, scale=  0.96] at ( 39.51, 57.00) {1000};

\node[text=drawColor,anchor=base east,inner sep=0pt, outer sep=0pt, scale=  0.96] at ( 39.51, 79.46) {2000};

\node[text=drawColor,anchor=base east,inner sep=0pt, outer sep=0pt, scale=  0.96] at ( 39.51,101.93) {3000};
\end{scope}
\begin{scope}
\path[clip] (  0.00,  0.00) rectangle (361.35,216.81);
\definecolor[named]{drawColor}{rgb}{0.50,0.50,0.50}

\path[draw=drawColor,line width= 0.6pt,line join=round] ( 42.35, 37.85) --
	( 46.62, 37.85);

\path[draw=drawColor,line width= 0.6pt,line join=round] ( 42.35, 60.31) --
	( 46.62, 60.31);

\path[draw=drawColor,line width= 0.6pt,line join=round] ( 42.35, 82.77) --
	( 46.62, 82.77);

\path[draw=drawColor,line width= 0.6pt,line join=round] ( 42.35,105.23) --
	( 46.62,105.23);
\end{scope}
\begin{scope}
\path[clip] (336.67,120.91) rectangle (349.30,204.77);
\definecolor[named]{fillColor}{rgb}{0.80,0.80,0.80}

\path[fill=fillColor] (336.67,120.91) rectangle (349.30,204.77);
\definecolor[named]{drawColor}{rgb}{0.00,0.00,0.00}

\node[text=drawColor,rotate=270.00,anchor=base,inner sep=0pt, outer sep=0pt, scale=  0.96] at (339.68,162.84) {khappa};
\end{scope}
\begin{scope}
\path[clip] (336.67, 34.03) rectangle (349.30,117.89);
\definecolor[named]{fillColor}{rgb}{0.80,0.80,0.80}

\path[fill=fillColor] (336.67, 34.03) rectangle (349.30,117.89);
\definecolor[named]{drawColor}{rgb}{0.00,0.00,0.00}

\node[text=drawColor,rotate=270.00,anchor=base,inner sep=0pt, outer sep=0pt, scale=  0.96] at (339.68, 75.96) {sign changes};
\end{scope}
\begin{scope}
\path[clip] (  0.00,  0.00) rectangle (361.35,216.81);
\definecolor[named]{drawColor}{rgb}{0.50,0.50,0.50}

\path[draw=drawColor,line width= 0.6pt,line join=round] ( 59.80, 29.77) --
	( 59.80, 34.03);

\path[draw=drawColor,line width= 0.6pt,line join=round] (111.00, 29.77) --
	(111.00, 34.03);

\path[draw=drawColor,line width= 0.6pt,line join=round] (162.20, 29.77) --
	(162.20, 34.03);

\path[draw=drawColor,line width= 0.6pt,line join=round] (213.40, 29.77) --
	(213.40, 34.03);

\path[draw=drawColor,line width= 0.6pt,line join=round] (264.61, 29.77) --
	(264.61, 34.03);

\path[draw=drawColor,line width= 0.6pt,line join=round] (315.81, 29.77) --
	(315.81, 34.03);
\end{scope}
\begin{scope}
\path[clip] (  0.00,  0.00) rectangle (361.35,216.81);
\definecolor[named]{drawColor}{rgb}{0.50,0.50,0.50}

\node[text=drawColor,anchor=base,inner sep=0pt, outer sep=0pt, scale=  0.96] at ( 59.80, 20.31) {0};

\node[text=drawColor,anchor=base,inner sep=0pt, outer sep=0pt, scale=  0.96] at (111.00, 20.31) {1000};

\node[text=drawColor,anchor=base,inner sep=0pt, outer sep=0pt, scale=  0.96] at (162.20, 20.31) {2000};

\node[text=drawColor,anchor=base,inner sep=0pt, outer sep=0pt, scale=  0.96] at (213.40, 20.31) {3000};

\node[text=drawColor,anchor=base,inner sep=0pt, outer sep=0pt, scale=  0.96] at (264.61, 20.31) {4000};

\node[text=drawColor,anchor=base,inner sep=0pt, outer sep=0pt, scale=  0.96] at (315.81, 20.31) {5000};
\end{scope}
\begin{scope}
\path[clip] (  0.00,  0.00) rectangle (361.35,216.81);
\definecolor[named]{drawColor}{rgb}{0.00,0.00,0.00}

\node[text=drawColor,anchor=base,inner sep=0pt, outer sep=0pt, scale=  1.20] at (191.64,  9.03) {Iterations};
\end{scope}
\end{tikzpicture}
}
	\caption{Influence of the \(\kappa\) parameter for two different problems using the fractional update, with noise applied to resolve ties. Only the first trial is shown.}
	\label{fig:khappa-plot}
\end{figure}

In particular, \cref{fig:khappa-plot:deer} shows a full run in which \(\kappa\) is varied throughout the entire range \([0,1]\).
As can be seen, \(\kappa\) is initiall kept at 0 for a number of iterations --- this is in effect an attempt to find guaranteed optimal solutions if possible, only attempting to solve the problem heuristically if this fails.
Then, the parameter is increased fairly quickly until reaching an upper limit (\SI{70}{\percent} of the final value of \(\kappa\)), after which it is increased more slowly.

In \cref{fig:khappa-plot:deer} the final value of \(\kappa\) is 1, but this is not always the case.
For instance, when a solution has been found in a previous trial, the \(\kappa\) for which that solution was found is used as a final value instead.
This means that in the next trial of the problem shown in \cref{fig:khappa-plot:comp} the final value of \(\kappa\) will be around \num{0.3}.

The purpose of increasing \(\kappa\) slowly near this value is to increase accuracy by avoiding overshoot, as a lower \(\alpha\) will result in a better approximative solution.

\subsubsection{Trials}
As briefly mentioned above, the optimization involves several \emph{trials}.
Before each trial all constraints, variables and costs are reset to their original state.
Then, the parameter sweep is performed and until the final \(\kappa\) value is reached or a feasible solution is found.
If \(\kappa=0\) (\emph{i.e.} the solution is optimal), no more trials are run.
Otherwise, the program moves on to the next trial.

The number of trials is configurable, but the current implementation moves on to a new trial unless the best solution hasn't been improved in the last 4 trials.

	\section{Benchmarking}
% [todo] - review this section
% [todo] - explain the conditions of the tests, compare to conditions in deGivry14
In order to determine the efficiency of the algorithm, and determine what problem paradigm the algorithm is most usefully applied to, extensive benchmark testing will be performed.
The algorithm was tested against (part of) the large problem set used by \textcite{deGivry14}, which includes problems from the \gls{mrf}, \gls{wpms}, \gls{cfn}, Max-\gls{csp}, \gls{cp} and \gls{cvpr} domains.
All problems in the set are available in the \textsc{wcsp} file format.
Exact data (elapsed time and obtained solution for every solver, as well as proven optima and upper bounds for every problem) for these data sets have been obtained directly from \citeauthor{deGivry14}.

This section will describe the method used to benchmark the algorithm, including the calculation of presented data.
It will also briefly present the problems used in the benchmark, to provide background that may explain the performance characteristics of the algorithm, as well as short introductions to other solvers with which the in-the-middle algorithm will be compared.

\subsection{Benchmarking method}
All problem instances were limited in runtime by the upper time limit \(t_{\text{max}} = \SI{1200}{\second}\), and the benchmarks were run on a single core of an Intel~Core~i5 processor at \SI{2.3}{\giga\hertz}, with \SI{8}{\gibi\byte} RAM.
This is comparable to the conditions of the benchmark performed by \textcite{deGivry14}, and should ensure that the comparisons are valid.

Since the algorithm when used with the corresponding heuristic is an inexact algorithm (while the solvers benchmarked by \textcite{deGivry14} are all exact solvers) the quality of the solution must be compared in addition to the elapsed time per problem.
While the time may be compared as-is (given the comparable hardware conditions), the found solution will have to be compared relative to the proven optimum for each problem.

\label{pg:bench-method}
The relative deviation of the obtained solution (an \emph{optimality gap}) may be calculated as \((f - \bar{f})/(\mathrm{UB}-\mathrm{LB})\), where \(f\) is the solution found by the algorithm, \(\bar{f}\) is the proven optimum and \(\mathrm{UB}, \mathrm{LB}\) is the trivial upper and lower bound of the problem (the upper bound is specified in the \textsc{wcsp} format, and the lower bound is trivially the lowest cost among all constraints).
When \(\bar{f}\) is unknown, the lower bound is used instead.

For the \enquote{push} operation, the constant \(\rho\) was set to \num{5}, the reduction of \(\kappa\) was set to \num{0.8} and the algorithm was run for at most \num{100} additional iterations.
This matches the parameters used by \textcite{Bastert10}.

\subsection{Problem sets}
The problem sets obtained from \textcite{deGivry14} belong to a number of different domains and represent different types of problems from industry, academia and random generation.
This section will briefly review each problem set used in the benchmark, and review both problem source, interpretation and size.
% [review] - wcsp same as cfn, make this clear? < should be made clear when presenting cfn/wcsp
All of these problems are directly representable as \gls{wcsp} problems, and hence as max-sum optimization problems, and \textcite{deGivry14} provide details on the translation from each field to the \gls{wcsp} formulation used in this benchmark.

Several sets from the benchmark performed by \textcite{deGivry14} have been omitted from this benchmark.
This is because the in-the-middle algorithm wasn't able to solve any instances in the set (due to time or memory constraints), making them uninteresting in the sense that the algorithm isn't a feasible alternative for those problems. 
The omitted problem sets are mostly from the \gls{cp} and \gls{wpms} categories (where only two resp. one problem(s) were kept), but the \emph{Chinese Characters}, \emph{Color Segmentation}, \emph{Matching Stereo} and \emph{Photo Montage} sets from \gls{cvpr} are also omitted from the benchmark.
% [todo] - review briefly the different fields?
% [todo] - mention mean/median number of hard/soft constraints in each problem set? use c program to find out! 

% [review] - table with problem data instead? #vars, #constrs, domain, arity, graph connectivity/fullness, %hard constr., %zero constr, reference? (+size = n*d?)
% [todo] - calculate graph girth/diameter if possible? could be a good measure of graph completeness
% [todo] - Calculate connectivity? The connectivity or vertex connectivity κ(G) (where G is not a complete graph) is the size of a minimal vertex cut. http://en.wikipedia.org/wiki/Connectivity_(graph_theory)
% [todo] - calculate density? D = 2 * e / (n^2 - n) when r = 2

% \begin{table}
% 	\centering
% 	% Updated 2014-05-25
% 	\caption{
% 		Representative properties of each problem set included in the benchmark.
% 		The number of variables \(n\), domain size \(D\), number of cost functions \(e\) and overall size \(n\times D\) of every set is represented by the maximum value in that set.
% 		The properties of the problem graph (diameter \(d\) and girth \(g\)) are represented by the mode of that variable in every set.
% 		The constraint type values are represented by their median in every set.
% 		%Finally, the original source of each problem set is presented.
% 	}
% 	% [review] - decide between this table and just writing stuff in text instead. or do this in results for the relevant problems when discussing good/bad problems?
% 	\label{tab:problem-properties}
% 	\begin{figcenter}
% 	\begin{tabular}{xy S[table-figures-decimal=0,table-figures-exponent=0,table-figures-integer=6,table-figures-uncertainty=0]
% 					   S[table-figures-decimal=0,table-figures-exponent=0,table-figures-integer=3,table-figures-uncertainty=0]
% 					   S[table-figures-decimal=0,table-figures-exponent=0,table-figures-integer=6,table-figures-uncertainty=0]
% 					   H%S[table-figures-decimal=0,table-figures-exponent=0,table-figures-integer=1,table-figures-uncertainty=0]
% 					   S[table-figures-decimal=0,table-figures-exponent=0,table-figures-integer=0,table-figures-uncertainty=0]% may need special attention to force exponents
% 					   S[table-figures-decimal=0,table-figures-exponent=0,table-figures-integer=0,table-figures-uncertainty=0]
% 					   S[table-figures-decimal=0,table-figures-exponent=0,table-figures-integer=0,table-figures-uncertainty=0]
% 					   S[table-figures-decimal=0,table-figures-exponent=0,table-figures-integer=0,table-figures-uncertainty=0]
% 					   S[table-figures-decimal=0,table-figures-exponent=0,table-figures-integer=0,table-figures-uncertainty=0]
% 					   S[table-figures-decimal=0,table-figures-exponent=0,table-figures-integer=0,table-figures-uncertainty=0]
% 					   H}
% 		\toprule
% 			{} & {} & \multicolumn{5}{c}{Problem size} & \multicolumn{2}{c}{Graph} & \multicolumn{3}{c}{Constr.~type (\si{\percent})} & {} \\
% 			\cmidrule(rl){3-7} \cmidrule(rl){8-9} \cmidrule(rl){10-12}
% 			% #vars, domain, #constr., constr arity, n*d, graph diameter, graph girth, ...
% 			{\normalsize Category} & {\normalsize Set} & {\(n\)} & {\(D\)} & {\(e\)} & {\(r\)} & {\(n\times D\)} & {\(d\)} & {\(g\)} & {Hard} & {Soft} & {Zero} & {\normalsize Source} \\
% 		\midrule
% \acrshort{cfn}	&	Auction			&    246 &   2 &  11528 & 2 & ? & ? & ? & \\
% 				&	CELAR			&    458 &  44 &   2335 & 2 & ? & ? & ? & \\
% 				&	Pedigree		&  10017 &  28 &  18875 & 3 & ? & ? & ? & \\
% 				&	ProteinDesign	&     18 & 198 &    171 & 2 & ? & ? & ? & \\
% 				&	SPOT5			&   1057 &   4 &  21786 & 3 & ? & ? & ? & \\
% 				&	Warehouse		&   1100 & 300 & 101100 & 2 & ? & ? & ? & \\
% \acrshort{cp}	&	OnCallRostering	&   2205 &  89 &   4513 & 4 & ? & ? & ? & \\
% 				&	ParityLearning	&    759 &  20 &   1440 & 4 & ? & ? & ? & \\
% \acrshort{cvpr}	&	GeomSurf		&   1133 &   7 &   5039 & 3 & ? & ? & ? & \\
% 				&	InPainting		&  14400 &   4 &  71282 & 2 & ? & ? & ? & \\
% 				&	Matching		&     20 &  20 &    210 & 2 & ? & ? & ? & \\
% 				&	ObjectSeg		& 166222 &  20 & 497849 & 2 & ? & ? & ? & \\
% 				&	SceneDecomp		&    208 &   8 &    769 & 2 & ? & ? & ? & \\
% Max-\acrshort{csp}	&	BlackHole	&    205 &  50 &   1651 & 2 & ? & ? & ? & \\
% 				&	Coloring		&    450 &   6 &   6164 & 2 & ? & ? & ? & \\
% 				&	Composed		&     83 &  10 &    785 & 2 & ? & ? & ? & \\
% 				&	EHI				&    315 &   7 &   4715 & 2 & ? & ? & ? & \\
% 				&	Geometric		&     50 &  20 &    605 & 2 & ? & ? & ? & \\
% 				&	Langford		&     33 &  29 &    517 & 2 & ? & ? & ? & \\
% 				&	QCP				&    264 &   9 &   2662 & 2 & ? & ? & ? & \\
% \acrshort{mrf}	&	DBN				&   1094 &   2 &  22793 & 2 & ? & ? & ? & \\
% 				&	Grid			&   6400 &   2 &  19200 & 2 & ? & ? & ? & \\
% 				&	ImageAlignment	&    400 &  93 &   3563 & 2 & ? & ? & ? & \\
% 				&	Linkage			&   1289 &   7 &   2184 & 5 & ? & ? & ? & \\
% 				&	ObjectDetection	&     60 &  21 &   1830 & 2 & ? & ? & ? & \\
% 				&	ProteinFolding	&   1972 & 503 &   8816 & 2 & ? & ? & ? & \\
% 				&	Segmentation	&    237 &  21 &    886 & 2 & ? & ? & ? & \\
% \acrshort{wpms}	&	MaxClique		&   3321 &   2 & 378247 & 2 & ? & ? & ? & \\
% 		\bottomrule
% 	\end{tabular}
% 	\end{figcenter}
% \end{table}

\subsubsection{Constraint Function Network (CFN)}
This category contains six problem sets, all from the CFLib collection mentioned by \textcite[\pno~3]{deGivry14}.
Most of them are real-world problems or generated to approximate such problems, and all of them are readily available in the WCSP file format mentioned earlier.

\begin{description}
	\item[Auction]
		The combinatorial auction problem has been previously used by \textcites{Larrosa08}{Sandholm99}.
		In summary, the problem allows bidders to bid for indivisible subsets of goods, and the optimization problem is to maximize the revenue of the bid-taker.
		The problems are generated, but inspired by real-world scenarios.
		All variables are binary (the original problem is a binary Max-SAT problem), and the problems contain up to \num{246} variables and \num{12000} constraints.

		The problem set includes \emph{scheduling} and \emph{path} distribution problems, but omits the \emph{regions} distribution mentioned by \textcite[\pno~228]{Larrosa08}.

	\item[CELAR]
		As detailed by \textcite{Cabon99} \parencite[and to some extent][\pno~315\psq]{Meseguer06}, this problem set concerns radio frequency assignment, \emph{i.e.} the problem of providing communication channels from limited resources while minimizing interference in the network.
		The problems where initially introduced in 1993 by \emph{Centre d’Electronique de l’Armement}, and are based on real-world data.

		The CELAR problems are fairly large, with variable domains ranging up to \num{44}, with up to \num{458} variables and \num{2400} constraints.
		Problems included in the benchmark are mainly the CELAR sub-instances \parencite[\pno~85]{Cabon99} and some GRAPH instances.

	\item[Pedigree]
		This category contains problems relating to the Mendelian error correction on complex pedigree \parencites{Sanchez08}[\pno~317\psq]{Meseguer06}, which is a real-world \gls{wcsp} problem.
		The problem may be described as surveying a pedigree (similar to a family tree), detecting individuals that are erroneous in the sense that they do not conform to the Mendelian laws of inheritance.
		Specifically, the problem formulation is to find the minimum number of errors needed to explain erroneous data.

		The problems are very large, with the number of variables reaching \num{10000} and almost \num{20000} constraints, with variable domains around \num{25}.

	\item[Protein Design]
		Computational Protein Design problems concern the identification of proteins performing given tasks. The actual problem statement is the optimization of a complex energy function over amino acid sequences, and it is described in length by \textcite{Allouche12}.

		The problems may be expressed by \gls{cfn} or integer \gls{lp} models — only the \gls{cfn} formulations are used in this benchmark. The problems contain few (roughly \num{20}) variables with very large domains (up to \num{200}), and around \num{170} constraints.

	\item[SPOT5]
		The SPOT5 problems are in essence planning problems, taken from real-world planning of earth optical observation satellites.
		Given a number of images to be taken during one day using one of three instruments, an associated importance and a set of imperative constraints (transition times, data flow limitations, on-board recording capacity \emph{etc.}), the problem is to find a feasible subset of images that maximize the sum of the associated weights \parencite{Bensana99}.

		The problems are large, with roughly \num{1000} variables and \num{22000} constraints, but the variables are all 4-ary.

	\item[Warehouse]
		Originally presented by \textcite{Kratica01}, the uncapacitated warehouse location instances are randomly generated instances of the facility location problem. In essence, the problem concerns the optimal placement of facilities (in this case warehouses) while minimizing transport costs. These instances were previously used by \textcite{deGivry05} in their evaluation of existential arc consistency for \glspl{csp}.

		These problem instances are very large. The variable domain reaches \num{300} for some problems, with \num{1100} variables and \num{101100} constraint functions.

\end{description}

\subsubsection{Constraint Programming (CP)}
Two problems (one real-world problem and one academic) from this category have been included in this benchmark.
All problems in these sets come from the MiniZinc Challenge\footnote{\url{http://www.minizinc.org/}} \parencite[\pno~5]{deGivry14}, and represent specific problems (representable as \gls{wcsp} instances) defined through Constraint Programming languages.
% [review] - page number above will have to change
Note that \gls{cp} problems are not generally representable as \glspl{wcsp}, which makes this category less interesting in the context of \gls{wcsp} or max-sum algorithms.

\begin{description}
%	\item[A Maze]
%	\item[Fast Food]
%	\item[Golomb]
	\item[On Call Rostering]
		This problem is a planning problem in which staff members are assigned to days in a rostering period.
		Requirements on staff (both forced rostering and staff being unavailable) affect the schedule, and work load should be even over the rostering period.
		Additionally, staff members are not allowed to be on call more than two days in a row, and prefer not to be on call for two consecutive days.
		Other, similar constraints may also be present.

		Problems in this set are generally fairly large in terms of domain size (up to \num{90}), but only contain up to \num{2200} variables and \num{4500} constraints, which is small compared to other sets.
	
	\item[Parity Learning]
		The Parity Learning set contains instances of an optimization variant of the Minimal Disagreement Parity problem \parencite{Crawford94}.
		A set of input/output samples of a Boolean function is given, where the function outputs the parity of an unkown subset of the input variables.
		A stated number of the input/output samples are incorrect with respect to the given function, and the goal of the parity learning problem is to find a subset of the input variables that minimizes the number of errors.

		In terms of problem size, instances of this set are fairly small. The number of variables is below \num{760}, and the number of constraints at most \num{1440}. The variable domain sizes are below \num{20}.
%	\item[VRP]
\end{description}

\subsubsection{Computer Vision and Pattern Recognition (CVPR)}
In this category there are nine problem sets containing \gls{mrf} instances from the OpenGM2 benchmark \parencite{Kappes13}.
The problems have been collected from various sources \parencite[\pno~1330]{Kappes13}, but all concern various computer vision tasks performed on real-world images.

The size of these problems vary, with \numrange{20}{500000} variables, \numrange{210}{2000000} constraints and variable domains reaching \num{20} for some sets.

\subsubsection{Max-CSP}
The seven max-\gls{csp} sets are restated binary \gls{csp} instances such that the optimal solution of each instance is the minimal number of unsatisfiable constraints in the original \gls{csp} problem.
The instances are from the 2008 max-\gls{csp} competition\multfootnote{\url{http://www.cril.univ-artois.fr/~lecoutre/benchmarks.html};\url{http://www.cril.univ-artois.fr/CPAI08/}}.
The problems are mostly academic and random, with a relatively small number of variables and constraints (below \num{450} and \num{6500}, respectively).

\begin{description}
	\item[Black Hole]
		\enquote{Black Hole} is a solitaire card game in which card from 17 piles are moved to a center pile according to certain rules.
		The instances in this set correspond to a simplification of this solitaire game, and were used in the 2005 \gls{csp} Solver Competition.
		\Textcite{Gent07} provide a theoretical background to the \gls{csp} formulation of this problem.
	\item[Coloring]
		In the well-known graph coloring (decision) problem, the objective is to decide whether a given graph is \(k\)-colorable, \emph{i.e.} whether each edge can be assigned one of \(k\) distinct colors such that no adjacent (connected) nodes have the same color.
		As such, the instances are crafted academic problems.
		The instances in this set originate from the \emph{Center for Discrete Mathematics and Theoretical Computer Science} (DIMACS), and have been previously studied by \textcite{Benhamou07}.
	\item[Composed]
		These are random instances composed of an unconstrained \gls{csp} core combined using binary constraints with auxillary fragments.
		Such problems have been previously used by \textcite{Lecoutre04,Jussien00}.
	\item[EHI]
		The EHI problem instances are random 3-\gls{sat} (\gls{sat} problems where every clause consists of exactly three literals) converted into \gls{csp} instances with binary constraints, then further restated as max-\gls{csp} problems. They have previously been considered by \textcite{Lecoutre04}.
	\item[Geometric]
		This set contains problems generated from random points in the unit square. For every pair of points, a hard constraint forbidding the pair is added if they are within a certain distance from each other.
		These instances were used in the 2005 max-\gls{csp} competition \parencite{Boussemart05}.
	\item[Langford]
		The Langford instances are academic instances solving the problem of arranging \(k\) sets of numbers ranging from \(1\) to \(n\) so that appearances of the number \(m\) are exactly \(m\) places apart.
		A general formulation of the problem has been presented by \textcite{Linek03}.
	\item[QCP]
		The Quasi-group Completion problems (QCP) are concerned with deciding if a partial Latin square can be filled in order to obtain a full Latin square.
		The set consists of 60 instances previously used in the 2005 \gls{csp} Solver Competition, and the general problem has previously been studied by \textcite{Gomes02}.
\end{description}

\subsubsection{Markov Random Fields (MRF)}
This category consists of seven sets, where the task is to estimate \gls{map} probabilities on \gls{mrf}. The sets represent different underlying problems such as image alignment, genetic linkage analysis, protein folding and other probabilistic problems.

All of these problems, except those in the \emph{Linkage} set (which was used in the UAI08 probabilistic inference evaluation and later by \textcite{Favier11} in their work on pairwise decomposition), are from the 2011 Probabilistic Inference Challenge\footnote{\url{http://www.cs.huji.ac.il/project/PASCAL/}}.
The problems were translated into their \gls{wcsp} equivalent using a \(-\log{}\) transformation \parencite[\pno~4]{deGivry14}.
% [review] - page number above will have to change

Most problems are modest in size, with \numrange{60}{2000} variables and \numrange{1000}{10000} constraints and variable domains below \num{30}. 
Some problems have variable domains approaching \num{500}, and some have up to \num{6400} variables and \num{20000} constraints.
Notably, the \emph{Segmentation} set contains both binary and 21-ary formulations of each problem.

% [review] - describe every set?

\subsubsection{Weighted Partial Max-SAT (WPMS)}
From the field of \gls{wpms}, only one problem set was kept.
Problems in this fiels contain a very large number of (binary, for obvious reasons) variables and cost functions with very large arity --- in fact, this is the only field in which cost function arity exceeds \num{5} (most other sets have cost functions of two variables only).
The only kept problem set, \emph{Max-Clique}, has a cost function arity of \num{2}.
Discarded sets were omitted due to memory concerns, likely caused by inefficient representation of the cost functions.
The instances were used in the eigth Max-SAT Evaluation\footnote{\url{http://maxsat.ia.udl.cat/13/benchmarks/}},

\begin{description}
%	\item[Haplotyping]
	\item[Max-Clique]
		The Max-Clique problem, which may be restated as a max-SAT problem \parencite{Heras08}, is the well-known problem of finding the largest \emph{clique} (complete subgraph) of a graph.
		It may be regarded as an academic problem, but has many applications to real-world problems and has seen extensive study when it comes to tailored algorithms for the original formulation.
		The original instances are from the second DIMACS challenge \parencite{Johnson96}, and have been previously used by \emph{e.g.} \textcite{Östergård02} for benchmarking max-clique algorithms.

		When translated into their max-SAT equivalent, max-clique instances become very large. While the number of variables is fairly low (below \num{3400}), the number of constraints is very large (approaching \num{380000}).
%	\item[MIPLib]
%	\item[Packup Weighted]
%	\item[Planning With Preferences]
%	\item[Timetabling]
%	\item[Upgradeability]
\end{description}

\subsection{Solvers}
Three solvers used by \textcite{deGivry14} are included in this benchmark.
This section will briefly describe their original field of application, give brief pointers to the method they employ, and relate them to the in-the-middle algorithm.

\subsubsection{Toulbar2}
The Toulbar2 solver is an exact anytime solver for \gls{wcsp} based on a depth-first branch-and-bound algorithm \parencite{Allouche10}.
In addition to using a strong arc consistency property\footnote{Existential directed arc consistency, as presented by \textcite{deGivry05}.}, the solver uses a sizable bag of tricks including variable elimination, dead-end elimination \parencite{deGivry13} and pairwise decomposition \parencite{Favier11}.

The Toulbar2 solver is by far the most advanced solver in the \gls{wcsp} field, with significant amounts of theory behind it especially with respect to strong arc consistency properties (the one implicitly employed by the in-the-middle algorithm, generalized arc consistency, is quite weak in comparison).
It is therefore an interesting benchmark \enquote{opponent}, and matching it in a benchmark (especially the \gls{mrf}, \gls{cfn} and max-\gls{csp} categories) could indicate potential for the in-the-middle algorithm in those categories.

\subsubsection{CPLEX}
The well-known CPLEX solver, which is an exact \gls{lp} solver rather than a \gls{wcsp} solver, is also included in the benchmark.
Using the \emph{direct} encoding described by \textcite[\pno~3]{deGivry14} of \gls{wcsp} into \gls{lp} problems, CPLEX was found to have very good performance for some problem categories an therefore it is also included in this benchmark.
CPLEX is highly optimized proprietary software, but uses the well-known simplex and barrier interior point methods to solve \gls{lp} problems.
Even so, previous results \parencite{Mason01,Ernst05} have shown that the \gls{lp} formulation of the in-the-middle algorithm is competitive with CPLEX, and it is therefore interesting to see how the two relate in the \gls{wcsp} field.

\subsubsection{MaxHS}
The MaxHS solver is a \gls{maxsat} and \gls{wpms} solver based on decomposing \gls{maxsat} problems into several smaller \gls{sat} instances, solving these using cooperation between an underlying \gls{sat} solver and a \gls{mip} solver \parencite{Davies11}.
It has been found to perform well in solving non-random \gls{maxsat} instances \parencite{Davies13}, and is therefore a prime candidate for comparison in the \gls{wpms} (and to some extent max-\gls{csp}) categories.


	\chapter{Results}
	This chapter will review the results of the benchmark, and compare these to previous results due to \textcite{deGivry14}.
Results from two modified variants, using the \enquote{push} operation and using the greedy \gls{dp} update, will also be reviewed and compared to the benchmark of the standard algorithm.
Problem sets which yield good performance will be identified and examined further.

The chapter is divided into three parts.
First, benchmarking results of the standard in-the-middle algorithm will be presented.
Then, results from both modified variants will be presented along with the chosen subset of benchmarking problems applied to these specific variants.
Finally, the results are discussed and interpreted in further depth.

	\section{Standard algorithm}
Benchmarking the standard algorithm on the large set of problems provided earlier produced mixed results.
The number of problems (out of the total number available from each set) that were solved by the in-the-middle algorithm indicate that the algorithm is able to solve the same types of problem as the \emph{Toulbar} and \emph{CPLEX} solvers, with a few exceptions.
This implies that the algorithm may be useful in most of the fields from which problems were drawn, but does not indicate whether it is useful in its current state, performance-wise.

However, as \cref{tab:comparative-results} shows, the algorithm performed very well when comparing runtime to other solvers and in fact it was the fastest for almost half of the sets after removing incomplete data (runtimes based on data where less than \SI{70}{\percent} of the problems were solved).
In most problem sets, the optimality gap was small as well --- for several sets optimal solutions were found --- with the notable exception of the \emph{Scene Decomposition} set\footnote{Note however that the Toulbar2 solver finds the same non-optimal solutions in this problem set.}.

It should be noted that Toulbar2, CPLEX and MaxHS may be at a slight disadvantage in terms of runtime presented in \cref{tab:comparative-results}.
Brief testing of the Toulbar2 solver indicated that the difference in hardware between the in-the-middle runtimes and those provided by \textcite{deGivry14} may skew the results in favour of the in-the-middle solver, with actual runtimes on the same hardware being \SIrange{0}{50}{\percent} lower for Toulbar2.
Fortunately, this difference only makes comparisons in a small number of the sets (\emph{Pedigree}, \emph{In-Painting} and \emph{Max-Clique}) potentially invalid, since the difference in runtime between solvers is large in most cases.

\begin{table}
	\centering
	% Updated 2014-05-25
	\caption{
		Optimality gap and runtime.
		For each problem instance used in the benchmark, the in-the-middle solver runtime is compared the other solvers included in the benchmark, and the objective value is compared to the best known optimum from \textcite{deGivry14}.
		Problem sets marked with \textdagger{} include unsolved problems (no feasible solution found by the in-the-middle solver), and n/a values indicate that none of the problems in the set were solved.
		Runtimes based on less than \SI{70}{\percent} of the problems are faded, while the best runtime of those remaining is emphasised.
	}
	\label{tab:comparative-results}
	\begin{figcenter}
	\begin{tabular}{xyHS[round-mode=places,round-precision=3,scientific-notation=fixed,fixed-exponent=0]
				    S[round-mode=places,round-precision=2,scientific-notation=fixed,fixed-exponent=0]
				    H%S[round-mode=places,round-precision=2,scientific-notation=fixed,fixed-exponent=0]
				    S[round-mode=places,round-precision=2,scientific-notation=fixed,fixed-exponent=0]
				    S[round-mode=places,round-precision=2,scientific-notation=fixed,fixed-exponent=0]
				    S[round-mode=places,round-precision=2,scientific-notation=fixed,fixed-exponent=0]}
		\toprule
			{} & {} & {} & {} & \multicolumn{5}{c}{Mean solution time (\si{\second})} \\
			\cmidrule(rl){5-9}
			{\normalsize Category} & {\normalsize Set} & {\(\#\) solved} & {Gap (\si{\percent})} & {ITM} & {MPLP2} & {Toulbar2} & {CPLEX} & {MaxHS} \\
		\midrule
\acrshort{cfn}	&	Auction\textdagger	&	{102/170}	&	0.000000e+00	&	\color{gray}82.8575	&	1200.00	&	8.195	&	\emshape 0.030	&	0.040 \\
				&	CELAR\textdagger	&	{10/16}	&	9.081260e-07	&	\color{gray}193.3445	&	1200.00	&	\emshape 22.375	&	1200.00	&	{\textcolor{gray}{n/a}} \\
				&	Pedigree	&	\emph{10/10}	&	1.804874e-00	&	\emshape 2.3750	&	{\textcolor{gray}{n/a}}	&	4.130	&	\color{gray}0.710	&	\color{gray}0.030 \\
				&	ProteinDesign	&	\emph{10/10}	&	0.000000e+00	&	43.3995	&	60.500	&	\emshape 2.330	&	1200.00	&	{\textcolor{gray}{n/a}} \\
				&	SPOT5\textdagger	&	{5/20}	&	4.977105e-03	&	\color{gray}6.4360	&	1200.00	&	1200.00	&	\emshape 0.465	&	\color{gray}0.820 \\
				&	Warehouse\textdagger	&	{38/55}	&	0.000000e+00	&	\color{gray}55.8550	&	57.970	&	0.160	&	\emshape 0.050	&	0.560 \\
%\acrshort{cp}	&	AMaze\textdagger	&	{0/6}	&	{\textcolor{gray}{n/a}}	&	{\textcolor{gray}{n/a}}	&	{\textcolor{gray}{n/a}}	&	544.545	&	{\textcolor{gray}{n/a}}	&	2.940 \\
%				&	FastFood\textdagger	&	{1/6}	&	0.000000e+00	&	0.0000	&	{\textcolor{gray}{n/a}}	&	0.0	&	0.010	&	0.0 \\
%\acrshort{cp}	&	Golomb\textdagger	&	{0/6}	&	{\textcolor{gray}{n/a}}	&	{\textcolor{gray}{n/a}}	&	{\textcolor{gray}{n/a}}	&	19.860	&	{\textcolor{gray}{n/a}}	&	42.670 \\
\acrshort{cp}	&	OnCallRostering\textdagger	&	{3/5}	&	4.000000e-06	&	\emshape 10.3540	&	{\textcolor{gray}{n/a}}	&	71.040	&	1200.0	&	18.950 \\
				&	ParityLearning	&	\emph{7/7}	&	1.800000e-05	&	\emshape 34.5300	&	{\textcolor{gray}{n/a}}	&	368.080	&	1200.0	&	\color{gray}222.690 \\
%				&	VRP\textdagger	&	{0/5}	&	{\textcolor{gray}{n/a}}	&	{\textcolor{gray}{n/a}}	&	{\textcolor{gray}{n/a}}	&	1200.0	&	{\textcolor{gray}{n/a}}	&	{\textcolor{gray}{n/a}} \\
%\acrshort{cvpr}	&	ChineseChars\textdagger	&	{0/100}	&	{\textcolor{gray}{n/a}}	&	{\textcolor{gray}{n/a}}	&	1200.00	&	1200.0	&	1200.0	&	{\textcolor{gray}{n/a}} \\
%				&	ColorSeg\textdagger	&	{0/21}	&	{\textcolor{gray}{n/a}}	&	{\textcolor{gray}{n/a}}	&	1200.00	&	1200.0	&	1200.0	&	{\textcolor{gray}{n/a}} \\
\acrshort{cvpr}	&	GeomSurf	&	\emph{600/600}	&	2.091307e-00	&	\emshape 0.0460	&	1.740	&	0.070	&	6.620	&	\color{gray}27.110 \\
				&	InPainting	&	\emph{4/4}	&	1.797097e-02	&	\emshape 1009.5145	&	1057.860	&	1200.0	&	1200.0	&	{\textcolor{gray}{n/a}} \\
				&	Matching	&	\emph{4/4}	&	0.000000e+00	&	17.9275	&	12.710	&	\emshape 4.120	&	1200.0	&	{\textcolor{gray}{n/a}} \\
%				&	MatchingStereo\textdagger	&	{0/2}	&	{\textcolor{gray}{n/a}}	&	{\textcolor{gray}{n/a}}	&	1200.00	&	1200.0	&	{\textcolor{gray}{n/a}}	&	{\textcolor{gray}{n/a}} \\
				&	ObjectSeg	&	\emph{5/5}	&	3.253700e-04	&	\emshape 1200.0	&	\emshape 1200.00	&	\emshape 1200.0	&	\emshape 1200.0	&	{\textcolor{gray}{n/a}} \\
%				&	PhotoMontage\textdagger	&	{0/2}	&	{\textcolor{gray}{n/a}}	&	{\textcolor{gray}{n/a}}	&	{\textcolor{gray}{n/a}}	&	{\textcolor{gray}{n/a}}	&	{\textcolor{gray}{n/a}}	&	{\textcolor{gray}{n/a}} \\
				&	SceneDecomp	&	\emph{715/715}	&	7.545481e+01	&	0.0210	&	0.110	&	\emshape 0.020	&	1200.0	&	\color{gray}521.160 \\
Max-\acrshort{csp}	&	BlackHole	&	\emph{37/37}	&	9.009009e-01	&	\emshape 58.8900	&	{\textcolor{gray}{n/a}}	&	1200.0	&	315.050	&	\color{gray}0.635 \\
				&	Coloring	&	\emph{22/22}	&	0.000000e+00	&	1.6860	&	{\textcolor{gray}{n/a}}	&	\emshape 0.405	&	1.275	&	\color{gray}0.030 \\
				&	Composed	&	\emph{80/80}	&	1.342282e-01	&	20.3400	&	1200.00	&	\emshape 0.115	&	5.755	&	32.690 \\
				&	EHI	&	{200/200}	&	9.000000e-01	&	\emshape 191.2190	&	{\textcolor{gray}{n/a}}	&	1200.0	&	1200.0	&	{\textcolor{gray}{n/a}} \\
				&	Geometric	&	\emph{100/100}	&	1.082434e-00	&	98.9760	&	{\textcolor{gray}{n/a}}	&	0.620	&	1200.0	&	\emshape 0.150 \\
				&	Langford	&	\emph{4/4}	&	1.311265e-00	&	\emshape 70.7775	&	1200.00	&	600.255	&	851.605	&	\color{gray}0.270 \\
				&	QCP	&	\emph{60/60}	&	1.292034e-00	&	43.2575	&	1200.00	&	1200.0	&	1200.0	&	\emshape 0.125 \\
\acrshort{mrf}	&	DBN	&	\emph{108/108}	&	0.000000e+00	&	37.9040	&	1200.00	&	\emshape 0.180	&	48.280	&	\color{gray}20.020 \\
				&	Grid\textdagger	&	{0/21}	&	{\textcolor{gray}{n/a}}	&	{\textcolor{gray}{n/a}}	&	1200.00	&	1200.0	&	\emshape 160.640	&	\color{gray}542.850 \\
				&	ImageAlignment	&	\emph{10/10}	&	0.000000e+00	&	\emshape 0.5815	&	4.855	&	1.800	&	1200.0	&	{\textcolor{gray}{n/a}} \\
				&	Linkage\textdagger	&	{8/22}	&	0.000000e+00	&	\color{gray}41.0700	&	1200.00	&	32.050	&	327.625	&	\emshape 16.520 \\
				&	ObjectDetection	&	\emph{37/37}	&	6.465565e-00	&	\emshape 279.8620	&	1200.00	&	1200.0	&	1200.0	&	{\textcolor{gray}{n/a}} \\
				&	ProteinFolding\textdagger	&	{20/21}	&	0.000000e+00	&	1200.0000	&	1200.00	&	\emshape 23.140	&	\color{gray}116.735	&	{\textcolor{gray}{n/a}} \\
				&	Segmentation	&	\emph{100/100}	&	0.000000e+00	&	\emshape 0.0310	&	0.355	&	0.150	&	600.070	&	\color{gray}0.300 \\
%\acrshort{wpms}	&	Haplotyping	&	{0/100}	&	{\textcolor{gray}{n/a}}	&	{\textcolor{gray}{n/a}}	&	{\textcolor{gray}{n/a}}	&	1200.0	&	1200.0	&	5.200 \\
\acrshort{wpms}	&	MaxClique\textdagger	&	{46/62}	&	2.583333e-00	&	\emshape 257.0920	&	1200.00	&	389.745	&	481.550	&	\color{gray}8.795 \\
%				&	MIPLib\textdagger	&	{0/12}	&	{\textcolor{gray}{n/a}}	&	{\textcolor{gray}{n/a}}	&	{\textcolor{gray}{n/a}}	&	193.990	&	533.220	&	0.360 \\
%				&	PackupWeighted\textdagger	&	{0/99}	&	{\textcolor{gray}{n/a}}	&	{\textcolor{gray}{n/a}}	&	{\textcolor{gray}{n/a}}	&	292.960	&	0.280	&	4.620 \\
%				&	PlanningWithPref	&	{0/29}	&	{\textcolor{gray}{n/a}}	&	{\textcolor{gray}{n/a}}	&	{\textcolor{gray}{n/a}}	&	1200.0	&	1200.0	&	1.270 \\
%				&	TimeTabling	&	{0/25}	&	{\textcolor{gray}{n/a}}	&	{\textcolor{gray}{n/a}}	&	{\textcolor{gray}{n/a}}	&	1200.0	&	1200.0	&	83.800 \\
%				&	Upgradeability\textdagger	&	{0/100}	&	{\textcolor{gray}{n/a}}	&	{\textcolor{gray}{n/a}}	&	{\textcolor{gray}{n/a}}	&	3.095	&	1.010	&	24.985 \\
		\bottomrule
	\end{tabular}
	\end{figcenter}
\end{table}

The algorithm shows promise especially in the Max-\gls{csp} and \gls{mrf} categories, where overall solution quality is good and the algorithm had the best performance for several sets.
Compared to both Toulbar2 and CPLEX, the in-the-middle algorithm appears to be a useful complement providing (good) approximative solutions to problems the other solvers have great difficulty in solving.
Results from the \gls{cp} category are also promising, but these problem sets are small and not all \gls{cp} problems have max-sum formulations.

\Cref{fig:cactus-std} shows accumulated runtimes for three of the sets in which the algorithm performed well.
The algorithm has a consistent advantage in the \emph{Segmentation} set (\cref{fig:cactus-std:segmentation}), which is reflected by the mean runtime in \cref{tab:comparative-results}.
\Cref{fig:cactus-std:dbn} highlights a more interesting situation.
It shows performance in the \emph{DBN} set, in which the algorithm has an advantage in total runtime across the whole set, almost entirely due to good performance on the more difficult problems.
In fact, the runtimes are fairly evenly distributed whereas the runtimes for CPLEX and Toulbar vary significantly between the difficult and easy problems of the set.
% [todo] - justify choice of median for the values (goes in method?)

\begin{figure}[tp]
	\begin{figcenter}
	\subfloat[The \emph{Segmentation} set of the \gls{mrf} category.\label{fig:cactus-std:segmentation}]{% Created by tikzDevice version 0.7.0 on 2014-06-01 21:54:24
% !TEX encoding = UTF-8 Unicode
\begin{tikzpicture}[x=1pt,y=1pt]
\definecolor[named]{fillColor}{rgb}{1.00,1.00,1.00}
\path[use as bounding box,fill=fillColor,fill opacity=0.00] (0,0) rectangle (433.62,144.54);
\begin{scope}
\path[clip] (  0.00,  0.00) rectangle (433.62,144.54);
\definecolor[named]{drawColor}{rgb}{1.00,1.00,1.00}
\definecolor[named]{fillColor}{rgb}{1.00,1.00,1.00}

\path[draw=drawColor,line width= 0.6pt,line join=round,line cap=round,fill=fillColor] (  0.00,  0.00) rectangle (433.62,144.54);
\end{scope}
\begin{scope}
\path[clip] ( 51.42, 34.03) rectangle (322.26,132.50);
\definecolor[named]{fillColor}{rgb}{0.90,0.90,0.90}

\path[fill=fillColor] ( 51.42, 34.03) rectangle (322.26,132.50);
\definecolor[named]{drawColor}{rgb}{0.95,0.95,0.95}

\path[draw=drawColor,line width= 0.3pt,line join=round] ( 51.42, 66.28) --
	(322.26, 66.28);

\path[draw=drawColor,line width= 0.3pt,line join=round] ( 51.42, 98.96) --
	(322.26, 98.96);

\path[draw=drawColor,line width= 0.3pt,line join=round] ( 51.42,131.64) --
	(322.26,131.64);

\path[draw=drawColor,line width= 0.3pt,line join=round] ( 92.33, 34.03) --
	( 92.33,132.50);

\path[draw=drawColor,line width= 0.3pt,line join=round] (154.51, 34.03) --
	(154.51,132.50);

\path[draw=drawColor,line width= 0.3pt,line join=round] (216.68, 34.03) --
	(216.68,132.50);

\path[draw=drawColor,line width= 0.3pt,line join=round] (278.86, 34.03) --
	(278.86,132.50);
\definecolor[named]{drawColor}{rgb}{1.00,1.00,1.00}

\path[draw=drawColor,line width= 0.6pt,line join=round] ( 51.42, 49.93) --
	(322.26, 49.93);

\path[draw=drawColor,line width= 0.6pt,line join=round] ( 51.42, 82.62) --
	(322.26, 82.62);

\path[draw=drawColor,line width= 0.6pt,line join=round] ( 51.42,115.30) --
	(322.26,115.30);

\path[draw=drawColor,line width= 0.6pt,line join=round] ( 61.24, 34.03) --
	( 61.24,132.50);

\path[draw=drawColor,line width= 0.6pt,line join=round] (123.42, 34.03) --
	(123.42,132.50);

\path[draw=drawColor,line width= 0.6pt,line join=round] (185.59, 34.03) --
	(185.59,132.50);

\path[draw=drawColor,line width= 0.6pt,line join=round] (247.77, 34.03) --
	(247.77,132.50);

\path[draw=drawColor,line width= 0.6pt,line join=round] (309.95, 34.03) --
	(309.95,132.50);
\definecolor[named]{drawColor}{rgb}{0.40,0.76,0.65}

\path[draw=drawColor,line width= 0.6pt,line join=round] ( 63.73, 41.32) --
	( 66.22, 41.48) --
	( 68.70, 45.66) --
	( 71.19, 45.72) --
	( 73.68, 48.20) --
	( 76.16, 48.23) --
	( 78.65, 49.80) --
	( 81.14, 49.82) --
	( 83.62, 50.03) --
	( 86.11, 50.23) --
	( 88.60, 54.35) --
	( 91.09, 54.36) --
	( 93.57, 54.63) --
	( 96.06, 54.63) --
	( 98.55, 55.05) --
	(101.03, 55.19) --
	(103.52, 55.78) --
	(106.01, 55.80) --
	(108.49, 56.03) --
	(110.98, 56.08) --
	(113.47, 56.65) --
	(115.96, 56.73) --
	(118.44, 57.82) --
	(120.93, 57.84) --
	(123.42, 58.12) --
	(125.90, 58.12) --
	(128.39, 58.69) --
	(130.88, 58.71) --
	(133.37, 58.78) --
	(135.85, 58.82) --
	(138.34, 59.02) --
	(140.83, 59.04) --
	(143.31, 59.74) --
	(145.80, 59.78) --
	(148.29, 59.89) --
	(150.77, 59.90) --
	(153.26, 59.98) --
	(155.75, 60.00) --
	(158.24, 60.11) --
	(160.72, 60.11) --
	(163.21, 60.17) --
	(165.70, 60.21) --
	(168.18, 60.34) --
	(170.67, 60.35) --
	(173.16, 60.62) --
	(175.65, 60.63) --
	(178.13, 60.95) --
	(180.62, 60.96) --
	(183.11, 61.03) --
	(185.59, 61.05) --
	(188.08, 66.43) --
	(190.57, 66.43) --
	(193.05, 66.57) --
	(195.54, 66.57) --
	(198.03, 66.67) --
	(200.52, 66.68) --
	(203.00, 66.74) --
	(205.49, 66.75) --
	(207.98, 66.80) --
	(210.46, 66.81) --
	(212.95, 66.82) --
	(215.44, 66.83) --
	(217.92, 66.85) --
	(220.41, 66.85) --
	(222.90, 66.89) --
	(225.39, 66.90) --
	(227.87, 67.07) --
	(230.36, 67.08) --
	(232.85, 67.10) --
	(235.33, 67.10) --
	(237.82, 67.18) --
	(240.31, 67.18) --
	(242.80, 74.84) --
	(245.28, 74.84) --
	(247.77, 75.43) --
	(250.26, 75.48) --
	(252.74, 75.52) --
	(255.23, 75.52) --
	(257.72, 75.54) --
	(260.20, 75.54) --
	(262.69, 75.58) --
	(265.18, 75.59) --
	(267.67, 75.60) --
	(270.15, 75.61) --
	(272.64, 75.68) --
	(275.13, 75.68) --
	(277.61, 75.74) --
	(280.10, 75.74) --
	(282.59, 75.77) --
	(285.08, 75.78) --
	(287.56, 75.79) --
	(290.05, 75.79) --
	(292.54, 75.81) --
	(295.02, 75.81) --
	(297.51, 75.87) --
	(300.00, 75.87) --
	(302.48, 75.90) --
	(304.97, 75.90) --
	(307.46, 75.91) --
	(309.95, 75.91);
\definecolor[named]{drawColor}{rgb}{0.99,0.55,0.38}

\path[draw=drawColor,line width= 0.6pt,line join=round] ( 63.73, 38.51) --
	( 66.22, 39.50) --
	( 68.70, 44.42) --
	( 71.19, 44.57) --
	( 73.68, 47.50) --
	( 76.16, 47.60) --
	( 78.65, 50.86) --
	( 81.14, 50.92) --
	( 83.62, 52.01) --
	( 86.11, 52.12) --
	( 88.60, 57.11) --
	( 91.09, 57.14) --
	( 93.57, 57.61) --
	( 96.06, 57.63) --
	( 98.55, 63.21) --
	(101.03, 63.22) --
	(103.52, 64.28) --
	(106.01, 64.29) --
	(108.49, 64.56) --
	(110.98, 64.68) --
	(113.47, 69.13) --
	(115.96, 69.15) --
	(118.44, 69.28) --
	(120.93, 69.28) --
	(123.42, 69.76) --
	(125.90, 69.76) --
	(128.39, 81.42) --
	(130.88, 81.42) --
	(133.37, 81.44) --
	(135.85, 81.44) --
	(138.34, 81.46) --
	(140.83, 81.46) --
	(143.31, 81.56) --
	(145.80, 81.57) --
	(148.29, 81.64) --
	(150.77, 81.64) --
	(153.26, 81.66) --
	(155.75, 81.66) --
	(158.24, 81.69) --
	(160.72, 81.69) --
	(163.21, 81.71) --
	(165.70, 81.71) --
	(168.18, 81.73) --
	(170.67, 81.73) --
	(173.16, 81.76) --
	(175.65, 81.76) --
	(178.13, 82.57) --
	(180.62, 82.57) --
	(183.11, 82.58) --
	(185.59, 82.58) --
	(188.08, 82.66) --
	(190.57, 82.67) --
	(193.05, 82.68) --
	(195.54, 82.68) --
	(198.03, 82.73) --
	(200.52, 82.73) --
	(203.00, 82.74) --
	(205.49, 82.74) --
	(207.98, 82.76) --
	(210.46, 82.77) --
	(212.95, 82.78) --
	(215.44, 82.78) --
	(217.92, 82.79) --
	(220.41, 82.79) --
	(222.90, 82.81) --
	(225.39, 82.81) --
	(227.87, 82.83) --
	(230.36, 82.83) --
	(232.85, 82.84) --
	(235.33, 82.85) --
	(237.82, 82.86) --
	(240.31, 82.86) --
	(242.80, 82.95) --
	(245.28, 82.95) --
	(247.77, 82.98) --
	(250.26, 82.98) --
	(252.74, 83.03) --
	(255.23, 83.03) --
	(257.72, 83.05) --
	(260.20, 83.05) --
	(262.69, 83.48) --
	(265.18, 83.48) --
	(267.67, 83.71) --
	(270.15, 83.72) --
	(272.64, 91.78) --
	(275.13, 91.78) --
	(277.61, 92.01) --
	(280.10, 92.01) --
	(282.59, 92.02) --
	(285.08, 92.02) --
	(287.56, 92.02) --
	(290.05, 92.02) --
	(292.54, 92.03) --
	(295.02, 92.03) --
	(297.51, 92.04) --
	(300.00, 92.04) --
	(302.48, 92.04) --
	(304.97, 92.04) --
	(307.46, 92.04) --
	(309.95, 92.04);
\definecolor[named]{drawColor}{rgb}{0.55,0.63,0.80}

\path[draw=drawColor,line width= 0.6pt,line join=round] ( 63.73,100.25) --
	( 66.22,100.25) --
	( 68.70,105.17) --
	( 71.19,105.17) --
	( 73.68,108.05) --
	( 76.16,108.05) --
	( 78.65,110.09) --
	( 81.14,110.09) --
	( 83.62,111.68) --
	( 86.11,111.68) --
	( 88.60,112.97) --
	( 91.09,112.97) --
	( 93.57,114.07) --
	( 96.06,114.07) --
	( 98.55,115.01) --
	(101.03,115.01) --
	(103.52,115.85) --
	(106.01,115.85) --
	(108.49,116.60) --
	(110.98,116.60) --
	(113.47,117.27) --
	(115.96,117.27) --
	(118.44,117.89) --
	(120.93,117.89) --
	(123.42,118.46) --
	(125.90,118.46) --
	(128.39,118.98) --
	(130.88,118.98) --
	(133.37,119.47) --
	(135.85,119.47) --
	(138.34,119.93) --
	(140.83,119.93) --
	(143.31,120.36) --
	(145.80,120.36) --
	(148.29,120.77) --
	(150.77,120.77) --
	(153.26,121.15) --
	(155.75,121.15) --
	(158.24,121.52) --
	(160.72,121.52) --
	(163.21,121.86) --
	(165.70,121.86) --
	(168.18,122.19) --
	(170.67,122.19) --
	(173.16,122.51) --
	(175.65,122.51) --
	(178.13,122.81) --
	(180.62,122.81) --
	(183.11,123.10) --
	(185.59,123.10) --
	(188.08,123.38) --
	(190.57,123.38) --
	(193.05,123.65) --
	(195.54,123.65) --
	(198.03,123.90) --
	(200.52,123.90) --
	(203.00,124.15) --
	(205.49,124.15) --
	(207.98,124.39) --
	(210.46,124.39) --
	(212.95,124.63) --
	(215.44,124.63) --
	(217.92,124.85) --
	(220.41,124.85) --
	(222.90,125.07) --
	(225.39,125.07) --
	(227.87,125.28) --
	(230.36,125.28) --
	(232.85,125.49) --
	(235.33,125.49) --
	(237.82,125.69) --
	(240.31,125.69) --
	(242.80,125.88) --
	(245.28,125.88) --
	(247.77,126.07) --
	(250.26,126.07) --
	(252.74,126.26) --
	(255.23,126.26) --
	(257.72,126.44) --
	(260.20,126.44) --
	(262.69,126.61) --
	(265.18,126.61) --
	(267.67,126.78) --
	(270.15,126.78) --
	(272.64,126.95) --
	(275.13,126.95) --
	(277.61,127.11) --
	(280.10,127.11) --
	(282.59,127.27) --
	(285.08,127.27) --
	(287.56,127.43) --
	(290.05,127.43) --
	(292.54,127.58) --
	(295.02,127.58) --
	(297.51,127.73) --
	(300.00,127.73) --
	(302.48,127.88) --
	(304.97,127.88) --
	(307.46,128.02) --
	(309.95,128.02);
\definecolor[named]{drawColor}{rgb}{0.91,0.54,0.76}

\path[draw=drawColor,line width= 0.6pt,line join=round] ( 63.73, 39.80) --
	( 66.22, 64.36) --
	( 68.70, 64.60) --
	( 71.19, 64.88) --
	( 73.68, 65.13) --
	( 76.16, 65.38) --
	( 78.65, 65.58) --
	( 81.14, 65.78) --
	( 83.62, 69.30) --
	( 86.11, 77.81) --
	( 88.60, 77.84) --
	( 91.09, 77.88) --
	( 93.57, 79.07) --
	( 96.06, 81.11) --
	( 98.55, 82.13) --
	(101.03, 82.15) --
	(103.52, 82.18) --
	(106.01, 82.36) --
	(108.49, 82.37) --
	(110.98, 82.39) --
	(113.47, 83.02) --
	(115.96, 83.03) --
	(118.44, 83.55) --
	(120.93, 83.91) --
	(123.42, 83.93) --
	(125.90, 83.94) --
	(128.39, 83.95) --
	(130.88, 84.69) --
	(133.37, 84.71) --
	(135.85, 84.72) --
	(138.34, 85.35) --
	(140.83, 85.36) --
	(143.31, 87.36) --
	(145.80, 87.36) --
	(148.29, 87.37) --
	(150.77, 87.38) --
	(153.26, 87.39) --
	(155.75, 87.41) --
	(158.24, 87.41) --
	(160.72, 87.71) --
	(163.21, 87.71) --
	(165.70, 88.39) --
	(168.18, 88.40) --
	(170.67, 88.41) --
	(173.16, 88.42) --
	(175.65, 88.63) --
	(178.13, 88.63) --
	(180.62, 88.92) --
	(183.11, 89.12) --
	(185.59, 89.27);
\definecolor[named]{fillColor}{rgb}{0.40,0.76,0.65}

\path[fill=fillColor] ( 63.73, 41.32) circle (  1.60);

\path[fill=fillColor] ( 66.22, 41.48) circle (  1.60);

\path[fill=fillColor] ( 68.70, 45.66) circle (  1.60);

\path[fill=fillColor] ( 71.19, 45.72) circle (  1.60);

\path[fill=fillColor] ( 73.68, 48.20) circle (  1.60);

\path[fill=fillColor] ( 76.16, 48.23) circle (  1.60);

\path[fill=fillColor] ( 78.65, 49.80) circle (  1.60);

\path[fill=fillColor] ( 81.14, 49.82) circle (  1.60);

\path[fill=fillColor] ( 83.62, 50.03) circle (  1.60);

\path[fill=fillColor] ( 86.11, 50.23) circle (  1.60);

\path[fill=fillColor] ( 88.60, 54.35) circle (  1.60);

\path[fill=fillColor] ( 91.09, 54.36) circle (  1.60);

\path[fill=fillColor] ( 93.57, 54.63) circle (  1.60);

\path[fill=fillColor] ( 96.06, 54.63) circle (  1.60);

\path[fill=fillColor] ( 98.55, 55.05) circle (  1.60);

\path[fill=fillColor] (101.03, 55.19) circle (  1.60);

\path[fill=fillColor] (103.52, 55.78) circle (  1.60);

\path[fill=fillColor] (106.01, 55.80) circle (  1.60);

\path[fill=fillColor] (108.49, 56.03) circle (  1.60);

\path[fill=fillColor] (110.98, 56.08) circle (  1.60);

\path[fill=fillColor] (113.47, 56.65) circle (  1.60);

\path[fill=fillColor] (115.96, 56.73) circle (  1.60);

\path[fill=fillColor] (118.44, 57.82) circle (  1.60);

\path[fill=fillColor] (120.93, 57.84) circle (  1.60);

\path[fill=fillColor] (123.42, 58.12) circle (  1.60);

\path[fill=fillColor] (125.90, 58.12) circle (  1.60);

\path[fill=fillColor] (128.39, 58.69) circle (  1.60);

\path[fill=fillColor] (130.88, 58.71) circle (  1.60);

\path[fill=fillColor] (133.37, 58.78) circle (  1.60);

\path[fill=fillColor] (135.85, 58.82) circle (  1.60);

\path[fill=fillColor] (138.34, 59.02) circle (  1.60);

\path[fill=fillColor] (140.83, 59.04) circle (  1.60);

\path[fill=fillColor] (143.31, 59.74) circle (  1.60);

\path[fill=fillColor] (145.80, 59.78) circle (  1.60);

\path[fill=fillColor] (148.29, 59.89) circle (  1.60);

\path[fill=fillColor] (150.77, 59.90) circle (  1.60);

\path[fill=fillColor] (153.26, 59.98) circle (  1.60);

\path[fill=fillColor] (155.75, 60.00) circle (  1.60);

\path[fill=fillColor] (158.24, 60.11) circle (  1.60);

\path[fill=fillColor] (160.72, 60.11) circle (  1.60);

\path[fill=fillColor] (163.21, 60.17) circle (  1.60);

\path[fill=fillColor] (165.70, 60.21) circle (  1.60);

\path[fill=fillColor] (168.18, 60.34) circle (  1.60);

\path[fill=fillColor] (170.67, 60.35) circle (  1.60);

\path[fill=fillColor] (173.16, 60.62) circle (  1.60);

\path[fill=fillColor] (175.65, 60.63) circle (  1.60);

\path[fill=fillColor] (178.13, 60.95) circle (  1.60);

\path[fill=fillColor] (180.62, 60.96) circle (  1.60);

\path[fill=fillColor] (183.11, 61.03) circle (  1.60);

\path[fill=fillColor] (185.59, 61.05) circle (  1.60);

\path[fill=fillColor] (188.08, 66.43) circle (  1.60);

\path[fill=fillColor] (190.57, 66.43) circle (  1.60);

\path[fill=fillColor] (193.05, 66.57) circle (  1.60);

\path[fill=fillColor] (195.54, 66.57) circle (  1.60);

\path[fill=fillColor] (198.03, 66.67) circle (  1.60);

\path[fill=fillColor] (200.52, 66.68) circle (  1.60);

\path[fill=fillColor] (203.00, 66.74) circle (  1.60);

\path[fill=fillColor] (205.49, 66.75) circle (  1.60);

\path[fill=fillColor] (207.98, 66.80) circle (  1.60);

\path[fill=fillColor] (210.46, 66.81) circle (  1.60);

\path[fill=fillColor] (212.95, 66.82) circle (  1.60);

\path[fill=fillColor] (215.44, 66.83) circle (  1.60);

\path[fill=fillColor] (217.92, 66.85) circle (  1.60);

\path[fill=fillColor] (220.41, 66.85) circle (  1.60);

\path[fill=fillColor] (222.90, 66.89) circle (  1.60);

\path[fill=fillColor] (225.39, 66.90) circle (  1.60);

\path[fill=fillColor] (227.87, 67.07) circle (  1.60);

\path[fill=fillColor] (230.36, 67.08) circle (  1.60);

\path[fill=fillColor] (232.85, 67.10) circle (  1.60);

\path[fill=fillColor] (235.33, 67.10) circle (  1.60);

\path[fill=fillColor] (237.82, 67.18) circle (  1.60);

\path[fill=fillColor] (240.31, 67.18) circle (  1.60);

\path[fill=fillColor] (242.80, 74.84) circle (  1.60);

\path[fill=fillColor] (245.28, 74.84) circle (  1.60);

\path[fill=fillColor] (247.77, 75.43) circle (  1.60);

\path[fill=fillColor] (250.26, 75.48) circle (  1.60);

\path[fill=fillColor] (252.74, 75.52) circle (  1.60);

\path[fill=fillColor] (255.23, 75.52) circle (  1.60);

\path[fill=fillColor] (257.72, 75.54) circle (  1.60);

\path[fill=fillColor] (260.20, 75.54) circle (  1.60);

\path[fill=fillColor] (262.69, 75.58) circle (  1.60);

\path[fill=fillColor] (265.18, 75.59) circle (  1.60);

\path[fill=fillColor] (267.67, 75.60) circle (  1.60);

\path[fill=fillColor] (270.15, 75.61) circle (  1.60);

\path[fill=fillColor] (272.64, 75.68) circle (  1.60);

\path[fill=fillColor] (275.13, 75.68) circle (  1.60);

\path[fill=fillColor] (277.61, 75.74) circle (  1.60);

\path[fill=fillColor] (280.10, 75.74) circle (  1.60);

\path[fill=fillColor] (282.59, 75.77) circle (  1.60);

\path[fill=fillColor] (285.08, 75.78) circle (  1.60);

\path[fill=fillColor] (287.56, 75.79) circle (  1.60);

\path[fill=fillColor] (290.05, 75.79) circle (  1.60);

\path[fill=fillColor] (292.54, 75.81) circle (  1.60);

\path[fill=fillColor] (295.02, 75.81) circle (  1.60);

\path[fill=fillColor] (297.51, 75.87) circle (  1.60);

\path[fill=fillColor] (300.00, 75.87) circle (  1.60);

\path[fill=fillColor] (302.48, 75.90) circle (  1.60);

\path[fill=fillColor] (304.97, 75.90) circle (  1.60);

\path[fill=fillColor] (307.46, 75.91) circle (  1.60);

\path[fill=fillColor] (309.95, 75.91) circle (  1.60);
\definecolor[named]{fillColor}{rgb}{0.99,0.55,0.38}

\path[fill=fillColor] ( 63.73, 38.51) circle (  1.60);

\path[fill=fillColor] ( 66.22, 39.50) circle (  1.60);

\path[fill=fillColor] ( 68.70, 44.42) circle (  1.60);

\path[fill=fillColor] ( 71.19, 44.57) circle (  1.60);

\path[fill=fillColor] ( 73.68, 47.50) circle (  1.60);

\path[fill=fillColor] ( 76.16, 47.60) circle (  1.60);

\path[fill=fillColor] ( 78.65, 50.86) circle (  1.60);

\path[fill=fillColor] ( 81.14, 50.92) circle (  1.60);

\path[fill=fillColor] ( 83.62, 52.01) circle (  1.60);

\path[fill=fillColor] ( 86.11, 52.12) circle (  1.60);

\path[fill=fillColor] ( 88.60, 57.11) circle (  1.60);

\path[fill=fillColor] ( 91.09, 57.14) circle (  1.60);

\path[fill=fillColor] ( 93.57, 57.61) circle (  1.60);

\path[fill=fillColor] ( 96.06, 57.63) circle (  1.60);

\path[fill=fillColor] ( 98.55, 63.21) circle (  1.60);

\path[fill=fillColor] (101.03, 63.22) circle (  1.60);

\path[fill=fillColor] (103.52, 64.28) circle (  1.60);

\path[fill=fillColor] (106.01, 64.29) circle (  1.60);

\path[fill=fillColor] (108.49, 64.56) circle (  1.60);

\path[fill=fillColor] (110.98, 64.68) circle (  1.60);

\path[fill=fillColor] (113.47, 69.13) circle (  1.60);

\path[fill=fillColor] (115.96, 69.15) circle (  1.60);

\path[fill=fillColor] (118.44, 69.28) circle (  1.60);

\path[fill=fillColor] (120.93, 69.28) circle (  1.60);

\path[fill=fillColor] (123.42, 69.76) circle (  1.60);

\path[fill=fillColor] (125.90, 69.76) circle (  1.60);

\path[fill=fillColor] (128.39, 81.42) circle (  1.60);

\path[fill=fillColor] (130.88, 81.42) circle (  1.60);

\path[fill=fillColor] (133.37, 81.44) circle (  1.60);

\path[fill=fillColor] (135.85, 81.44) circle (  1.60);

\path[fill=fillColor] (138.34, 81.46) circle (  1.60);

\path[fill=fillColor] (140.83, 81.46) circle (  1.60);

\path[fill=fillColor] (143.31, 81.56) circle (  1.60);

\path[fill=fillColor] (145.80, 81.57) circle (  1.60);

\path[fill=fillColor] (148.29, 81.64) circle (  1.60);

\path[fill=fillColor] (150.77, 81.64) circle (  1.60);

\path[fill=fillColor] (153.26, 81.66) circle (  1.60);

\path[fill=fillColor] (155.75, 81.66) circle (  1.60);

\path[fill=fillColor] (158.24, 81.69) circle (  1.60);

\path[fill=fillColor] (160.72, 81.69) circle (  1.60);

\path[fill=fillColor] (163.21, 81.71) circle (  1.60);

\path[fill=fillColor] (165.70, 81.71) circle (  1.60);

\path[fill=fillColor] (168.18, 81.73) circle (  1.60);

\path[fill=fillColor] (170.67, 81.73) circle (  1.60);

\path[fill=fillColor] (173.16, 81.76) circle (  1.60);

\path[fill=fillColor] (175.65, 81.76) circle (  1.60);

\path[fill=fillColor] (178.13, 82.57) circle (  1.60);

\path[fill=fillColor] (180.62, 82.57) circle (  1.60);

\path[fill=fillColor] (183.11, 82.58) circle (  1.60);

\path[fill=fillColor] (185.59, 82.58) circle (  1.60);

\path[fill=fillColor] (188.08, 82.66) circle (  1.60);

\path[fill=fillColor] (190.57, 82.67) circle (  1.60);

\path[fill=fillColor] (193.05, 82.68) circle (  1.60);

\path[fill=fillColor] (195.54, 82.68) circle (  1.60);

\path[fill=fillColor] (198.03, 82.73) circle (  1.60);

\path[fill=fillColor] (200.52, 82.73) circle (  1.60);

\path[fill=fillColor] (203.00, 82.74) circle (  1.60);

\path[fill=fillColor] (205.49, 82.74) circle (  1.60);

\path[fill=fillColor] (207.98, 82.76) circle (  1.60);

\path[fill=fillColor] (210.46, 82.77) circle (  1.60);

\path[fill=fillColor] (212.95, 82.78) circle (  1.60);

\path[fill=fillColor] (215.44, 82.78) circle (  1.60);

\path[fill=fillColor] (217.92, 82.79) circle (  1.60);

\path[fill=fillColor] (220.41, 82.79) circle (  1.60);

\path[fill=fillColor] (222.90, 82.81) circle (  1.60);

\path[fill=fillColor] (225.39, 82.81) circle (  1.60);

\path[fill=fillColor] (227.87, 82.83) circle (  1.60);

\path[fill=fillColor] (230.36, 82.83) circle (  1.60);

\path[fill=fillColor] (232.85, 82.84) circle (  1.60);

\path[fill=fillColor] (235.33, 82.85) circle (  1.60);

\path[fill=fillColor] (237.82, 82.86) circle (  1.60);

\path[fill=fillColor] (240.31, 82.86) circle (  1.60);

\path[fill=fillColor] (242.80, 82.95) circle (  1.60);

\path[fill=fillColor] (245.28, 82.95) circle (  1.60);

\path[fill=fillColor] (247.77, 82.98) circle (  1.60);

\path[fill=fillColor] (250.26, 82.98) circle (  1.60);

\path[fill=fillColor] (252.74, 83.03) circle (  1.60);

\path[fill=fillColor] (255.23, 83.03) circle (  1.60);

\path[fill=fillColor] (257.72, 83.05) circle (  1.60);

\path[fill=fillColor] (260.20, 83.05) circle (  1.60);

\path[fill=fillColor] (262.69, 83.48) circle (  1.60);

\path[fill=fillColor] (265.18, 83.48) circle (  1.60);

\path[fill=fillColor] (267.67, 83.71) circle (  1.60);

\path[fill=fillColor] (270.15, 83.72) circle (  1.60);

\path[fill=fillColor] (272.64, 91.78) circle (  1.60);

\path[fill=fillColor] (275.13, 91.78) circle (  1.60);

\path[fill=fillColor] (277.61, 92.01) circle (  1.60);

\path[fill=fillColor] (280.10, 92.01) circle (  1.60);

\path[fill=fillColor] (282.59, 92.02) circle (  1.60);

\path[fill=fillColor] (285.08, 92.02) circle (  1.60);

\path[fill=fillColor] (287.56, 92.02) circle (  1.60);

\path[fill=fillColor] (290.05, 92.02) circle (  1.60);

\path[fill=fillColor] (292.54, 92.03) circle (  1.60);

\path[fill=fillColor] (295.02, 92.03) circle (  1.60);

\path[fill=fillColor] (297.51, 92.04) circle (  1.60);

\path[fill=fillColor] (300.00, 92.04) circle (  1.60);

\path[fill=fillColor] (302.48, 92.04) circle (  1.60);

\path[fill=fillColor] (304.97, 92.04) circle (  1.60);

\path[fill=fillColor] (307.46, 92.04) circle (  1.60);

\path[fill=fillColor] (309.95, 92.04) circle (  1.60);
\definecolor[named]{fillColor}{rgb}{0.55,0.63,0.80}

\path[fill=fillColor] ( 63.73,100.25) circle (  1.60);

\path[fill=fillColor] ( 66.22,100.25) circle (  1.60);

\path[fill=fillColor] ( 68.70,105.17) circle (  1.60);

\path[fill=fillColor] ( 71.19,105.17) circle (  1.60);

\path[fill=fillColor] ( 73.68,108.05) circle (  1.60);

\path[fill=fillColor] ( 76.16,108.05) circle (  1.60);

\path[fill=fillColor] ( 78.65,110.09) circle (  1.60);

\path[fill=fillColor] ( 81.14,110.09) circle (  1.60);

\path[fill=fillColor] ( 83.62,111.68) circle (  1.60);

\path[fill=fillColor] ( 86.11,111.68) circle (  1.60);

\path[fill=fillColor] ( 88.60,112.97) circle (  1.60);

\path[fill=fillColor] ( 91.09,112.97) circle (  1.60);

\path[fill=fillColor] ( 93.57,114.07) circle (  1.60);

\path[fill=fillColor] ( 96.06,114.07) circle (  1.60);

\path[fill=fillColor] ( 98.55,115.01) circle (  1.60);

\path[fill=fillColor] (101.03,115.01) circle (  1.60);

\path[fill=fillColor] (103.52,115.85) circle (  1.60);

\path[fill=fillColor] (106.01,115.85) circle (  1.60);

\path[fill=fillColor] (108.49,116.60) circle (  1.60);

\path[fill=fillColor] (110.98,116.60) circle (  1.60);

\path[fill=fillColor] (113.47,117.27) circle (  1.60);

\path[fill=fillColor] (115.96,117.27) circle (  1.60);

\path[fill=fillColor] (118.44,117.89) circle (  1.60);

\path[fill=fillColor] (120.93,117.89) circle (  1.60);

\path[fill=fillColor] (123.42,118.46) circle (  1.60);

\path[fill=fillColor] (125.90,118.46) circle (  1.60);

\path[fill=fillColor] (128.39,118.98) circle (  1.60);

\path[fill=fillColor] (130.88,118.98) circle (  1.60);

\path[fill=fillColor] (133.37,119.47) circle (  1.60);

\path[fill=fillColor] (135.85,119.47) circle (  1.60);

\path[fill=fillColor] (138.34,119.93) circle (  1.60);

\path[fill=fillColor] (140.83,119.93) circle (  1.60);

\path[fill=fillColor] (143.31,120.36) circle (  1.60);

\path[fill=fillColor] (145.80,120.36) circle (  1.60);

\path[fill=fillColor] (148.29,120.77) circle (  1.60);

\path[fill=fillColor] (150.77,120.77) circle (  1.60);

\path[fill=fillColor] (153.26,121.15) circle (  1.60);

\path[fill=fillColor] (155.75,121.15) circle (  1.60);

\path[fill=fillColor] (158.24,121.52) circle (  1.60);

\path[fill=fillColor] (160.72,121.52) circle (  1.60);

\path[fill=fillColor] (163.21,121.86) circle (  1.60);

\path[fill=fillColor] (165.70,121.86) circle (  1.60);

\path[fill=fillColor] (168.18,122.19) circle (  1.60);

\path[fill=fillColor] (170.67,122.19) circle (  1.60);

\path[fill=fillColor] (173.16,122.51) circle (  1.60);

\path[fill=fillColor] (175.65,122.51) circle (  1.60);

\path[fill=fillColor] (178.13,122.81) circle (  1.60);

\path[fill=fillColor] (180.62,122.81) circle (  1.60);

\path[fill=fillColor] (183.11,123.10) circle (  1.60);

\path[fill=fillColor] (185.59,123.10) circle (  1.60);

\path[fill=fillColor] (188.08,123.38) circle (  1.60);

\path[fill=fillColor] (190.57,123.38) circle (  1.60);

\path[fill=fillColor] (193.05,123.65) circle (  1.60);

\path[fill=fillColor] (195.54,123.65) circle (  1.60);

\path[fill=fillColor] (198.03,123.90) circle (  1.60);

\path[fill=fillColor] (200.52,123.90) circle (  1.60);

\path[fill=fillColor] (203.00,124.15) circle (  1.60);

\path[fill=fillColor] (205.49,124.15) circle (  1.60);

\path[fill=fillColor] (207.98,124.39) circle (  1.60);

\path[fill=fillColor] (210.46,124.39) circle (  1.60);

\path[fill=fillColor] (212.95,124.63) circle (  1.60);

\path[fill=fillColor] (215.44,124.63) circle (  1.60);

\path[fill=fillColor] (217.92,124.85) circle (  1.60);

\path[fill=fillColor] (220.41,124.85) circle (  1.60);

\path[fill=fillColor] (222.90,125.07) circle (  1.60);

\path[fill=fillColor] (225.39,125.07) circle (  1.60);

\path[fill=fillColor] (227.87,125.28) circle (  1.60);

\path[fill=fillColor] (230.36,125.28) circle (  1.60);

\path[fill=fillColor] (232.85,125.49) circle (  1.60);

\path[fill=fillColor] (235.33,125.49) circle (  1.60);

\path[fill=fillColor] (237.82,125.69) circle (  1.60);

\path[fill=fillColor] (240.31,125.69) circle (  1.60);

\path[fill=fillColor] (242.80,125.88) circle (  1.60);

\path[fill=fillColor] (245.28,125.88) circle (  1.60);

\path[fill=fillColor] (247.77,126.07) circle (  1.60);

\path[fill=fillColor] (250.26,126.07) circle (  1.60);

\path[fill=fillColor] (252.74,126.26) circle (  1.60);

\path[fill=fillColor] (255.23,126.26) circle (  1.60);

\path[fill=fillColor] (257.72,126.44) circle (  1.60);

\path[fill=fillColor] (260.20,126.44) circle (  1.60);

\path[fill=fillColor] (262.69,126.61) circle (  1.60);

\path[fill=fillColor] (265.18,126.61) circle (  1.60);

\path[fill=fillColor] (267.67,126.78) circle (  1.60);

\path[fill=fillColor] (270.15,126.78) circle (  1.60);

\path[fill=fillColor] (272.64,126.95) circle (  1.60);

\path[fill=fillColor] (275.13,126.95) circle (  1.60);

\path[fill=fillColor] (277.61,127.11) circle (  1.60);

\path[fill=fillColor] (280.10,127.11) circle (  1.60);

\path[fill=fillColor] (282.59,127.27) circle (  1.60);

\path[fill=fillColor] (285.08,127.27) circle (  1.60);

\path[fill=fillColor] (287.56,127.43) circle (  1.60);

\path[fill=fillColor] (290.05,127.43) circle (  1.60);

\path[fill=fillColor] (292.54,127.58) circle (  1.60);

\path[fill=fillColor] (295.02,127.58) circle (  1.60);

\path[fill=fillColor] (297.51,127.73) circle (  1.60);

\path[fill=fillColor] (300.00,127.73) circle (  1.60);

\path[fill=fillColor] (302.48,127.88) circle (  1.60);

\path[fill=fillColor] (304.97,127.88) circle (  1.60);

\path[fill=fillColor] (307.46,128.02) circle (  1.60);

\path[fill=fillColor] (309.95,128.02) circle (  1.60);
\definecolor[named]{fillColor}{rgb}{0.91,0.54,0.76}

\path[fill=fillColor] ( 63.73, 39.80) circle (  1.60);

\path[fill=fillColor] ( 66.22, 64.36) circle (  1.60);

\path[fill=fillColor] ( 68.70, 64.60) circle (  1.60);

\path[fill=fillColor] ( 71.19, 64.88) circle (  1.60);

\path[fill=fillColor] ( 73.68, 65.13) circle (  1.60);

\path[fill=fillColor] ( 76.16, 65.38) circle (  1.60);

\path[fill=fillColor] ( 78.65, 65.58) circle (  1.60);

\path[fill=fillColor] ( 81.14, 65.78) circle (  1.60);

\path[fill=fillColor] ( 83.62, 69.30) circle (  1.60);

\path[fill=fillColor] ( 86.11, 77.81) circle (  1.60);

\path[fill=fillColor] ( 88.60, 77.84) circle (  1.60);

\path[fill=fillColor] ( 91.09, 77.88) circle (  1.60);

\path[fill=fillColor] ( 93.57, 79.07) circle (  1.60);

\path[fill=fillColor] ( 96.06, 81.11) circle (  1.60);

\path[fill=fillColor] ( 98.55, 82.13) circle (  1.60);

\path[fill=fillColor] (101.03, 82.15) circle (  1.60);

\path[fill=fillColor] (103.52, 82.18) circle (  1.60);

\path[fill=fillColor] (106.01, 82.36) circle (  1.60);

\path[fill=fillColor] (108.49, 82.37) circle (  1.60);

\path[fill=fillColor] (110.98, 82.39) circle (  1.60);

\path[fill=fillColor] (113.47, 83.02) circle (  1.60);

\path[fill=fillColor] (115.96, 83.03) circle (  1.60);

\path[fill=fillColor] (118.44, 83.55) circle (  1.60);

\path[fill=fillColor] (120.93, 83.91) circle (  1.60);

\path[fill=fillColor] (123.42, 83.93) circle (  1.60);

\path[fill=fillColor] (125.90, 83.94) circle (  1.60);

\path[fill=fillColor] (128.39, 83.95) circle (  1.60);

\path[fill=fillColor] (130.88, 84.69) circle (  1.60);

\path[fill=fillColor] (133.37, 84.71) circle (  1.60);

\path[fill=fillColor] (135.85, 84.72) circle (  1.60);

\path[fill=fillColor] (138.34, 85.35) circle (  1.60);

\path[fill=fillColor] (140.83, 85.36) circle (  1.60);

\path[fill=fillColor] (143.31, 87.36) circle (  1.60);

\path[fill=fillColor] (145.80, 87.36) circle (  1.60);

\path[fill=fillColor] (148.29, 87.37) circle (  1.60);

\path[fill=fillColor] (150.77, 87.38) circle (  1.60);

\path[fill=fillColor] (153.26, 87.39) circle (  1.60);

\path[fill=fillColor] (155.75, 87.41) circle (  1.60);

\path[fill=fillColor] (158.24, 87.41) circle (  1.60);

\path[fill=fillColor] (160.72, 87.71) circle (  1.60);

\path[fill=fillColor] (163.21, 87.71) circle (  1.60);

\path[fill=fillColor] (165.70, 88.39) circle (  1.60);

\path[fill=fillColor] (168.18, 88.40) circle (  1.60);

\path[fill=fillColor] (170.67, 88.41) circle (  1.60);

\path[fill=fillColor] (173.16, 88.42) circle (  1.60);

\path[fill=fillColor] (175.65, 88.63) circle (  1.60);

\path[fill=fillColor] (178.13, 88.63) circle (  1.60);

\path[fill=fillColor] (180.62, 88.92) circle (  1.60);

\path[fill=fillColor] (183.11, 89.12) circle (  1.60);

\path[fill=fillColor] (185.59, 89.27) circle (  1.60);
\end{scope}
\begin{scope}
\path[clip] (  0.00,  0.00) rectangle (433.62,144.54);
\definecolor[named]{drawColor}{rgb}{0.50,0.50,0.50}

\node[text=drawColor,anchor=base east,inner sep=0pt, outer sep=0pt, scale=  0.96] at ( 44.30, 46.63) {1};

\node[text=drawColor,anchor=base east,inner sep=0pt, outer sep=0pt, scale=  0.96] at ( 44.30, 79.31) {100};

\node[text=drawColor,anchor=base east,inner sep=0pt, outer sep=0pt, scale=  0.96] at ( 44.30,112.00) {10000};
\end{scope}
\begin{scope}
\path[clip] (  0.00,  0.00) rectangle (433.62,144.54);
\definecolor[named]{drawColor}{rgb}{0.50,0.50,0.50}

\path[draw=drawColor,line width= 0.6pt,line join=round] ( 47.15, 49.93) --
	( 51.42, 49.93);

\path[draw=drawColor,line width= 0.6pt,line join=round] ( 47.15, 82.62) --
	( 51.42, 82.62);

\path[draw=drawColor,line width= 0.6pt,line join=round] ( 47.15,115.30) --
	( 51.42,115.30);
\end{scope}
\begin{scope}
\path[clip] (  0.00,  0.00) rectangle (433.62,144.54);
\definecolor[named]{drawColor}{rgb}{0.50,0.50,0.50}

\path[draw=drawColor,line width= 0.6pt,line join=round] ( 61.24, 29.77) --
	( 61.24, 34.03);

\path[draw=drawColor,line width= 0.6pt,line join=round] (123.42, 29.77) --
	(123.42, 34.03);

\path[draw=drawColor,line width= 0.6pt,line join=round] (185.59, 29.77) --
	(185.59, 34.03);

\path[draw=drawColor,line width= 0.6pt,line join=round] (247.77, 29.77) --
	(247.77, 34.03);

\path[draw=drawColor,line width= 0.6pt,line join=round] (309.95, 29.77) --
	(309.95, 34.03);
\end{scope}
\begin{scope}
\path[clip] (  0.00,  0.00) rectangle (433.62,144.54);
\definecolor[named]{drawColor}{rgb}{0.50,0.50,0.50}

\node[text=drawColor,anchor=base,inner sep=0pt, outer sep=0pt, scale=  0.96] at ( 61.24, 20.31) {0};

\node[text=drawColor,anchor=base,inner sep=0pt, outer sep=0pt, scale=  0.96] at (123.42, 20.31) {25};

\node[text=drawColor,anchor=base,inner sep=0pt, outer sep=0pt, scale=  0.96] at (185.59, 20.31) {50};

\node[text=drawColor,anchor=base,inner sep=0pt, outer sep=0pt, scale=  0.96] at (247.77, 20.31) {75};

\node[text=drawColor,anchor=base,inner sep=0pt, outer sep=0pt, scale=  0.96] at (309.95, 20.31) {100};
\end{scope}
\begin{scope}
\path[clip] (  0.00,  0.00) rectangle (433.62,144.54);
\definecolor[named]{drawColor}{rgb}{0.00,0.00,0.00}

\node[text=drawColor,anchor=base,inner sep=0pt, outer sep=0pt, scale=  1] at (186.84,  8.53) {Number of problems};
\end{scope}
\begin{scope}
\path[clip] (  0.00,  0.00) rectangle (433.62,144.54);
\definecolor[named]{drawColor}{rgb}{0.00,0.00,0.00}

\node[text=drawColor,rotate= 90.00,anchor=base,inner sep=0pt, outer sep=0pt, scale=  1] at ( 16.80, 83.26) {Cumulative CPU time};
\end{scope}
\begin{scope}
\path[clip] (  0.00,  0.00) rectangle (433.62,144.54);
\definecolor[named]{fillColor}{rgb}{1.00,1.00,1.00}

\path[fill=fillColor] (331.12, 44.97) rectangle (412.71,121.56);
\end{scope}
\begin{scope}
\path[clip] (  0.00,  0.00) rectangle (433.62,144.54);
\definecolor[named]{drawColor}{rgb}{0.00,0.00,0.00}

\node[text=drawColor,anchor=base west,inner sep=0pt, outer sep=0pt, scale=  0.96] at (335.39,110.67) {\bfseries Solver};
\end{scope}
\begin{scope}
\path[clip] (  0.00,  0.00) rectangle (433.62,144.54);
\definecolor[named]{drawColor}{rgb}{1.00,1.00,1.00}
\definecolor[named]{fillColor}{rgb}{0.95,0.95,0.95}

\path[draw=drawColor,line width= 0.6pt,line join=round,line cap=round,fill=fillColor] (335.39, 92.60) rectangle (349.85,107.05);
\end{scope}
\begin{scope}
\path[clip] (  0.00,  0.00) rectangle (433.62,144.54);
\definecolor[named]{drawColor}{rgb}{0.40,0.76,0.65}

\path[draw=drawColor,line width= 0.6pt,line join=round] (336.84, 99.83) -- (348.40, 99.83);
\end{scope}
\begin{scope}
\path[clip] (  0.00,  0.00) rectangle (433.62,144.54);
\definecolor[named]{fillColor}{rgb}{0.40,0.76,0.65}

\path[fill=fillColor] (342.62, 99.83) circle (  1.60);
\end{scope}
\begin{scope}
\path[clip] (  0.00,  0.00) rectangle (433.62,144.54);
\definecolor[named]{drawColor}{rgb}{1.00,1.00,1.00}
\definecolor[named]{fillColor}{rgb}{0.95,0.95,0.95}

\path[draw=drawColor,line width= 0.6pt,line join=round,line cap=round,fill=fillColor] (335.39, 78.15) rectangle (349.85, 92.60);
\end{scope}
\begin{scope}
\path[clip] (  0.00,  0.00) rectangle (433.62,144.54);
\definecolor[named]{drawColor}{rgb}{0.99,0.55,0.38}

\path[draw=drawColor,line width= 0.6pt,line join=round] (336.84, 85.37) -- (348.40, 85.37);
\end{scope}
\begin{scope}
\path[clip] (  0.00,  0.00) rectangle (433.62,144.54);
\definecolor[named]{fillColor}{rgb}{0.99,0.55,0.38}

\path[fill=fillColor] (342.62, 85.37) circle (  1.60);
\end{scope}
\begin{scope}
\path[clip] (  0.00,  0.00) rectangle (433.62,144.54);
\definecolor[named]{drawColor}{rgb}{1.00,1.00,1.00}
\definecolor[named]{fillColor}{rgb}{0.95,0.95,0.95}

\path[draw=drawColor,line width= 0.6pt,line join=round,line cap=round,fill=fillColor] (335.39, 63.69) rectangle (349.85, 78.15);
\end{scope}
\begin{scope}
\path[clip] (  0.00,  0.00) rectangle (433.62,144.54);
\definecolor[named]{drawColor}{rgb}{0.55,0.63,0.80}

\path[draw=drawColor,line width= 0.6pt,line join=round] (336.84, 70.92) -- (348.40, 70.92);
\end{scope}
\begin{scope}
\path[clip] (  0.00,  0.00) rectangle (433.62,144.54);
\definecolor[named]{fillColor}{rgb}{0.55,0.63,0.80}

\path[fill=fillColor] (342.62, 70.92) circle (  1.60);
\end{scope}
\begin{scope}
\path[clip] (  0.00,  0.00) rectangle (433.62,144.54);
\definecolor[named]{drawColor}{rgb}{1.00,1.00,1.00}
\definecolor[named]{fillColor}{rgb}{0.95,0.95,0.95}

\path[draw=drawColor,line width= 0.6pt,line join=round,line cap=round,fill=fillColor] (335.39, 49.24) rectangle (349.85, 63.69);
\end{scope}
\begin{scope}
\path[clip] (  0.00,  0.00) rectangle (433.62,144.54);
\definecolor[named]{drawColor}{rgb}{0.91,0.54,0.76}

\path[draw=drawColor,line width= 0.6pt,line join=round] (336.84, 56.46) -- (348.40, 56.46);
\end{scope}
\begin{scope}
\path[clip] (  0.00,  0.00) rectangle (433.62,144.54);
\definecolor[named]{fillColor}{rgb}{0.91,0.54,0.76}

\path[fill=fillColor] (342.62, 56.46) circle (  1.60);
\end{scope}
\begin{scope}
\path[clip] (  0.00,  0.00) rectangle (433.62,144.54);
\definecolor[named]{drawColor}{rgb}{0.00,0.00,0.00}

\node[text=drawColor,anchor=base west,inner sep=0pt, outer sep=0pt, scale=  0.96] at (351.65, 96.52) {In-the-middle};
\end{scope}
\begin{scope}
\path[clip] (  0.00,  0.00) rectangle (433.62,144.54);
\definecolor[named]{drawColor}{rgb}{0.00,0.00,0.00}

\node[text=drawColor,anchor=base west,inner sep=0pt, outer sep=0pt, scale=  0.96] at (351.65, 82.07) {Toulbar2};
\end{scope}
\begin{scope}
\path[clip] (  0.00,  0.00) rectangle (433.62,144.54);
\definecolor[named]{drawColor}{rgb}{0.00,0.00,0.00}

\node[text=drawColor,anchor=base west,inner sep=0pt, outer sep=0pt, scale=  0.96] at (351.65, 67.61) {CPLEX};
\end{scope}
\begin{scope}
\path[clip] (  0.00,  0.00) rectangle (433.62,144.54);
\definecolor[named]{drawColor}{rgb}{0.00,0.00,0.00}

\node[text=drawColor,anchor=base west,inner sep=0pt, outer sep=0pt, scale=  0.96] at (351.65, 53.16) {MaxHS};
\end{scope}
\end{tikzpicture}
}
	\end{figcenter}
	\\
	\begin{figcenter}
	\subfloat[The \emph{DBN} set of the \gls{mrf} category.\label{fig:cactus-std:dbn}]{% Created by tikzDevice version 0.7.0 on 2014-06-05 13:06:24
% !TEX encoding = UTF-8 Unicode
\begin{tikzpicture}[x=1pt,y=1pt]
\definecolor[named]{fillColor}{rgb}{1.00,1.00,1.00}
\path[use as bounding box,fill=fillColor,fill opacity=0.00] (0,0) rectangle (433.62,144.54);
\begin{scope}
\path[clip] (  0.00,  0.00) rectangle (433.62,144.54);
\definecolor[named]{drawColor}{rgb}{1.00,1.00,1.00}
\definecolor[named]{fillColor}{rgb}{1.00,1.00,1.00}

\path[draw=drawColor,line width= 0.6pt,line join=round,line cap=round,fill=fillColor] (  0.00,  0.00) rectangle (433.62,144.54);
\end{scope}
\begin{scope}
\path[clip] ( 51.42, 34.03) rectangle (322.26,132.50);
\definecolor[named]{fillColor}{rgb}{0.90,0.90,0.90}

\path[fill=fillColor] ( 51.42, 34.03) rectangle (322.26,132.50);
\definecolor[named]{drawColor}{rgb}{0.95,0.95,0.95}

\path[draw=drawColor,line width= 0.3pt,line join=round] ( 51.42, 44.29) --
	(322.26, 44.29);

\path[draw=drawColor,line width= 0.3pt,line join=round] ( 51.42, 73.36) --
	(322.26, 73.36);

\path[draw=drawColor,line width= 0.3pt,line join=round] ( 51.42,102.43) --
	(322.26,102.43);

\path[draw=drawColor,line width= 0.3pt,line join=round] ( 51.42,131.49) --
	(322.26,131.49);

\path[draw=drawColor,line width= 0.3pt,line join=round] ( 95.94, 34.03) --
	( 95.94,132.50);

\path[draw=drawColor,line width= 0.3pt,line join=round] (164.98, 34.03) --
	(164.98,132.50);

\path[draw=drawColor,line width= 0.3pt,line join=round] (234.01, 34.03) --
	(234.01,132.50);

\path[draw=drawColor,line width= 0.3pt,line join=round] (303.04, 34.03) --
	(303.04,132.50);
\definecolor[named]{drawColor}{rgb}{1.00,1.00,1.00}

\path[draw=drawColor,line width= 0.6pt,line join=round] ( 51.42, 58.83) --
	(322.26, 58.83);

\path[draw=drawColor,line width= 0.6pt,line join=round] ( 51.42, 87.89) --
	(322.26, 87.89);

\path[draw=drawColor,line width= 0.6pt,line join=round] ( 51.42,116.96) --
	(322.26,116.96);

\path[draw=drawColor,line width= 0.6pt,line join=round] ( 61.43, 34.03) --
	( 61.43,132.50);

\path[draw=drawColor,line width= 0.6pt,line join=round] (130.46, 34.03) --
	(130.46,132.50);

\path[draw=drawColor,line width= 0.6pt,line join=round] (199.49, 34.03) --
	(199.49,132.50);

\path[draw=drawColor,line width= 0.6pt,line join=round] (268.53, 34.03) --
	(268.53,132.50);
\definecolor[named]{drawColor}{rgb}{0.40,0.76,0.65}

\path[draw=drawColor,line width= 0.6pt,line join=round] ( 63.73, 74.14) --
	( 66.03, 78.58) --
	( 68.33, 81.30) --
	( 70.63, 83.21) --
	( 72.93, 84.72) --
	( 75.23, 85.96) --
	( 77.53, 87.01) --
	( 79.84, 87.91) --
	( 82.14, 88.70) --
	( 84.44, 89.45) --
	( 86.74, 90.12) --
	( 89.04, 90.74) --
	( 91.34, 91.30) --
	( 93.64, 91.82) --
	( 95.94, 92.31) --
	( 98.24, 92.77) --
	(100.55, 93.20) --
	(102.85, 93.61) --
	(105.15, 94.00) --
	(107.45, 94.37) --
	(109.75, 94.73) --
	(112.05, 95.06) --
	(114.35, 95.39) --
	(116.65, 95.69) --
	(118.95, 95.99) --
	(121.26, 96.27) --
	(123.56, 96.54) --
	(125.86, 96.81) --
	(128.16, 97.06) --
	(130.46, 97.32) --
	(132.76, 97.71) --
	(135.06, 98.09) --
	(137.36, 98.45) --
	(139.66, 98.79) --
	(141.97, 99.13) --
	(144.27, 99.44) --
	(146.57, 99.75) --
	(148.87,100.04) --
	(151.17,100.32) --
	(153.47,100.59) --
	(155.77,100.85) --
	(158.07,101.10) --
	(160.37,101.35) --
	(162.68,101.59) --
	(164.98,101.82) --
	(167.28,102.05) --
	(169.58,102.27) --
	(171.88,102.48) --
	(174.18,102.69) --
	(176.48,102.90) --
	(178.78,103.10) --
	(181.08,103.31) --
	(183.39,103.51) --
	(185.69,103.71) --
	(187.99,103.90) --
	(190.29,104.09) --
	(192.59,104.27) --
	(194.89,104.45) --
	(197.19,104.65) --
	(199.49,104.84) --
	(201.79,105.18) --
	(204.09,105.50) --
	(206.40,105.81) --
	(208.70,106.11) --
	(211.00,106.39) --
	(213.30,106.66) --
	(215.60,106.92) --
	(217.90,107.18) --
	(220.20,107.42) --
	(222.50,107.71) --
	(224.80,108.02) --
	(227.11,108.45) --
	(229.41,108.85) --
	(231.71,109.24) --
	(234.01,109.61) --
	(236.31,109.96) --
	(238.61,110.30) --
	(240.91,110.62) --
	(243.21,110.93) --
	(245.51,111.23) --
	(247.82,111.53) --
	(250.12,111.82) --
	(252.42,112.09) --
	(254.72,112.38) --
	(257.02,112.66) --
	(259.32,112.92) --
	(261.62,113.18) --
	(263.92,113.44) --
	(266.22,113.69) --
	(268.53,113.98) --
	(270.83,114.38) --
	(273.13,114.74) --
	(275.43,115.11) --
	(277.73,115.46) --
	(280.03,115.79) --
	(282.33,116.11) --
	(284.63,116.43) --
	(286.93,116.75) --
	(289.24,117.07) --
	(291.54,117.40) --
	(293.84,117.73) --
	(296.14,118.13) --
	(298.44,118.51) --
	(300.74,118.89) --
	(303.04,119.25) --
	(305.34,119.62) --
	(307.64,119.98) --
	(309.95,120.40);
\definecolor[named]{drawColor}{rgb}{0.99,0.55,0.38}

\path[draw=drawColor,line width= 0.6pt,line join=round] ( 63.73, 38.51) --
	( 66.03, 42.89) --
	( 68.33, 45.95) --
	( 70.63, 48.00) --
	( 72.93, 49.55) --
	( 75.23, 50.79) --
	( 77.53, 51.83) --
	( 79.84, 52.72) --
	( 82.14, 53.50) --
	( 84.44, 54.19) --
	( 86.74, 54.82) --
	( 89.04, 55.39) --
	( 91.34, 56.01) --
	( 93.64, 56.58) --
	( 95.94, 57.09) --
	( 98.24, 57.57) --
	(100.55, 58.02) --
	(102.85, 58.44) --
	(105.15, 58.83) --
	(107.45, 59.19) --
	(109.75, 59.60) --
	(112.05, 59.98) --
	(114.35, 60.34) --
	(116.65, 60.67) --
	(118.95, 61.00) --
	(121.26, 61.30) --
	(123.56, 61.59) --
	(125.86, 61.87) --
	(128.16, 62.14) --
	(130.46, 62.40) --
	(132.76, 62.68) --
	(135.06, 62.94) --
	(137.36, 63.20) --
	(139.66, 63.45) --
	(141.97, 63.69) --
	(144.27, 63.92) --
	(146.57, 64.17) --
	(148.87, 64.43) --
	(151.17, 64.69) --
	(153.47, 64.95) --
	(155.77, 65.21) --
	(158.07, 65.46) --
	(160.37, 65.70) --
	(162.68, 65.95) --
	(164.98, 66.19) --
	(167.28, 66.42) --
	(169.58, 66.66) --
	(171.88, 66.89) --
	(174.18, 67.14) --
	(176.48, 67.37) --
	(178.78, 67.61) --
	(181.08, 67.85) --
	(183.39, 68.09) --
	(185.69, 68.35) --
	(187.99, 68.60) --
	(190.29, 68.85) --
	(192.59, 69.15) --
	(194.89, 69.49) --
	(197.19, 69.85) --
	(199.49, 70.52) --
	(201.79, 74.59) --
	(204.09, 77.50) --
	(206.40, 79.91) --
	(208.70, 82.12) --
	(211.00, 86.15) --
	(213.30, 88.63) --
	(215.60, 90.53) --
	(217.90, 92.73) --
	(220.20, 95.24) --
	(222.50, 97.27) --
	(224.80, 98.95) --
	(227.11,100.92) --
	(229.41,102.44) --
	(231.71,104.38) --
	(234.01,105.88) --
	(236.31,107.39) --
	(238.61,108.88) --
	(240.91,111.15) --
	(243.21,112.81) --
	(245.51,114.12) --
	(247.82,115.21) --
	(250.12,116.14) --
	(252.42,116.95) --
	(254.72,117.67) --
	(257.02,118.31) --
	(259.32,118.89) --
	(261.62,119.43) --
	(263.92,119.92) --
	(266.22,120.38) --
	(268.53,120.80) --
	(270.83,121.20) --
	(273.13,121.58) --
	(275.43,121.93) --
	(277.73,122.27) --
	(280.03,122.59) --
	(282.33,122.89) --
	(284.63,123.18) --
	(286.93,123.46) --
	(289.24,123.72) --
	(291.54,123.97) --
	(293.84,124.22) --
	(296.14,124.45) --
	(298.44,124.68) --
	(300.74,124.90) --
	(303.04,125.11) --
	(305.34,125.32) --
	(307.64,125.52) --
	(309.95,125.71);
\definecolor[named]{drawColor}{rgb}{0.55,0.63,0.80}

\path[draw=drawColor,line width= 0.6pt,line join=round] ( 63.73, 64.91) --
	( 66.03, 69.35) --
	( 68.33, 72.09) --
	( 70.63, 74.15) --
	( 72.93, 75.71) --
	( 75.23, 77.03) --
	( 77.53, 78.21) --
	( 79.84, 79.22) --
	( 82.14, 80.17) --
	( 84.44, 81.02) --
	( 86.74, 81.80) --
	( 89.04, 82.58) --
	( 91.34, 83.32) --
	( 93.64, 83.99) --
	( 95.94, 84.63) --
	( 98.24, 85.21) --
	(100.55, 85.78) --
	(102.85, 86.31) --
	(105.15, 86.87) --
	(107.45, 87.38) --
	(109.75, 87.88) --
	(112.05, 88.47) --
	(114.35, 89.01) --
	(116.65, 89.51) --
	(118.95, 89.99) --
	(121.26, 90.46) --
	(123.56, 90.98) --
	(125.86, 91.53) --
	(128.16, 92.16) --
	(130.46, 92.75) --
	(132.76, 93.30) --
	(135.06, 93.82) --
	(137.36, 94.31) --
	(139.66, 94.78) --
	(141.97, 95.22) --
	(144.27, 95.64) --
	(146.57, 96.06) --
	(148.87, 96.47) --
	(151.17, 96.87) --
	(153.47, 97.29) --
	(155.77, 97.69) --
	(158.07, 98.07) --
	(160.37, 98.43) --
	(162.68, 98.80) --
	(164.98, 99.16) --
	(167.28, 99.51) --
	(169.58, 99.84) --
	(171.88,100.17) --
	(174.18,100.48) --
	(176.48,100.79) --
	(178.78,101.09) --
	(181.08,101.40) --
	(183.39,101.71) --
	(185.69,102.01) --
	(187.99,102.36) --
	(190.29,102.70) --
	(192.59,103.11) --
	(194.89,103.56) --
	(197.19,104.07) --
	(199.49,104.59) --
	(201.79,106.63) --
	(204.09,109.00) --
	(206.40,110.73) --
	(208.70,112.41) --
	(211.00,113.80) --
	(213.30,114.94) --
	(215.60,115.91) --
	(217.90,116.74) --
	(220.20,117.48) --
	(222.50,118.14) --
	(224.80,118.74) --
	(227.11,119.29) --
	(229.41,119.79) --
	(231.71,120.26) --
	(234.01,120.69) --
	(236.31,121.10) --
	(238.61,121.48) --
	(240.91,121.84) --
	(243.21,122.18) --
	(245.51,122.50) --
	(247.82,122.81) --
	(250.12,123.10) --
	(252.42,123.38) --
	(254.72,123.65) --
	(257.02,123.91) --
	(259.32,124.16) --
	(261.62,124.39) --
	(263.92,124.62) --
	(266.22,124.84) --
	(268.53,125.06) --
	(270.83,125.26) --
	(273.13,125.46) --
	(275.43,125.66) --
	(277.73,125.85) --
	(280.03,126.03) --
	(282.33,126.21) --
	(284.63,126.38) --
	(286.93,126.55) --
	(289.24,126.71) --
	(291.54,126.87) --
	(293.84,127.03) --
	(296.14,127.18) --
	(298.44,127.33) --
	(300.74,127.47) --
	(303.04,127.61) --
	(305.34,127.75) --
	(307.64,127.89) --
	(309.95,128.02);
\definecolor[named]{drawColor}{rgb}{0.91,0.54,0.76}

\path[draw=drawColor,line width= 0.6pt,line join=round] ( 63.73, 71.65) --
	( 66.03, 77.07) --
	( 68.33, 80.43) --
	( 70.63, 82.72) --
	( 72.93, 84.46) --
	( 75.23, 85.86) --
	( 77.53, 87.01) --
	( 79.84, 88.00) --
	( 82.14, 88.95) --
	( 84.44, 89.82) --
	( 86.74, 90.61) --
	( 89.04, 91.32) --
	( 91.34, 91.97) --
	( 93.64, 92.58) --
	( 95.94, 93.13) --
	( 98.24, 93.68) --
	(100.55, 94.18) --
	(102.85, 94.65) --
	(105.15, 95.09) --
	(107.45, 95.52) --
	(109.75, 95.92) --
	(112.05, 96.31) --
	(114.35, 96.69) --
	(116.65, 97.06) --
	(118.95, 97.43) --
	(121.26, 97.80) --
	(123.56, 98.16) --
	(125.86, 98.57) --
	(128.16, 99.01) --
	(130.46, 99.47);
\definecolor[named]{fillColor}{rgb}{0.40,0.76,0.65}

\path[fill=fillColor] ( 63.73, 74.14) circle (  1.60);

\path[fill=fillColor] ( 66.03, 78.58) circle (  1.60);

\path[fill=fillColor] ( 68.33, 81.30) circle (  1.60);

\path[fill=fillColor] ( 70.63, 83.21) circle (  1.60);

\path[fill=fillColor] ( 72.93, 84.72) circle (  1.60);

\path[fill=fillColor] ( 75.23, 85.96) circle (  1.60);

\path[fill=fillColor] ( 77.53, 87.01) circle (  1.60);

\path[fill=fillColor] ( 79.84, 87.91) circle (  1.60);

\path[fill=fillColor] ( 82.14, 88.70) circle (  1.60);

\path[fill=fillColor] ( 84.44, 89.45) circle (  1.60);

\path[fill=fillColor] ( 86.74, 90.12) circle (  1.60);

\path[fill=fillColor] ( 89.04, 90.74) circle (  1.60);

\path[fill=fillColor] ( 91.34, 91.30) circle (  1.60);

\path[fill=fillColor] ( 93.64, 91.82) circle (  1.60);

\path[fill=fillColor] ( 95.94, 92.31) circle (  1.60);

\path[fill=fillColor] ( 98.24, 92.77) circle (  1.60);

\path[fill=fillColor] (100.55, 93.20) circle (  1.60);

\path[fill=fillColor] (102.85, 93.61) circle (  1.60);

\path[fill=fillColor] (105.15, 94.00) circle (  1.60);

\path[fill=fillColor] (107.45, 94.37) circle (  1.60);

\path[fill=fillColor] (109.75, 94.73) circle (  1.60);

\path[fill=fillColor] (112.05, 95.06) circle (  1.60);

\path[fill=fillColor] (114.35, 95.39) circle (  1.60);

\path[fill=fillColor] (116.65, 95.69) circle (  1.60);

\path[fill=fillColor] (118.95, 95.99) circle (  1.60);

\path[fill=fillColor] (121.26, 96.27) circle (  1.60);

\path[fill=fillColor] (123.56, 96.54) circle (  1.60);

\path[fill=fillColor] (125.86, 96.81) circle (  1.60);

\path[fill=fillColor] (128.16, 97.06) circle (  1.60);

\path[fill=fillColor] (130.46, 97.32) circle (  1.60);

\path[fill=fillColor] (132.76, 97.71) circle (  1.60);

\path[fill=fillColor] (135.06, 98.09) circle (  1.60);

\path[fill=fillColor] (137.36, 98.45) circle (  1.60);

\path[fill=fillColor] (139.66, 98.79) circle (  1.60);

\path[fill=fillColor] (141.97, 99.13) circle (  1.60);

\path[fill=fillColor] (144.27, 99.44) circle (  1.60);

\path[fill=fillColor] (146.57, 99.75) circle (  1.60);

\path[fill=fillColor] (148.87,100.04) circle (  1.60);

\path[fill=fillColor] (151.17,100.32) circle (  1.60);

\path[fill=fillColor] (153.47,100.59) circle (  1.60);

\path[fill=fillColor] (155.77,100.85) circle (  1.60);

\path[fill=fillColor] (158.07,101.10) circle (  1.60);

\path[fill=fillColor] (160.37,101.35) circle (  1.60);

\path[fill=fillColor] (162.68,101.59) circle (  1.60);

\path[fill=fillColor] (164.98,101.82) circle (  1.60);

\path[fill=fillColor] (167.28,102.05) circle (  1.60);

\path[fill=fillColor] (169.58,102.27) circle (  1.60);

\path[fill=fillColor] (171.88,102.48) circle (  1.60);

\path[fill=fillColor] (174.18,102.69) circle (  1.60);

\path[fill=fillColor] (176.48,102.90) circle (  1.60);

\path[fill=fillColor] (178.78,103.10) circle (  1.60);

\path[fill=fillColor] (181.08,103.31) circle (  1.60);

\path[fill=fillColor] (183.39,103.51) circle (  1.60);

\path[fill=fillColor] (185.69,103.71) circle (  1.60);

\path[fill=fillColor] (187.99,103.90) circle (  1.60);

\path[fill=fillColor] (190.29,104.09) circle (  1.60);

\path[fill=fillColor] (192.59,104.27) circle (  1.60);

\path[fill=fillColor] (194.89,104.45) circle (  1.60);

\path[fill=fillColor] (197.19,104.65) circle (  1.60);

\path[fill=fillColor] (199.49,104.84) circle (  1.60);

\path[fill=fillColor] (201.79,105.18) circle (  1.60);

\path[fill=fillColor] (204.09,105.50) circle (  1.60);

\path[fill=fillColor] (206.40,105.81) circle (  1.60);

\path[fill=fillColor] (208.70,106.11) circle (  1.60);

\path[fill=fillColor] (211.00,106.39) circle (  1.60);

\path[fill=fillColor] (213.30,106.66) circle (  1.60);

\path[fill=fillColor] (215.60,106.92) circle (  1.60);

\path[fill=fillColor] (217.90,107.18) circle (  1.60);

\path[fill=fillColor] (220.20,107.42) circle (  1.60);

\path[fill=fillColor] (222.50,107.71) circle (  1.60);

\path[fill=fillColor] (224.80,108.02) circle (  1.60);

\path[fill=fillColor] (227.11,108.45) circle (  1.60);

\path[fill=fillColor] (229.41,108.85) circle (  1.60);

\path[fill=fillColor] (231.71,109.24) circle (  1.60);

\path[fill=fillColor] (234.01,109.61) circle (  1.60);

\path[fill=fillColor] (236.31,109.96) circle (  1.60);

\path[fill=fillColor] (238.61,110.30) circle (  1.60);

\path[fill=fillColor] (240.91,110.62) circle (  1.60);

\path[fill=fillColor] (243.21,110.93) circle (  1.60);

\path[fill=fillColor] (245.51,111.23) circle (  1.60);

\path[fill=fillColor] (247.82,111.53) circle (  1.60);

\path[fill=fillColor] (250.12,111.82) circle (  1.60);

\path[fill=fillColor] (252.42,112.09) circle (  1.60);

\path[fill=fillColor] (254.72,112.38) circle (  1.60);

\path[fill=fillColor] (257.02,112.66) circle (  1.60);

\path[fill=fillColor] (259.32,112.92) circle (  1.60);

\path[fill=fillColor] (261.62,113.18) circle (  1.60);

\path[fill=fillColor] (263.92,113.44) circle (  1.60);

\path[fill=fillColor] (266.22,113.69) circle (  1.60);

\path[fill=fillColor] (268.53,113.98) circle (  1.60);

\path[fill=fillColor] (270.83,114.38) circle (  1.60);

\path[fill=fillColor] (273.13,114.74) circle (  1.60);

\path[fill=fillColor] (275.43,115.11) circle (  1.60);

\path[fill=fillColor] (277.73,115.46) circle (  1.60);

\path[fill=fillColor] (280.03,115.79) circle (  1.60);

\path[fill=fillColor] (282.33,116.11) circle (  1.60);

\path[fill=fillColor] (284.63,116.43) circle (  1.60);

\path[fill=fillColor] (286.93,116.75) circle (  1.60);

\path[fill=fillColor] (289.24,117.07) circle (  1.60);

\path[fill=fillColor] (291.54,117.40) circle (  1.60);

\path[fill=fillColor] (293.84,117.73) circle (  1.60);

\path[fill=fillColor] (296.14,118.13) circle (  1.60);

\path[fill=fillColor] (298.44,118.51) circle (  1.60);

\path[fill=fillColor] (300.74,118.89) circle (  1.60);

\path[fill=fillColor] (303.04,119.25) circle (  1.60);

\path[fill=fillColor] (305.34,119.62) circle (  1.60);

\path[fill=fillColor] (307.64,119.98) circle (  1.60);

\path[fill=fillColor] (309.95,120.40) circle (  1.60);
\definecolor[named]{fillColor}{rgb}{0.99,0.55,0.38}

\path[fill=fillColor] ( 63.73, 38.51) circle (  1.60);

\path[fill=fillColor] ( 66.03, 42.89) circle (  1.60);

\path[fill=fillColor] ( 68.33, 45.95) circle (  1.60);

\path[fill=fillColor] ( 70.63, 48.00) circle (  1.60);

\path[fill=fillColor] ( 72.93, 49.55) circle (  1.60);

\path[fill=fillColor] ( 75.23, 50.79) circle (  1.60);

\path[fill=fillColor] ( 77.53, 51.83) circle (  1.60);

\path[fill=fillColor] ( 79.84, 52.72) circle (  1.60);

\path[fill=fillColor] ( 82.14, 53.50) circle (  1.60);

\path[fill=fillColor] ( 84.44, 54.19) circle (  1.60);

\path[fill=fillColor] ( 86.74, 54.82) circle (  1.60);

\path[fill=fillColor] ( 89.04, 55.39) circle (  1.60);

\path[fill=fillColor] ( 91.34, 56.01) circle (  1.60);

\path[fill=fillColor] ( 93.64, 56.58) circle (  1.60);

\path[fill=fillColor] ( 95.94, 57.09) circle (  1.60);

\path[fill=fillColor] ( 98.24, 57.57) circle (  1.60);

\path[fill=fillColor] (100.55, 58.02) circle (  1.60);

\path[fill=fillColor] (102.85, 58.44) circle (  1.60);

\path[fill=fillColor] (105.15, 58.83) circle (  1.60);

\path[fill=fillColor] (107.45, 59.19) circle (  1.60);

\path[fill=fillColor] (109.75, 59.60) circle (  1.60);

\path[fill=fillColor] (112.05, 59.98) circle (  1.60);

\path[fill=fillColor] (114.35, 60.34) circle (  1.60);

\path[fill=fillColor] (116.65, 60.67) circle (  1.60);

\path[fill=fillColor] (118.95, 61.00) circle (  1.60);

\path[fill=fillColor] (121.26, 61.30) circle (  1.60);

\path[fill=fillColor] (123.56, 61.59) circle (  1.60);

\path[fill=fillColor] (125.86, 61.87) circle (  1.60);

\path[fill=fillColor] (128.16, 62.14) circle (  1.60);

\path[fill=fillColor] (130.46, 62.40) circle (  1.60);

\path[fill=fillColor] (132.76, 62.68) circle (  1.60);

\path[fill=fillColor] (135.06, 62.94) circle (  1.60);

\path[fill=fillColor] (137.36, 63.20) circle (  1.60);

\path[fill=fillColor] (139.66, 63.45) circle (  1.60);

\path[fill=fillColor] (141.97, 63.69) circle (  1.60);

\path[fill=fillColor] (144.27, 63.92) circle (  1.60);

\path[fill=fillColor] (146.57, 64.17) circle (  1.60);

\path[fill=fillColor] (148.87, 64.43) circle (  1.60);

\path[fill=fillColor] (151.17, 64.69) circle (  1.60);

\path[fill=fillColor] (153.47, 64.95) circle (  1.60);

\path[fill=fillColor] (155.77, 65.21) circle (  1.60);

\path[fill=fillColor] (158.07, 65.46) circle (  1.60);

\path[fill=fillColor] (160.37, 65.70) circle (  1.60);

\path[fill=fillColor] (162.68, 65.95) circle (  1.60);

\path[fill=fillColor] (164.98, 66.19) circle (  1.60);

\path[fill=fillColor] (167.28, 66.42) circle (  1.60);

\path[fill=fillColor] (169.58, 66.66) circle (  1.60);

\path[fill=fillColor] (171.88, 66.89) circle (  1.60);

\path[fill=fillColor] (174.18, 67.14) circle (  1.60);

\path[fill=fillColor] (176.48, 67.37) circle (  1.60);

\path[fill=fillColor] (178.78, 67.61) circle (  1.60);

\path[fill=fillColor] (181.08, 67.85) circle (  1.60);

\path[fill=fillColor] (183.39, 68.09) circle (  1.60);

\path[fill=fillColor] (185.69, 68.35) circle (  1.60);

\path[fill=fillColor] (187.99, 68.60) circle (  1.60);

\path[fill=fillColor] (190.29, 68.85) circle (  1.60);

\path[fill=fillColor] (192.59, 69.15) circle (  1.60);

\path[fill=fillColor] (194.89, 69.49) circle (  1.60);

\path[fill=fillColor] (197.19, 69.85) circle (  1.60);

\path[fill=fillColor] (199.49, 70.52) circle (  1.60);

\path[fill=fillColor] (201.79, 74.59) circle (  1.60);

\path[fill=fillColor] (204.09, 77.50) circle (  1.60);

\path[fill=fillColor] (206.40, 79.91) circle (  1.60);

\path[fill=fillColor] (208.70, 82.12) circle (  1.60);

\path[fill=fillColor] (211.00, 86.15) circle (  1.60);

\path[fill=fillColor] (213.30, 88.63) circle (  1.60);

\path[fill=fillColor] (215.60, 90.53) circle (  1.60);

\path[fill=fillColor] (217.90, 92.73) circle (  1.60);

\path[fill=fillColor] (220.20, 95.24) circle (  1.60);

\path[fill=fillColor] (222.50, 97.27) circle (  1.60);

\path[fill=fillColor] (224.80, 98.95) circle (  1.60);

\path[fill=fillColor] (227.11,100.92) circle (  1.60);

\path[fill=fillColor] (229.41,102.44) circle (  1.60);

\path[fill=fillColor] (231.71,104.38) circle (  1.60);

\path[fill=fillColor] (234.01,105.88) circle (  1.60);

\path[fill=fillColor] (236.31,107.39) circle (  1.60);

\path[fill=fillColor] (238.61,108.88) circle (  1.60);

\path[fill=fillColor] (240.91,111.15) circle (  1.60);

\path[fill=fillColor] (243.21,112.81) circle (  1.60);

\path[fill=fillColor] (245.51,114.12) circle (  1.60);

\path[fill=fillColor] (247.82,115.21) circle (  1.60);

\path[fill=fillColor] (250.12,116.14) circle (  1.60);

\path[fill=fillColor] (252.42,116.95) circle (  1.60);

\path[fill=fillColor] (254.72,117.67) circle (  1.60);

\path[fill=fillColor] (257.02,118.31) circle (  1.60);

\path[fill=fillColor] (259.32,118.89) circle (  1.60);

\path[fill=fillColor] (261.62,119.43) circle (  1.60);

\path[fill=fillColor] (263.92,119.92) circle (  1.60);

\path[fill=fillColor] (266.22,120.38) circle (  1.60);

\path[fill=fillColor] (268.53,120.80) circle (  1.60);

\path[fill=fillColor] (270.83,121.20) circle (  1.60);

\path[fill=fillColor] (273.13,121.58) circle (  1.60);

\path[fill=fillColor] (275.43,121.93) circle (  1.60);

\path[fill=fillColor] (277.73,122.27) circle (  1.60);

\path[fill=fillColor] (280.03,122.59) circle (  1.60);

\path[fill=fillColor] (282.33,122.89) circle (  1.60);

\path[fill=fillColor] (284.63,123.18) circle (  1.60);

\path[fill=fillColor] (286.93,123.46) circle (  1.60);

\path[fill=fillColor] (289.24,123.72) circle (  1.60);

\path[fill=fillColor] (291.54,123.97) circle (  1.60);

\path[fill=fillColor] (293.84,124.22) circle (  1.60);

\path[fill=fillColor] (296.14,124.45) circle (  1.60);

\path[fill=fillColor] (298.44,124.68) circle (  1.60);

\path[fill=fillColor] (300.74,124.90) circle (  1.60);

\path[fill=fillColor] (303.04,125.11) circle (  1.60);

\path[fill=fillColor] (305.34,125.32) circle (  1.60);

\path[fill=fillColor] (307.64,125.52) circle (  1.60);

\path[fill=fillColor] (309.95,125.71) circle (  1.60);
\definecolor[named]{fillColor}{rgb}{0.55,0.63,0.80}

\path[fill=fillColor] ( 63.73, 64.91) circle (  1.60);

\path[fill=fillColor] ( 66.03, 69.35) circle (  1.60);

\path[fill=fillColor] ( 68.33, 72.09) circle (  1.60);

\path[fill=fillColor] ( 70.63, 74.15) circle (  1.60);

\path[fill=fillColor] ( 72.93, 75.71) circle (  1.60);

\path[fill=fillColor] ( 75.23, 77.03) circle (  1.60);

\path[fill=fillColor] ( 77.53, 78.21) circle (  1.60);

\path[fill=fillColor] ( 79.84, 79.22) circle (  1.60);

\path[fill=fillColor] ( 82.14, 80.17) circle (  1.60);

\path[fill=fillColor] ( 84.44, 81.02) circle (  1.60);

\path[fill=fillColor] ( 86.74, 81.80) circle (  1.60);

\path[fill=fillColor] ( 89.04, 82.58) circle (  1.60);

\path[fill=fillColor] ( 91.34, 83.32) circle (  1.60);

\path[fill=fillColor] ( 93.64, 83.99) circle (  1.60);

\path[fill=fillColor] ( 95.94, 84.63) circle (  1.60);

\path[fill=fillColor] ( 98.24, 85.21) circle (  1.60);

\path[fill=fillColor] (100.55, 85.78) circle (  1.60);

\path[fill=fillColor] (102.85, 86.31) circle (  1.60);

\path[fill=fillColor] (105.15, 86.87) circle (  1.60);

\path[fill=fillColor] (107.45, 87.38) circle (  1.60);

\path[fill=fillColor] (109.75, 87.88) circle (  1.60);

\path[fill=fillColor] (112.05, 88.47) circle (  1.60);

\path[fill=fillColor] (114.35, 89.01) circle (  1.60);

\path[fill=fillColor] (116.65, 89.51) circle (  1.60);

\path[fill=fillColor] (118.95, 89.99) circle (  1.60);

\path[fill=fillColor] (121.26, 90.46) circle (  1.60);

\path[fill=fillColor] (123.56, 90.98) circle (  1.60);

\path[fill=fillColor] (125.86, 91.53) circle (  1.60);

\path[fill=fillColor] (128.16, 92.16) circle (  1.60);

\path[fill=fillColor] (130.46, 92.75) circle (  1.60);

\path[fill=fillColor] (132.76, 93.30) circle (  1.60);

\path[fill=fillColor] (135.06, 93.82) circle (  1.60);

\path[fill=fillColor] (137.36, 94.31) circle (  1.60);

\path[fill=fillColor] (139.66, 94.78) circle (  1.60);

\path[fill=fillColor] (141.97, 95.22) circle (  1.60);

\path[fill=fillColor] (144.27, 95.64) circle (  1.60);

\path[fill=fillColor] (146.57, 96.06) circle (  1.60);

\path[fill=fillColor] (148.87, 96.47) circle (  1.60);

\path[fill=fillColor] (151.17, 96.87) circle (  1.60);

\path[fill=fillColor] (153.47, 97.29) circle (  1.60);

\path[fill=fillColor] (155.77, 97.69) circle (  1.60);

\path[fill=fillColor] (158.07, 98.07) circle (  1.60);

\path[fill=fillColor] (160.37, 98.43) circle (  1.60);

\path[fill=fillColor] (162.68, 98.80) circle (  1.60);

\path[fill=fillColor] (164.98, 99.16) circle (  1.60);

\path[fill=fillColor] (167.28, 99.51) circle (  1.60);

\path[fill=fillColor] (169.58, 99.84) circle (  1.60);

\path[fill=fillColor] (171.88,100.17) circle (  1.60);

\path[fill=fillColor] (174.18,100.48) circle (  1.60);

\path[fill=fillColor] (176.48,100.79) circle (  1.60);

\path[fill=fillColor] (178.78,101.09) circle (  1.60);

\path[fill=fillColor] (181.08,101.40) circle (  1.60);

\path[fill=fillColor] (183.39,101.71) circle (  1.60);

\path[fill=fillColor] (185.69,102.01) circle (  1.60);

\path[fill=fillColor] (187.99,102.36) circle (  1.60);

\path[fill=fillColor] (190.29,102.70) circle (  1.60);

\path[fill=fillColor] (192.59,103.11) circle (  1.60);

\path[fill=fillColor] (194.89,103.56) circle (  1.60);

\path[fill=fillColor] (197.19,104.07) circle (  1.60);

\path[fill=fillColor] (199.49,104.59) circle (  1.60);

\path[fill=fillColor] (201.79,106.63) circle (  1.60);

\path[fill=fillColor] (204.09,109.00) circle (  1.60);

\path[fill=fillColor] (206.40,110.73) circle (  1.60);

\path[fill=fillColor] (208.70,112.41) circle (  1.60);

\path[fill=fillColor] (211.00,113.80) circle (  1.60);

\path[fill=fillColor] (213.30,114.94) circle (  1.60);

\path[fill=fillColor] (215.60,115.91) circle (  1.60);

\path[fill=fillColor] (217.90,116.74) circle (  1.60);

\path[fill=fillColor] (220.20,117.48) circle (  1.60);

\path[fill=fillColor] (222.50,118.14) circle (  1.60);

\path[fill=fillColor] (224.80,118.74) circle (  1.60);

\path[fill=fillColor] (227.11,119.29) circle (  1.60);

\path[fill=fillColor] (229.41,119.79) circle (  1.60);

\path[fill=fillColor] (231.71,120.26) circle (  1.60);

\path[fill=fillColor] (234.01,120.69) circle (  1.60);

\path[fill=fillColor] (236.31,121.10) circle (  1.60);

\path[fill=fillColor] (238.61,121.48) circle (  1.60);

\path[fill=fillColor] (240.91,121.84) circle (  1.60);

\path[fill=fillColor] (243.21,122.18) circle (  1.60);

\path[fill=fillColor] (245.51,122.50) circle (  1.60);

\path[fill=fillColor] (247.82,122.81) circle (  1.60);

\path[fill=fillColor] (250.12,123.10) circle (  1.60);

\path[fill=fillColor] (252.42,123.38) circle (  1.60);

\path[fill=fillColor] (254.72,123.65) circle (  1.60);

\path[fill=fillColor] (257.02,123.91) circle (  1.60);

\path[fill=fillColor] (259.32,124.16) circle (  1.60);

\path[fill=fillColor] (261.62,124.39) circle (  1.60);

\path[fill=fillColor] (263.92,124.62) circle (  1.60);

\path[fill=fillColor] (266.22,124.84) circle (  1.60);

\path[fill=fillColor] (268.53,125.06) circle (  1.60);

\path[fill=fillColor] (270.83,125.26) circle (  1.60);

\path[fill=fillColor] (273.13,125.46) circle (  1.60);

\path[fill=fillColor] (275.43,125.66) circle (  1.60);

\path[fill=fillColor] (277.73,125.85) circle (  1.60);

\path[fill=fillColor] (280.03,126.03) circle (  1.60);

\path[fill=fillColor] (282.33,126.21) circle (  1.60);

\path[fill=fillColor] (284.63,126.38) circle (  1.60);

\path[fill=fillColor] (286.93,126.55) circle (  1.60);

\path[fill=fillColor] (289.24,126.71) circle (  1.60);

\path[fill=fillColor] (291.54,126.87) circle (  1.60);

\path[fill=fillColor] (293.84,127.03) circle (  1.60);

\path[fill=fillColor] (296.14,127.18) circle (  1.60);

\path[fill=fillColor] (298.44,127.33) circle (  1.60);

\path[fill=fillColor] (300.74,127.47) circle (  1.60);

\path[fill=fillColor] (303.04,127.61) circle (  1.60);

\path[fill=fillColor] (305.34,127.75) circle (  1.60);

\path[fill=fillColor] (307.64,127.89) circle (  1.60);

\path[fill=fillColor] (309.95,128.02) circle (  1.60);
\definecolor[named]{fillColor}{rgb}{0.91,0.54,0.76}

\path[fill=fillColor] ( 63.73, 71.65) circle (  1.60);

\path[fill=fillColor] ( 66.03, 77.07) circle (  1.60);

\path[fill=fillColor] ( 68.33, 80.43) circle (  1.60);

\path[fill=fillColor] ( 70.63, 82.72) circle (  1.60);

\path[fill=fillColor] ( 72.93, 84.46) circle (  1.60);

\path[fill=fillColor] ( 75.23, 85.86) circle (  1.60);

\path[fill=fillColor] ( 77.53, 87.01) circle (  1.60);

\path[fill=fillColor] ( 79.84, 88.00) circle (  1.60);

\path[fill=fillColor] ( 82.14, 88.95) circle (  1.60);

\path[fill=fillColor] ( 84.44, 89.82) circle (  1.60);

\path[fill=fillColor] ( 86.74, 90.61) circle (  1.60);

\path[fill=fillColor] ( 89.04, 91.32) circle (  1.60);

\path[fill=fillColor] ( 91.34, 91.97) circle (  1.60);

\path[fill=fillColor] ( 93.64, 92.58) circle (  1.60);

\path[fill=fillColor] ( 95.94, 93.13) circle (  1.60);

\path[fill=fillColor] ( 98.24, 93.68) circle (  1.60);

\path[fill=fillColor] (100.55, 94.18) circle (  1.60);

\path[fill=fillColor] (102.85, 94.65) circle (  1.60);

\path[fill=fillColor] (105.15, 95.09) circle (  1.60);

\path[fill=fillColor] (107.45, 95.52) circle (  1.60);

\path[fill=fillColor] (109.75, 95.92) circle (  1.60);

\path[fill=fillColor] (112.05, 96.31) circle (  1.60);

\path[fill=fillColor] (114.35, 96.69) circle (  1.60);

\path[fill=fillColor] (116.65, 97.06) circle (  1.60);

\path[fill=fillColor] (118.95, 97.43) circle (  1.60);

\path[fill=fillColor] (121.26, 97.80) circle (  1.60);

\path[fill=fillColor] (123.56, 98.16) circle (  1.60);

\path[fill=fillColor] (125.86, 98.57) circle (  1.60);

\path[fill=fillColor] (128.16, 99.01) circle (  1.60);

\path[fill=fillColor] (130.46, 99.47) circle (  1.60);
\end{scope}
\begin{scope}
\path[clip] (  0.00,  0.00) rectangle (433.62,144.54);
\definecolor[named]{drawColor}{rgb}{0.50,0.50,0.50}

\node[text=drawColor,anchor=base east,inner sep=0pt, outer sep=0pt, scale=  0.96] at ( 44.30, 55.52) {1};

\node[text=drawColor,anchor=base east,inner sep=0pt, outer sep=0pt, scale=  0.96] at ( 44.30, 84.59) {100};

\node[text=drawColor,anchor=base east,inner sep=0pt, outer sep=0pt, scale=  0.96] at ( 44.30,113.66) {10000};
\end{scope}
\begin{scope}
\path[clip] (  0.00,  0.00) rectangle (433.62,144.54);
\definecolor[named]{drawColor}{rgb}{0.50,0.50,0.50}

\path[draw=drawColor,line width= 0.6pt,line join=round] ( 47.15, 58.83) --
	( 51.42, 58.83);

\path[draw=drawColor,line width= 0.6pt,line join=round] ( 47.15, 87.89) --
	( 51.42, 87.89);

\path[draw=drawColor,line width= 0.6pt,line join=round] ( 47.15,116.96) --
	( 51.42,116.96);
\end{scope}
\begin{scope}
\path[clip] (  0.00,  0.00) rectangle (433.62,144.54);
\definecolor[named]{drawColor}{rgb}{0.50,0.50,0.50}

\path[draw=drawColor,line width= 0.6pt,line join=round] ( 61.43, 29.77) --
	( 61.43, 34.03);

\path[draw=drawColor,line width= 0.6pt,line join=round] (130.46, 29.77) --
	(130.46, 34.03);

\path[draw=drawColor,line width= 0.6pt,line join=round] (199.49, 29.77) --
	(199.49, 34.03);

\path[draw=drawColor,line width= 0.6pt,line join=round] (268.53, 29.77) --
	(268.53, 34.03);
\end{scope}
\begin{scope}
\path[clip] (  0.00,  0.00) rectangle (433.62,144.54);
\definecolor[named]{drawColor}{rgb}{0.50,0.50,0.50}

\node[text=drawColor,anchor=base,inner sep=0pt, outer sep=0pt, scale=  0.96] at ( 61.43, 20.31) {0};

\node[text=drawColor,anchor=base,inner sep=0pt, outer sep=0pt, scale=  0.96] at (130.46, 20.31) {30};

\node[text=drawColor,anchor=base,inner sep=0pt, outer sep=0pt, scale=  0.96] at (199.49, 20.31) {60};

\node[text=drawColor,anchor=base,inner sep=0pt, outer sep=0pt, scale=  0.96] at (268.53, 20.31) {90};
\end{scope}
\begin{scope}
\path[clip] (  0.00,  0.00) rectangle (433.62,144.54);
\definecolor[named]{drawColor}{rgb}{0.00,0.00,0.00}

\node[text=drawColor,anchor=base,inner sep=0pt, outer sep=0pt, scale=  1] at (186.84,  8.53) {Number of problems};
\end{scope}
\begin{scope}
\path[clip] (  0.00,  0.00) rectangle (433.62,144.54);
\definecolor[named]{drawColor}{rgb}{0.00,0.00,0.00}

\node[text=drawColor,rotate= 90.00,anchor=base,inner sep=0pt, outer sep=0pt, scale=  1] at ( 16.80, 83.26) {Cumulative CPU time};
\end{scope}
\begin{scope}
\path[clip] (  0.00,  0.00) rectangle (433.62,144.54);
\definecolor[named]{fillColor}{rgb}{1.00,1.00,1.00}

\path[fill=fillColor] (331.12, 44.97) rectangle (412.71,121.56);
\end{scope}
\begin{scope}
\path[clip] (  0.00,  0.00) rectangle (433.62,144.54);
\definecolor[named]{drawColor}{rgb}{0.00,0.00,0.00}

\node[text=drawColor,anchor=base west,inner sep=0pt, outer sep=0pt, scale=  0.96] at (335.39,110.67) {\bfseries Solver};
\end{scope}
\begin{scope}
\path[clip] (  0.00,  0.00) rectangle (433.62,144.54);
\definecolor[named]{drawColor}{rgb}{1.00,1.00,1.00}
\definecolor[named]{fillColor}{rgb}{0.95,0.95,0.95}

\path[draw=drawColor,line width= 0.6pt,line join=round,line cap=round,fill=fillColor] (335.39, 92.60) rectangle (349.85,107.05);
\end{scope}
\begin{scope}
\path[clip] (  0.00,  0.00) rectangle (433.62,144.54);
\definecolor[named]{drawColor}{rgb}{0.40,0.76,0.65}

\path[draw=drawColor,line width= 0.6pt,line join=round] (336.84, 99.83) -- (348.40, 99.83);
\end{scope}
\begin{scope}
\path[clip] (  0.00,  0.00) rectangle (433.62,144.54);
\definecolor[named]{fillColor}{rgb}{0.40,0.76,0.65}

\path[fill=fillColor] (342.62, 99.83) circle (  1.60);
\end{scope}
\begin{scope}
\path[clip] (  0.00,  0.00) rectangle (433.62,144.54);
\definecolor[named]{drawColor}{rgb}{1.00,1.00,1.00}
\definecolor[named]{fillColor}{rgb}{0.95,0.95,0.95}

\path[draw=drawColor,line width= 0.6pt,line join=round,line cap=round,fill=fillColor] (335.39, 78.15) rectangle (349.85, 92.60);
\end{scope}
\begin{scope}
\path[clip] (  0.00,  0.00) rectangle (433.62,144.54);
\definecolor[named]{drawColor}{rgb}{0.99,0.55,0.38}

\path[draw=drawColor,line width= 0.6pt,line join=round] (336.84, 85.37) -- (348.40, 85.37);
\end{scope}
\begin{scope}
\path[clip] (  0.00,  0.00) rectangle (433.62,144.54);
\definecolor[named]{fillColor}{rgb}{0.99,0.55,0.38}

\path[fill=fillColor] (342.62, 85.37) circle (  1.60);
\end{scope}
\begin{scope}
\path[clip] (  0.00,  0.00) rectangle (433.62,144.54);
\definecolor[named]{drawColor}{rgb}{1.00,1.00,1.00}
\definecolor[named]{fillColor}{rgb}{0.95,0.95,0.95}

\path[draw=drawColor,line width= 0.6pt,line join=round,line cap=round,fill=fillColor] (335.39, 63.69) rectangle (349.85, 78.15);
\end{scope}
\begin{scope}
\path[clip] (  0.00,  0.00) rectangle (433.62,144.54);
\definecolor[named]{drawColor}{rgb}{0.55,0.63,0.80}

\path[draw=drawColor,line width= 0.6pt,line join=round] (336.84, 70.92) -- (348.40, 70.92);
\end{scope}
\begin{scope}
\path[clip] (  0.00,  0.00) rectangle (433.62,144.54);
\definecolor[named]{fillColor}{rgb}{0.55,0.63,0.80}

\path[fill=fillColor] (342.62, 70.92) circle (  1.60);
\end{scope}
\begin{scope}
\path[clip] (  0.00,  0.00) rectangle (433.62,144.54);
\definecolor[named]{drawColor}{rgb}{1.00,1.00,1.00}
\definecolor[named]{fillColor}{rgb}{0.95,0.95,0.95}

\path[draw=drawColor,line width= 0.6pt,line join=round,line cap=round,fill=fillColor] (335.39, 49.24) rectangle (349.85, 63.69);
\end{scope}
\begin{scope}
\path[clip] (  0.00,  0.00) rectangle (433.62,144.54);
\definecolor[named]{drawColor}{rgb}{0.91,0.54,0.76}

\path[draw=drawColor,line width= 0.6pt,line join=round] (336.84, 56.46) -- (348.40, 56.46);
\end{scope}
\begin{scope}
\path[clip] (  0.00,  0.00) rectangle (433.62,144.54);
\definecolor[named]{fillColor}{rgb}{0.91,0.54,0.76}

\path[fill=fillColor] (342.62, 56.46) circle (  1.60);
\end{scope}
\begin{scope}
\path[clip] (  0.00,  0.00) rectangle (433.62,144.54);
\definecolor[named]{drawColor}{rgb}{0.00,0.00,0.00}

\node[text=drawColor,anchor=base west,inner sep=0pt, outer sep=0pt, scale=  0.96] at (351.65, 96.52) {In-the-middle};
\end{scope}
\begin{scope}
\path[clip] (  0.00,  0.00) rectangle (433.62,144.54);
\definecolor[named]{drawColor}{rgb}{0.00,0.00,0.00}

\node[text=drawColor,anchor=base west,inner sep=0pt, outer sep=0pt, scale=  0.96] at (351.65, 82.07) {Toulbar2};
\end{scope}
\begin{scope}
\path[clip] (  0.00,  0.00) rectangle (433.62,144.54);
\definecolor[named]{drawColor}{rgb}{0.00,0.00,0.00}

\node[text=drawColor,anchor=base west,inner sep=0pt, outer sep=0pt, scale=  0.96] at (351.65, 67.61) {CPLEX};
\end{scope}
\begin{scope}
\path[clip] (  0.00,  0.00) rectangle (433.62,144.54);
\definecolor[named]{drawColor}{rgb}{0.00,0.00,0.00}

\node[text=drawColor,anchor=base west,inner sep=0pt, outer sep=0pt, scale=  0.96] at (351.65, 53.16) {MaxHS};
\end{scope}
\end{tikzpicture}
}
	\end{figcenter}
	\caption{Accumulated runtime of the algorithms in three different sets, sorted by runtime individually for every solver. Note the logarithmic scale of the \(y\) axis.}
	\label{fig:cactus-std}
\end{figure}

% [todo] - explain SceneDecomp anomaly? Maybe deGivry14 has something on it?

The largest problem solved optimally by the in-the-middle algorithm was an instance of in the \emph{In-Painting} set with search space \(d^n = 4^{\num{14400}}\).
The smallest unsolved instances are in the \emph{Auction} set, with seach spaces \(d^n \approx 2^{80}\).

% \begin{table}
% 	\centering
% 	% Updated 2014-05-25
% 	\caption{
% 		Solved problems.
% 		For each problem instance used in the benchmark, the number of problems solved by each solver is listed.
% 		%Some problem sets have been omitted.
% 	}
% 	% [review] - should this be in an appendix?
% 	\label{tab:number-solved}
% 	\begin{figcenter}
% 	\begin{tabular}{xyccHccc}
% 		\toprule
% 			{} & {} & {} & \multicolumn{5}{c}{\(\#\) solved} \\
% 			\cmidrule(rl){4-8}
% 			{\normalsize Category} & {\normalsize Set} & {\(\#\) of problems} & {ITM} & {MPLP2} & {Toulbar2} & {CPLEX} & {MaxHS} \\
% 		\midrule
% \acrshort{cfn}	&	Auction	&	170	&	102	&	97	&	170	&	170	&	170 \\
% 				&	CELAR	&	16	&	10	&	16	&	16	&	16	&	0 \\
% 				&	Pedigree	&	10	&	10	&	0	&	10	&	5	&	5 \\
% 				&	ProteinDesign	&	10	&	10	&	9	&	10	&	10	&	0 \\
% 				&	SPOT5	&	20	&	5	&	1	&	20	&	20	&	5 \\
% 				&	Warehouse	&	55	&	38	&	54	&	55	&	55	&	47 \\
% %\acrshort{cp}	&	AMaze	&	6	&	0	&	0	&	2	&	0	&	6 \\
% %				&	FastFood	&	6	&	1	&	0	&	1	&	1	&	1 \\
% %\acrshort{cp}	&	Golomb	&	6	&	0	&	0	&	3	&	0	&	3 \\
% \acrshort{cp}	&	OnCallRostering	&	5	&	3	&	0	&	3	&	3	&	3 \\
% 				&	ParityLearning	&	7	&	7	&	0	&	7	&	7	&	3 \\
% %				&	VRP	&	5	&	0	&	0	&	5	&	0	&	0 \\
% %\acrshort{cvpr}	&	ChineseChars	&	100	&	0	&	100	&	100	&	100	&	0 \\
% %				&	ColorSeg	&	21	&	0	&	16	&	21	&	5	&	0 \\
% \acrshort{cvpr}	&	GeomSurf	&	600	&	600	&	324	&	600	&	600	&	310 \\
% 				&	InPainting	&	4	&	4	&	4	&	4	&	3	&	0 \\
% 				&	Matching	&	4	&	4	&	1	&	4	&	4	&	0 \\
% %				&	MatchingStereo	&	2	&	0	&	2	&	2	&	0	&	0 \\
% 				&	ObjectSeg	&	5	&	5	&	5	&	5	&	4	&	0 \\
% %				&	PhotoMontage	&	2	&	0	&	0	&	0	&	0	&	0 \\
% 				&	SceneDecomp	&	715	&	715	&	715	&	715	&	715	&	1 \\
% Max-\acrshort{csp}	&	BlackHole	&	37	&	37	&	0	&	37	&	37	&	10 \\
% 				&	Coloring	&	22	&	22	&	0	&	20	&	22	&	12 \\
% 				&	Composed	&	80	&	80	&	80	&	80	&	80	&	80 \\
% 				&	EHI	&	200	&	200	&	0	&	200	&	200	&	0 \\
% 				&	Geometric	&	100	&	100	&	0	&	100	&	100	&	88 \\
% 				&	Langford	&	4	&	4	&	1	&	4	&	4	&	2 \\
% 				&	QCP	&	60	&	60	&	4	&	60	&	60	&	46 \\
% \acrshort{mrf}	&	DBN	&	108	&	108	&	108	&	108	&	108	&	30 \\
% 				&	Grid	&	21	&	0	&	21	&	21	&	21	&	4 \\
% 				&	ImageAlignment	&	10	&	10	&	10	&	10	&	8	&	0 \\
% 				&	Linkage	&	22	&	8	&	5	&	22	&	22	&	20 \\
% 				&	ObjectDetection	&	37	&	37	&	23	&	37	&	37	&	0 \\
% 				&	ProteinFolding	&	21	&	20	&	21	&	21	&	10	&	0 \\
% 				&	Segmentation	&	100	&	100	&	100	&	100	&	100	&	50 \\
% %\acrshort{wpms}	&	Haplotyping	&	100	&	0	&	0	&	100	&	96	&	25 \\
% \acrshort{wpms}	&	MaxClique	&	62	&	46	&	56	&	62	&	62	&	36 \\
% %				&	MIPLib	&	12	&	0	&	0	&	5	&	5	&	3 \\
% %				&	PackupWeighted	&	99	&	0	&	0	&	98	&	99	&	85 \\
% %				&	PlanningWithPref	&	29	&	0	&	0	&	24	&	21	&	27 \\
% %				&	TimeTabling	&	25	&	0	&	0	&	1	&	1	&	1 \\
% %				&	Upgradeability	&	100	&	0	&	0	&	100	&	100	&	100 \\
% 		\bottomrule
% 	\end{tabular}
% 	\end{figcenter}
% \end{table}

	% [todo] - move table/figures around
\section{Extensions and improvements}
Two variants of the algorithm have been discussed earlier in the thesis, with different purpose and functionality.
Benchmark data was produced for these variants as well, but on a limited subset of the problems, chosen specifically to test the variants and their intended purpose.
Results for both variants will be reviewed in in order to determine if they have the desired effect, and if their function is satisfactory with respect to the drawbacks they introduce.
Results will be presented by comparison with the original benchmark only, without the solvers benchmarked by \textcite{deGivry14}.

% [todo] - explaining why sets were chosen
% [todo] - analyzing the results
% [todo] - were results as expected?
% [todo] - is a given improvement useful?

\subsection{The \enquote{push} operation}
The purpose of the \enquote{push} operation is to increase solution quality of the approximative algorithm once a feasible solution has been found.
Six problem sets from \cref{tab:comparative-results} were therefore chosen for this benchmark, all with comparatively bad solutions (solution differences close to or above \SI{1}{\percent}) but competitive runtimes.
The expectation was to obtain better solutions while maintaining good runtimes.

\Cref{tab:push-results} shows the results of benchmarking the \enquote{push} operation on the selected problems.
Surprisingly, the solution quality did not improve for a majority of the problems.
In fact, for some sets the solution quality was decreased, and the runtime of the algorithm improved instead (which given the already competitive runtime of the standard algorithm is an unwanted result).
In fact, the only problem set for which the excpected result was obained is the \gls{cfn} \emph{Pedigree} set.

The reason for this is likely that the algorithm, in trials after the \enquote{push} operation has been applied, is more conservative than the original algorithm in that the maximum \(\kappa\) value will be lower. This means more trials fail to force an integer solution, reducing the number of trials and as a consequence reducing the runtime as well as the solution quality.
A more correct implementation would take into consideration the reduction of \(\kappa\) caused by the push operation when calculating the maximum \(\kappa\) value of subsequent trials.
% [review] - maybe not true
% [fix] - introduce this fix and re-run trials?

Due to these results, the \enquote{push} operation is not as interesting when applied to the max-sum algorithm as it is in the original \gls{lp} formulation.
% [todo] - analyze why it works on the CFN/Pedigree problem!

% Interesting cactus plots: Mainly MaxCSP/BlackHole when compared to standard

\begin{table}
	\centering
	% Updated 2014-05-24
	\caption{
		Solution quality and runtime using the \enquote{push} operation.
		For several chosen problem sets, the \enquote{push} variant runtime is compared to the results obtained by the standard algorithm (see \cref{tab:comparative-results}).
	}
	% [todo] - double-check all table data!
	\label{tab:push-results}
	\begin{figcenter}
	\begin{tabular}{xySSS[round-mode=places,round-precision=3,scientific-notation=fixed,fixed-exponent=0]
				     S[round-mode=places,round-precision=3,scientific-notation=fixed,fixed-exponent=0]
				     S[round-mode=places,round-precision=2,scientific-notation=fixed,fixed-exponent=0]
				     S[round-mode=places,round-precision=2,scientific-notation=fixed,fixed-exponent=0]}
		\toprule
			{} & {} & \multicolumn{2}{c}{\(\#\) solved} & \multicolumn{2}{c}{Sol. diff. (\si{\percent})} & \multicolumn{2}{c}{Mean time (\si{\second})} \\
			\cmidrule(rl){3-4} \cmidrule(rl){5-6} \cmidrule(rl){7-8}
			{\normalsize Category} & {\normalsize Set} & {Std.} & {\enquote{Push}} & {Std.} & {\enquote{Push}} & {Std.} & {\enquote{Push}} \\
		\midrule
\acrshort{cfn}	&	Pedigree	&	10	&	10	&	1.804874e-00	&	1.51249400	&	2.3750	&	5.6070 \\
\acrshort{cvpr}	&	GeomSurf	&	600	&	600	&	2.091307e-00	&	2.09130700	&	0.0460	&	0.0425 \\
				&	SceneDecomp	&	715	&	715	&	7.545481e+01	&	75.4547890	&	0.0210	&	0.0170 \\
Max-\acrshort{csp}	&	BlackHole	&	37	&	37	&	9.009009e-01	&	1.08108100	&	58.8900	&	31.3310 \\
				&	Langford	&	4	&	4	&	1.311265e-00	&	1.55417600	&	70.7775	&	56.2925 \\
				&	QCP	&	60	&	60	&	1.292034e-00	&	1.30359500	&	43.2575	&	38.0705 \\
\acrshort{mrf}	&	ObjectDetection	&	37	&	37	&	6.465565e-00	&	6.46556500	&	279.8620	&	167.0170 \\
		\bottomrule
	\end{tabular}
	\end{figcenter}
\end{table}


\subsection{The greedy DP update}
The greedy DP update, obtained by fixing \(\alpha=1\) of the fractional DP update, should theoretically improve convergence at the expense of the optimality guarantee.
Here, a large number of problems from \cref{tab:comparative-results} were selected, all exhibiting good solution quality and a reasonable but uncompetitive runtime.
The expectation was to decrease runtime at the expense of solution quality.
Additionally, some sets with perfect solution quality and competitive runtime were also included, to observe the effects of this variant on already well-performing problem sets.

\Cref{tab:greedy-dp-results} shows the results of benchmarking the greedy DP algorithm against the selected problems.
As expected, the runtime of all sets (except the \emph{Auction} set) was improved significantly --- between \num{4} and \num{350} times --- with little or no decrease in solution quality.
The runtime improvements are most significant for the \gls{cfn} and \gls{mrf} problems, where the greedy DP variant is competitive with all three other solvers.

In situations where finding a solution quickly is more important than finding a guaranteed optimal solution, the greedy DP update is therefore a viable alternative.
There is no problem set where solution quality is significantly lowered, and only three sets contain problems which could be solved by the standard algorithm but not using greedy DP updates.
Three problem sets from the max-\gls{csp} category additionally show improved solution quality as well as better runtimes.

\Cref{fig:cactus-greedy} illustrates the utility of the greedy DP update.
While it is slower for very small problems in the \emph{Black Hole} set, it is significantly faster in solving the more difficult problems.

% [todo] - theorize on reason of CFN/Auction not working!

With these results in mind, the greedy algorithm may be very useful when exact solutions aren't required.
% [todo] - mention fields where a "good enough" solution is satisfactory and why
This makes the algorithm with greedy updates extremely competitive in such cases.
The greedy update variant may also be useful as part of a broader strategy, by for instance providing fast and good upper bounds or by providing fast unproven solutions while waiting for exact solvers.
Another interesting direction for the greedy DP update could be to run it in parallel with the standard algorithm (sharing the immutable constraint components in memory), again providing quick approximative solutions as well as good solutions.

% Note: for some problems, including CFN/ProteinDesign, no loss in accuracy but great decrease in time - discuss this!

% Good as part of a broader strategy, to quickly obtain upper bounds?

% Interesting cactus plots: MaxCSP/BlackHole, MaxCSP/QCP, MRF/DBN

\begin{table}
	\centering
	% Updated 2014-05-24
	\caption{
		Solution quality and runtime using the greedy DP update (setting \(\alpha=1\)).
		For several chosen problem sets, the greedy DP runtime is compared to the results obtained by the standard algorithm (see \cref{tab:comparative-results}).
		Problem sets marked with \textdagger{} include unsolved problems (no feasible solution found by the greedy DP update), and n/a values indicate that none of the problems in the set were solved.
		Runtimes based on less than \SI{70}{\percent} of the problems are faded.
	}
	\label{tab:greedy-dp-results}
	\begin{figcenter}
	\begin{tabular}{xySSS[round-mode=places,round-precision=3,scientific-notation=fixed,fixed-exponent=0]
				     S[round-mode=places,round-precision=3,scientific-notation=fixed,fixed-exponent=0]
				     S[round-mode=places,round-precision=2,scientific-notation=fixed,fixed-exponent=0]
				     S[round-mode=places,round-precision=2,scientific-notation=fixed,fixed-exponent=0]}
		\toprule
			{} & {} & \multicolumn{2}{c}{\(\#\) solved} & \multicolumn{2}{c}{Sol. diff. (\si{\percent})} & \multicolumn{2}{c}{Mean time (\si{\second})} \\
			\cmidrule(rl){3-4} \cmidrule(rl){5-6} \cmidrule(rl){7-8}
			{\normalsize Category} & {\normalsize Set} & {Std.} & {\(\alpha=1\)} & {Std.} & {\(\alpha=1\)} & {Std.} & {\(\alpha=1\)} \\
		\midrule
\acrshort{cfn}	&	Auction\textdagger	&	102	&	0	&	0.000000e+00	&	{\textcolor{gray}{n/a}}	&	82.8575	&	{\textcolor{gray}{n/a}} \\
%				&	CELAR\textdagger	&	10	&	4	&	9.081260e-07	&	0.00000000	&	\color{gray}193.3445	&	\color{gray}3.8775 \\
				&	ProteinDesign\textdagger	&	10	&	9	&	0.000000e+00	&	0.00000000	&	43.3995	&	0.7220 \\
				&	Warehouse\textdagger	&	38	&	53	&	0.000000e+00	&	0.00000000	&	\color{gray}55.8550	&	0.6780 \\
\acrshort{cp}	&	ParityLearning	&	7	&	7	&	1.800000e-05	&	0.000013	&	34.5300	&	3.1260 \\
\acrshort{cvpr}	&	Matching	&	4	&	4	&	0.000000e+00	&	0.00000000	&	17.9275	&	4.5525 \\
Max-\acrshort{csp}	&	BlackHole\textdagger	&	37	&	36	&	9.009009e-01	&	1.081081	&	58.8900	&	13.1665 \\
				&	Coloring	&	22	&	22	&	0.000000e+00	&	0.000000	&	1.6860	&	0.2080 \\
				&	Composed	&	80	&	80	&	1.342282e-01	&	0.00000000	&	20.3400	&	0.9980 \\
				&	Geometric	&	100	&	100	&	1.082434e-00	&	0.94082100	&	98.9760	&	13.3705 \\
				&	Langford	&	4	&	4	&	1.311265e-00	&	0.96711800	&	70.7775	&	7.3910 \\
				&	QCP	&	60	&	60	&	1.292034e-00	&	2.12260500	&	43.2575	&	4.4240 \\
\acrshort{mrf}	&	DBN	&	108	&	108	&	0.000000e+00	&	0.00000000	&	37.9040	&	0.4465 \\
%				&	Linkage\textdagger	&	8	&	10	&	0.000000e+00	&	0.00000000	&	\color{gray}41.0700	&	\color{gray}3.6005 \\
				&	ObjectDetection	&	37	&	37	&	6.465565e-00	&	6.46556500	&		279.8620	&	0.8380 \\
				&	Segmentation	&	100	&	100	&	0.000000e+00	&	0.000000	&	0.0310	&	0.1295 \\
		\bottomrule
	\end{tabular}
	\end{figcenter}
\end{table}

\begin{figure}[tp]
	\begin{figcenter}
	% Created by tikzDevice version 0.7.0 on 2014-06-05 14:10:37
% !TEX encoding = UTF-8 Unicode
\begin{tikzpicture}[x=1pt,y=1pt]
\definecolor[named]{fillColor}{rgb}{1.00,1.00,1.00}
\path[use as bounding box,fill=fillColor,fill opacity=0.00] (0,0) rectangle (433.62,144.54);
\begin{scope}
\path[clip] (  0.00,  0.00) rectangle (433.62,144.54);
\definecolor[named]{drawColor}{rgb}{1.00,1.00,1.00}
\definecolor[named]{fillColor}{rgb}{1.00,1.00,1.00}

\path[draw=drawColor,line width= 0.6pt,line join=round,line cap=round,fill=fillColor] (  0.00,  0.00) rectangle (433.62,144.54);
\end{scope}
\begin{scope}
\path[clip] ( 51.42, 34.03) rectangle (340.63,132.50);
\definecolor[named]{fillColor}{rgb}{0.90,0.90,0.90}

\path[fill=fillColor] ( 51.42, 34.03) rectangle (340.63,132.50);
\definecolor[named]{drawColor}{rgb}{0.95,0.95,0.95}

\path[draw=drawColor,line width= 0.3pt,line join=round] ( 51.42, 57.08) --
	(340.63, 57.08);

\path[draw=drawColor,line width= 0.3pt,line join=round] ( 51.42, 85.64) --
	(340.63, 85.64);

\path[draw=drawColor,line width= 0.3pt,line join=round] ( 51.42,114.19) --
	(340.63,114.19);

\path[draw=drawColor,line width= 0.3pt,line join=round] ( 93.78, 34.03) --
	( 93.78,132.50);

\path[draw=drawColor,line width= 0.3pt,line join=round] (166.81, 34.03) --
	(166.81,132.50);

\path[draw=drawColor,line width= 0.3pt,line join=round] (239.84, 34.03) --
	(239.84,132.50);

\path[draw=drawColor,line width= 0.3pt,line join=round] (312.87, 34.03) --
	(312.87,132.50);
\definecolor[named]{drawColor}{rgb}{1.00,1.00,1.00}

\path[draw=drawColor,line width= 0.6pt,line join=round] ( 51.42, 42.81) --
	(340.63, 42.81);

\path[draw=drawColor,line width= 0.6pt,line join=round] ( 51.42, 71.36) --
	(340.63, 71.36);

\path[draw=drawColor,line width= 0.6pt,line join=round] ( 51.42, 99.91) --
	(340.63, 99.91);

\path[draw=drawColor,line width= 0.6pt,line join=round] ( 51.42,128.46) --
	(340.63,128.46);

\path[draw=drawColor,line width= 0.6pt,line join=round] ( 57.26, 34.03) --
	( 57.26,132.50);

\path[draw=drawColor,line width= 0.6pt,line join=round] (130.29, 34.03) --
	(130.29,132.50);

\path[draw=drawColor,line width= 0.6pt,line join=round] (203.32, 34.03) --
	(203.32,132.50);

\path[draw=drawColor,line width= 0.6pt,line join=round] (276.36, 34.03) --
	(276.36,132.50);
\definecolor[named]{drawColor}{rgb}{0.40,0.76,0.65}

\path[draw=drawColor,line width= 0.6pt,line join=round] ( 64.56, 38.51) --
	( 71.87, 43.40) --
	( 79.17, 46.45) --
	( 86.47, 49.41) --
	( 93.78, 55.46) --
	(101.08, 58.90) --
	(108.38, 61.56) --
	(115.69, 64.82) --
	(122.99, 67.07) --
	(130.29, 69.03) --
	(137.60, 92.09) --
	(144.90, 96.96) --
	(152.20,100.00) --
	(159.51,102.06) --
	(166.81,103.79) --
	(174.11,105.37) --
	(181.41,106.66) --
	(188.72,107.75) --
	(196.02,108.70) --
	(203.32,109.59) --
	(210.63,110.39) --
	(217.93,111.12) --
	(225.23,111.85) --
	(232.54,112.59) --
	(239.84,113.31) --
	(247.14,114.00) --
	(254.45,114.62) --
	(261.75,115.21) --
	(269.05,115.77) --
	(276.36,116.40) --
	(283.66,119.68) --
	(290.96,121.82) --
	(298.27,123.58) --
	(305.57,125.00) --
	(312.87,126.17) --
	(320.18,127.16) --
	(327.48,128.02);
\definecolor[named]{drawColor}{rgb}{0.99,0.55,0.38}

\path[draw=drawColor,line width= 0.6pt,line join=round] ( 64.56, 55.85) --
	( 71.87, 60.26) --
	( 79.17, 62.86) --
	( 86.47, 64.74) --
	( 93.78, 66.26) --
	(101.08, 67.50) --
	(108.38, 74.30) --
	(115.69, 79.25) --
	(122.99, 81.98) --
	(130.29, 84.65) --
	(137.60, 86.51) --
	(144.90, 88.87) --
	(152.20, 90.77) --
	(159.51, 92.33) --
	(166.81, 93.82) --
	(174.11, 95.19) --
	(181.41, 96.40) --
	(188.72, 97.69) --
	(196.02, 98.76) --
	(203.32, 99.69) --
	(210.63,100.51) --
	(217.93,101.33) --
	(225.23,102.06) --
	(232.54,102.93) --
	(239.84,103.96) --
	(247.14,104.97) --
	(254.45,105.85) --
	(261.75,106.70) --
	(269.05,107.74) --
	(276.36,110.53) --
	(283.66,113.29) --
	(290.96,115.31) --
	(298.27,117.53) --
	(305.57,119.21) --
	(312.87,121.17) --
	(320.18,123.13);
\definecolor[named]{fillColor}{rgb}{0.40,0.76,0.65}

\path[fill=fillColor] ( 64.56, 38.51) circle (  1.60);

\path[fill=fillColor] ( 71.87, 43.40) circle (  1.60);

\path[fill=fillColor] ( 79.17, 46.45) circle (  1.60);

\path[fill=fillColor] ( 86.47, 49.41) circle (  1.60);

\path[fill=fillColor] ( 93.78, 55.46) circle (  1.60);

\path[fill=fillColor] (101.08, 58.90) circle (  1.60);

\path[fill=fillColor] (108.38, 61.56) circle (  1.60);

\path[fill=fillColor] (115.69, 64.82) circle (  1.60);

\path[fill=fillColor] (122.99, 67.07) circle (  1.60);

\path[fill=fillColor] (130.29, 69.03) circle (  1.60);

\path[fill=fillColor] (137.60, 92.09) circle (  1.60);

\path[fill=fillColor] (144.90, 96.96) circle (  1.60);

\path[fill=fillColor] (152.20,100.00) circle (  1.60);

\path[fill=fillColor] (159.51,102.06) circle (  1.60);

\path[fill=fillColor] (166.81,103.79) circle (  1.60);

\path[fill=fillColor] (174.11,105.37) circle (  1.60);

\path[fill=fillColor] (181.41,106.66) circle (  1.60);

\path[fill=fillColor] (188.72,107.75) circle (  1.60);

\path[fill=fillColor] (196.02,108.70) circle (  1.60);

\path[fill=fillColor] (203.32,109.59) circle (  1.60);

\path[fill=fillColor] (210.63,110.39) circle (  1.60);

\path[fill=fillColor] (217.93,111.12) circle (  1.60);

\path[fill=fillColor] (225.23,111.85) circle (  1.60);

\path[fill=fillColor] (232.54,112.59) circle (  1.60);

\path[fill=fillColor] (239.84,113.31) circle (  1.60);

\path[fill=fillColor] (247.14,114.00) circle (  1.60);

\path[fill=fillColor] (254.45,114.62) circle (  1.60);

\path[fill=fillColor] (261.75,115.21) circle (  1.60);

\path[fill=fillColor] (269.05,115.77) circle (  1.60);

\path[fill=fillColor] (276.36,116.40) circle (  1.60);

\path[fill=fillColor] (283.66,119.68) circle (  1.60);

\path[fill=fillColor] (290.96,121.82) circle (  1.60);

\path[fill=fillColor] (298.27,123.58) circle (  1.60);

\path[fill=fillColor] (305.57,125.00) circle (  1.60);

\path[fill=fillColor] (312.87,126.17) circle (  1.60);

\path[fill=fillColor] (320.18,127.16) circle (  1.60);

\path[fill=fillColor] (327.48,128.02) circle (  1.60);
\definecolor[named]{fillColor}{rgb}{0.99,0.55,0.38}

\path[fill=fillColor] ( 64.56, 55.85) circle (  1.60);

\path[fill=fillColor] ( 71.87, 60.26) circle (  1.60);

\path[fill=fillColor] ( 79.17, 62.86) circle (  1.60);

\path[fill=fillColor] ( 86.47, 64.74) circle (  1.60);

\path[fill=fillColor] ( 93.78, 66.26) circle (  1.60);

\path[fill=fillColor] (101.08, 67.50) circle (  1.60);

\path[fill=fillColor] (108.38, 74.30) circle (  1.60);

\path[fill=fillColor] (115.69, 79.25) circle (  1.60);

\path[fill=fillColor] (122.99, 81.98) circle (  1.60);

\path[fill=fillColor] (130.29, 84.65) circle (  1.60);

\path[fill=fillColor] (137.60, 86.51) circle (  1.60);

\path[fill=fillColor] (144.90, 88.87) circle (  1.60);

\path[fill=fillColor] (152.20, 90.77) circle (  1.60);

\path[fill=fillColor] (159.51, 92.33) circle (  1.60);

\path[fill=fillColor] (166.81, 93.82) circle (  1.60);

\path[fill=fillColor] (174.11, 95.19) circle (  1.60);

\path[fill=fillColor] (181.41, 96.40) circle (  1.60);

\path[fill=fillColor] (188.72, 97.69) circle (  1.60);

\path[fill=fillColor] (196.02, 98.76) circle (  1.60);

\path[fill=fillColor] (203.32, 99.69) circle (  1.60);

\path[fill=fillColor] (210.63,100.51) circle (  1.60);

\path[fill=fillColor] (217.93,101.33) circle (  1.60);

\path[fill=fillColor] (225.23,102.06) circle (  1.60);

\path[fill=fillColor] (232.54,102.93) circle (  1.60);

\path[fill=fillColor] (239.84,103.96) circle (  1.60);

\path[fill=fillColor] (247.14,104.97) circle (  1.60);

\path[fill=fillColor] (254.45,105.85) circle (  1.60);

\path[fill=fillColor] (261.75,106.70) circle (  1.60);

\path[fill=fillColor] (269.05,107.74) circle (  1.60);

\path[fill=fillColor] (276.36,110.53) circle (  1.60);

\path[fill=fillColor] (283.66,113.29) circle (  1.60);

\path[fill=fillColor] (290.96,115.31) circle (  1.60);

\path[fill=fillColor] (298.27,117.53) circle (  1.60);

\path[fill=fillColor] (305.57,119.21) circle (  1.60);

\path[fill=fillColor] (312.87,121.17) circle (  1.60);

\path[fill=fillColor] (320.18,123.13) circle (  1.60);
\end{scope}
\begin{scope}
\path[clip] (  0.00,  0.00) rectangle (433.62,144.54);
\definecolor[named]{drawColor}{rgb}{0.50,0.50,0.50}

\node[text=drawColor,anchor=base east,inner sep=0pt, outer sep=0pt, scale=  0.96] at ( 44.30, 39.50) {0,01};

\node[text=drawColor,anchor=base east,inner sep=0pt, outer sep=0pt, scale=  0.96] at ( 44.30, 68.05) {1};

\node[text=drawColor,anchor=base east,inner sep=0pt, outer sep=0pt, scale=  0.96] at ( 44.30, 96.61) {100};

\node[text=drawColor,anchor=base east,inner sep=0pt, outer sep=0pt, scale=  0.96] at ( 44.30,125.16) {10000};
\end{scope}
\begin{scope}
\path[clip] (  0.00,  0.00) rectangle (433.62,144.54);
\definecolor[named]{drawColor}{rgb}{0.50,0.50,0.50}

\path[draw=drawColor,line width= 0.6pt,line join=round] ( 47.15, 42.81) --
	( 51.42, 42.81);

\path[draw=drawColor,line width= 0.6pt,line join=round] ( 47.15, 71.36) --
	( 51.42, 71.36);

\path[draw=drawColor,line width= 0.6pt,line join=round] ( 47.15, 99.91) --
	( 51.42, 99.91);

\path[draw=drawColor,line width= 0.6pt,line join=round] ( 47.15,128.46) --
	( 51.42,128.46);
\end{scope}
\begin{scope}
\path[clip] (  0.00,  0.00) rectangle (433.62,144.54);
\definecolor[named]{drawColor}{rgb}{0.50,0.50,0.50}

\path[draw=drawColor,line width= 0.6pt,line join=round] ( 57.26, 29.77) --
	( 57.26, 34.03);

\path[draw=drawColor,line width= 0.6pt,line join=round] (130.29, 29.77) --
	(130.29, 34.03);

\path[draw=drawColor,line width= 0.6pt,line join=round] (203.32, 29.77) --
	(203.32, 34.03);

\path[draw=drawColor,line width= 0.6pt,line join=round] (276.36, 29.77) --
	(276.36, 34.03);
\end{scope}
\begin{scope}
\path[clip] (  0.00,  0.00) rectangle (433.62,144.54);
\definecolor[named]{drawColor}{rgb}{0.50,0.50,0.50}

\node[text=drawColor,anchor=base,inner sep=0pt, outer sep=0pt, scale=  0.96] at ( 57.26, 20.31) {0};

\node[text=drawColor,anchor=base,inner sep=0pt, outer sep=0pt, scale=  0.96] at (130.29, 20.31) {10};

\node[text=drawColor,anchor=base,inner sep=0pt, outer sep=0pt, scale=  0.96] at (203.32, 20.31) {20};

\node[text=drawColor,anchor=base,inner sep=0pt, outer sep=0pt, scale=  0.96] at (276.36, 20.31) {30};
\end{scope}
\begin{scope}
\path[clip] (  0.00,  0.00) rectangle (433.62,144.54);
\definecolor[named]{drawColor}{rgb}{0.00,0.00,0.00}

\node[text=drawColor,anchor=base,inner sep=0pt, outer sep=0pt, scale=  1] at (196.02,  8.53) {Number of problems};
\end{scope}
\begin{scope}
\path[clip] (  0.00,  0.00) rectangle (433.62,144.54);
\definecolor[named]{drawColor}{rgb}{0.00,0.00,0.00}

\node[text=drawColor,rotate= 90.00,anchor=base,inner sep=0pt, outer sep=0pt, scale=  1] at ( 16.80, 83.26) {Cumulative CPU time};
\end{scope}
\begin{scope}
\path[clip] (  0.00,  0.00) rectangle (433.62,144.54);
\definecolor[named]{fillColor}{rgb}{1.00,1.00,1.00}

\path[fill=fillColor] (349.49, 59.42) rectangle (412.71,107.11);
\end{scope}
\begin{scope}
\path[clip] (  0.00,  0.00) rectangle (433.62,144.54);
\definecolor[named]{drawColor}{rgb}{0.00,0.00,0.00}

\node[text=drawColor,anchor=base west,inner sep=0pt, outer sep=0pt, scale=  0.96] at (353.76, 96.21) {\bfseries Solver};
\end{scope}
\begin{scope}
\path[clip] (  0.00,  0.00) rectangle (433.62,144.54);
\definecolor[named]{drawColor}{rgb}{1.00,1.00,1.00}
\definecolor[named]{fillColor}{rgb}{0.95,0.95,0.95}

\path[draw=drawColor,line width= 0.6pt,line join=round,line cap=round,fill=fillColor] (353.76, 78.15) rectangle (368.22, 92.60);
\end{scope}
\begin{scope}
\path[clip] (  0.00,  0.00) rectangle (433.62,144.54);
\definecolor[named]{drawColor}{rgb}{0.40,0.76,0.65}

\path[draw=drawColor,line width= 0.6pt,line join=round] (355.21, 85.37) -- (366.77, 85.37);
\end{scope}
\begin{scope}
\path[clip] (  0.00,  0.00) rectangle (433.62,144.54);
\definecolor[named]{fillColor}{rgb}{0.40,0.76,0.65}

\path[fill=fillColor] (360.99, 85.37) circle (  1.60);
\end{scope}
\begin{scope}
\path[clip] (  0.00,  0.00) rectangle (433.62,144.54);
\definecolor[named]{drawColor}{rgb}{1.00,1.00,1.00}
\definecolor[named]{fillColor}{rgb}{0.95,0.95,0.95}

\path[draw=drawColor,line width= 0.6pt,line join=round,line cap=round,fill=fillColor] (353.76, 63.69) rectangle (368.22, 78.15);
\end{scope}
\begin{scope}
\path[clip] (  0.00,  0.00) rectangle (433.62,144.54);
\definecolor[named]{drawColor}{rgb}{0.99,0.55,0.38}

\path[draw=drawColor,line width= 0.6pt,line join=round] (355.21, 70.92) -- (366.77, 70.92);
\end{scope}
\begin{scope}
\path[clip] (  0.00,  0.00) rectangle (433.62,144.54);
\definecolor[named]{fillColor}{rgb}{0.99,0.55,0.38}

\path[fill=fillColor] (360.99, 70.92) circle (  1.60);
\end{scope}
\begin{scope}
\path[clip] (  0.00,  0.00) rectangle (433.62,144.54);
\definecolor[named]{drawColor}{rgb}{0.00,0.00,0.00}

\node[text=drawColor,anchor=base west,inner sep=0pt, outer sep=0pt, scale=  0.96] at (370.02, 82.07) {Standard};
\end{scope}
\begin{scope}
\path[clip] (  0.00,  0.00) rectangle (433.62,144.54);
\definecolor[named]{drawColor}{rgb}{0.00,0.00,0.00}

\node[text=drawColor,anchor=base west,inner sep=0pt, outer sep=0pt, scale=  0.96] at (370.02, 67.61) {\(\alpha=1\)};
\end{scope}
\end{tikzpicture}

	\end{figcenter}
	\caption{Accumulated runtime of the standard and greedy algorithm in the \emph{Black Hole} set, sorted by runtime individually for each variant. Note the logarithmic scale of the \(y\) axis.}
	\label{fig:cactus-greedy}
\end{figure}

	\section{Discussion}
% [todo] - introductory blurb



% [review] - review this section/file, it may be a bit osammanhängande

Analyzing the shortcomings of the algorithm, it is clear from \cref{tab:comparative-results} that it had significant difficulties in solving the problems from the \gls{cfn} category.
% large domains, large number of cost functions. memory issues? need background on toulbar first?
% most cost functions have arity 2, would better constraint update be useful?
% does cfn have a structure toulbar/cplex is using?

% [todo] - analyze which problems itm is bad at, an whether specialized updates may solve this
% [todo] - identify what problems ITM is good at

Compared to the results of CPLEX, the \gls{lp}-based competitor in the benchmark, the in-the-middle algorithm performs better in most sets.
This indicates that the translation of the algorithm to a max-sum based approach is a better approach than translating max-sum problems to \gls{lp} instances.
The only sets in which CPLEX performs decisively better (those from \gls{cfn}) mainly contain problems of an operations research character (resource allocation, planning \emph{etc.}).

The algorithm actually shows some promise in the \gls{wpms} category.
While most problem sets in this category were omitted due to memory concerns, the algorithm has good runtime performance in the remaining set.
The current implementation represents binary variables of the problem using two distinct variables internally, which could be improved by representing binary variables using only one variable.
This could improve both memory use and runtime, making the algorithm more interesting for approximative applications in the \gls{wpms} field.
The same variable representation issue applies to problems in the \emph{Auction} and \emph{DBN} sets.

One would expect the algorithm to perform well on problems with many hard clauses (due to the sparse representation of constraint components including these implicitly, resulting in fewer clauses to operate on in the constraint update), and this is the case for those sets included in the benchmark.
The algorithm performs well in all sets where a majority of the clauses are hard (\emph{On Call Rostering}, \emph{Parity Learning} and \emph{Pedigree}), but performs worse in sets where only \SIrange{25}{50}{\percent} of the clauses are hard (\emph{Max-Clique}, \emph{Protein Design} and \emph{Linkage}).

The number of zero-value clauses does not seem to have any effect on the performance of the algorithm, suggesting that the tie-breaking method employed efficiently resolves any resulting ties.


% [todo] - maybe mention that large cost function arities weren't tested, but may be interesting?

\subsection{Extensions and improvements}
% [todo] - introductory blurb


% [todo] - analyzing the results
% [todo] - were results as expected?
% [todo] - is a given improvement useful?


%%% THIS IS FOR THE PUSH VARIANT, MAKE THIS CLEAR
The reason for this is likely that the algorithm, in trials after the \enquote{push} operation has been applied, is more conservative than the original algorithm in that the maximum \(\kappa\) value will be lower. This means more trials fail to force an integer solution, reducing the number of trials and as a consequence reducing the runtime as well as increasing the optimality gap.
A more correct implementation would take into consideration the reduction of \(\kappa\) caused by the push operation when calculating the maximum \(\kappa\) value of subsequent trials.
% [review] - maybe not true
% [fix] - introduce this fix and re-run trials?

Due to these results, the \enquote{push} operation is not as interesting when applied to the max-sum algorithm as it is in the original \gls{lp} formulation.
% [todo] - analyze why it works on the CFN/Pedigree problem!



%%% THIS IS FOR THE GREEDY DP UPDATE, MAKE THIS CLEAR
In situations where finding a solution quickly is more important than finding a guaranteed optimal solution, the greedy DP update is therefore a viable alternative.
There is no problem set where the optimality gap is significantly increased, and only three sets contain problems which could be solved by the standard algorithm but not using greedy DP updates.


It seems that the reason for this is that the algorithm never reaches a situation where the solution doesn't change between two iterations, instead oscillating between a set of solutions.
This could potentially be resolved by increasing the threshold \(\epsilon\) and amplitude \(\zeta\) of the tie-breaking noise.

With these results in mind, the greedy algorithm may be very useful when exact solutions aren't required.
% [todo] - mention fields where a "good enough" solution is satisfactory and why
This makes the algorithm with greedy updates extremely competitive in such cases.
The greedy update variant may also be useful as part of a broader strategy, by for instance providing fast and good upper bounds or by providing fast unproven solutions while waiting for exact solvers.
Another interesting direction for the greedy DP update could be to run it in parallel with the standard algorithm (sharing the immutable constraint components in memory), again providing quick approximative solutions as well as good solutions.


	\chapter{Conclusions}
	In this final chapter I will attempt to consolidate the results of the thesis work, conclusively addressing the objective and purpose of the thesis itself.
I will do this by identifying already apparent applications of the in-the-middle algorithm within the field of graphical models while also identifying the limitations and weaknesses of the algorithm, and by pointing out some areas of further research which I believe are tractable and meaningful to pursue when it comes to applying the algorithm to optimization within that field.

\section{Applications}
Since the performance of the algorithm varies widely between different problem sets, one cannot immediately identify any single field in which the algorithm would be decisively competitive.
However, the good overall performance and low optimality gap of the algorithm --- especially when applied to problem sets in which other solvers struggle --- suggest that the algorithm may be a good candidate for a so-called portfolio approach, wherein several solvers are applied in parallel.

The same properties, especially the good optimality gap, suggest that the algorithm may be useful in applications where good upper bounds on the solution are required.
Since the greedy \gls{dp} variant of the algorithm provides such upper bounds quickly, that variant of the algorithm may have applications in branch-and-bound applications.
In combination with an exact solver, the in-the-middle algorithm may serve as an approximative anytime solver providing good approximative solutions while waiting for the exact solver to supply truly optimal solutions.

\section{Further research}
The algorithm shows promise in several categories of those included in the benchmark.
In these fields it would therefore be of interest to extend the algorithm and develop the underlying theory to further improve performance in these fields.

Previous work, mainly from \gls{inra}, has provided strong theory on graphical model optimization especially with respect to arc consistency \parencite{Cooper10,Cooper08,deGivry06,deGivry05}.
The in-the-middle algorithm, as previously detailed, uses a relatively weak arc consistency property (general arc consistency).
If stronger notions such as existential \parencite{deGivry05} or optimal soft \parencite{Cooper10} arc consistency can be extended to the framework used by the in-the-middle algorithm, it may increase runtime performance and solution quality.

Another area requiring further research is the design of the constraint updates, especially their effect on the optimality guarantee of the algorithm.
A common situation for the algorithm is that the heuristic finds optimal solutions but cannot guarantee the optimality, which in turn means the algorithm wastes time in subsequent trials.
Constructing constraint updates with better guarantees or providing stronger theory on other parts of the algorithm may therefore improve the performance of the algorithm.
Additionally, common constraint types such as the \emph{all-different} constraint may benefit from specialized constraint component updates.

There are also many tricks, some of them used in \gls{lp} implementations of the in-the-middle heuristic, that may be applied to the algorithm.
In this benchmark, only tie-breaking using noise was introduced.
Futher modifications that would require further research include not updating constraint components for which the associated variable components are unchanged and using the existing \gls{lp} or set-partitioning constraint update \parencite[\pno~102]{Wedelin08} for constraints of that type.
Both these modifications may improve runtime for some problems.

Finally, specializations of the implementation may prove interesting in some areas.
For example, the \gls{wpms} category of problems would benefit from a formulation where the binary variables are stored in a single, shared variable component element instead of two different ones as in the current implementation.
Similar simplifications based on various assumptions of the problem structure may be introduced in the constraint component updates.





	% [fix] - find published variants of unpublished works (incl. strange pubstates)
	% [fix] - check report for the ibidem thing, make sure it doesn't make things strange
	\printbibliography[heading=bibintoc]
	%\printindex
	\backmatter
\end{document}
