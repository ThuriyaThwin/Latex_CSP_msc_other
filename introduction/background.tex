\section{Background}
The in-the-middle algorithm \parencite{Wedelin95} for \gls{lp} problems can be interpreted as a coordinate descent algorithm, and with a simple heuristic it has been shown to solve integer programming problems with very good performance and quality in large scale applications.
The algorithm has been extended to also apply to max-sum problems \parencites{Wedelin08}{Wedelin13}, but this application has not yet been benchmarked against other algorithms and software in that field.
% [fix] - Wedelin has a comment saying "non-trivial?" here. Ask about it or try to understand what he means.
The primary aim of this thesis is therefore to implement and benchmark the in-the-middle algorithm applied to max-sum problems.

When formulated to solve max-sum problems, the algorithm is applicable to a host of problems including but not limited to (weighted, valued and regular) \glspl{csp}, \glspl{cfn} and \gls{wpms} problems.
Additionally, though a simple logarithmic transformation of the problem, \gls{map} optimization problems on \glspl{mrf} may also be restated as max-sum problems.
This makes the algorithm (including the approximate heuristic variant) interesting in fields ranging from scheduling, protein folding and combinatorial puzzles to classification and pattern recognition.

The \gls{csp} and \gls{maxsat} communities have been very active in recent years, producing benchmarks of solvers in those fields at the \emph{International Conference on Theory and Applications of Satisfiability Testing} \parencite{Argelich11} and the \emph{Annual Congress of the French Society of Operations Research and Decision Support} \parencite{Allouche14b}.
Using the results of such benchmarks, the in-the-middle algorithm on max-sum problems may be compared to existing algorithms in the mentioned fields, determining the interest of developing, analysing or specializing the algorithm further for such problems.
