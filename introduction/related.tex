\section{Related work}

% [review] - is this background?

\subsection{The in-the-middle algorithm}
Although the in-the-middle algorithm has mostly been discussed by its principal author \parencites{Wedelin95}{Wedelin08}{Wedelin13}{Alefragis00}, there is some litterature evaluating and improving upon the algorithm available.

\Textcite{Bastert10} notes the practical relevance of the field of heuristics for integer programming in recent years, and evaluates the in-the-middle algorithm applies to such problems. In addition to explicitly providing the generalizations mentioned by \textcite{Wedelin95}, they also introduce a \enquote{push} operation intended to improve the quality of the approximative solutions. Finally, they also compare the algorithm with commercial software, with fairly favourable results.

% [todo] - more related work?

\subsection{Graphical models and related problems}
Graphical models allow the modelling of a range of NP-hard optimization problems in a consistent manner \parencite{deGivry14}, and recent articles on the subject expose strong connections between linear programming and graphical models \parencites{Werner07}{Kolmogorov13}. As such, there is reason to believe that the application of linear programming algorithms, especially approximative ones, to graphical models may have practical relevance.

\Textcite{deGivry14} evaluate several exact optimizers to graphical model optimization problems, and the results indicate that there may be several problem domains (most notably \gls{wpms} and Max-\acrshort{csp}) which could benefit from fast, approximative solvers.
