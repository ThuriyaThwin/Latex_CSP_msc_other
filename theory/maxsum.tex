\section{Max-sum problems and constraint satisfaction}
\subsection{Max-sum problems}
The general max-sum problem is an NP-hard optimization problem with many applications in fields ranging from statistical physics to artificial intelligence and pattern recognition.
Problems including set partitioning \parencite[\pno~107]{Wedelin08}, max-flow/min-cut and (as we will see later) several variants of constraint satisfaction may be restated as max-sum problems.
Formally, we may use the following definition:
\begin{definition}[Max-sum problem] \label{def:max-sum}
	The max-sum problem is the optimization problem
	\begin{equation*}
		\max*[w] f(w) = \sum_k g_k(w^k) + C,
	\end{equation*}
	where \(g_k(w^k) \in \R\) are distinct arbitrary functions over \(w^k \subseteq w\) and \(C\) is a constant.
\end{definition}

% [review]  more terminology?

There are several algorithms available for solving max-sum problems, and \textcite{Werner07} mentions the \emph{augmented DAG algorithm}, the \emph{max-sum diffusion algorithm} \parencite{Flach98} and a \gls{lp} relaxation method.
In addition to those direct methods, the relation to \gls{csp} provides many more (which will be mentioned later), and algorithms such as \emph{belief propagation} and \emph{message passing} are also applicable to some max-sum problems (in particular, max-sum problems without loops).

% [todo] - more text

\subsubsection{Markov Random Fields}
A restricted variant of the max-sum problem called the \emph{binary max-sum labeling problem} has direct applications to artificial intelligence and pattern regocnition, where the problem is known as computing the \gls{map} configuration of \acrlongpl{mrf} \parencite[\pno~1165]{Werner07}.

% [todo] - more text

\subsection{Constraint satisfaction problems}
% http://costfunction.org/mobyle/htdocs/portal/help/CFN_intro.html
% http://carlit.toulouse.inra.fr/cgi-bin/awki.cgi/CpWcspFormats

\subsubsection{Weighted partial max-SAT}

\subsection{Translating CSP to max-sum}
