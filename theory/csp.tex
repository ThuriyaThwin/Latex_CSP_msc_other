\section{(Weighted) Constraint Satisfaction Problems}
Informally, a \gls{csp} may be described as the problem of assigning values to a set of \emph{variables} such that a number of \emph{constraints} are satisfied.
By annotating each constraint of a \gls{csp} with a value denoting the impact of its violation, a \gls{wcsp} is obtained --- the problem becomes to assign values such that the impact is minimized \parencite[\pno~219]{Bistarelli99}.
Each constraint will take different values (\emph{costs}) depending on the variable assignment, and the cost function will depend on these in some unspecified manner (commonly as the sum of all constraint costs).
Such problems are also called \glspl{cfn} in some litterature \parencite{Allouche14}.

A \gls{wcsp}, when defined as a maximization problem, has an associated semiring\footnote{In fact, even regular \glspl{csp} have an associated semiring: \(\langle \{0,1\}, \vee, \wedge, 0, 1 \rangle\).} \(\langle \Rminus\cup\{-\infty\}, \max*, +, -\infty, 0 \rangle\) \parencite[\pno~211]{Bistarelli99} which will be discussed further on.
Such a problem also includes a distinction between \emph{soft} and \emph{hard} constraints.
We will use a formal definition of \gls{wcsp} similar to the one given by \textcite[\pno~3]{Allouche14}\multfootnote{In fact, our definition is equivalent to the triple \(\langle X, W, \infty \rangle\) using the definition given by \textcite{Allouche14}, adapted for maximization instead of minimization.;Note that this definition may also be used to define regular \glspl{csp}, by only including hard constraints.}.

\begin{definition}[\Acrlong{wcsp}]
	A \gls{wcsp} is a tuple \(\langle X, W \rangle\) where \(X=\{1,\dotsc,n\}\) is a set of \(n\) discrete variables and \(W\) is a set of non-positive functions.
	Each variable \(i \in X\) has a finite domain \(D_i\) of values that can be assigned to it.
	A function \(w_S \in W\), with scope \(S \subseteq X\), is a function \(w_S : D_S \mapsto \{\alpha \in \Rminus \cup \{-\infty\}\}\).
\end{definition}

Using this definition, we can see that \(X\) defines the aforementioned \emph{variables} while \(W\) defines the \emph{constraints}.
Constraints for which \(w_S({x}) = -\infty\) for some input \({x}\) are \emph{hard} constraints, which must be satisfied.
Constraints with \(w_S({x}) < -\infty, \forall {x}\) are \emph{soft} constraints.

Formally, we may now express a solution to the \gls{wcsp} as an assignment \({x} \in D_X\), and we may note that all such solutions have a corresponding cost, defined by \(\cost{{x}} = \sum w_S({x})\).
We may then define a \emph{feasible solution} as a solution \({x}\) that satisfies \(\cost{{x}} > -\infty\).
In some cases we are only interested in finding a feasible solution, but whenever soft constraints are involved it is of interest to find the \emph{optimal solution}, \emph{i.e.} a solution \(\bar{{x}}\) for which \(\cost{\bar{{x}}} \geq \cost{{x}}, \forall {x}\) holds.

% [todo] - how does this affect things?
% [todo] - find a better reference, Bistarelli seems relevant as does Schiex95
\gls{csp}, and as a consequence \gls{wcsp}, is known to be NP-complete \parencite{Mackworth93}.

\subsubsection*{Binary representation of \acrshortpl{wcsp}}
\label{term:direct-encoding}
In order to solve \glspl{wcsp} using the in-the-middle method, the variable set \(X\) has to be transformed --- the in-the-middle algorithm was originally designed to solve binary \gls{lp} instances \parencite{Wedelin95}, and the variant discussed in this thesis is designed to operate on binary representations as well.

The \gls{wcsp} is transformed by \emph{encoding} each variable \(i\in X\) using what \textcite[\pno~5]{Allouche14} refers to as \emph{direct encoding}, which is a more general variation of the \emph{variable component} defined by \textcite[\pno~5]{Wedelin08}.

The encoding may be summarized as follows: for every variable \(i\in X\), introduce \(\abs{D_i}\) binary variables \(z_i^r, r\in D_i\) (and modify the affected constraints \(w_S\) in the obvious manner). 
Then, each variable \(z_i^r\) corresponds to the \(i\)th variable taking the \(r\)th value of its domain \(D_i\).
To ensure that exactly one value from each domain is chosen, additional constraints \(\sum_{r\in D_i} x_i^r = 1, \forall i\) are added --- we will see that these are implicitly enforced by the in-the-middle algorithm.

\subsection{Max-sum formulation of \acrshortpl{wcsp}}
A problem closely related to \glspl{wcsp} is the max-sum labeling problem.
\Textcite{Werner07} provides several links between the max-sum labeling problem and \gls{wcsp}, as well as providing link to the \gls{lp} approach to the max-sum problem (and thus implicitly to \gls{wcsp}).
The definition of a max-sum labeling problem is intuitive, and we will see that a \gls{wcsp} according to our definition may be rewritten as a max-sum problem.

% [review] - use a more compatible definition? is there one in Bistarelli97 or associated papers?
% there is also the "relaxed max-sum problem" on page 5 of Werner07
% the Werner07 explanation later (semirings) illustrates the connection between T/F <-> 0/-inf ands AND/OR <-> +/max from p. 3 of Wedelin08, which is useful but perhaps needs to be more explicit?
% Werner07 only compares max-sum and CSP; try to argue from the respective semirings directly instead of refering to Werner07?
\begin{definition}[Max-sum problem \parencite{Werner07}]
	A binary max-sum problem labeling problem is defined as maximizing a sum of unary and binary functions of discrete variables, \emph{i.e.} as computing
	\begin{equation*}
		\max[x\in X^T]*\left[
			\sum_{t\in T}g_t(x_t) + \sum_{\{t,t'\}\in E}g_{tt'}(x_t,x_{t'})
		\right],
	\end{equation*}
	where an undirected graph \((T,E)\), a finite set \(X\) and numbers \(g_t(x_t), g_{tt'}(x_t,x_{t'}) \in \R\cup\{-\infty\}\) are given.
\end{definition}

It is not immediately obvious that the \gls{wcsp} may be expressed as a max-sum problem.
\Textcite{Bistarelli97} provide a framework for \emph{semiring-based \glspl{csp}}, which \textcite{Werner07} uses to indicate that the regular \gls{csp} is closely related to the max-sum problem.

% [review] - Bistarelli99, page 211, defines the c-semiring exactly as in our case!
Even with our modified definition of \glspl{wcsp}, the reasoning in \textcite[\pno~230]{Bistarelli97} holds insofar that the semiring formulation of our definition, \(\langle \Rminus\cup\{-\infty\}, \max*, +, -\infty, 0 \rangle\), defines a c-semiring.
As mentioned by \textcite[\pno~3]{Werner07}, the max-sum problem is defined by the semiring \(\langle \R\cup\{-\infty\}, \max*, +, -\infty, 0 \rangle\), which means a \gls{wcsp} should be expressible in terms of a max-sum problem since the only difference is in their domain.
We will now describe an explicit transformation from our definition of \glspl{wcsp} into a max-sum problem, in order to use the in-the-middle algorithm on \glspl{wcsp} in general.

Consider the \enquote{original} \gls{wcsp} \(\langle X, W \rangle\). Now, let the finite set \(Z = \{z_i^r \mid i\in X, r\in D_i\}\) as given by the direct encoding described on \cpageref{term:direct-encoding}, and let the undirected graph \((T,E)\) ...
% [todo] - continue with explicit conversion

The resulting max-sum problem may be expressed in more intuitively as
\begin{equation*}
	\max*[x\in Z] \sum_{S} w_S(x).
\end{equation*}
