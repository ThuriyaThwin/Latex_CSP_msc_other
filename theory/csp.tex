\section{(Weighted) Constraint Satisfaction Problems}
Informally, a \gls{csp} may be described as the problem of assigning values to a set of \emph{variables} such that a number of \emph{constraints} are satisfied.
Allowing feasible solutions to be differentiated through costs yields a \gls{vcsp}, and the extended problem is then to assign values that satisfy the constraints while also minimizing a \emph{cost function}.
Each constraint will take different values (\emph{costs}) depending on the values chosen, and the cost function will depend on these in some unspecified manner.
Such a problem includes a distinction between \emph{soft} and \emph{hard} constraints.
% [todo] - check that this relation between VCSP and WCSP is correct

The cost function of a \gls{vcsp} may be restricted to be a linear combination of constraint costs, resulting in a \gls{wcsp} --- these are also called \glspl{cfn} in some litterature \parencite{Allouche14}. We will use a formal definition of \gls{wcsp} similar to the one given by \textcite[\pno~3]{Allouche14}\footnote{In fact, our definition is equivalent to the triple \(\langle X, W, \infty \rangle\) using the definition given by \textcite{Allouche14}, adapted for maximization instead of minimization.}\footnote{Note that this definition may also be used to define regular \glspl{csp}, by restricting the functions \(w_S\) such that \(w_S : D_S \mapsto \{\alpha \in \{0, -\infty\}\}\).}.

\begin{definition}[\Acrlong{wcsp}]
	A \gls{wcsp} is a tuple \(\langle X, W \rangle\) where \(X=\{1,\dotsc,n\}\) is a set of \(n\) discrete variables and \(W\) is a set of non-positive functions.
	Each variable \(i \in X\) has a finite domain \(D_i\) of values that can be assigned to it.
	A function \(w_S \in W\), with scope \(S \subseteq X\), is a function \(w_S : D_S \mapsto \{\alpha \in \Rminus \cup \{-\infty\}\}\).
\end{definition}

Using this definition, we can see that \(X\) defines the aforementioned \emph{variables} while \(W\) defines the \emph{constraints}.
Constraints for which \(w_S({x}) = -\infty\) for some input \({x}\) are \emph{hard} constraints, which must be satisfied.
Constraints with \(w_S({x}) < -\infty, \forall {x}\) are \emph{soft} constraints.

Formally, we may now express a solution to the \gls{wcsp} as an assignment \({x} \in D\), and we may note that all such solutions have a corresponding cost, defined by \(\cost{{x}} = \sum w_S({x})\).
We may then define a \emph{feasible solution} as a solution \({x}\) that satisfies \(\cos{{x}} > -\infty\).
In some cases we are only interested in finding a feasible solution, but whenever soft constraints are involved it is of interest to find the \emph{optimal solution}, \emph{i.e.} a solution \(\bar{{x}}\) for which \(\cost{\bar{{x}}} \geq \cost{{x}}, \forall {x}\) holds.

% [todo] binary formulation/encoding of the CSP

% [todo] - how does this affect things?
% [todo] - find a better reference, Bistarelli seems relevant as does Schiex95
\gls{csp}, and as a consequence \gls{wcsp}, is known to be NP-complete \parencite{Mackworth93}.

\subsection{Relation to max-sum problems}
A problem closely related to \glspl{wcsp} is the max-sum labeling problem.
\Textcite{Werner07} provides several links between the max-sum labeling problem and \gls{wcsp}, as well as providing link to the \gls{lp} approach to the max-sum problem (and thus implicitly to \gls{wcsp}).
The definition of a max-sum labeling problem is intuitive, and we will see that a \gls{wcsp} according to our definition may be rewritten as a max-sum problem.

\begin{definition}[Max-sum problem \parencite{Werner07}]
	A binary max-sum problem labeling problem is defined as maximizing a sum of unary and binary functions of discrete variables, \emph{i.e.} as computing
	\begin{equation*}
		\max_{x\in X^T}\left[
			\sum_{t\in T}g_t(x_t) + \sum_{\{t,t'\}\in E}g_{tt'}(x_t,x_{t'})
		\right],
	\end{equation*}
	where an undirected graph \((T,E)\), a finite set \(X\) and numbers \(g_t(x_t), g_{tt'}(x_t,x_{t'}) \in \R\cup\{-\infty\}\) are given.
\end{definition}

It is not immediately obvious that the \gls{wcsp} may be expressed as a max-sum problem.
\Textcite{Bistarelli97} provide a framework for \emph{semiring-based \glspl{csp}}, which \textcite{Werner07} uses to indicate that the regular \gls{csp} is closely related to the max-sum problem.
% [todo] - check the following claim!
Even with our modified definition of \glspl{wcsp}, the reasoning in \textcite[\pno~230]{Bistarelli97} holds insofar that the semiring formulation of our definition, \(\langle \Rminus\cup\{-\infty\}, \max*, +, -\infty, 0 \rangle\), defines a c-semiring.
As mentioned by \textcite[\pno~3]{Werner07}, the max-sum problem is defined by the semiring \(\langle \R\cup\{-\infty\}, \max*, +, -\infty, 0 \rangle\), which means a \gls{wcsp} should be expressible in terms of a max-sum problem since the only difference is in their domain.
We will now describe an explicit transformation from our definition of \glspl{wcsp} into a max-sum problem, in order to use the in-the-middle algorithm on \glspl{wcsp} in general.

% [todo] - provide explicit conversion
