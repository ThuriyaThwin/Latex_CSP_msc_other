\section[Linear programming theory]{\acrlong{lp} theory and the original in-the-middle algorithm}
The in-the-middle algorithm applied to \gls{lp} problems was originally described by \textcite{Wedelin95}, and has since been extensively applied to large \acrlong{lp} problems (specifically set covering problems) in the airline industry.
We will briefly describe the algorithm and an associated heuristic, in order to provide the reader with a good understanding of the algorithm, before we apply it to \gls{csp} and max-sum problems further on. A more recent description of the algorithm is also available \parencite{Wedelin13}, which considers some of the applications to max-sum problems.

As detailed by \textcite{Wedelin95}, the algorithm is designed to solve large scale binary integer programming problems.
In this thesis, we choose to consider problems
\begin{equation}
	\label{eq:lp-ilp}
	\begin{gathered}
		\max* cx \\
		\text{s.t. } Ax = b, \\
		x_j \in \{0,1\},
	\end{gathered}
\end{equation}
where \(A\) is binary\footnote{This may be extended to \(a_{ij}\in\{-1,0,1\}\) \parencite{Wedelin13}.} and \(b\) is integer.
This problem is NP-hard, and many difficult problems such as set partitioning and set covering may be conveniently stated in this way.

Observing the Lagrangian relaxation of \eqref{eq:lp-ilp}, \emph{i.e.}
\begin{equation}
	\label{eq:lp-ilp-relax}
	\max*[0\leq x\leq1] cx + y(b-Ax),
\end{equation}
we may define the \emph{reduced cost vector} \(\bar{c}=c-yA\).
It is then obvious that we may express the solution to \eqref{eq:lp-ilp-relax} as
\begin{equation*}
	x_j = 
	\begin{cases}
		1, &\quad \text{if \(\bar{c}_j > 0\)}, \\
		0\text{ or }1, &\quad \text{if \(\bar{c}_j = 0\)}, \\
		0, &\quad \text{if \(\bar{c}_j < 0\)}.
	\end{cases}
\end{equation*}
A unique integer solution to \eqref{eq:lp-ilp} is guaranteed if \(\bar{c}\) is nonzero (otherwise, finding a feasible solution is difficult), and it is therefore interesting to find some \(y\) for which this is true.
In cases where such \(y\) may not exist, the algorithm applies a heuristic (which will be described later).

\subsection{The in-the-middle algorithm}
The algorithm, when expressed without its approximative heuristic, may be viewed as a simple coordinate search.
Iterating over \(y\), each \(y_i\) is chosen so that the solution \(x\) to \eqref{eq:lp-ilp-relax} satisfies the \(i\)th constraint of \(Ax=b\).

The general idea is to choose \(y_i\) so that the corresponding elements of \(\bar{c}\), which we will denote \(\bar{c}^i\) (\emph{i.e.} \(\bar{c}^i = \{\bar{c}_j \mid a_{ij} = 1\}\)), are non-zero with exactly \(b_i\) positive elements.
Considering the case of binary \(a_{ij}\), we may choose two \emph{critical values} \(r^+\) and \(r^-\), where \(r^+\) is the \(b_i\)th largest element of \(\bar{c}^i\) and \(r^+\) is the \((b_i+1)\)th largest.
The new value \(y_i\) is now chosen as \(y_i \coloneq y_i + \frac{r^+ + r^-}{2}\).
By iteratively applying this update to each constraint until sign changes in \(\bar{c}\) no longer occur, a solution \(x\) which satisfies \eqref{eq:lp-ilp} may be obtained (assuming \(\bar{c}_j\neq0\)).
Pseudo-code describing these operations may be seen in \cref{alg:itm-lp}.

\begin{algorithm}[tbp]
	\Repeat{no sign changes in \(\bar{c}\)}{
		\ForAll{constraints \(i\) in \(A\)}{
			\(r^+ \leftarrow b_i\text{th largest element of }\bar{c}^i\) \;
			\(r^- \leftarrow (b_i+1)\text{th largest element of }\bar{c}^i\) \;
			\(y_i \leftarrow y_i + \frac{r^+ + r^-}{2}\) \;
		}
	}

	\caption{The in-the-middle algorithm without approximation.}
	\label{alg:itm-lp}
\end{algorithm}

For simple problems this algorithm is very efficient, but for many problems the algorithm converges to a situation where many elements in \(\bar{c}\) are \(0\), and no integer solution is found.
To remedy this problem, a heuristic is applied (thus making the algorithm approximative instead of exact).

\subsection{The in-the-middle heuristic}
The heuristic used in the in-the-middle algorithm is designed to enforce \(\bar{c}_j\neq0\) and as a consequence guarantee that the algorithm finds integer solutions to \eqref{eq:lp-ilp}.
Ideally, the modifications to \(\bar{c}\) should be as small as possible to avoid distorting the problem too much (and providing poor approximative solutions).
A parameter \(\alpha\) is therefore introduced, which will control the degree of approximation introduced by the heuristic.
When \(\alpha=0\), no approximation is applied.

In essence, the heuristic consists of a modification of the update procedure described above, where the dual solutions \(\bar{c}_j\) are modified with invariant offsets \(s_j\).
Before each update is performed, previous offsets \(s^i\) (where \(s^i\) is defined analogously to \(\bar{c}^i\)) are removed from the reduced costs \(\bar{c}^i\).
After the update, new offsets are re-applied to \(\bar{c}^i\), in effect moving the reduced costs away from \(0\).
The offsets are based on the difference between the two critical values used by the exact algorithm, so that
\begin{equation*}
	\bar{c}_j \coloneq \bar{c}_j + \alpha\sgn{\bar{c}_j}\frac{r^+-r^-}{2}.
\end{equation*}
The resulting pseudo-code may be seen in \cref{alg:itm-lp-heur}.

\begin{algorithm}[tbp]
	% [review] - express loops over j more elegantly
	\Repeat{no sign changes in \(\bar{c}\)}{
		\ForAll{constraints \(i\) in \(A\)}{
			\lForAll{\(j\)}{\(\bar{c}^i_j \leftarrow \bar{c}^i_j - s^i_j\)}
			\(r^+ \leftarrow b_i\text{th largest element of }\bar{c}^i\) \;
			\(r^- \leftarrow (b_i+1)\text{th largest element of }\bar{c}^i\) \;
			\(y_i \leftarrow y_i + \frac{r^+ + r^-}{2}\) \;
			\ForAll{\(j\)}{
				\(s^i_j \leftarrow \alpha\sgn{\bar{c}^i_j}\frac{r^+-r^-}{2}\) \;
				\(\bar{c}^i_j \leftarrow \bar{c}^i + s^i\) \;
			}
		}
	}

	\caption{The in-the-middle algorithm with approximation.}
	\label{alg:itm-lp-heur}
\end{algorithm}

The parameter \(\alpha\) is hard to choose, and it is difficult to know in advance what value of \(\alpha\) may work well for any given problem instance.
To address this issue, a sweep strategy may be applied, which starts at \(\alpha=0\) and progressively raises this value until an integer solution is found. 
In addition, ties in the problem may be resolved by adding random offsets to the reduced costs if deemed necessary.
